%!TEX root = thesis_single.tex

\chapter{Stratified Morse Theory}
\label{chp:Stratified Morse Theory}

Summary of the definitions and results of stratified Morse Theory.

\section{Stratified Spaces}
\label{sec:Stratified Spaces}
A decomposition of a topological space is a locally finite collection of disjoint locally closed subsets called pieces, such that the space is the union of its pieces and whenever the intersection of a piece $X$ with the closure of a piece $\overline{Y}$ is nonempty then $X$ is a subset of $Y$. A stratified space $Z$ is a subset of a manifold $M$ that is a subset that is decomposed into pieces which are pairwise disjoint smooth submanifolds of $M$. Each submanifold is called a stratum of the space. Some examples include all manifolds with boundary, which have two strata. The interior is a submanifold of the same dimension as $M$, and the boundary is one of dimension one lower. Similarly for manifolds with corners. The nodal cubic is also an example with two strata. The smooth part of the curve is a one dimensional submanifold, and the singular point is taken to be a stratum on its own. More generally algebraic varieties with singularities are stratified spaces. Every submanifold is trivially a stratified space with one strata.

In the following, let $p$ lie in the stratum $S$ in the stratified space $Z$ in the manifold $M$.

Whitney stratifications are a class of well behaved stratified space. They are ones that meet the following conditions.
\begin{enumerate}
\item
Locally finite - each point of $M$ has a neighborhood meeting only finitely many strata.

\item
Condition on the Boundary - for each stratum $S$, its boundary $(\overline{S} - S) \cap Z$ is the union of strata. Something that fails this condition is a two strata space of a plane with a ray emanating from it. The boundary of the ray is a point in the plane, and cannot be written as the union of strata. To `fix' this, one would have to decompose the space as a plane minus a point, the point, and the ray emanating from that point.

\item
Condition b - Whitney's condition b is a condition on how the tangent spaces of different strata fit together along joins. It is a local condition independent of coordinate chart, and so we formulate it for submanifolds of $\R^n$. Define the secant of two points as the line through the origin parallel to the line joining those two points. Take a pair $(X,Y)$ of submanifolds of $\R^n$, a point $y\in Y$ and let $m = \dim X$. Consider two sequences $\{x_i\}$ in $X$ and $\{y_i\}$ in $Y$, both converging to $y$ with $x_i \neq y_i$ for all $i$. Suppose that $T_{x_i}X$ converges to an $m$-plane $τ \subset \R^n$ and that the secants converge to some line $l$. The pair $(X,Y)$ satisfies condition b at $y$ if $l \subseteq τ$ for all such sequences. The pair is said to satisfy the condition if it does so at every point of $Y$.
\end{enumerate}

Mather\cite[Example 2.5]{Mather1970} gives a counterexample (though gives the wrong equation) of the logarithmic spiral as $X$ and the origin as $Y$. This spiral is defined as the one where the tangent line makes a constant angle with a radial line. Fix a line through the origin $l$ and take $l\cap X$ for the sequence of points $x_i$ converging to the origin. They all have the same tangent line $τ$. The secants are also all the same; they are just $l$. But $l$ and $τ$ are at an angle, and so there is no containment.

One may guess that there is also a condition a, but it turned out that condition b implies condition a, so we may focus on the latter.

A Whitney stratified space must be locally finite, meet the condition on the boundary and have condition b hold for every pair of strata. As the boundary of a stratum must be the union of strata, we need only check condition b for pairs where one is in the boundary of the other (otherwise there is are no sequences of points with a common limit). Whitney stratified spaces are well behaved in that they are all triangulable and the transverse intersection of two Whitney stratified spaces is again a Whitney stratified space.

The first result we wish to use is Thom's first isotopy lemma. Let $M$ and $P$ be smooth manifolds, $f: M \to P$ a smooth map, and $Z$ a Whitney stratified space. Suppose that $f|_Z : Z \to P$ is proper and for each stratum $S$, $f|_S$ is a submersion, then it is called a proper stratified submersion. The bundle $(Z,P,f)$ is locally trivial \cite[Prop 11.1]{Mather1970}. Another version\cite[p41]{Goresky1989} of the lemma, where $P=\R^n$, says there is a stratum preserving homeomorphism
\[
h : Z \to \R^n\times(f^{-1}(0) \cap Z)
\]
which is smooth on each stratum. In particular, the fibres of $f|_Z$ are homeomorphic.

Two submanifolds are transverse to one another if the direct sum of their tangent spaces span the tangent space of the ambient space. Two stratified spaces are transverse if every pair of their strata are transverse.  Let $N'$ be a smooth submanifold of $M$ such that it is transverse to $Z$, intersects $S$ at only $p$ and $\dim N' + \dim S = \dim M$. The normal slice $N(p$ of $p$ is the intersection $ N' \cap Z \cap B_δ(p)$, where $B_δ(p)$ is a small ball. By condition b, it is possible to choose $δ$ small enough that the sphere $\partial B_δ(p)$ is transverse to $Z$.

The link of a point is similar. It is defined to be $L(p) = N' \cap Z \cap \partial B_δ(p)$. Both of these spaces have a canonical Whitney stratification. Further, if one stratifies the cone of $L(p)$ in the obvious way, then there is a stratum preserving homeomorphism between the cone and the normal slice. For sufficiently small $δ$, the topological types of the normal slice and the link are independent of choice of $δ$, the metric used for the ball, the transverse submanifold $N'$ or even the point $p$ within the same connected component of the stratum $S$.

A generalised tangent space at $p$ is any subspace of $T_pM$ that is the limit of $T_{p_i}S'$ for a sequence $p_i$ in a stratum $S'$, such that $p_i \to p$.

A smooth function on $Z$ is one which is the restriction of a smooth function $\tilde f$ on $M$. A critical point $p$ of $f$ is one where $d\tilde f_p (T_p S) = 0$. That is, the differential annihilates the tangent space of the stratum. A Morse function is a smooth function on $Z$ such that
\begin{enumerate}
\item it is proper, and has distinct critical values.
\item for each stratum $S$, the critical points of $f|_S$ are nondegenerate.
\item for each generalised tangent space $Q$ at $p$, with the possible exception of $Q=T_pS$, we have $d\tilde f _p (Q) \neq 0$.
\end{enumerate}

The first two conditions are familiar from regular Morse theory. The third condition is a type of nondegeneracy between strata. Note in particular the special case of a zero-dimensional stratum. Then all its tangent spaces are the trivial vector space, and so every point in this strata is a critical point. We now suppose that $f$ is a Morse function with critical point $p$ and $v = f(p)$. The sublevel set $Z_{\leq a}$ is the space $Z \cap f^{-1}((-\infty,a])$. Likewise $Z_{[a,b]} = Z \cap f^{-1}([a,b])$ and $Z_a = Z \cap f^{-1}(a)$.

A pair $(A,B)$ is Morse data for $f$ at $p$ if there is an embedding $h : B \to Z_{\leq v-ε}$ such that $Z_{\leq v+ε}$ is homeomorphic to $Z_{\leq v-ε} \cup_B A$. There are two immediate candidates for Morse data, namely coarse and local Morse data, which are respectively $(Z_{[v-ε,v+ε]}, Z_{v-ε})$ and the intersection of these with a small ball centered at $p$. The local Morse data is well defined \cite[Prop 3.5.3]{Goresky1989} and is Morse data \cite[Thm 3.5.4]{Goresky1989}. It turns out though that the local Morse data can be computed in two parts; from a tangential and a normal component. Tangential Morse data is local Morse data for $f|_S$. And from regular Morse theory, this is homotopic to a disc and its boundary of dimension $λ$, $(D^λ, \partial D^{λ-1})$, where $λ$ is the index of the critical point (the number of negative eigenvalues of the Hessian).

The Normal Morse data comes from the normal slice, it is the local Morse data for $f|_{N(p)}$. It does not fall out of a computation as nice as Tangential Morse data, but can be described in terms of the half link. The half link $\ell$ is $L(p) \cap f^{-1}(v-ε)$. The Normal Morse data is homotopic to the pair $({\rm cone} (\ell), \ell)$.

Finally, the local Morse data of $f$ at $p$ is product of the Tangential and Normal Morse data. By product, we mean in the sense of pairs of spaces,
\[
(A,B) \times (C,D) = (A \times C, A \times D \cup B \times C).
\]



\begin{lem}
    \label{lem:cond_b_pair}
Suppose that $Z$ is a stratified space. Suppose that $Z$ is an (embedded) submanifold of $M$. Then $(X,Y)$ satisfies Whitney's condition b at $y$ whenever $X$ is open in $Z$.
\begin{proof}
Denote the dimension of $Z$ by $n$ and the dimension of $M$ by $m$. By the local immersion theorem, there is a coordinate neighbourhood $U'$ of $y$ in $Z$ and a coordinate neighbourhood $V$ of $y$ in $M$ such that the natural inclusion takes the form
\[
i : U' \to V, \;\; (x_1, \dots, x_n) \mapsto (x_1, \dots, x_n, 0, \dots, 0)
\]
and $y=0$ in both charts.

We now use the embedded submanifold assumption. To be embedded entails that the manifold topology of the submanifold is the same as subspace topology. For $U'$ to be open in $Z$, there must be an open set $U$ in $M$ such that $U' = U \cap Z$. Recall a sequence $(x_n)$ converges to a point, if for every open set containing that point there is a cutoff $N$ so that the tail sequence with $n>N$ lies in the open set. Suppose there is a sequence of points of $Z$ that converges to $y$ in the topology of $M$, then consider the open set $U$. It contains $y$ and so eventually the sequence must come to lie in $U$. As the sequence lies in $Z$ by hypothesis, it comes to lie in $U'$. Therefore, for sequences in $Z$ converging to $y$, we need only check sequences with points from $U'$.

We use the previously found coordinates to identify $V$ as an open subset in $\R^m$ and $U'$ is contained in $\{x_{n+1}=\dots = x_{m} = 0\} = \R^n \subset \R^m$; an open subset of $\R^n$.

Suppose we have the setup for condition b; two strata $X,Y$ of $Z$ with sequences $x_i, y_i$ converging to $0\in Y$. By earlier remarks, points of $X$ can only converge to $y$ if they eventually lie in $U'$ also. We can now assume that $x_i \in X\cap U'$. As $X$ and $U'$ are open in $Z$, so $T_{x_i} X = T_{x_i} Z = T_{x_i} U' = \text{span}_\R \{ \partial_{u_1}, \dots, \partial_{u_n} \} = \R^n \subset \R^m$. Further, $U'$ is contained in $u_{n+1}=\dots = u_{N} = 0$, so all secants are contained in $\R^n$. Hence if a limiting secant exists, it must be contained in the tangent planes, and so Whitney's condition b is fulfilled.

\end{proof}
\end{lem}


Note the necessity of the condition the submanifold has the subspace topology. In particular, \emph{immersed} submanifolds may fail the condition. Such a counterexample is the space $M = \R^2$, $X = \bigcup_{n\in \N} \{1/n\} \times \R$, $Y = \{0\}\times \R$, $y=(0,0)$. This is a collection of parallel lines spaced at the reciprocal integers. The sequence of points $x_n = (1/n, 0)$ in $X$ converges to $y$, but the limit of the tangent planes ($\{0\}\times\R$) does not contain the limit of the secant ($ \R\times\{0\} $). The proof fails in this case because it is no longer true that a sequence of points of $Z$ converging to $y$ need lie in $U'$. Here $U'$ is the line $Y$, and there is no open set $U\subset \R^2$ with $U' = U \cap Z$.





\begin{lem}
    \label{lem:startified_pair}
Suppose that $Z$ is a closed submanifold of $M$. Let $Y$ be a closed submanifold of $Z$ and $X := Z \setminus Y$. Then $X\cup Y$ is a Whitney stratification of $Z$.

\begin{proof}
Because $Z$ is closed in $M$, $\overline{X} = Z$ and so its boundary is $Y$. Also $X$ is open in $Z$ and so by the previous lemma $Z$ satisfies Whitney's condition b.
\end{proof}
\end{lem}







\begin{lem}
    \label{lem:stratified_split}
Suppose that a stratum $X$ is decomposed into the finite disjoint union of pieces $X_i$, each of which is open and closed in $X$ and have the same boundary as $X$. Then the stratified space formed where each is considered as a separate stratum is Whitney stratified.

\begin{proof}
We have turned one stratum into a finite number of strata, so the result is still locally finite. If a stratum had $X$ in its boundary, now it has $\bigcup X_i$ in its boundary, and so continues to be a union of strata. And the boundary of each $X_i$ is the same as the boundary of $X$, which was a union of strata already. Finally, we no not need to check Condition b for pairs $X_i,X_j$ as no sequence in one can have a limit in another, as they are closed in $X$. As for pairs where one of the members is an $X_i$, consider the same sequence in $X$, for which condition b is known to hold.
\end{proof}
\end{lem}





\begin{lem}
    \label{lem:stratified_remove}
If one takes a Whitney stratifies space and removes a stratum that is not the boundary of any other, then the result is again a Whitney stratified space.

\begin{proof}
The result is clearly locally finite, as we have only removed strata. Condition b still holds between all remaining pairs. Finally, the condition on the boundary holds for the remaining strata, as the stratum removed is not in the boundary of any remaining piece.
\end{proof}
\end{lem}






\begin{lem}
    \label{lem:stratified_level_set}
Suppose $\overline B = B \cup \partial B$ is a closed half space in $M$. That is, $B$ is an open submanifold of $M$, $\partial B$ (the boundary of $B$) is a closed submanifold and both $B$ and $M\setminus \overline{B}$ are simply connected. If $q$ is a regular value of $T$ and $T^{-1}(q)$ is transverse to $\partial B$, then $Z = T^{-1}(q) \cap \overline B$ is a Whitney stratified space.

\begin{proof}
Let $Z = T^{-1}(q)$. It is a closed submanifold of $M$. By the transverse assumption, $Y = Z\cap \partial B$ is a closed submanifold of $Z$. Hence we are in the situation of Lemma \ref{lem:startified_pair} and letting $X = Z\setminus Y$, we have that $Z = X \cup Y$ is a Whitney stratified space. Now split $X$ into the two pieces $X_1 = X \cap B$ and $X_2 = X \cap (M \setminus \overline{B})$. Then by Lemma \ref{lem:stratified_split}, $Z = X_1 \cup X_2 \cup Y$ is a Whitney stratified space and by Lemma \ref{lem:stratified_remove}, $Z \cap \overline{B} = X_1 \cup Y$ is too.
\end{proof}
\end{lem}


% \section{Implicit Function Theorems}
% \label{sec:Implicit Function Theorems}
%
%
%
% \begin{lem}
% Suppose, as for the implicit function theorem, that $c$ is a regular value of $F : \R^{n+m} \to \R^n$. Let $p \in \R^{n+m}$ be a point in the level set $Z = F^{-1}(c)$. Then every open neighbourhood of $p$ in $Z$ has an open subset containing $p$ such that this subset is the intersection of $Z$ and an open set in $\R^{n+m}$.
%
% \begin{proof}
% We start by using the implicit function theorem at $p$. After possibly reordering the coordinates, there exists open sets $U \subset \R^n$ and $V \subset \R^m$ and a smooth function $g : U \to V$ such that $p \in U\times V$ and
% \[
% \Set{(x,g(x))}{x \in U} = \Set{ (x,y) \in U\times V}{F(x,y) = c}.
% \]
% If we rephrase this into a chart for $Z$ at $p$, let $W = Z \cap (U\times V)$ be the coordinate neighbourhood and define the chart $φ$ by
% \begin{align*}
% φ : W &\to U            &          φ^{-1} : U &\to W \\
% (x,y) &\mapsto x        &                   x &\mapsto (x, g(x)).
% \end{align*}
% Now take any open neighbourhood $A$ of $p$ in $Z$. By intersecting it with $W$ we have subset, $A' := A \cap W$, that fits inside the coordinate chart. We will show that $A'$ is the restriction of an open subset of ambient space $\R^{n+m}$. Symbolically, we will show that
% \[
% A' = Z \cap (φ(A') \times V).
% \]
% Take any point $q$ in $A'$. By the definition of $A$ as a subset of $Z$, $q$ lies in $Z$, so must show it lies in the product. As $q$ lies in $W$,  we may apply the chart map to it to conclude that $q = φ^{-1} \circ φ(q) = (φ(q), g(φ(x)))$, which plainly lies in $φ(A') \times V$.
%
% To see the reverse inclusion, now suppose that $q \in Z \cap (φ(A') \times V)$. Initially, we can say $q$ is in $W$, as is seen from the progression $A' \subset W$, $φ(A') \subset φ(W) = U$, $φ(A') \times V \subset U \times V$ and finally $Z \cap (φ(A') \times V) \subset Z \cap (U\times V) = W $. Let $q = (x,y)$ for some $x \in φ(A')$ and $x = φ(q)$ as $φ$ acts by projection onto the first component. Let $q' \in A' \subset W$ be the preimage of $x$, ie $x = φ(q')$. $φ$ is a bijection on $W$ so $q=q'$ and the equality is proved.
%
% This exhibits $A'$, an open subset of $A$, as the intersection of $Z$ and an open set $φ(A') \times V$ in $\R^{n+m}$.
% \end{proof}
% \end{lem}

\section{Morse Theory}
\label{sec:Morse Theory}

\begin{lem}
Given functions $F, h: U \subset \R^n \to \R$. Suppose that $c$ is a regular value of $F$. $h$ has a critical point on $F^{-1}(c)$ at $p$ exactly when $\nabla h(p) || \nabla F(p)$.
\begin{proof}
First note that $\nabla F(p)$ is nonzero and $F^{-1}(c)$ is a manifold by the implicit function theorem. This provides the existence of $n-1$ local coordinates $u^j$ on $F^{-1}(c)$ near $p$. Let $\mathbf x = (x^i)$ be the standard coordinates of $\R^n$. By the chain rule
\[
\Partial{F}{x^i} \Partial{x^i}{u^j} = \nabla F \cdot \Partial{\mathbf{x}}{u^j} = 0
\]
So in one direction, if $\nabla h (p) = λ \nabla F(p)$, then immediately we have that
\[
0 = λ \nabla F \cdot \Partial{\mathbf{x}}{u^j} = \nabla h(p) \cdot \Partial{\mathbf{x}}{u^j} = \Partial{h}{x^i} \Partial{x^i}{u^j}(p) = \frac{d h}{d u^j}(p)
\]
so $h$ has a critical point at $p$. Conversely, suppose that $h$ has a critical point at $p$. Then consider the matrix $A$,
\[
A = \begin{pmatrix}
\nabla h \\
\nabla F
\end{pmatrix}
\]
with the two gradients as rows. By assumption, this matrix acting on the tangent vectors $\partial \mathbf{x} / \partial u^j$ yields zero. Hence the dimension of the kernel is at least $n-1$ (the number of tangent vectors). On the other hand, $\nabla F$ is non-zero so it has an image of dimension one or greater. As we are mapping from $\R^n$, it must be that $A$ is exactly rank $1$ and its rows have a linear dependence. Therefore $\nabla h(p) || \nabla F(p)$.
\end{proof}
\end{lem}
