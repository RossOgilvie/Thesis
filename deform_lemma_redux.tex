\section{Deformation Theory}
\label{sec:Deformation Theory}

\subsection{Ideas for improvment of this section}
\label{sec:Ideas for improvment}
I need to incorporate tha arguments that the solutions are real polynomials. Perhaps should investigate whether a coordinate change to one on which the real involution acts as complex conjugation would simplify that without throwing out the degree calculations.

Should change the focus to that of solving \eqref{eq:EMPDi}, with \eqref{eq:Q} recast as a necessary condition. Emphasise that it ensures the two solutions for $P$ are equal.

Factor the generic lemma into two parts, one getting the c's and other to solve the $\dot b$ equation, for greater/cleaner reuse in the bajillion variation lemmas.

Beef up the variations to cover higher order roots and multiple roots in common.

Switch to consistently writing the common factor as $F$.

Think up of better notation for reduced (currently tilde), base solutions (currently bold), things with $(ζ^2-1)$ included/not (currently ).

Include statement of Bezout's identity in the form we want it.

\subsection{Set Up} % (fold)
Let the spectral curve be $\eta^2 = P(\zeta)$, where $P$ is a real polynomial of degree $2g+2$ (so it has genus $g$). Let the roots of $P$ be $\alpha_i$. They occur in conjugate inverse pairs. We use the convention that a lower index on a polynomial/power series denotes its coefficents. eg $P = P_0 + P_1\zeta + P_2\zeta^2 + \dots$. Dashes denote differentiation with respect to $\zeta$ and dot for differentiation with respect to $t$, the deformation parameter, evaluated at $t=0$.

The meromorphic differentials $Θ^i = dq^i$ have double poles above $0$ and $\infty$, and so can be written as
\[
dq^i = \frac{1}{\zeta^2\eta}b^i(\zeta) d\zeta
\]
for a polynomial $b^i$ of degree $g+3$. Further, the reality condition of $\Theta^i$ means that $b^i$ is a real polynomial. $q^i = \log \mu^i$ is locally defined up to periods and a constant. If the deformation flow preserves the periods, then the integrality of the periods means that their $t$-derivative is zero. Thus $\dot q^i$ is a well defined meromorphic function. At $ζ=0$ (or $\infty$), we can differentiate the power series expansion for $q^i$ to obtain (letting $f$ be a holomorphic function)
\begin{align*}
q^i &= \frac{1}{\zeta\eta}f(\zeta) \\
\dot q^i &= \frac{1}{\zeta\eta} \left(-\frac{1}{2}\frac{\dot P}{P}f + \dot f\right),
\end{align*}
showing that $\dot q^i$ has a simple pole at $0$, and similiarly showing that is has simple poles at the roots of $P$ and thus is of the form
\[
\dot{q}^i = \frac{1}{\zeta\eta}\hat c^i(\zeta)
\]
for some degree $g+3$ polynomial $\hat c^i$. The reality condition on $q^i$ implies that this is an imaginary polynomial (ie that $i \hat c^i$ is a real polynmomial).

The Sym point condition, required to have an actual harmonic map, is that $\mu^i(\pi^{-1}(\pm 1)) = 1$, and can be expressed as a period-type condition. Let $γ^+$ and $γ_-$ be a path in the spectral curve connecting the two points above $ζ=1,-1$ respectively then
\begin{align*}
\int_{\gamma^i} dq^i = q^i(1^+) - q^i(1^-) \in 2\pi i \Z
\end{align*}
Hence the t-derivative of this integral is 0. Applying this to the middle term means that $\dot q^i(1^+) = \dot q^i(1^-)$. Thus $\hat c^i$ must have a fator of $\zeta^2-1$. Let $\hat c^i(\zeta) = (\zeta^2 - 1) c^i(\zeta)$, with $c^i$ a real polynomial of degree $g+1$.


\subsection{Forward direction}
From the equality of mixed partial derivatives
\begin{align*}
\dot{(dq^i)} &= \frac{d\zeta}{\zeta^2\eta}\left( -\frac{1}{2}\frac{\dot P}{P}b^i + \dot b^i \right) \\
d\dot q^i & = \frac{1}{\zeta^2\eta}\left( -\hat c^i -\frac{1}{2}\frac{P'}{P}\zeta\hat c^i + \zeta\hat {c^i}'\right)d\zeta \\
-\dot P b^i + 2P \dot b^i & = -2P \hat c^i -P'\zeta\hat c^i + 2P\zeta\hat {c^i}' \\
\dot P b^i - 2P\dot b^i &= 2P\left( \hat c^i - \zeta\hat {c^i}'\right) + P'\zeta\hat c^i \labelthis{eq:EMPDi}
\end{align*}
This gives two equations. If we multiply by alternatively by $\hat c^2$ and $\hat c^1$ and subtracting
\begin{align*}
\dot P b^1\hat c^2 - 2P \dot b^1\hat c^2 & = 2P \hat c^1\hat c^2 +P'\zeta\hat c^1\hat c^2 - 2P\zeta\hat {c^1}'\hat c^2 \\
\dot P b^2\hat c^1 - 2P \dot b^2\hat c^1 & = 2P \hat c^2\hat c^1 +P'\zeta\hat c^2\hat c^1 - 2P\zeta\hat {c^2}'\hat c^1 \\
\dot P (b^1\hat c^2 - b^2\hat c^1) - 2P (\dot b^1\hat c^2 - \dot b^2\hat c^1) & = - 2P\zeta(\hat {c^1}'\hat c^2 - \hat {c^2}'\hat c^1) \\
\dot P (b^1\hat c^2 - b^2\hat c^1) & =  2P(\dot b^1\hat c^2 - \dot b^2\hat c^1 - \zeta\hat {c^1}'\hat c^2 + \zeta\hat {c^2}'\hat c^1)\labelthis{eq:diffed}
\end{align*}

We now make the assumption that the spectral curve is nonsingular. This is to say that $P$ has only simple roots, or in terms of the factorisation of polynomials (which we shall shortly care about), we assume that $gcd(P,P') = 1$. This assumption always holds in low spectral genus.

From \eqref{eq:diffed} we can conclude that $P$ divides $b^1\hat c^2 - b^2\hat c^1$ in the following way. If $P$ and $\dot P$ have a common root $\alpha$, then from \eqref{eq:EMPDi} we get that
\begin{align*}
\dot P(\alpha) b^i(\alpha) &= 2P(\alpha)\left( \dot b^i(\alpha) + \hat c^i(\alpha) - \alpha\hat {c^i}'(\alpha)\right) +P'(\alpha)\alpha\hat c^i(\alpha) \\
0 &= 0 + P'(\alpha)\alpha\hat c^i(\alpha)
\end{align*}
As $P(0)\neq 0$, we cannot have $\alpha=0$. And by the assumption of nonsingularity, $P'(\alpha)\neq 0$. Thus we can conclude that $c^i(\alpha)=0$. We note that any other roots of $P$ must be roots of $b^1\hat c^2 - b^2 \hat c^1$ directly from \eqref{eq:diffed}. Hence $P$ divides $b^1\hat c^2 - b^2 \hat c^1$ and we can conclude that there is some degree 4 polynomial $\hat Q$ such that
\[
b^1 \hat c^2 - b^2 \hat c^1 = \hat Q P
\]
As $\zeta^2-1$ is a factor of both $c$'s, and $P$ has no zeroes on the unit circle, it must also be a factor of $\hat Q$. Define $\hat Q = (\zeta^2-1)Q$ to give
\[
b^1 c^2 - b^2 c^1 = Q P \labelthis{eq:Q}
\]
for some real quadratic polynomial $Q$.

The next question that arises is whether at a given point $(P,b^1,b^2)$ the $\hat c$'s and $Q$ are uniquely determined by a tangent vector $(\dot P, \dot b^1, \dot b^2)$. Since the equations are linear in the $t$-derivative, we need only consider the zero tangent vector. But then
\[
0 = 2P\left( \hat c^i - \zeta\hat {c^i}'\right) + P'\zeta\hat c^i
\]
and the assumption of a nonsingular spectral curve means that $P$ and $P'$ have no common factors and by evaluation at the roots of $P$ implies that $\hat c^i=0$. This immediately implies that $Q=0$ also. So every real quadratic $Q$ is associated to at most one tangent vector to the space of spectral data.


\subsection{Reversing}
\begin{lem}[Generic Version]
Take an intial $(P,b^1,b^2)$, with all roots distinct, and any real quadratic polynomial $Q$. Let the roots of $b^i$ be denoted as $\{\beta^i_j\}$ and let
\[
R^i = \sum_{j=1}^{g+3} \frac{Q(\beta^i_j) P(\beta^i_j)}{b^{3-i}(\beta^i_j) {b^i}'(\beta^i_j)}.
\]
Then for every such $Q$ with $R^i = 0$ for $i=1,2$, there is a tangent to the space of spectral data $(\dot P, \dot b^1, \dot b^2)$.
\end{lem}

Proof

We first attempt to solve \eqref{eq:Q} for $c^i$. But if we consider this as a linear system on the coefficents of $c^i$, then it is overdetermined, but nicely seaprates into two indepenedent subsystems by considering it at the values at the roots of the $b$'s. Suppose then for a moment that the degree of each $c^i$ was $g+2$. Then the matrix is the Vandemonde matrix, and is nonzero exactly when the roots of the $b$'s are distinct. Thus since we have distinct roots, there is a unique solution for the $c$'s. This solution is given by a linear combination the Legrende polynomials. The $j$-th Legrende polynomial $L^i_j(\zeta)$ is the unique polynomial that takes the value $1$ at $\beta^i_j$ and is zero at the other roots. We can write
\[
c^i(\zeta) = - \sum_{j=1}^{g+3} \left( \frac{QP\zeta}{b^{3-i}(b^i)'}\right)(\beta^i_j) \frac{b^i(\zeta)}{\zeta-\beta^i_j}
\]
The $g+2$ degree coeffient of $c^i$ is therefore $-b^2_{g+3}R^i$. But under the assumptions, this is zero, therefore we have found a unique solution for $c^i$ of degree $g+1$.

Moving on to \eqref{eq:EMPDi}, we now attempt to solve for the dotted quantities. Using Bezout's Lemma, it has a solution if and only if $\gcd(P,b^i)$ divides the right hand side. But under our asusmptions, this is one and so there is always a solution. The first concern is that the two equations for $i=1,2$ may give different solutions for $\dot P$. But this cannot be, since $\dot P$ is determined by the value of the equation at the roots of $P$, namely
\[
-\dot P(\alpha) b^i(\alpha) = -P'(\alpha)\alpha\hat c^i(\alpha)
\]
The polynomials $\hat c^i$ are determined by \eqref{eq:Q} and so if $\dot P^1$ and $\dot P^2$ are the two solutions arrising from the two equations then
\begin{align*}
b^1(\alpha) c^2(\alpha) - b^2(\alpha) c^1(\alpha) &= Q(\alpha) P(\alpha) = 0 \\
\dot P^1(\alpha)
&= -P'(\alpha)\alpha \frac{\hat c^1(\alpha)}{b^1(\alpha)} \\
&= -P'(\alpha)\alpha \frac{\hat c^2(\alpha)}{b^2(\alpha)} \\
&= \dot P^1(\alpha).
\end{align*}

The second consideration is that this solution may not be unique. Indeed it is not. Let $\dot {\bf P}, \dot {\bf b}^1, \dot {\bf b}^2$ be a solution. Then the other solutions are given by, for $r\in \R$,
\begin{align*}
\dot P &= \dot {\bf P} + 2rP \\
\dot b^i &= \dot {\bf b}^i + rP \\
\end{align*}
But if we fix a scaling of $P$, say $\abs{P_0} = const$, then differentiating we obtain
\[
\arg \dot P_0 = \pm \arg i P_0
\]
which it leads to a picture

\begin{center}
\begin{tikzpicture}[scale=1.5]
    % Draw axes
    \draw[<->] (-3,0) -- (3,0) node[right] {$z\in\C$};
	  \draw[<->] (0,-2) -- (0,2);
    % Draw P_0
    \draw [style=help lines] (0,0) -- (2,1) ;
		\fill (2,1) circle (1pt) node[right]{$P_0$};
		% draw the lines
    \draw (-2,0) -- (2,2) node[right]{$\{z=\dot {\bf P}_0 + 2rP_0\}$};
		\fill (1,1.5) circle (1pt) node[above]{$\dot {\bf P}_0$};

    \draw (-1,2)  -- (1,-2) node[right]{$\{\arg z = \pm i \frac{P_0}{const}\}$};

		\fill (-0.4,0.8) circle (2pt) node[above left=2pt]{$\dot P_0$};
\end{tikzpicture}
\end{center}
And so we see if we fix a scaling of the spectral curve, then there is a unique solution to \eqref{eq:EMPDi}, and hence a unique infinitesimal deformation. This data then amounts to a first order linear ODE, which can be solved to give a unique deformation of the spectral data.
\todo{show the resulting stuff are real polys}
\qed









\begin{lem}[$b^1$ and $P$]
Suppose that $b^1$ and $P$ share $d$ common roots $α_i\in D$ and that $b^2(α_i)\neq 0$. Then for every real quadratic polynomial $Q$ satisfying $R^2=0$ as before, there is a solution $(\dot P, \dot b^1, \dot b^2)$ to \eqref{eq:EMPDi}.
\end{lem}

Proof.

As $b^1$ is real and $α_i$ cannot lie on the unit circle, the common factor must be $F=\Pi_i (ζ-α_i)(1-\bar{α_i}ζ)$ and the polynomials factor as $b^1 = F\tilde{b^1}$ and write $P = F\tilde{P}$. Consider the equation
\begin{align}
\tilde{b^1}c^2 - b^2\tilde{c^1} = Q\tilde{P}.
\end{align}
With the common factor excluded from this modified equation, Bezout's identity again says that there is a unique solution for $\tilde{c}^1,c^2$ for degrees at most $g+2-2d$ and $g+2$ respectively. But the $R^2=0$ condition is exactly the one that will ensure that $c^2$ is infact of degree $g+1$. Considering the $ζ^{2g+5-2d}$ terms of this equation it implies that $\tilde{c}^1$ is of degree $g+1-2d$. Let $c^1 :=  F\tilde{c}^1$. Then $c^1, c^2$ solve \eqref{eq:Q}, a necessary condition to solving \eqref{eq:EMPDi}. As before, there is a 1 dimensional space of $\dotP $ and $\dot b^2$ that solve these equations. But the previous coprimality assumption on the $i=1$ equation is no longer true. Every term however contains the factor $F$, so we may pass to an equation with tildes.
\[
\dot P \tilde b^1 - 2\tilde P \dot b^1 = 2\tilde P\left(\hat c^1 - ζ{\hat {c}^1}'\right) + P'ζ(ζ^2-1)\tilde c^1
\]
With the common factor removed, there must exist a solution to this for $\dot {\tilde{b}}^1, \dot {\tilde{P}}$, call it ${\bf\dot b^1}, {\bf\dot P}$. Then for any degree $2d$ polynomial $q$ another solution is
\[
{\bf\dot b^1} + q \tilde b^1, {\bf\dot P} + 2q \tilde P
\]
The space of solutions to the nonreduced equation is
\[
F{\bf\dot b^1} + q \tilde b^1, F{\bf\dot P} + 2q \tilde P
\]
This does not mean however that there is a large space of solutions, because $\dot P$ must satisfy both equations. As before, $\dot P$ is determined by its values at the roots of $P$ which are still consistent between the two equations. So there for there must be some $q$, unique up to scaling, such that for any real $r$ the solutions to $\dot b^1$ are of the form
\[
F{\bf\dot b^1} + rq \tilde b^1.
\]
The choice of scalar is again linked to the scaling of $P$, so there is an essentially unique vector $(\dot P, \dot b^1, \dot b^2)$ for every choice $Q$.
\qed










\begin{lem}[$b^1$ and $b^2$, on the unit circle]
Suppose that $b^1$ and $b^2$ share a single common root $β\in\S^1$ and that $P(β)\neq 0$. Then for either every real linear polynomial $\tilde Q$ or pair of real numbers $(r,\tilde Q)$ there is a solution $(\dot P, \dot b^1, \dot b^2)$ to \eqref{eq:EMPDi}.
\end{lem}

Proof.

Suppose first that $β$ is a simple root of at least one the polynomials. As before, begin by defining versions of the data with the common factor removed $b^i = \sqrt{-\bar{β}}(ζ-β) \tilde b^i$. Take any real linear polynomial $\tilde Q$ and endevour to solve
\[
\tilde b^1 c^2 - \tilde b^2 c^1 = \tilde Q P.
\]
Bezout's identity says that there is a unique solution to this equation $(c^1,c^2)$ where the polynomials are both of degree $g+1$. Multiplying the above equation by the factor $\sqrt{-\bar{β}}(ζ-β)$ gives a solution to \eqref{eq:Q}.

Suppose next that $β$ is a root of order 2 or more for both polynomials. Then we can see that it is only possible to satisfy the necessary condition of the Q-equation the $b$'s have at most an order 2 zero not in common with $P$, because $Q$ is quadratic and could not accomedate any higher order. Therefore consider $F=\bar{β}(ζ-β)^2$, $b^i = F \tilde b^i$ and the equation
\[
\tilde b^1 c^2 - \tilde b^2 c^1 = \tilde Q P.
\]
for a real scalar $\tilde Q$. Bezout's identity give fundemental solutions of degree $g+1$. The space of solutions to \eqref{eq:Q} are then
\[
F{\bf c^1} + r \tilde b^1, F{\bf c^2} + r \tilde b^2
\]

In either case, having found $c$'s that satisfy the necessary condition, the second half of the proof of the generic lemma applies to give a tangent vector.
\qed







\begin{lem}[$b^1$ and $b^2$, off the unit circle]
Suppose that $b^1$ and $b^2$ share a single pair of common roots $β,\bar{β}^{-1}\not\in\S^1$ and that $P(β)\neq 0$. Then for every pair of real numbers $(\tilde Q,r)$ there is a solution $(\dot P, \dot b^1, \dot b^2)$ to \eqref{eq:EMPDi}.
\end{lem}

Proof.

First observe, similiar to the previous case, that because $Q$ is quadratic and $P$ does not share the common root of the $b$'s, $β$ and $\cji{β}$ are first order zeroes. Write $b^i = (ζ-β)(1-\bar{β}ζ) \tilde b^i$ and consider the equation
\[
\tilde b^1 c^2 - \tidlde b^2 c^1 = \tilde Q P.
\]
In this situation, Bezout's identity says that there is a family of solutions
\[
{\bf c^1} + r \tilde b^1, {\bf c^2} + r \tilde b^2
\]
where the $c$'s are degree $g+1$. $r$ is an affine coordinate for this space, but we may choose an origin for it such that ${\bf c^1}$ has degree $g$ for the sake of definiteness. Multiplying by the factor $(ζ-β)(1-\bar{β}ζ)$ gives a solution to \eqref{eq:Q} and a solution to \eqref{eq:EMPDi} follows easily. Different choices for $r$ lead to different $c$'s and therefore to different tangent vectors by the remark earlier that the relationship of tangent vectors to $c$'s is one-to-one.
\qed













\begin{lem}[$b^1$, $b^2$ and $P$] IN PROGRESS, NOT CERTAIN
Suppose that $b^1$, $b^2$ and $P$ share a single pair of common roots $α,\bar{α}^{-1}\not\in\S^1$ and $P$ has some constraint \todo{ find such constraint}. Then for every pair of real numbers $(\tilde Q,r)$ and real quadratic polynomial $s$ there is a solution $(\dot P, \dot b^1, \dot b^2)$ to \eqref{eq:EMPDi}.
\end{lem}

Proof.

Write $b^i = (ζ-α)(1-\bar{α}ζ) \tilde b^i$, $P = (ζ-α)(1-\bar{α}ζ) \tilde P$ and consider the equation
\[
\tilde b^1 \tilde c^2 - \tilde b^2 \tilde c^1 = \tilde Q \tilde P.
\]
In this situation, Bezout's identity says that there is a family of solutions
\[
{\bf \tilde c^1} + r \tilde b^1, {\bf \tilde c^2} + r \tilde b^2
\]
where the $c$'s are degree $g$. Use constraint to bump them down a degree. Let $c^1 = (ζ-α)(1-\bar{α}ζ) \tilde c^i$. Multiplying by the factor $(ζ-α)^2(1-\bar{α}ζ)^2$ gives a solution to \eqref{eq:Q}. Space of solutions is now
\[
{\bf c^1} + r \tilde b^1, {\bf c^2} + r \tilde b^2
\]
Because we forced the $c$'s to have the facotr, we can now remove the factor from every term in \eqref{eq:EMPDi}. Bezout says there is a space of solutions that look like
\[
{\bf\dot {\tilde b}^1} + q \tilde b^1, {\bf\dot {\tilde b}^2} + q \tilde b^2, {\bf\dot{\tilde P}} + 2q \tilde P,
\]
for real quadratic $q$. Choose $q$ such that the degrees are as low as possible, so then we may multiple the solutions to the reduced equation but the factor to get solutions to the nonreduced equation.
