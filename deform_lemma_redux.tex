%!TEX root = thesis.tex

\section{Deformation Theory}
\label{sec:Deformation Theory}

\subsection{Ideas for improvment of this section}
\label{sec:Ideas for improvment}
Beef up the variations to cover higher order roots and multiple roots in common. ie Singular cases

Switch to consistently writing the common factor as $F$.

Think up of better notation for reduced (currently tilde), base solutions (currently bold), things with $(ζ^2-1)$ included/not (currently ).






















\subsection{Bezout's Identity}
\label{sub:Bezout's Identity}
Suppose that $A$, $B$ and $C$ are single variable polynomials over the complex numbers of degrees $a,b,c$ respectively. A natural and long studied question is the solution to the linear equation
\[
AX - BY = C
\labelthis{eqn:Bezout}
\]
in the unknowns $X$ and $Y$. Its complete solution for polynomials is due to \`Etienne B\'ezout REF. Let us recall the method of solution.

Without loss of generality, we may assume that $\gcd(A,B,C) = 1$, otherwise we may divide out the common factor. Next, there cannot be a solution unless $\gcd(A',B') = 1$. If they have a common factor, then evaluation at a common root will cause the left hand side to vanish, but the right hand side will not. We will first solve $AX-BY = 1$ and then use this to give a complete solution. Evaluate the equation at the roots of $A$ to give values of $Y$, and at the roots of $B$ to learn similiarly about $X$. If either $A$ or $B$ has multiple roots, then differentiate the equation the appropriate number of times to generate further independent conditions. Hence we obtain a solution $(\tilde{X},\tilde{Y})$ of degrees at most $(b-1,a-1)$ by polynomial interpolation. $(C\tilde{X}, C\tilde{Y})$ is then a solution to $AX - BY = C$.

If we have two solutions $(X,Y)$ and $(X',Y')$ to the equation, their difference satisfies $A(X-X') - B(Y-Y') = 0$. Again, evaluating at the roots of $A$ and $B$ show that the differences must be multiples of $B$ and $A$ respectively. Hence, given the solution $(\tilde{X},\tilde{Y})$ above, all other solutions are given by
\[
\Set {(C\tilde{X} + rB, C\tilde{Y} + rA) }{ r \text{ a polynomial} }.
\]

What can be said about the degree of these solutions? We can choose an $r$ in such a way that $X = CX' + rB$ is reduced to at most degree $b-1$. There are two regimes to consider for what then happens to the corresponding $Y$ solution. If $c < a + \deg X$ then consideration of the leading terms of the equation forces $Y$ to be degree $a + \deg X - b$. Whereas if $c > a + \deg X$ the degree of $Y$ must stay high, $\deg Y = c-b$ and it is not possible to simultaneously lower the degrees of the solutions.

Because of the reality condition that the spectral curve and differentials must satisfy, we are particularly interested in real solutions to Bezout equations. Take a real involution be $ρ$. We say that a polynomial $p$ is real if $ρ^* p = p$ and imaginary if $ρ^* p = -p$. Suppose that $A,B,C$ are real polynomials and $(X,Y)$ is a solution to \eqref{eqn:Bezout}. Then applying $ρ$ and taking the sum, we see that
\[
A(X+ρ^*X) - B(Y+ρ^*Y) = 2C
\]
Hence the real parts of $(X,Y)$, namely $(\tfrac{1}{2}(X+ρ^*X), \tfrac{1}{2}(Y+ρ^*Y))$, are also a solution. All other real solutions can then be obtained by adding real multiples of $B$ and $A$ respectively. Explicitly, the space of real solutions to $AX-BY = C$ is
\[
\Set{
    \bra { \frac{1}{2}C(\tilde{X}+ρ^*\tilde{X}) + rB, \frac{1}{2}C(\tilde{Y}+ρ^*\tilde{Y})) + rA) }
}{
    r \text{ a real polynomial}
}.
\]





















\subsection{Set Up}
This chapter considers the infinitesimal deformations of the spectral data, so called Whitham deformations. HISTORY OF WHITHAM DEFROMATIONS\todo{}. We have seen that the spectral data $(Σ,Θ^1,Θ^2)$ can be described more concretely as follows. The marked curve $Σ$ is described by $η^2 = P(ζ)$, for $P$ a real polynomial of degree $2g+2$, and each differential can be described by a real polynomial $b$ of degree $g+3$. Thus, we consider the moduli space of spectral data $\mathcal{M}$ as a subspace of the affine space of polynomials $\R^{4g+11} = \R^{2g+3}\times\R^{g+4}\times\R^{g+4}$.

A deformation of spectral data is a path $\ell:(-ε,ε) \to \mathcal{M}$, paramterised by $t$. An infinitesimal deformation is the tangent vector of such a curve at $t=0$. We shall use dashes to denote differentiation with respect to $\zeta$ and dots for differentiation with respect to $t$ evaluated at $t=0$. We may write $\ell(t) = (P(t),b^1(t),b^2(t))$ and $\ell(0) = (P,b^1,b^2)$.

We have seen that the differentials of the spectral data are the derivative of the logarithms of the eigenvalues $μ,\tilde{μ}$. With this in mind, introduce the notation $q^i$ for the functions $q^1 = \log μ$ and $q^2 = \log \tilde{μ}$, which are defined locally up to a constant and globally up to periods. For points along the curve $\ell$, by definition
\[
dq^i(t) = Θ^1(t) = \frac{1}{\zeta^2\eta}b^i(t,\zeta) d\zeta.
\]
Again, if we omit the parameter $t$, we take this to mean the value at $t=0$.


The periods of the differentials of a triple of spectral data satify an integrality condtion. Along a deformation, this forces that the periods have a fixed value. In particular then, their $t$-derivative is zero. Thus $\dot q^i$ is a well defined meromorphic function. Suppose we are in the nonconformal case, and $ζ=0$ is not a branch point of the curve $Σ$. In a neighbourhood of a point $ξ_0$ over zero, because $dq^i$ has a double pole without residue, $q^i$ has the form
\[
q^i = \frac{1}{\zeta\eta}f(\zeta),
\]
for some holomorphic function $f$. Differentiating this with respect to $t$,
\[
\dot q^i = \frac{1}{\zeta\eta} \left(-\frac{1}{2}\frac{\dot P}{P}f + \dot f\right),
\]
showing that $\dot q^i$ has a simple pole at $ξ_0$. By the reality condition REF, it has simple poles over $ζ=\infty$ also. At a (simple) nonzero root $α$ of $P$, let $ξ(t)^2 = ζ-α(t)$ be a local coordinate in a neighbourhood of $ζ=α,t=0$. Its derivative with respect to $t$ is given by $2ξ\dot{ξ} = - \dot{α}$. Removing the $(ζ-α)$ factor from $P$, we write $P = ξ^2 \tilde{P}$, for some polynomial in $ζ$ that does not vanish at $ζ=α$. We compute
\begin{align*}
\dot P &= -\dot{α} \tilde{P} + O(ξ^2)\\
\frac{\dot P}{P} &= -\frac{\dot{α}}{ξ^2} + O(ξ^0).
\labelthis{eqn:t-derivative P}
\end{align*}
At $α$, $Θ^i$ is holomorphic hence $q^i$ is too. For some holomorphic function $f$, in a neighbourhood of $α$ we may write
\[
q^i = \frac{ξ}{η}f(ξ).
\]
Differentiating,
\begin{align*}
\dot{q}^i
&= \frac{1}{η}\bra{\dot{ξ}f(ξ) -\frac{1}{2}\frac{\dot P}{P}ξf(ξ) + ξ\dot{f}(ξ) + ξf'(ξ)\dot{ξ}} \\
&= \frac{1}{η}\bra{-\frac{1}{2}\frac{\dot{α}}{ξ}f(ξ) +\frac{1}{2}\bra{\frac{\dot{α}}{ξ^2} + O(ξ^0) }ξf(ξ) + ξ\dot{f}(ξ) - \frac{1}{2}\dot{α}f'(ξ)}\\
&= \frac{1}{η}\bra{\frac{1}{2}O(ξ^1)f(ξ) + ξ\dot{f}(ξ) -\frac{1}{2}\dot{α}f'(ξ)},
\end{align*}
which shows that $\dot{q}^i$ may have a simple pole at the nonzero roots of $P$. Finally, we consider the conformal case over $ζ=0$. Suppose that some root $α$ of $P$ is zero at $t=0$. Let $ξ(t)^2 = ζ-α(t)$ be a local coordinate in a neighbourhood of $ζ=0,t=0$. $q^i$ takes the form
\[
q^i = \frac{1}{(ξ^2+α)}\frac{ξ}{η} f(t,ξ),
\]
for some function $f$, holomorphic in $ξ$, that has a simple zero at $ξ=0$ when $t=0$. Differentiating
\[
\dot{q}^i
= \frac{1}{(ξ^2+α)}\frac{1}{η}\left[ -\frac{1}{2}\frac{1}{ξ}\dot{α}f - \frac{1}{2}ξ\frac{\dot{P}}{P}f + ξ\dot{f} -\frac{1}{2}\dot{α}f' \right].
\]
The relation \eqref{eqn:t-derivative P} holds in this situation, so the second term cancels the pole of the first, reducing the equation to
\[
\dot{q}^i
= \frac{1}{ζη}\left[ - O(ξ)f + ξ\dot{f} -\frac{1}{2}\dot{α}f' \right].
\]
The bracketed term is now holomorphic in $ξ$. At any other points of the curve $Σ$, $\dot{q}^i$ is holomorphic. Using the form of meromorphic functions on a hyperelliptic curve described in REF MIRANDA, we can deduce that
\[
\dot{q}^i = \frac{1}{\zeta\eta}\hat c^i(\zeta)
\]
for some degree $g+3$ polynomial $\hat c^i$. The reality condition on $q^i$ implies that this is an imaginary polynomial. These polynomials encode the infinitesimal deformations of the differentials that preserve the periods.

Finally, consider the closing condition in its integral form REF. Using the notation we have introduced in this chapter, for some consistent choice of $q^i$ along $γ_+$
\begin{align*}
\int_{γ_+} dq^i = q^i(σ(ξ_1)) - q^i(ξ_1) \in 2\pi i \Z
\end{align*}
Hence, just like for the periods, the t-derivative of this integral is 0. Differentiating shows that $\dot q^i(σ(ξ_1)) = \dot q^i(ξ_1)$. But
\[
\dot q^i(σ(ξ_1)) = σ^* q^i (ξ_1) = - q^i(ξ_1).
\]
Thus $\hat c^i$ has a factor of $ζ-1$. The same reasoning applied to $γ_-$ leads to a factor of $ζ+1$. Let $\hat c^i(\zeta) = (\zeta^2 - 1) c^i(\zeta)$, for $c^i$ a real polynomial of degree $g+1$.




















\subsection{Further Necessary Conditions}
Naturally, there is a realtionship between the tangent vector to $\mathcal{M}$ at a point and these polynomials $c^i$. From the equality of mixed partial derivatives
\begin{align*}
\dot{(dq^i)} &= \frac{d\zeta}{\zeta^2\eta}\left( -\frac{1}{2}\frac{\dot P}{P}b^i + \dot b^i \right) \\
d\dot q^i & = \frac{1}{\zeta^2\eta}\left( -\hat c^i -\frac{1}{2}\frac{P'}{P}\zeta\hat c^i + \zeta\hat {c^i}'\right)d\zeta \\
\dot P b^i - 2P\dot b^i &= 2P\left( \hat c^i - \zeta\hat {c^i}'\right) + P'\zeta\hat c^i \labelthis{eqn:EMPDi}
\end{align*}
This yields equations linking $\dot{b}^i$ to $\hat{c}^i$ for each of $i=1,2$. The two equations are not however independent of one another, for they both contain $P$ and its derivatives. If we multiply the equations by $\hat c^2$ and $\hat c^1$ respectively and take the difference, we arrive at
\[
\dot P (b^1\hat c^2 - b^2\hat c^1) =  2P(\dot b^1\hat c^2 - \dot b^2\hat c^1 - \zeta\hat {c^1}'\hat c^2 + \zeta\hat {c^2}'\hat c^1).\labelthis{eqn:diffed}
\]
From \eqref{eqn:diffed} we will conclude that $P$ divides $b^1\hat c^2 - b^2\hat c^1$ by showing that it vanishes at every root of $P$. If $α$ is a root of $P$ and not a root of $\dot{P}$, we see it is a root of $b^1\hat c^2 - b^2 \hat c^1$ directly from \eqref{eqn:diffed}. Suppose that $P$ and $\dot P$ have a common root $α$. If $α=0$, then we know from REF that $b^i_0=0$ and so $ζ$ divides $b^i$. If $α\neq 0$, from \eqref{eqn:EMPDi} we have that
\begin{align*}
\dot P(α) b^i(α) &= 2P(α)\left( \dot b^i(α) + \hat c^i(α) - α\hat {c^i}'(α)\right) +P'(α)α\hat c^i(α) \\
0 &= 0 + P'(α)α\hat c^i(α)
\end{align*}
But the assumption that the spectral curve is nonsingular forces $P'(α)\neq 0$. Thus we can conclude that $\hat{c}^i(α)=0$. Hence $P$ divides $b^1\hat c^2 - b^2 \hat c^1$ and we can conclude that there is some degree 4 polynomial $\hat Q$ such that
\[
b^1 \hat c^2 - b^2 \hat c^1 = \hat Q P
\]
As $\zeta^2-1$ is a factor of both $\hat{c}$'s, and $P$ has no zeroes on the unit circle, it must be a factor of $\hat Q$. Define $\hat Q = (\zeta^2-1)Q$ to give
\[
b^1 c^2 - b^2 c^1 = Q P \labelthis{eqn:Q}
\]
for some real quadratic polynomial $Q$. This is a necessary condition that the polynomials $c^i$ must satisfy if they correspond to a tangent vector.

The construction of the polynomials $\hat{c}^i$ took into account the order of the poles, the real structure, the periods and the closing conditions of the differenitals, but we have not considered the residue free condition. For that, we need to preserve $P_1b_0 - 2P_0b_1 = 0$. Taking derivatives, we will need to show that
\[
\dot{P}_1 b_0 + P_1 \dot{b}_0 - 2 \dot{P}_0 b_1 - 2 P_0 \dot{b}_1 = 0 \labelthis{eqn:residueTangent}
\]
holds at every point of the path.

Finally, it will not be possible to solve for a tangent vector given an arbitrary polynomial $Q$. Obverse that we would expect generically that given polynomials $b^1,b^2,Q$ and $P$, the solutions $c^1,c^2$ to \eqref{eqn:Q} would be degree $g+2$, but infact we know that they must be a degree lower than this. To this end, let the $n+1$ distinct roots of $b^2$ be denoted as $\{\beta_j\}$ and have multiplicities $r_j +1$. For any real quadratic polynomial $Q$ define
\[
% R^i = \sum_{j=1}^{p+3} \frac{Q(\beta^i_j) P(\beta^i_j)}{b^{3-i}(\beta^i_j) {b^i}'(\beta^i_j)}.
R(Q) = \sum_{j=0}^{n} \sum_{k=0}^{r_j} \frac{g_j^{(r_j-k)}({β}_j)} {k! (r_j-k)!} \bra{\frac{Q(ζ) P(ζ)}{b^1(ζ)}}^{(k)}({β}_j),
\]
for
\[
g_j(ζ) = \frac{(ζ-{β}_j)^{r_j+1}}{b^2(ζ)} ,
\]
with bracketed superscripts denoting differentiation with respect to $ζ$. The important feature of this equation is that it is linear in the coefficients of $Q$.

To close this section, we consider whether at a given point $(P,b^1,b^2)$ the polynomials $\hat{c}^i$ and $Q$ are uniquely determined by a tangent vector $(\dot P, \dot b^1, \dot b^2)$. Since the equations are linear in the components of the tangent vector, we need only consider this question for the zero tangent vector. But then from \eqref{eqn:EMPDi},
\[
0 = 2P\left( \hat c^i - \zeta\hat {c^i}'\right) + P'\zeta\hat c^i.
\]
The assumption of a nonsingular spectral curve requires that $P$ and $P'$ have no common factors and by evaluation at the roots of $P$ the above equation implies that $\hat c^i=0$. This immediately implies that $Q=0$ also. So every real quadratic $Q$ is associated to at most one tangent vector to the space of spectral data.













\subsection{Reconstructing a tangent vector}
Above we saw how to distill an infinitesimal deformation into a real quadratic polynomial $Q$. In this section we will show how to do the reverse; for a given $Q$, generate a tangent vector to $\mathcal{M}$. There are two steps to this process. Firstly, we construct polynomials $c^i$ solving \eqref{eqn:Q}. The second step is separate the information about the deformation contained in $c^i$ into $\dot{b}^i$ and $\dot{P}$ using the equations \eqref{eqn:EMPDi}. The essential method is to use Bezout's identity to produce solutions to each equations with the necessary properties. One must however do this in several different cases to handle the possibility of common factors, even though morally each case tells the same story.

Suppose that $\ell(t) = (P(t),b^1(t),b^2(t))$ is a path in the space of spectral data, and that at $t=0$ we have the following common factors
\[
\gcd(P,b^1,b^2) = F,\;\; \gcd(P/F,b^i/F) = F^i,\;\; \gcd(b^1/FF^1, b^2/FF^2) = G,
\labelthis{eqn:common factors}
\]
so that we may write
\[
P = F F^1 F^2 \tilde{P},\;\; b^i = F F^i G \tilde{b}^i.
\]
Inserting these into \eqref{eqn:EMPDi}, we observe that
\[
\dot{P} F F^i G \tilde{b}^i = 2 F F^1 F^2 \tilde{P} (\dot{b}^i + \hat{c}^i - ζ\hat{c}^{i\prime}) + ζP' \hat{c}^i.
\]
Assuming that the spectral curve is nonsingular, $P'$ does not share any common factors with $P$. Hence we see that $FF^i$ divides $ζ\hat{c}^i$. If we assume that the spectral curve is nonconformal, $ζ$ is not a factor of $P$, so $FF^i$ divides $\hat{c}^i$. We write $\hat{c}^i = (ζ^2-1)FF^i\tilde{c}^i$. We then then remove the common factors from the above equation,
\[
\dot{P} G \tilde{b}^i - 2 F^j \tilde{P} \dot{b}^i = 2 F^j \tilde{P} (\hat{c}^i - ζ\hat{c}^{i\prime}) + ζ(ζ^2-1)P' \tilde{c}^i,
\labelthis{eqn:EMPDi reduced}
\]
with $j\neq i$ (ie if $i=1$, then $j=2$). Inserting these expressions for $\hat{c}^i$ into the $Q$-equation \eqref{eqn:Q} on the other hand produces
\begin{align*}
FF^1G\tilde{b}^1 FF^2\tilde{c}^2 - FF^2G\tilde{b}^2 FF^1\tilde{c}^1 &= Q FF^1F^2\tilde{P} \\
FG\bra{\tilde{b}^1 \tilde{c}^2 - \tilde{b}^2 \tilde{c}^1} &= Q \tilde{P}
\end{align*}
By definition, neither $F$ nor $G$ divide $\tilde{P}$ so they must divide $Q$. This provides a limit on the number of coincident roots that are allowed; $Q$ is quadratic so $FG$ is degree two or less. Moreover, because all of $P,b^1,b^2$ are real, and $P$ has no roots on the unit circle, any common roots of the three polynomials must come in conjugate inverse pairs and so $F$ is even degree. Hence at most one of $G$ and $F$ is not trivial. We write $Q = FG\tilde{Q}$. With all these common factors removed, we have the equation
\[
\tilde{b}^1 \tilde{c}^2 - \tilde{b}^2 \tilde{c}^1 = \tilde{Q} \tilde{P}.
\labelthis{eqn:Q reduced}
\]

In the conformal case, we know that $P_0, b^1_0$ and $b^2_0$ vanish at $t=0$. Thus $F$ is sure to include a factor of $ζ$. We write $F = ζ\tilde{F}$. The same reasoning as above then implies that $\tilde{F}G$ divides $Q$, with the same restriction that at most one of $\tilde{F}$ and $G$ can be nontrivial. Thus we have divided our analysis into six cases; conformal and nonconformal, and whether $F$, $G$ or neither is nontrivial.

BETTER LINKAGE\todo{}
We require real solutions. We now from REF that this is possible if the factors are all real polynomials. We check this now. For \eqref{eqn:Q} it is obvious. It is not obvious that the right hand side of \eqref{eqn:EMPDi} is real however. In particular, we must see how to take the real involution of a derivative. To do so, we compute the following, supposing $f$ is real of degree $k$
\begin{align*}
ζ^k \bar{f}(ζ^{-1}) &= f(ζ) \\
kζ^{k-1} \bar{f}(ζ^{-1}) - ζ^{k-2} \bar{f}'(ζ^{-1}) &= f'(ζ) \\
ρ^*(ζf') = ζ^{k-1}\bar{f}'(ζ^{-1}) &= k f(ζ) - ζf'(ζ)
\end{align*}
Thus we can compute the involution of the right hand side of \eqref{eqn:EMPDi}.
\begin{align*}
ρ^*\bra{ 2P\hat{c}^i - 2Pζ\hat{c}^{i\prime} + ζP'\hat{c}^i }
&= -2P\hat{c}^i - 2P\bra{-(g+3)\hat{c}^i + ζ\hat{c}^{i\prime}} - \bra{(2p+2)P - ζP'}\hat{c}^i \\
&= (-2+2g+6-2g-2)P\hat{c}^i - 2Pζ\hat{c}^{i\prime} + ζP'\hat{c}^i \\
&= 2P\hat{c}^i - 2Pζ\hat{c}^{i\prime} + ζP'\hat{c}^i
\end{align*}
Remarkably then, this is a real polynomial. To whit, for the equations that we are concerned with, if there are solutions the there will be real solutions of the same degree.

Before we proceed, it may be prudent to make a comment about the notation choices that have been made. We have attempted to use ligatures in a consistent way to indicate the factors that a polynomial does or does not have. For example, a hat indicates a factor of $ζ^2-1$, and a tilde indicates that common factors have been removed. Bold will signify a particular solution to an equation. To choice of this particular solution may or may not be unique. When a solution to an equation appears without bold, it signifies a solution from the set of potential solutions. We shall use $d$ with various subscripts to indicate the degrees of polynomials. Finally, we shall use $i$ and $j$ for indices ranging over $1$ and $2$, with the understanding that they are not equal. For example, if $i=1$, then we take $j=2$ and vice versa.




\begin{lem}[Nonconformal, $F=G=1$]
Take a triple of nonconformal spectral data $(P,b^1,b^2)$, with a nonsingular spectral curve given by $η^2 = P$ of genus $g$. Suppose that, with reference to \eqref{eqn:common factors}, $F=G=1$. For every $Q$ with $R(Q) = 0$, there exist unique real polynomials $c^i$ of degree $g+1$ that satisfy \eqref{eqn:Q} and, further, there is a unique tangent vector to the space of spectral data $(\dot P, \dot b^1, \dot b^2)$.

\begin{proof}
In order to solve \eqref{eqn:EMPDi} or \eqref{eqn:Q}, one must first solve their reduced counterparts \eqref{eqn:EMPDi reduced} and \eqref{eqn:Q reduced}.

If we consider \eqref{eqn:Q reduced} as a linear system on the coefficents of $\tilde{c}^i$, then it is overdetermined, but by evaluating it at the values at the roots of the polynomials $\tilde{b}^i$ it nicely seaprates into two independent subsystems. Consider \eqref{eqn:Q reduced} at the roots of $\tilde{b}^2$. If some of the roots are repeated, then consider the expansion near a $k$-th order root $β$ of $\tilde{b}^2$ and differentiate $(k-1)$ times. This yields a system of $g+3$ equations. The coefficent matrix in this system is a confluent Vandemonde matrix, and is always nonsingular. Likewise at the roots of $\tilde{b}^2$.

Thus there is a unique solution ${\bf \tilde{c}}^1, {\bf \tilde{c}}^2$ of degrees $g+2-d_1, g+2-d_2$, where $d_i = \deg F^i$. Explicitly, Hermite interpolation allows one to match the value of a function at $n+1$ points to order $m$ using a polynomial of degree $(n+1)(m+1)-1$. Take for $m$ the order of the highest order root of $\tilde{b}^1$ and consider the $(n+1)(m+1)$ polynomials such that\todo{don't use i j here, reserve them for indices.}
\[
H_{j,k}^{(l)}(β_i) = δ_{j,i}δ_{l,k}\;\;\text{ for }  l \leq m
\]

That is, $H_{j,k}$ is a polynomial that is zero and has zero derivatives to the first $m$ orders at every root $β_i$ of $\tilde{b}^1$, with the exception of its $k$-th derivative at $β_j$ which has a value of $1$. Each of these is degree $(n+1)(m+1)-1$. $g+3-d_1$ of these basis polynomials can be used to fit the data of the problem. The remaining $(n+1)(m+1)-(g+3-d_1)$ can be used to reduce degree of the solution by $1$ each, resulting in a solution of exactly degree $g+2-d_1$ as expected. The full formulae can be found in \cite{Spitzbart1960} for general confluent Vandermode matrices. We note only that ${\bf\tilde{c}}^1_{g+2-d_1}$ is $b^1_{g+3-d_1}R(Q)$, which by assumption is zero. By considering the leading order of \eqref{eqn:Q reduced}, if ${\bf\tilde{c}}^1_{g+2-d_1}$ vanishes, so too must ${\bf\tilde{c}}^2_{g+2-d_2}$.

Multiplying \eqref{eqn:Q reduced} through by $F^1F^2$, we arrive at unique solutions $c^i = F^i {\bf\tilde{c}}^i$ for \eqref{eqn:Q}. Both of these polynomials are degree $g+1$. We similiarly define $\hat{c}^i = (ζ^2 -1)c^i$.

Next we must solve \eqref{eqn:EMPDi reduced}, for which \eqref{eqn:Q reduced} was a necessary condition.
\[
\dot{P} \tilde{b}^i - 2 F^j \tilde{P} \dot{b}^i = 2 F^j \tilde{P} (\hat{c}^i - ζ\hat{c}^{i\prime}) + ζ(ζ^2-1)P' \tilde{c}^i.
\labelthis{eqn:EMPDi nc FG=1}
\]
Using Bezout's Identity, it has a solution because $\gcd(F^j\tilde{P},\tilde{b}^i) = 1$ The first concern is that the two equations for $i=1,2$ may give different solutions for $\dot P$, and indeed in general they do. However, for neither are the solutions unique, and we shall use the freedom in the space of solutions to find a common solution to both. Let a solution to each equation be $({\bf \dot{P}}^1, {\bf \dot{b}}^1)$ and $({\bf \dot{P}}^1, {\bf \dot{b}}^1)$. The sets of solutions are
\[
\Set { ({\bf \dot{P}}^1 + 2rF^2\tilde{P}, {\bf \dot{b}}^1) + r\tilde{b}^1 }
{ r \text{ a real polynomial of degree } d_1 },
\]
and
\[
\Set { ({\bf \dot{P}}^2 + 2sF^1\tilde{P}, {\bf \dot{b}}^2) + r\tilde{b}^2 }
{ s \text{ a real polynomial of degree } d_2 },
\]
respectively. First note that every element of both of these sets are in agreement at any root $α$ of $\tilde{P}$. From \eqref{eqn:EMPDi nc FG=1},
\[
\dot P^i(α) \tilde{b}^i(α) = P'(α)α(α^2 -1)\tilde{c}^i(α).
\]
By definition, roots of $\tilde{P}$ are roots of $P$ that are not common to either $\tilde{b}^1$ or $\tilde{b}^2$. If neither of them has a root at at $α$, then from \eqref{eqn:Q reduced} we see that
\[
\tilde{b}^1(α)\tilde{c}^2(α) - \tilde{b}^2(α)\tilde{c}^1(α) = Q(α)\tilde{P}(α) = 0,
\]
and
\[
\dot P^1(α)
= -P'(α)α (α^2 - 1)\frac{\tilde{c}^1(α)}{\tilde{b}^1(α)}
= -P'(α)α (α^2 - 1)\frac{\tilde{c}^2(α)}{\tilde{b}^2(α)}
= \dot P^2(α).
\]
At the $d_1$ roots of $F^1$, we see that every solution $\dot{P}^2$ takes the same value. Let $β$ be such a root, then
\[
\dot{P}^2(β)
= {\bf \dot{P}}^2(β) + 2sF^1(β)\tilde{P}(β)
= {\bf \dot{P}}^2(β)
= β (β^2-1) P'(β) \frac{\tilde{c}^2(β)}{\tilde{b}^2(β)},
\]
where we can be sure that $\tilde{b}^2(β) \neq 0$ because it cannot be a root of $b^2$ (if it were, $F\neq 1$). Also, $β$ is not a root of $\tilde{P}$ or $F^2$ by the assumption of nonsingularity. Thus this provides a constraint on the choice of $r$. Specifically we require that
\[
{\bf \dot{P}}^1(β) + 2r(β)F^2(β)\tilde{P}(β) = β (β^2-1) P'(β) \frac{\tilde{c}^2(β)}{\tilde{b}^2(β)}.
\]
Likewise, at the $d_2$ roots of $F^2$, we aquire constraints on the choice of $s$. It is always possible to satisfy such constraints, so we see that there is common solution $({\bf \dot{P}}, {\bf \dot{b}}^1, {\bf \dot{b}}^2)$ to \eqref{eqn:EMPDi reduced}. It is also a solution to \eqref{eqn:EMPDi}.

Still though, this solution is not unique. There remains one degree of freedom. For any real number $s$, we have solutions to \eqref{eqn:EMPDi} of the form
\begin{align*}
\dot P &= {\bf \dot{P}} + 2sP  \labelthis{eqn:P soln}\\
\dot b^i &= {\bf \dot{b}}^i + sb^i
\end{align*}
However, this freedom is simply the freedom to rescale $P$. We have chosen a preferred scaling of $P$, so our choice of $s$ is determined. Explicitly, if we were to allow other scalings, the formula for $P$ would be
\[
P = r(t) \prod_i (ζ-α_i)(1- \bar{α}_iζ),
\]
where $α_i$ are the roots inside the unit circle and $r(t)$ some real function. Then from any solution $\dot{P}$ (irrespective of the choice of $s$) we can determine how the roots $α_i$ are changing at $t=0$. Simply differentiate and evaluate at $α_i$
\begin{align*}
\dot{P} = \dot{r} \prod_i (ζ-α_i)(1- \bar{α}_iζ) + r(0) \sum_i (-\dot{α} &+ (\dot{α}\bar{α}+α\dot{\bar{α}})ζ - \dot{\bar{α}}ζ^2) \prod_{j\neq i} (ζ-α_j)(1- \bar{α}_jζ) \\
\dot{P}(α_i) &= -\dot{α}_i(1+α_i\bar{α}_i) \prod_{j\neq i} (α_i-α_j)(1- \bar{α}_j α_i).
\end{align*}
Thus we know the values of $\dot{α}_i$ independent of the scaling from our solution \eqref{eqn:P soln}. Alternatively, if we take the lowest order of this equation
\[
\dot{P}_0 = \dot {\bf P}_0 + 2sP_0 = \dot r P_0 + \sum_i (-\dot{α}_i)\prod_i (-α_j),
\]
so we may ensure that $r\equiv 1$ by choosing $s$ so that $\dot r = 0$. In summary, if we fix a scaling of the spectral curve, then there is a unique solution to \eqref{eqn:EMPDi}.

Finally then there is a second necessary condition that must be satisfied by our solution $(\dot{P},\dot{b}^1,\dot{b}^2)$. We must satisfy \eqref{eqn:residueTangent}, so that \eqref{eqn:residue} holds along the path. But this condition is satisfied already.
\begin{align*}
\dot{P}_1 b_0 + P_1 \dot{b}_0 - 2(\dot{P}_0 b_1 + P_0 \dot{b}_1)
&= P_1\dot{b}_0 - 2\dot{P}_0b_1 + 3P_1\hat{c}_0 - P_0\dot{b}_1 + 2P_1\dot{b}_0 \\
&= 3\bra{ P_1\dot{b}_0 - \dot{P}_0b_1 + P_1\hat{c}_0} \\
&= \frac{3}{P_0}\bra{ P_0P_1\dot{b}_0 - P_0\dot{P}_0b_1 + P_1\bra{ \frac{1}{2}\dot{P_0}b_0 - P_0\dot{b_0} }} \\
&= \frac{3\dot{P}_0}{P_0}\bra{ - P_0b_1 + \frac{1}{2}P_1b_0 }\\
&= 0
\end{align*}
The substitution in the first line comes from the $ζ^1$ terms of \eqref{eqn:Q}, the third line from the constant terms of that equation and the last line comes from the fact that the quantity in the bracket is exactly the residue at $ζ=0$, which is zero by the assumption that $(b^1,b^2,P)$ is spectal data.

Hence $(\dot{P},\dot{b}^1,\dot{b}^2)$ is a tangent vector to $\mathcal{M}$ at $(b^1,b^2,P)$.
\end{proof}
\end{lem}







The moral of these two lemmata is that at any point of spectral data there is a plane of deformation directions. Note that the condition on $Q$ is linear on its coefficents, so there is a plane of such $Q$ and each can then be used to first find a pair $(c^1,c^2)$ and then further a unique tangent vector. Thus, given a section of $Q$'s and an intial piece of spectral data, this data of tangent vectors then amounts to a first order linear ODE, which can be solved to give a deformation of the spectral data. (One may be concerned there is a catch-22 at play here: how is one to apply the condition $R=0$ on $Q$ without already having knowledge of the moduli of spectral data? Fear not. There are frames of meromorphic differentials defined for every hyperelliptic curve such that if the curve is actually a spectral curve, any admissible differentials are in the lattice generated by this frame. Thus we can use this frame to apply the condition and chose a section of $Q$'s, and if we start at a point in the moduli we will never flow outside it accidentally. )






What additional constraints are there if there are common zeroes amoung the polynomials? Suppose that $F = \gcd(P,b^1,b^2)$ is such a common factor. Take any root $α$ of $F$. Then from \eqref{eqn:EMPDi} we have that
\begin{align}
\dot P b^i - 2P\dot b^i &= 2P\left( \hat c^i - \zeta\hat {c^i}'\right) + P'\zeta\hat c^i \\
0 &= 0 + P'(α)α\hat c^i(α)
\end{align}
In the nonsingular and nonconformal case, this implies that $\hat c^i(α) = 0$ so that $F$ also divides both $\hat c$'s. The Q-equation \eqref{eqn:Q} then implies that
\[
F^2 \bar{\tilde{b}^1\tilde{c}^2 - \tilde{b}^2\tilde{c}^1 = F Q \tilde{P}}
\]
where tildes repesent polynomials with the factor of $F$ removed. Again by nonsingularity, $F$ cannot divide $\tilde{P}$ (else $P$ would be divisible by $F$, and hence have multiple roots) so must divide $Q$. Thus we see that $F$ can either be a constant or a real quadratic polynomial, but not any higher degree. If $b^1$ and $b^2$ have a common factor that is not shared by $P$, it must also divide $Q$. Hence if $b^1$ and $b^2$ have a nontrivial common factor, regardless of whether it is common to $P$, it is at most quadratic (a common root on the unit circle would be a linear example) and divides $Q$.














\begin{lem}[Nonsigular, nonconformal, no common factor]
Take a slight generalisation and assume $\gcd(P,b^1)=F^1$ and $\gcd(P,b^2)=F^2$ but $\gcd(b^1,b^2)=1$. Then for every real quadratic polynomial $Q$ satisfying $R=0$ (with the sum now taken over the roots of $b^1/\gcd(P,b^1)$), there is a solution $(\dot P, \dot b^1, \dot b^2)$ to \eqref{eqn:EMPDi}.
\begin{proof}

As $b^i$ is real and the roots of $P$ cannot lie on the unit circle, the common factors must be even degree $2d_i$, ie $F^i=\Pi_j (ζ-α_j)(1-\bar{α_j}ζ)$ and the polynomials factor as $b^i = F^i\tilde{b^i}$ and write $P = F^1F^2\tilde{P}$. Consider the equation
\begin{align}
\tilde{b^1}\tilde{c^2} - \tilde{b^2}\tilde{c^1} = Q\tilde{P}.\labelthis{eqn:Qreduced}
\end{align}

With the common factors excluded from this modified equation, Bezout's identity again says that there is a unique solution for $\tilde{c^1},\tilde{c^2}$ for degrees at most $p+2-2d_1$ and $p+2-2d_2$ respectively. But the $R=0$ condition is exactly the one that will ensure that $c^2$ is infact of degree $p+1-2d_2$. Considering the $ζ^{2p+5-2d_1-2d_2}$ terms of this equation it implies that $\tilde{c}^1$ is of degree $p+1-2d_1$. Let $c^i :=  F^i\tilde{c}^i$. Then $c^1, c^2$ solve \eqref{eqn:Q}, a necessary condition to solving \eqref{eqn:EMPDi}. The previous coprimality assumption is no longer true. Every term however contains the factor $F^i$, so the equation is still solvable. For each $i$, call it $({\bf\dot P^i}, {\bf\dot b^i})$. Then for any degree $2d_i$ polynomial $s^i$ another solution is
\[
{\bf\dot P^i} + 2s^i F^{3-i}\tilde P, {\bf\dot b^i} + s^i \tilde b^1
\]
This does not mean however that there is a large space of solutions, because $\dot P$ must satisfy both equations. As before, $\dot P$ is determined by its values at the roots of $P$. If there were a common root of $b^1$, $b^2$ and $P$ it would completely factor from both equations; it wouldn't provide any constraint on $\dot P$. The assumption says though that there are no such common roots. Thus every root of $P$ constrains $\dot P$, and \eqref{eqn:Qreduced} means that the two equations are consistent at the roots of $\tilde P$. So there for there must be some $s^i$'s that make the two solutions consistent. Having harmonised the solutions we are still free to add real multiples $P$, ie
\[
\{ ({\bf \dot P^1} + 2s^1 F^2 \tilde{P} + 2u F^1F^2\tilde{P}, {\bf \dot b^1} + s^1\tilde{b}^1 + u F^1\tilde{b}^1, {\bf \dot b^2} + s^2\tilde{b}^2 + uF^2 \tilde{b}^2) \mid u \in \R\}.
\]
The choice of scalar $u$ is again linked to the scaling of $P$, so there is an essentially unique vector $(\dot P, \dot b^1, \dot b^2)$ for every choice $Q$.
\end{proof}
\end{lem}












\begin{lem}[Nonsigular, nonconformal, some common factors]
Next assume $\gcd(P,b^1)=F^1$ and $\gcd(P,b^2)=F^2$ and $\gcd(b^1,b^2)=F$ but $\gcd(P,b^1,b^2)=1$. If $F$ is linear, take any real linear polynomial $Q$ or if $F$ is quadratic then take a pair of real numbers $\tilde Q$ and $r$. In either case, for any such choice there is a solution $(\dot P, \dot b^1, \dot b^2)$ to \eqref{eqn:EMPDi}.
\begin{proof}

Write $P = F^1F^2\tilde{P}$ and $b^i = FF^i\tilde b^i$. Let the degrees be $2d_i = \deg F^i$, $d = \deg F$. If $d=1$ consider the equation
\begin{align}
\tilde{b^1}\tilde{c^2} - \tilde{b^2}\tilde{c^1} = Q\tilde{P}.
\end{align}

This is becoming familiar. There is a unique solution for $\tilde{c^1},\tilde{c^2}$ for degrees at most $p+1-2d_1$ and $p+1-2d_2$ respectively. Let $c^i :=  F^i\tilde{c}^i$. Then $c^1, c^2$ solve \eqref{eqn:Q}.

If $d=2$ then consider instead
\begin{align}
\tilde{b^1}\tilde{c^2} - \tilde{b^2}\tilde{c^1} = \tilde{Q}\tilde{P}.
\end{align}
Real solutions for $\tilde{c^1},\tilde{c^2}$ of degrees at most $g+1-2d_1$ and $g+1-2d_2$ are now given by
\[
\{ ({\bf \tilde c^1} + r \tilde{b}^1, {\bf \tilde c^2} + r \tilde{b}^1) \mid r\in \R\}.
\]
This is naturally an affine space, so there is no `correct' choice for the origin. But for the purposes of specifying the parameter $r$ in the lemma, we may for example take $r=0$ to be the point where the constant coefficent of $\tilde c^1$ is of minimum modulus. Let $c^i :=  F^i\tilde{c}^i$. Then $c^1, c^2$ solve \eqref{eqn:Q}.

The first step in solving \eqref{eqn:EMPDi} is to factor from every term $F^i$. For each $i$, call the solution to the reduced equation $({\bf\dot P^i}, {\bf\dot b^i})$. Then for any degree $2d_i$ polynomial $s^i$ another solution is
\[
{\bf\dot P^i} + 2s^i F^{3-i}\tilde P, {\bf\dot b^i} + s^i \tilde b^1
\]
This does not mean however that there is a large space of solutions, because $\dot P$ must satisfy both equations. As before, $\dot P$ is determined by its values at the roots of $P$. Again by assumption there is no common root of $b^1$, $b^2$ and $P$, so there for there must be some $s^i$'s that make the two solutions consistent. Solutions are therefore
\[
\{ ({\bf \dot P^1} + 2s^1 F^2 \tilde{P} + 2u F^1F^2\tilde{P}, {\bf \dot b^1} + s^1\tilde{b}^1 + u F^1\tilde{b}^1, {\bf \dot b^2} + s^2\tilde{b}^2 + uF^2 \tilde{b}^2) \mid u \in \R\}.
\]
As before, the choice of $u$ is just scaling so we have constructed an essentially unique vector $(\dot P, \dot b^1, \dot b^2)$.
\end{proof}
\end{lem}













\begin{lem}[Nonsigular, nonconformal, triple common factor]
Assume $\gcd(P,b^1,b^2) = F$ is a real quadratic, $\gcd(P/F,b^1/F)=F^1$ and $\gcd(P/F,b^2/F)=F^2$. Take a pair of real numbers $\tilde Q$ and $r$. For any such choice there is a solution $(\dot P, \dot b^1, \dot b^2)$ to \eqref{eqn:EMPDi}. NEEDS CONDITION ON P \todo{figure out the condition}
\begin{proof}

Write $P = F^1F^2F\tilde{P}$ and $b^i = FF^i\tilde b^i$. Let the degrees be $2d_i = \deg F^i$. As was alluded to in the prior computation, the Q-equation is dramatically reduced.
\begin{align}
\tilde{b^1}\tilde{c^2} - \tilde{b^2}\tilde{c^1} = \tilde{Q}\tilde{P}.
\end{align}
There is a unique solution for $\tilde{c^1},\tilde{c^2}$ for degrees at most $p-2d_1$ and $p-2d_2$ respectively. By the condition CONDITION\todo{me}, the soltuions are actually one degree less. Let $c^i :=  FF^i\tilde{c}^i$. Then $c^1, c^2$ solve \eqref{eqn:Q} and are both degree $p+1$

We now progree to solving \eqref{eqn:EMPDi}. Factor $FF^i$ from every term. For each $i$, call the solution to the reduced equation $({\bf\dot P^i}, {\bf\dot b^i})$. Then for any degree $2d_i$ polynomial $s^i$ another solution is
\[
{\bf\dot P^i} + 2s^i F^{3-i}\tilde P, {\bf\dot b^i} + s^i \tilde b^1
\]
Unlike before, $\dot P$ is not fully determined by its values at the roots of $P$. FIXME \todo{this} Solutions are therefore
\[
\{ ({\bf \dot P^1} + 2s^1 F^2 \tilde{P} + 2u F^1F^2\tilde{P}, {\bf \dot b^1} + s^1\tilde{b}^1 + u F^1\tilde{b}^1, {\bf \dot b^2} + s^2\tilde{b}^2 + uF^2 \tilde{b}^2) \mid u \in \R\}.
\]
As before, the choice of $u$ is just scaling so we have constructed an essentially unique vector $(\dot P, \dot b^1, \dot b^2)$.
\end{proof}
\end{lem}












\begin{lem}[Nonsigular, conformal, generic]
Take a triple of conformal spectral data $(P,b^1,b^2)$, with a nonsingular spectral curve given by $η^2 = P$ of genus $p$. For any real numbers $Q_1$ and $r$ there is a unique vector to the space of spectral data.

\begin{proof}
Conformal spectral data is distinguished by $P(0)=0$. From the residue condition, or directly by consideration of the order of the poles, $b^i_0 = 0$ also. We may write therefore that $P= ζ\tilde{P}$ and $b^i = ζ \tilde{b}^i$, where $\tilde{P}$ is a real polynomial of degree $2p$ and both $b^i$ are real polynomials of degree $p+1$ (note that $ζ$ is a real quadratic).

Factoring the common root, consider the necessary condition
\[
\tilde{b}^1 c^2 - \tilde{b}^2 c^1 = Q\tilde{P}.\labelthis{eqn:QConformal}
\]
There is a unique solution to this where both $c^i$ polynomials are degree $p$. The space of solutions of at most degree $p+1$ is therefore the same with an arbitrary multiple of $\tilde{b}^i$ added. Within this space, by consideration of degrees, there must be a real solution $(\bf c^1, \bf c^2)$ and all other real solutions are given by
\[
\{ ({\bf c^1} + r \tilde{b}^1, {\bf c^2} + r \tilde{b}^1) \mid r\in \R\}.
\]

Moving on, we attempt to solve a factored version of \eqref{eqn:EMPDi}, namely
\[
\dot P \tilde{b}^i - 2\tilde{P}\dot b^i = 2\tilde{P}\left( \hat c^i - \zeta\hat {c^i}'\right) + P'\hat c^i. \labelthis{eqn:EMPDiConformal}
\]
For the dotted quatities $\dot P$ and $\dot b^1$ there are unique solutions of degree $2p-1$ and $p$ respectively. Mulitplying the whole equation by $ζ$ gives back \eqref{eqn:EMPDi}, so the right hand side is real and there are real solutions of degree $2p+2$ and $p+3$. The space of solutions is therefore
\[
\{ ({\bf \dot P} + 2s \tilde{P}, {\bf \dot b^1} + s \tilde{b}^1, {\bf \dot b^2} + s \tilde{b}^2) \mid s \text { a real quadratic polynomial}\}.
\]
This space appears to be too large, but there are additional constraints to satisfy. The derivate of the residue condition in the conformal case requires that
\[
2 \dot{P}_0 b^i_1 - P_1 \dot{b}^i_0 = 0
\]
This is not automatically true as it was in the nonconformal case. Let $s = s_0 + s_1ζ + \bar{s}_0 ζ^2$ we see that this condition for $i=1$ implies that
\[
2 {\bf\dot{P}_0} b^1_1 - P_1 {\bf \dot{b}^1_0} + 3 s_0 P_1 b^1_1 = 0
\]
which fully determines $s_0$. Can we therefore simultaneously satisfy the condition for $i=2$? Note that \eqref{eqn:EMPDiConformal} in the lowest degree says that
\[
\dot{P}_0 b^i_1 - 2P_1\dot{b}^i_0 = -3 P_1 c^i_0
\]
and \eqref{eqn:QConformal} in the lowest degree yields
\begin{align*}
b^1_1 c^2_0 &= Q_0 P_1 + b^2_1 c^1_0 \\
b^1_1 \bra{{\bf\dot{P}_0} b^2_1 - 2P_1{\bf\dot{b}^2_0}} &= -3Q_0 (P_1)^2 + b^2_1 \bra{{\bf\dot{P}_0} b^1_1 - 2P_1{\bf\dot{b}^1_0}} \\
2b^1_1 {\bf\dot{b}^2_0} &= 3Q_0 P_1 + 2 b^2_1 {\bf\dot{b}^1_0} \\
~\\
b^1_1 \bra{2 {\bf\dot{P}_0} b^2_1 - P_1 {\bf \dot{b}^2_0} + 3 s_0 P_1 b^2_1}
&= 2 {\bf\dot{P}_0} b^1_1b^2_1 - P_1 b^1_1{\bf \dot{b}^2_0} + 3 s_0 P_1 b^1_1b^2_1 \\
&= 2 {\bf\dot{P}_0} b^1_1b^2_1 - P_1 \bra{\frac{3}{2}Q_0 P_1 + b^2_1 {\bf\dot{b}^1_0}} + 3 s_0 P_1 b^1_1b^2_1 \\
&= b^2_1\bra{2 {\bf\dot{P}_0} b^1_1 - P_1 {\bf\dot{b}^1_0} + 3 s_0 P_1 b^1_1 } - \frac{3}{2}Q_0 (P_1)^2 \\
&= - \frac{3}{2}Q_0 (P_1)^2
\end{align*}
So the second condition is also true if and only if $Q_0=0$. Hence we may only choose $Q$ to be of the form $Q = Q_1 ζ$ for some real scalar.

Having passed this check, there is still one free parameter, namely for any $Q_1$ and $r$, the corresponding tangent vectors are
\[
\{ ({\bf \dot P} + 2(s_0+\bar{s}_0ζ^2) \tilde{P} + 2s_1ζ \tilde{P}, {\bf \dot b^1} + (s_0+\bar{s}_0ζ^2) \tilde{b}^1 + s_1ζ \tilde{b}^1, {\bf \dot b^2} + (s_0+\bar{s}_0ζ^2) \tilde{b}^2 + s_1ζ \tilde{b}^2) \mid s_1 \in \R\}.
\]
But our free choice of $s_1\in\R$ is only adding multiples of $(2P,b^1,b^2)$, which as in the nonconformal case is simply a rescaling of the spectral curve.
\end{proof}
\end{lem}















\begin{lem}[Nonsigular, conformal, common zeroes]
The final nonsingular case to consider is to make a slight generalisation and assume again $\gcd(P,b^1)=ζF^1$ and $\gcd(P,b^2)=ζF^2$. Then for every pair of real numbers $(Q_1,r)$, there is a solution $(\dot P, \dot b^1, \dot b^2)$ to \eqref{eqn:EMPDi}.

\begin{proof}
The proof of this is a hybrid of the previous two proofs. Necessarily $c^i = F^i \tilde{c}^i$, but then the argument from the conformal case applies to give a solution to \eqref{eqn:Q} for every $Q = Q_1ζ$ and $r\in\R$. The \eqref{eqn:EMPDiConformal} is massively underdetermined, however the argument from the nonconformal case applied here shows that $\dot P$ needs to have certain values at the nonzero roots of $P$, which there are $2p$ (not $2p+2$ as in the nonconformal case). Hence there is only a real quadratic freedom in the solution to that equation. Finally the argument about the residue condition applies to reduce this to just rescaling of spectral curve.
\end{proof}
\end{lem}












\subsection{Singular cases}
\label{sub:Singular cases}
These are incomplete. But there are no singular spectral curves in genus zero, one or two; hence it's a nice to have but not of consequence to later sections which focus on these genii (genuses?).


\begin{lem}[$b^1$ and $b^2$, on the unit circle]
Suppose that $b^1$ and $b^2$ share a single common root $β\in\S^1$ and that $P(β)\neq 0$. Then for either every real linear polynomial $\tilde Q$ or pair of real numbers $(r,\tilde Q)$ there is a solution $(\dot P, \dot b^1, \dot b^2)$ to \eqref{eqn:EMPDi}.
\end{lem}

Proof.

Suppose first that $β$ is a simple root of at least one the polynomials. As before, begin by defining versions of the data with the common factor removed $b^i = \sqrt{-\bar{β}}(ζ-β) \tilde b^i$. Take any real linear polynomial $\tilde Q$ and endevour to solve
\[
\tilde b^1 c^2 - \tilde b^2 c^1 = \tilde Q P.
\]
Bezout's identity says that there is a unique solution to this equation $(c^1,c^2)$ where the polynomials are both of degree $g+1$. Multiplying the above equation by the factor $\sqrt{-\bar{β}}(ζ-β)$ gives a solution to \eqref{eqn:Q}.

Suppose next that $β$ is a root of order 2 or more for both polynomials. Then we can see that it is only possible to satisfy the necessary condition of the Q-equation the $b$'s have at most an order 2 zero not in common with $P$, because $Q$ is quadratic and could not accomedate any higher order. Therefore consider $F=\bar{β}(ζ-β)^2$, $b^i = F \tilde b^i$ and the equation
\[
\tilde b^1 c^2 - \tilde b^2 c^1 = \tilde Q P.
\]
for a real scalar $\tilde Q$. Bezout's identity give fundemental solutions of degree $g+1$. The space of solutions to \eqref{eqn:Q} are then
\[
F{\bf c^1} + r \tilde b^1, F{\bf c^2} + r \tilde b^2
\]

In either case, having found $c$'s that satisfy the necessary condition, the second half of the proof of the generic lemma applies to give a tangent vector.
\qed







\begin{lem}[$b^1$ and $b^2$, off the unit circle]
Suppose that $b^1$ and $b^2$ share a single pair of common roots $β,\bar{β}^{-1}\not\in\S^1$ and that $P(β)\neq 0$. Then for every pair of real numbers $(\tilde Q,r)$ there is a solution $(\dot P, \dot b^1, \dot b^2)$ to \eqref{eqn:EMPDi}.
\end{lem}

Proof.

First observe, similiar to the previous case, that because $Q$ is quadratic and $P$ does not share the common root of the $b$'s, $β$ and $\cji{β}$ are first order zeroes. Write $b^i = (ζ-β)(1-\bar{β}ζ) \tilde b^i$ and consider the equation
\[
\tilde b^1 c^2 - \tilde b^2 c^1 = \tilde Q P.
\]
In this situation, Bezout's identity says that there is a family of solutions
\[
{\bf c^1} + r \tilde b^1, {\bf c^2} + r \tilde b^2
\]
where the $c$'s are degree $g+1$. $r$ is an affine coordinate for this space, but we may choose an origin for it such that ${\bf c^1}$ has degree $g$ for the sake of definiteness. Multiplying by the factor $(ζ-β)(1-\bar{β}ζ)$ gives a solution to \eqref{eqn:Q} and a solution to \eqref{eqn:EMPDi} follows easily. Different choices for $r$ lead to different $c$'s and therefore to different tangent vectors by the remark earlier that the relationship of tangent vectors to $c$'s is one-to-one.
\qed













\begin{lem}[$b^1$, $b^2$ and $P$] IN PROGRESS, NOT CERTAIN
Suppose that $b^1$, $b^2$ and $P$ share a single pair of common roots $α,\bar{α}^{-1}\not\in\S^1$ and $P$ has some constraint \todo{ find such constraint}. Then for every pair of real numbers $(\tilde Q,r)$ and real quadratic polynomial $s$ there is a solution $(\dot P, \dot b^1, \dot b^2)$ to \eqref{eqn:EMPDi}.
\end{lem}

Proof.

Write $b^i = (ζ-α)(1-\bar{α}ζ) \tilde b^i$, $P = (ζ-α)(1-\bar{α}ζ) \tilde P$ and consider the equation
\[
\tilde b^1 \tilde c^2 - \tilde b^2 \tilde c^1 = \tilde Q \tilde P.
\]
In this situation, Bezout's identity says that there is a family of solutions
\[
{\bf \tilde c^1} + r \tilde b^1, {\bf \tilde c^2} + r \tilde b^2
\]
where the $c$'s are degree $g$. Use constraint to bump them down a degree. Let $c^1 = (ζ-α)(1-\bar{α}ζ) \tilde c^i$. Multiplying by the factor $(ζ-α)^2(1-\bar{α}ζ)^2$ gives a solution to \eqref{eqn:Q}. Space of solutions is now
\[
{\bf c^1} + r \tilde b^1, {\bf c^2} + r \tilde b^2
\]
Because we forced the $c$'s to have the facotr, we can now remove the factor from every term in \eqref{eqn:EMPDi}. Bezout says there is a space of solutions that look like
\[
{\bf\dot {\tilde b}^1} + q \tilde b^1, {\bf\dot {\tilde b}^2} + q \tilde b^2, {\bf\dot{\tilde P}} + 2q \tilde P,
\]
for real quadratic $q$. Choose $q$ such that the degrees are as low as possible, so then we may multiple the solutions to the reduced equation but the factor to get solutions to the nonreduced equation.


\subsection{Possibility of common factors}
Suppose that $(P,b^1,b^2)$ is a path in the space of spectral data, and that at $t=0$ we have the following common factors
\[
\gcd(P,b^1,b^2) = F,\;\; \gcd(P/F,b^i/F) = F^i,\;\; \gcd(b^1/FF^1, b^2/FF^2) = G,
\]
so that we may write
\[
P = F F^1 F^2 \tilde{P},\;\; b^i = F F^i G \tilde{b}^i.
\]
Inserting these into REF, we observe that
\[
\dot{P} F F^i G \tilde{b}^i = 2 F F^1 F^2 \tilde{P} (\dot{b}^i + \hat{c}^i - ζ\hat{c}^{i\prime}) + ζP' \hat{c}^i.
\]
Assuming that the spectral curve is nonsingular, $P'$ does not share any common factors with $P$. Hence we see that $FF^i$ divides $ζ\hat{c}^i$. If we assume that the spectral curve is nonconformal, $ζ$ is not a factor of $P$, so $FF^i$ divides $\hat{c}^i$. We write $\hat{c}^i = (ζ^2-1)FF^i\tilde{c}^i$. Putting these into the $Q$-equation
\begin{align*}
FF^1G\tilde{b}^1 FF^2\tilde{c}^2 - FF^2G\tilde{b}^2 FF^1\tilde{c}^1 &= Q FF^1F^2\tilde{P} \\
FG\bra{\tilde{b}^1 \tilde{c}^2 - \tilde{b}^2 \tilde{c}^1} &= Q \tilde{P}
\end{align*}
By definition, neither $F$ nor $G$ divide $\tilde{P}$ so they must divide $Q$. This provides a limit on the number of coincident roots that are allowed; $Q$ is quadratic so $FG$ degree two or less. Moreover, because all of $P,b^1,b^2$ are real, and $P$ has no roots on the unit circle, any common roots must come in conjugate inverse pairs. Hence at most one of $G$ and $F$ is not trivial. We write $Q = FG\tilde{Q}$.

In the conformal case, we know that $P_0, b^1_0$ and $b^2_0$ vanish at $t=0$. Thus $F$ is sure to include a factor of $ζ$. We write $F = ζ\tilde{F}$. The same reasoning as above then implies that $\tilde{F}G$ divides $Q$, with the same restriction that at most one of $F$ and $G$ can be nontrivial. Thus we have divided our analysis into six cases; where $F$, $G$ or neither is nontrivial, and conformal and nonconformal.
