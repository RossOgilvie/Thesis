%!TEX root = thesis.tex

\section{Deformation Theory}
\label{sec:Deformation Theory}

\subsection{Ideas for improvment of this section}
\label{sec:Ideas for improvment}
Beef up the variations to cover higher order roots and multiple roots in common. ie Singular cases

Switch to consistently writing the common factor as $F$.

Think up of better notation for reduced (currently tilde), base solutions (currently bold), things with $(ζ^2-1)$ included/not (currently ).






















\subsection{Bezout's Identity}
\label{sub:Bezout's Identity}
Suppose that $A$, $B$ and $C$ are polynomials of degrees $a,b,c$ respectively. We consider only single variable polynomials over the complex numbers. What is the space of solutions $X,Y$to the equation $AX - BY = C$? Equations of this form are known as Bezout's identity and solutions exist in great generality beyond integers and polynomials. But we shall focus on polynomials and what can be said of the degree of the solutions. Firstly, let us assume that $\gcd(A,B,C) = G$ of degree $g$. Denote $A' = A/G$ and similarly for $B',C'$.

Next, there cannot be a solution unless $\gcd(A',B') = 1$. If they have a common factor, then evaluation at a common root will cause the left hand side to vanish, but the right hand side will not. We first attempt to solve $A'X-B'Y = 1$. One can solve this for a solution $(X,Y)$ of degrees $(b-g-1,a-g-1)$ by your favourite method of polynomial interpolation. Simply evaluate the equation at the roots of $A'$ to give values $Y$ must obtain, and at the roots of $B'$ to learn about $X$. If either $A$ or $B$ has multiple roots, then differentiate the equation the appropriate number of times to generate enough conditions. Having found this solution of least degree, all other solutions are given by
\[
\{(X' + rB', Y' + rA') \mid r \text{ a polynomial} \}
\]

Solutions to $A'X - B'Y = C'$ therefore must be of the form
\[
\{(X'C' + rB', Y'C' + rA') \mid r \text{ a polynomial} \}
\]
What can be said about the degree of the solution? One can use the freedom of adding say a multiple of $B'$ to ensure that the degree of the `$X$' solution is reduced to at most degree $b-g-1$, essentially using polynomial division with remainder. There are two regimes to consider though for what then happens to the `$Y$' solution. If $c \leq a + \deg X$ (which is generically $a+b-g-1$) then consideration of the top degree means that $Y$ is degree $a-b +\ deg X$. Where as if $c > a + \deg X$, then if the degree of $X$ is lowered, then the degree of $Y$ must stay high to compensate, namely $c-b$. It is not possible to simultaneously lower the degrees of the solutions in this case, and generically the degrees of the solutions $(X,Y)$ will be greater than or equal to $(c-a,c-b)$ respectively.

Finally, the space of solutions to the oringal equation $AX-BY=C$ is just this last set of solutions
\[
\{(X'C' + rB', Y'C' + rA') \mid r \text{ a polynomial} \}
\]
and the comments about degrees also applies.

We are particularly interested in real solutions to equations of this form. Suppose that $A,B,C$ are all real polynomials and $X,Y$ are solutions such that $\deg AX = \deg BY = \deg C$ (which may or may not exist from the discussion above). Then applying the real involution $ρ$ and subtracting from the original equation we get that
\[
A(X-ρ^*X) - B(Y-ρ^*Y) = 0
\]
The polynomials in the bracket are purely imaginary, indeed they are the twice imaginary part of $X$,$Y$ respectively in the sense that $X = \dfrac{1}{2}(X+ρ^*X) + \dfrac{1}{2}(X-ρ^*X)$, and we can deduce that they must be an imaginary multiple of $B',A'$ resp. But that exactly the sort of thing we are permitted to add and subtract from solutions, hence we can subtract off the imaginary part of our solution to be left with a purely real solution. Other real solutions can then be obtained by adding real multiples of $B',A'$.






















\subsection{Set Up}
We wish to consider infinitesimal deformations of the spectral data. To do this, we will consider small paths paramterised by $t$ and consider their tangents at $t=0$. Let the roots of $P$ be $\alpha_i$. They occur in conjugate inverse pairs, with half inside the unit disc $D$ and half outside. Dashes denote differentiation with respect to $\zeta$ and a dot is for differentiation with respect to $t$ evaluated at $t=0$.

We have already seen how the meromorphic differentials should be considered as the derivative of logarithms. With this in mind, we consider the function $q^i = \log μ^i$ defined locally up to a constant, and globally up to periods. Let $Θ^i = dq^i$ so recall then that
\[
dq^i = \frac{1}{\zeta^2\eta}b^i(\zeta) d\zeta
\]
for a polynomial $b^i$ of degree $p+3$. If the deformation flow preserves the periods, then the integrality of the periods means that their $t$-derivative is zero. Thus $\dot q^i$ is a well defined meromorphic function. At $ζ=0$ (or $\infty$), we can differentiate the power series expansion for $q^i$ to obtain (letting $f$ be a holomorphic function)
\begin{align*}
q^i &= \frac{1}{\zeta\eta}f(\zeta) \\
\dot q^i &= \frac{1}{\zeta\eta} \left(-\frac{1}{2}\frac{\dot P}{P}f + \dot f\right),
\end{align*}
showing that $\dot q^i$ has a simple pole at $0$. In a similiar manner, it has a simple zero at $\infty$. At a simple root $α$ of $P$, $Θ^i$ is holomorphic hence $q^i$ is too, so letting $ξ^2 = ζ-α$ be a local coordinate
\begin{align*}
q^i &= \frac{1}{η}ξf(ξ) \\
2ξ\dot{ξ} &= - \dot{α} \\
P(ζ) &= ξ^{2} P_{2p+2}\prod_{α_i\neq α} (ζ-α_i) \\
P(ζ=α) &\sim ξ^{2} P_{2p+2}\prod_{α_i\neq α} (α-α_i) \\
\dot P(ζ) &= -\dot{α} P_{2p+2} \prod_{α_i\neq α} (α-α_i) + O(ξ^2)\\
\dot P(ζ=α) &\sim -\dot{α} P_{2p+2} \prod_{α_i\neq α} (α-α_i) \\
\dot{q}^i &= \frac{1}{η}\bra{-\frac{1}{2}\frac{\dot P}{P}ξf(ξ) + \dot{ξ}f(ξ) + ξf'(ξ)\dot{ξ}} \\
&\sim \frac{1}{ξ}\bra{\frac{1}{2}\frac{\dot{α}}{ξ^2}ξf(0) - \frac{1}{2}\frac{\dot{α}}{ξ}f(0) - \frac{1}{2}\dot{α} f'(0)}
\end{align*}
Which shows that $\dot{q}^i$ may have a simple pole at the simple roots of $P$. Likewise at a higher, but still odd, order root $α$ of $P$, neighbouring spectral curves along the path may not have a zero to the same order as it may be formed by several roots coming together, but at least one of those other roots must be odd order; if a sum is odd, at least one summand is odd. Let $α_i(t)$ be the roots that collide (ie $α_i(0)=α$) and $α_1(t)$ be an odd order root. Write $P(t,ζ) = \prod_i^{2k+1} (ζ-α_i) \tilde{P}$, and make a change of coordinate $ξ^2=ζ-α_1$. As $dq^i$ is holomorphic near $α$, so must be $q^i$ and so for some holomorphic function $f$
\begin{align*}
q(t,ξ) &= \frac{1}{\sqrt{\tilde{P}}}f(ξ) \\
\dot q(t,ξ) &= -\frac{1}{2}\frac{1}{\sqrt{\tilde{P}}^3}f(ξ) + \frac{1}{\sqrt{\tilde{P}}}\frac{d}{dt}f(ξ)\\
\dot q(0,ξ) &= -\frac{1}{ξ}\frac{\dot{α}}{\sqrt{\tilde{P}}}f'(ξ) +\bra{ -\frac{1}{2}\frac{1}{\sqrt{\tilde{P}}^3}f(ξ) + \frac{1}{\sqrt{\tilde{P}}}\dot{f}(ξ)}
\end{align*}
Again showing that it may have a simple pole at a (general) branch point. For zeroes of $P$ that are not branch points (ie are even order zeroes), and other non-branch points more general, one does not need to be concerned with finding a local coordinate on the curve as $ζ$ already fills this role. Let $α$ be a non-branch point where $P$ vanishes to order $2k$. Then again we split $P(t,ζ) = \prod_i^{2k} (ζ-α_i) \tilde{P}$ into vanishing and nonvanishing factors. Simply
\begin{align}
q(t,ξ) &= \frac{1}{\sqrt{\tilde{P}}}f(ξ) \\
\dot q &= \frac{1}{\sqrt{\tilde{P}}}\bra{-\frac{1}{2}\frac{\dot{\tilde{P}}}{\tilde{P}} f + \dot f},
\end{align}
which is just holomorphic. Hence we can deduce that
\[
\dot{q}^i = \frac{1}{\zeta\eta}\hat c^i(\zeta)
\]
for some degree $p+3$ polynomial $\hat c^i$, which will have zeroes at singular points of the spectral curve. The reality condition on $q^i$ implies that this is an imaginary polynomial (ie that $i \hat c^i$ is a real polynmomial). These polynomials encode the infinitesimal deformations of the differentials that preserve the periods.

Consider the closing condition in its integral form. Recall that it is
\begin{align*}
\int_{\gamma_{1}} dq^i = q^i(1^+) - q^i(1^-) \in 2\pi i \Z
\end{align*}
Hence, just like for the periods, the t-derivative of this integral is 0. Applying this to the middle term of the above equation means that $\dot q^i(1^+) = \dot q^i(1^-)$. Thus $\hat c^i$ must have a fator of $ζ-1$. The same reasoning for $γ_{-1}$ leads to a factor of $ζ+1$. Let $\hat c^i(\zeta) = (\zeta^2 - 1) c^i(\zeta)$, with $c^i$ a real polynomial of degree $p+1$.




















\subsection{Two necessary conditions}
From the equality of mixed partial derivatives
\begin{align*}
\dot{(dq^i)} &= \frac{d\zeta}{\zeta^2\eta}\left( -\frac{1}{2}\frac{\dot P}{P}b^i + \dot b^i \right) \\
d\dot q^i & = \frac{1}{\zeta^2\eta}\left( -\hat c^i -\frac{1}{2}\frac{P'}{P}\zeta\hat c^i + \zeta\hat {c^i}'\right)d\zeta \\
-\dot P b^i + 2P \dot b^i & = -2P \hat c^i -P'\zeta\hat c^i + 2P\zeta\hat {c^i}' \\
\dot P b^i - 2P\dot b^i &= 2P\left( \hat c^i - \zeta\hat {c^i}'\right) + P'\zeta\hat c^i \labelthis{eq:EMPDi}
\end{align*}
This gives two equations. If we multiply by alternatively by $\hat c^2$ and $\hat c^1$ and subtracting
\begin{align*}
\dot P b^1\hat c^2 - 2P \dot b^1\hat c^2 & = 2P \hat c^1\hat c^2 +P'\zeta\hat c^1\hat c^2 - 2P\zeta\hat {c^1}'\hat c^2 \\
\dot P b^2\hat c^1 - 2P \dot b^2\hat c^1 & = 2P \hat c^2\hat c^1 +P'\zeta\hat c^2\hat c^1 - 2P\zeta\hat {c^2}'\hat c^1 \\
\dot P (b^1\hat c^2 - b^2\hat c^1) - 2P (\dot b^1\hat c^2 - \dot b^2\hat c^1) & = - 2P\zeta(\hat {c^1}'\hat c^2 - \hat {c^2}'\hat c^1) \\
\dot P (b^1\hat c^2 - b^2\hat c^1) & =  2P(\dot b^1\hat c^2 - \dot b^2\hat c^1 - \zeta\hat {c^1}'\hat c^2 + \zeta\hat {c^2}'\hat c^1)\labelthis{eq:diffed}
\end{align*}

USING NONSINGUALR ASSUMPTION:
We now make the assumption that the spectral curve is nonsingular. This is to say that $P$ has only simple roots, or in terms of the factorisation of polynomials (which we shall shortly care about), we assume that $gcd(P,P') = 1$. This assumption always holds in low spectral genus.

From \eqref{eq:diffed} we can conclude that $P$ divides $b^1\hat c^2 - b^2\hat c^1$ in the following way. If $P$ and $\dot P$ have a common root $\alpha$, then from \eqref{eq:EMPDi} we get that
\begin{align*}
\dot P(\alpha) b^i(\alpha) &= 2P(\alpha)\left( \dot b^i(\alpha) + \hat c^i(\alpha) - \alpha\hat {c^i}'(\alpha)\right) +P'(\alpha)\alpha\hat c^i(\alpha) \\
0 &= 0 + P'(\alpha)\alpha\hat c^i(\alpha)
\end{align*}
By the assumption of nonsingularity, $P'(\alpha)\neq 0$. In the nonconformal case $P(0) \neq 0$ and therefore the root $α$ cannot be $0$ either. Thus we can conclude that $\hat{c}^i(\alpha)=0$. If we are in the conformal case, then we have already noted that $b^i(0)=0$. We note that any other roots of $P$ must be roots of $b^1\hat c^2 - b^2 \hat c^1$ directly from \eqref{eq:diffed}.

ALTERNATIVE DERIVATION WITHOUT USING NONSINGUALR ASSUMPTION:
From \eqref{eq:diffed} we can conclude that $P$ divides $b^1\hat c^2 - b^2\hat c^1$ in the following way. Suppose that $α$ is a $k^{th}$ order root of $P$. Write $P = (ζ-α)^k \tilde{P}(ζ)$, with $\tilde{P}(α)\neq 0$. Then $\dot{P} = (ζ-α)^{k-1}((ζ-α)\dot{\tilde{P}}-k\dot{α}\tilde{P})$. Substituting this into \eqref{eq:EMPDi} and factoring the common factor of $(ζ-α)^{k-1}$ from every term we arrive at
\[
\bra{(ζ-α)\dot{\tilde{P}}-k\dot{α}\tilde{P}} b^i - 2(ζ-α)\tilde{P}\dot b^i = 2(ζ-α)\tilde{P}\left( \hat c^i - \zeta\hat {c^i}'\right) + \bra{k\tilde{P} + (ζ-α)\tilde{P}'}\zeta\hat c^i.
\]
Evaluating this at $ζ=α$ we deduce that
\begin{align*}
-k\dot{α}\tilde{P}(α) b^i(α) &= k\tilde{P}(α) α \hat c^i(α) \\
-\dot{α} b^i(α) &= α \hat c^i(α)
\end{align*}
If $α\neq 0$ then ...... DOESN'T SHOW WHAT I WANTED, PUT A BETTER THING HERE.


BACK TO NORMAL:
Hence $P$ divides $b^1\hat c^2 - b^2 \hat c^1$ and we can conclude that there is some degree 4 polynomial $\hat Q$ such that
\[
b^1 \hat c^2 - b^2 \hat c^1 = \hat Q P
\]
As $\zeta^2-1$ is a factor of both $\hat{c}$'s, and $P$ has no zeroes on the unit circle, it must also be a factor of $\hat Q$. Define $\hat Q = (\zeta^2-1)Q$ to give
\[
b^1 c^2 - b^2 c^1 = Q P \labelthis{eq:Q}
\]
for some real quadratic polynomial $Q$. This is a necessary condition implied by any tangent vector, and we will see that it is required because the $\hat{c}$'s contain information both about the deformation of the differentials, but also of the spectral curve, and this is the condition that ensure the information given is compatible.

The construction of the $\hat{c}$'s baked into it the preservation of the poles, the real structure, the periods and the closing condition, but we have not considered the resdue free condition. For that, we need to preserve $P_1b_0 - 2P_0b_1 = 0$. Taking derivatives, we will need to show that
\[
\dot{P}_1 b_0 + P_1 \dot{b}_0 - 2 \dot{P}_0 b_1 - 2 P_0 \dot{b}_1 = 0 \labelthis{eq:residueTangent}
\]
holds at every point of the path.















\subsection{Reconstructing a tangent vector}
By the construction of the spectral curve from the hyperelliptic curve, singular points of the spectral curve arise from coincidences of the eigenspaces of the holonomy matrix. At a singular point where $P$ has a root of order $2m$ or $1+2m$, the differentials have a common zero of order $m-1$. Also, the geometric and arithmetic genus must have the same parity (any singular points must occur in real pairs). Together these constraints serverly restrict the possibilities in low spectral genus. THIS IS NOW AN ODD THING TO HAVE HERE \todo{this}

Above we saw how to distill an infinitesimal deformation into a real quadratic polynomial $Q$. In this part we show how to do the reverse; to reconstruct a tangent vector to the space of spectral data given the $Q$. There are two parts to this process. Firstly, we construct $c$'s which come from deformations of the differentials. These polynomials however also contain the information of how the spectral curve is changing. The second step is therefore to split out the information from the $c$'s into the $t$-derviatives of the $b$'s and $P$. The situation is complicated by the necessary use of Bezout's identity, which behaves differently when terms have common factors. We are forced to split and handle several cases separately, even though morally each case tells the same story.

The cases divide on several lines. We deal first with the nonsinular case and distinguish conformal and nonconformal. Within each, generically the triple $(P,b^1,b^2)$ are pairwise coprime, but ENUMERATE COMMON ZERO CASES \todo{this}

\begin{lem}[Q-equation: Nonsingular, Nonconformal, Generic]
Take a triple of nonconformal spectral data $(P,b^1,b^2)$, with a nonsingular spectral curve given by $η^2 = P$ of genus $p$ and assume that the $b$ polynomials are coprime. Let the $n+1$ roots of $b^2$ be denoted as $\{\beta_j\}$ and have multiplicity $r_j +1$. For any real quadratic polynomial $Q$ define
\[
% R^i = \sum_{j=1}^{p+3} \frac{Q(\beta^i_j) P(\beta^i_j)}{b^{3-i}(\beta^i_j) {b^i}'(\beta^i_j)}.
R = \sum_{j=0}^{n} \sum_{k=0}^{r_j} \frac{g_j^{(r_j-k)}({β}_j)} {k! (r_j-k)!} \bra{\frac{Q(ζ) P(ζ)}{b^1(ζ)}}^{(k)}({β}_j)
\]
for
\[
g_j = \frac{(ζ-{β}_j)^{r_j+1}}{b^2(ζ)} ,
\]
with bracketed supscripts denoting multiple derivatives with respect to $ζ$. Then for every $Q$ with $R = 0$, there exist real polynomials $c^i$ of degree $p+1$ that satisfy \eqref{eq:Q}.

\begin{proof}
If we consider \eqref{eq:Q} as a linear system on the coefficents of $c^i$, then it is overdetermined, but nicely seaprates into two independent subsystems by considering it at the values at the roots of the $b$'s. If some of the roots are repeated, then consider the expansion near a $k$-th order root $β$ of $b^2$ and differentiate
\begin{align*}
c^2(ζ) &= \frac{Q(ζ)P(ζ)}{b^1(ζ)} + O((ζ-β)^k)\\
(c^2)'(ζ) &= \bra{\frac{Q(ζ)P(ζ)}{b^1(ζ)}}' + O((ζ-β)^{k-1})\\
(c^2)^{(l)}(ζ) &= \bra{\frac{Q(ζ)P(ζ)}{b^1(ζ)}}^{(l)} + O((ζ-β)^{k-l})\\
(c^2)^{(l)}(β) &= \bra{\frac{Q(ζ)P(ζ)}{b^1(ζ)}}^{(l)}(β)
\end{align*}

This yield a system of $p+3$ equations. Suppose then for a moment that the degree of $c^2$ was $p+2$, so that is has $p+3$ unknown coefficents. Then the coefficent matrix in this problem is the confluent Vandemonde matrix, and is always nonsingular. Thus there is a unique solution for $c^2$. If all the roots are simple, then this solution is given by a linear combination the Legrende polynomials. The $j$-th Legrende polynomial $L_j(\zeta)$ is the unique polynomial that takes the value $1$ at $\beta_j$ and is zero at the other roots. We can write
\[
c^2(\zeta) = - \sum_{j=0}^{p+2} \left( \frac{QP}{b^{1}(b^2)'}\right)(\beta_j) \frac{b^2(\zeta)}{\zeta-\beta_j}
\]

The $p+2$ degree coeffient of $c^2$ is therefore $-b^2_{p+3}R$. But under the assumptions, this is zero, therefore we have found a unique solution for $c^2$ of degree $p+1$. The analogous construction works to find $c^1$ at the roots of $b^1$, and in fact we do not need an additional constraint on $Q$ as consideration of the $2p+5$-th degree of the equation implies that $c^1_{p+2}=0$ as well.

However, if there are higher order roots, a more sophisticed technique is needed. Hermite interpolation allows one to match the value of a function at $n+1$ points to order $m$ each using a polynomial of degree $(n+1)(m+1)-1$. Take for $m$ the order of the highest order root of $b^2$ and consider the $(n+1)(m+1)$ polynomials such that
\[
H_{j,k}^{(l)}(β_i) = δ_{j,i}δ_{l,k}\;\;\text{ for }  l \leq m
\]

That is $H_{j,k}$ is a polynomial that is zero and has zero derivative to to the first $m$ orders, except its $k$-th derivative at $β_j$ is $1$. Each of these is degree $(n+1)(m+1)-1$. $p+3$ of these basis polynomials can be used to fit the data of the problem. The remainding $(n+1)(m+1)-p-3$ can be used to reduce degree of the solution by $1$ each, resulting in a solution of exactly $p+2$ as expected. The precise formulas can be found in  \cite{Spitzbart1960} for general confluent Vandermode matrices. We are not interested in the precise form, but note that $c^2_{p+2}$ is $-b^2_{p+3}R$, which by assumption is zero. As before, the same approach works for $c^1$, and hence we have built a pair of degree $p+1$ polynomials. By the comment earlier, they must be real polynomials.

\end{proof}
\end{lem}












\begin{lem}[Tangent vector equation,  Nonsingular, Nonconformal, generic version]
Take a triple $(P,b^1,b^2)$, corresponding to a nonsingular spectral curve. Assume additionally that $\gcd(P,b^1)=\gcd(P,b^2)=1$. Given a pair $(c^1,c^2)$ of degree $p+1$ real polynomials that satisfy \eqref{eq:Q} for some $Q$ with $R=0$, there is a unique tangent vector to the space of spectral data $(\dot P, \dot b^1, \dot b^2)$.
\begin{proof}

We are attempting to solve \eqref{eq:EMPDi} for the dotted quantities. Using Bezout's Identity, it has a solution if and only if $\gcd(P,b^i)$ divides the right hand side. But under our asusmptions, this is one and so there is always a solution. The first concern is that the two equations for $i=1,2$ may give different solutions for $\dot P$. But this cannot be, since $\dot P$ is determined by the value of the equation at the distict roots of $P$, namely
\[
-\dot P(\alpha) b^i(\alpha) = -P'(\alpha)\alpha\hat c^i(\alpha)
\]
The polynomials $\hat c^i$ are determined by \eqref{eq:Q} and so if $\dot P^1$ and $\dot P^2$ are the two solutions arrising from the two equations then
\begin{align*}
b^1(\alpha) c^2(\alpha) - b^2(\alpha) c^1(\alpha) &= Q(\alpha) P(\alpha) = 0 \\
\dot P^1(\alpha)
&= -P'(\alpha)\alpha \frac{\hat c^1(\alpha)}{b^1(\alpha)} \\
&= -P'(\alpha)\alpha \frac{\hat c^2(\alpha)}{b^2(\alpha)} \\
&= \dot P^2(\alpha).
\end{align*}

The second consideration is the reality of the solutions. It is not obvious that the right hand side is real, but if it is then so will be the solutions. To see how to take the real involution of a derivative, we compute the following, supposing $q$ is real of degree $k$
\begin{align*}
ζ^k \bar{q}(ζ^{-1}) &= q(ζ) \\
kζ^{k-1} \bar{q}(ζ^{-1}) - ζ^{k-2} \bar{q}'(ζ^{-1}) &= q'(ζ) \\
ρ^*(ζq') = ζ^{k-1}\bar{q}'(ζ^{-1}) &= k q(ζ) - ζq'(ζ)
\end{align*}
For an imaginary polynomial, the right hand side changes sign. Thus we can compute the involution of the right hand side of \eqref{eq:EMPDi}.
\begin{align*}
ρ^*(2P\hat{c^i} - 2Pζ\hat{c^i}' + ζP'\hat{c^i})
&= -2P\hat{c^i} - 2P\bra{-(p+3)\hat{c^i} + ζ\hat{c^i}'} - \bra{(2p+2)P - ζP'}\hat{c^i} \\
&= (-2+2p+6-2p-2)P\hat{c^i} - 2Pζ\hat{c^i}' + ζP'\hat{c^i} \\
&= 2P\hat{c^i} - 2Pζ\hat{c^i}' + ζP'\hat{c^i}
\end{align*}
Remarkably then, this is a real polynomial and there are solutions of degree $2p+2$ and $p+3$ respectively that are real.

The next worry is that this solution may not be unique. Indeed it is not. Let $\dot {\bf P}, \dot {\bf b}^1, \dot {\bf b}^2$ be a solution. Then the other solutions are given by, for $s\in \R$,
\begin{align*}
\dot P &= \dot {\bf P} + 2sP \\
\dot b^i &= \dot {\bf b}^i + sb^i \\
\end{align*}
But if we fix a scaling of $P$, say $\abs{P_0} = const$, then differentiating we obtain
\[
\arg \dot P_0 = \pm \arg i P_0
\]
which it leads to a picture

\begin{center}
\begin{tikzpicture}[scale=1.5]
    % Draw axes
    \draw[<->] (-3,0) -- (3,0) node[right] {$z\in\C$};
	  \draw[<->] (0,-2) -- (0,2);
    % Draw P_0
    \draw [style=help lines] (0,0) -- (2,1) ;
		\fill (2,1) circle (1pt) node[right]{$P_0$};
		% draw the lines
    \draw (-2,0) -- (2,2) node[right]{$\{z=\dot {\bf P}_0 + 2sP_0\}$};
		\fill (1,1.5) circle (1pt) node[above]{$\dot {\bf P}_0$};

    \draw (-1,2)  -- (1,-2) node[right]{$\{\arg z = \pm i \frac{P_0}{const}\}$};

		\fill (-0.4,0.8) circle (2pt) node[above left=2pt]{$\dot P_0$};
\end{tikzpicture}
\end{center}

And so we see if we fix a scaling of the spectral curve, then there is a unique solution to \eqref{eq:EMPDi}, and hence a unique tangent vector.

Finally then there is a second necessary condition that we derived. Recall \eqref{eq:residueTangent}. We must satisfy this infinitesimally, so that if we were to integrate out a path, the condition of no-residues would hold along the curve. We will see that this condition is satisfied already.
\begin{align*}
\dot{P}_1 b_0 + P_1 \dot{b}_0 - 2(\dot{P}_0 b_1 + P_0 \dot{b}_1)
&= P_1\dot{b}_0 - 2\dot{P}_0b_1 + 3P_1\hat{c}_0 - P_0\dot{b}_1 + 2P_1\dot{b}_0 \\
&= 3\bra{ P_1\dot{b}_0 - \dot{P}_0b_1 + P_1\hat{c}_0} \\
&= \frac{3}{P_0}\bra{ P_0P_1\dot{b}_0 - P_0\dot{P}_0b_1 + P_1\bra{ \frac{1}{2}\dot{P_0}b_0 - P_0\dot{b_0} }} \\
&= \frac{3\dot{P}_0}{P_0}\bra{ - P_0b_1 + \frac{1}{2}P_1b_0 }\\
&= 0
\end{align*}
The substitution in the first line comes from the $ζ$ linear terms of \eqref{eq:Q}, the third line from the constant terms of that equation and the last line coem from the fact that the quantity in the bracket is exactly the residue at $ζ=0$, which is zero by the assumption that $(b^1,b^2,P)$ is spectal data already.

\end{proof}
\end{lem}

The next question that arises is whether at a given point $(P,b^1,b^2)$ the $\hat c$'s and $Q$ are uniquely determined by a tangent vector $(\dot P, \dot b^1, \dot b^2)$. Since the equations are linear in the $t$-derivative quantities, we need only consider the zero tangent vector. But then
\[
0 = 2P\left( \hat c^i - \zeta\hat {c^i}'\right) + P'\zeta\hat c^i
\]
and the assumption of a nonsingular spectral curve means that $P$ and $P'$ have no common factors and by evaluation at the roots of $P$ implies that $\hat c^i=0$. This immediately implies that $Q=0$ also. So every real quadratic $Q$ is associated to at most one tangent vector to the space of spectral data.

The moral of these two lemmata is that at any point of spectral data there is a plane of deformation directions. Note that the condition on $Q$ is linear on its coefficents, so there is a plane of such $Q$ and each can then be used to first find a pair $C^1,c^2$ and then further a unique tangent vector. Thus, given a section of $Q$'s and an intial piece of spectral data, this data of tangent vectors then amounts to a first order linear ODE, which can be solved to give a deformation of the spectral data. (One may be concerned there is a catch-22 at play here: how is one to apply the condition $R=0$ on $Q$ without already having knowledge of the moduli of spectral data? Fear not. There are frames of meromorphic differentials defined for every hyperelliptic curve such that if the curve is actually a spectral curve, any admissible differentials are in the lattice generated by this frame. Thus we can use this frame to apply the condition and chose a section of $Q$'s, and if we start at a point in the moduli we will never flow outside it accidentally. )

IDEA: Can we say some sort of foliation result/ Frobenius theorem here to show the space of spectral data is an integral submanifold? \todo{this}







What additional constraints are there if there are common zeroes amoung the polynomials? Suppose that $F = \gcd(P,b^1,b^2)$ is such a common factor. Take any root $α$ of $F$. Then from \eqref{eq:EMPDi} we have that
\begin{align}
\dot P b^i - 2P\dot b^i &= 2P\left( \hat c^i - \zeta\hat {c^i}'\right) + P'\zeta\hat c^i \\
0 &= 0 + P'(α)α\hat c^i(α)
\end{align}
In the nonsingular and nonconformal case, this implies that $\hat c^i(α) = 0$ so that $F$ also divides both $\hat c$'s. The Q-equation \eqref{eq:Q} then implies that
\[
F^2 \bar{\tilde{b}^1\tilde{c}^2 - \tilde{b}^2\tilde{c}^1 = F Q \tilde{P}}
\]
where tildes repesent polynomials with the factor of $F$ removed. Again by nonsingularity, $F$ cannot divide $\tilde{P}$ (else $P$ would be divisible by $F$, and hence have multiple roots) so must divide $Q$. Thus we see that $F$ can either be a constant or a real quadratic polynomial, but not any higher degree. If $b^1$ and $b^2$ have a common factor that is not shared by $P$, it must also divide $Q$. Hence if $b^1$ and $b^2$ have a nontrivial common factor, regardless of whether it is common to $P$, it is at most quadratic (a common root on the unit circle would be a linear example) and divides $Q$.














\begin{lem}[Nonsigular, nonconformal, no common factor]
Take a slight generalisation and assume $\gcd(P,b^1)=F^1$ and $\gcd(P,b^2)=F^2$ but $\gcd(b^1,b^2)=1$. Then for every real quadratic polynomial $Q$ satisfying $R=0$ (with the sum now taken over the roots of $b^1/\gcd(P,b^1)$), there is a solution $(\dot P, \dot b^1, \dot b^2)$ to \eqref{eq:EMPDi}.
\begin{proof}

As $b^i$ is real and the roots of $P$ cannot lie on the unit circle, the common factors must be even degree $2d_i$, ie $F^i=\Pi_j (ζ-α_j)(1-\bar{α_j}ζ)$ and the polynomials factor as $b^i = F^i\tilde{b^i}$ and write $P = F^1F^2\tilde{P}$. Consider the equation
\begin{align}
\tilde{b^1}\tilde{c^2} - \tilde{b^2}\tilde{c^1} = Q\tilde{P}.\labelthis{eq:Qreduced}
\end{align}

With the common factors excluded from this modified equation, Bezout's identity again says that there is a unique solution for $\tilde{c^1},\tilde{c^2}$ for degrees at most $p+2-2d_1$ and $p+2-2d_2$ respectively. But the $R=0$ condition is exactly the one that will ensure that $c^2$ is infact of degree $p+1-2d_2$. Considering the $ζ^{2p+5-2d_1-2d_2}$ terms of this equation it implies that $\tilde{c}^1$ is of degree $p+1-2d_1$. Let $c^i :=  F^i\tilde{c}^i$. Then $c^1, c^2$ solve \eqref{eq:Q}, a necessary condition to solving \eqref{eq:EMPDi}. The previous coprimality assumption is no longer true. Every term however contains the factor $F^i$, so the equation is still solvable. For each $i$, call it $({\bf\dot P^i}, {\bf\dot b^i})$. Then for any degree $2d_i$ polynomial $s^i$ another solution is
\[
{\bf\dot P^i} + 2s^i F^{3-i}\tilde P, {\bf\dot b^i} + s^i \tilde b^1
\]
This does not mean however that there is a large space of solutions, because $\dot P$ must satisfy both equations. As before, $\dot P$ is determined by its values at the roots of $P$. If there were a common root of $b^1$, $b^2$ and $P$ it would completely factor from both equations; it wouldn't provide any constraint on $\dot P$. The assumption says though that there are no such common roots. Thus every root of $P$ constrains $\dot P$, and \eqref{eq:Qreduced} means that the two equations are consistent at the roots of $\tilde P$. So there for there must be some $s^i$'s that make the two solutions consistent. Having harmonised the solutions we are still free to add real multiples $P$, ie
\[
\{ ({\bf \dot P^1} + 2s^1 F^2 \tilde{P} + 2u F^1F^2\tilde{P}, {\bf \dot b^1} + s^1\tilde{b}^1 + u F^1\tilde{b}^1, {\bf \dot b^2} + s^2\tilde{b}^2 + uF^2 \tilde{b}^2) \mid u \in \R\}.
\]
The choice of scalar $u$ is again linked to the scaling of $P$, so there is an essentially unique vector $(\dot P, \dot b^1, \dot b^2)$ for every choice $Q$.
\end{proof}
\end{lem}












\begin{lem}[Nonsigular, nonconformal, some common factors]
Next assume $\gcd(P,b^1)=F^1$ and $\gcd(P,b^2)=F^2$ and $\gcd(b^1,b^2)=F$ but $\gcd(P,b^1,b^2)=1$. If $F$ is linear, take any real linear polynomial $Q$ or if $F$ is quadratic then take a pair of real numbers $\tilde Q$ and $r$. In either case, for any such choice there is a solution $(\dot P, \dot b^1, \dot b^2)$ to \eqref{eq:EMPDi}.
\begin{proof}

Write $P = F^1F^2\tilde{P}$ and $b^i = FF^i\tilde b^i$. Let the degrees be $2d_i = \deg F^i$, $d = \deg F$. If $d=1$ consider the equation
\begin{align}
\tilde{b^1}\tilde{c^2} - \tilde{b^2}\tilde{c^1} = Q\tilde{P}.
\end{align}

This is becoming familiar. There is a unique solution for $\tilde{c^1},\tilde{c^2}$ for degrees at most $p+1-2d_1$ and $p+1-2d_2$ respectively. Let $c^i :=  F^i\tilde{c}^i$. Then $c^1, c^2$ solve \eqref{eq:Q}.

If $d=2$ then consider instead
\begin{align}
\tilde{b^1}\tilde{c^2} - \tilde{b^2}\tilde{c^1} = \tilde{Q}\tilde{P}.
\end{align}
Real solutions for $\tilde{c^1},\tilde{c^2}$ of degrees at most $g+1-2d_1$ and $g+1-2d_2$ are now given by
\[
\{ ({\bf \tilde c^1} + r \tilde{b}^1, {\bf \tilde c^2} + r \tilde{b}^1) \mid r\in \R\}.
\]
This is naturally an affine space, so there is no `correct' choice for the origin. But for the purposes of specifying the parameter $r$ in the lemma, we may for example take $r=0$ to be the point where the constant coefficent of $\tilde c^1$ is of minimum modulus. Let $c^i :=  F^i\tilde{c}^i$. Then $c^1, c^2$ solve \eqref{eq:Q}.

The first step in solving \eqref{eq:EMPDi} is to factor from every term $F^i$. For each $i$, call the solution to the reduced equation $({\bf\dot P^i}, {\bf\dot b^i})$. Then for any degree $2d_i$ polynomial $s^i$ another solution is
\[
{\bf\dot P^i} + 2s^i F^{3-i}\tilde P, {\bf\dot b^i} + s^i \tilde b^1
\]
This does not mean however that there is a large space of solutions, because $\dot P$ must satisfy both equations. As before, $\dot P$ is determined by its values at the roots of $P$. Again by assumption there is no common root of $b^1$, $b^2$ and $P$, so there for there must be some $s^i$'s that make the two solutions consistent. Solutions are therefore
\[
\{ ({\bf \dot P^1} + 2s^1 F^2 \tilde{P} + 2u F^1F^2\tilde{P}, {\bf \dot b^1} + s^1\tilde{b}^1 + u F^1\tilde{b}^1, {\bf \dot b^2} + s^2\tilde{b}^2 + uF^2 \tilde{b}^2) \mid u \in \R\}.
\]
As before, the choice of $u$ is just scaling so we have constructed an essentially unique vector $(\dot P, \dot b^1, \dot b^2)$.
\end{proof}
\end{lem}













\begin{lem}[Nonsigular, nonconformal, triple common factor]
Assume $\gcd(P,b^1,b^2) = F$ is a real quadratic, $\gcd(P/F,b^1/F)=F^1$ and $\gcd(P/F,b^2/F)=F^2$. Take a pair of real numbers $\tilde Q$ and $r$. For any such choice there is a solution $(\dot P, \dot b^1, \dot b^2)$ to \eqref{eq:EMPDi}. NEEDS CONDITION ON P \todo{figure out the condition}
\begin{proof}

Write $P = F^1F^2F\tilde{P}$ and $b^i = FF^i\tilde b^i$. Let the degrees be $2d_i = \deg F^i$. As was alluded to in the prior computation, the Q-equation is dramatically reduced.
\begin{align}
\tilde{b^1}\tilde{c^2} - \tilde{b^2}\tilde{c^1} = \tilde{Q}\tilde{P}.
\end{align}
There is a unique solution for $\tilde{c^1},\tilde{c^2}$ for degrees at most $p-2d_1$ and $p-2d_2$ respectively. By the condition CONDITION\todo{me}, the soltuions are actually one degree less. Let $c^i :=  FF^i\tilde{c}^i$. Then $c^1, c^2$ solve \eqref{eq:Q} and are both degree $p+1$

We now progree to solving \eqref{eq:EMPDi}. Factor $FF^i$ from every term. For each $i$, call the solution to the reduced equation $({\bf\dot P^i}, {\bf\dot b^i})$. Then for any degree $2d_i$ polynomial $s^i$ another solution is
\[
{\bf\dot P^i} + 2s^i F^{3-i}\tilde P, {\bf\dot b^i} + s^i \tilde b^1
\]
Unlike before, $\dot P$ is not fully determined by its values at the roots of $P$. FIXME \todo{this} Solutions are therefore
\[
\{ ({\bf \dot P^1} + 2s^1 F^2 \tilde{P} + 2u F^1F^2\tilde{P}, {\bf \dot b^1} + s^1\tilde{b}^1 + u F^1\tilde{b}^1, {\bf \dot b^2} + s^2\tilde{b}^2 + uF^2 \tilde{b}^2) \mid u \in \R\}.
\]
As before, the choice of $u$ is just scaling so we have constructed an essentially unique vector $(\dot P, \dot b^1, \dot b^2)$.
\end{proof}
\end{lem}












\begin{lem}[Nonsigular, conformal, generic]
Take a triple of conformal spectral data $(P,b^1,b^2)$, with a nonsingular spectral curve given by $η^2 = P$ of genus $p$. For any real numbers $Q_1$ and $r$ there is a unique vector to the space of spectral data.

\begin{proof}
Conformal spectral data is distinguished by $P(0)=0$. From the residue condition, or directly by consideration of the order of the poles, $b^i_0 = 0$ also. We may write therefore that $P= ζ\tilde{P}$ and $b^i = ζ \tilde{b}^i$, where $\tilde{P}$ is a real polynomial of degree $2p$ and both $b^i$ are real polynomials of degree $p+1$ (note that $ζ$ is a real quadratic).

Factoring the common root, consider the necessary condition
\[
\tilde{b}^1 c^2 - \tilde{b}^2 c^1 = Q\tilde{P}.\labelthis{eq:QConformal}
\]
There is a unique solution to this where both $c^i$ polynomials are degree $p$. The space of solutions of at most degree $p+1$ is therefore the same with an arbitrary multiple of $\tilde{b}^i$ added. Within this space, by consideration of degrees, there must be a real solution $(\bf c^1, \bf c^2)$ and all other real solutions are given by
\[
\{ ({\bf c^1} + r \tilde{b}^1, {\bf c^2} + r \tilde{b}^1) \mid r\in \R\}.
\]

Moving on, we attempt to solve a factored version of \eqref{eq:EMPDi}, namely
\[
\dot P \tilde{b}^i - 2\tilde{P}\dot b^i = 2\tilde{P}\left( \hat c^i - \zeta\hat {c^i}'\right) + P'\hat c^i. \labelthis{eq:EMPDiConformal}
\]
For the dotted quatities $\dot P$ and $\dot b^1$ there are unique solutions of degree $2p-1$ and $p$ respectively. Mulitplying the whole equation by $ζ$ gives back \eqref{eq:EMPDi}, so the right hand side is real and there are real solutions of degree $2p+2$ and $p+3$. The space of solutions is therefore
\[
\{ ({\bf \dot P} + 2s \tilde{P}, {\bf \dot b^1} + s \tilde{b}^1, {\bf \dot b^2} + s \tilde{b}^2) \mid s \text { a real quadratic polynomial}\}.
\]
This space appears to be too large, but there are additional constraints to satisfy. The derivate of the residue condition in the conformal case requires that
\[
2 \dot{P}_0 b^i_1 - P_1 \dot{b}^i_0 = 0
\]
This is not automatically true as it was in the nonconformal case. Let $s = s_0 + s_1ζ + \bar{s}_0 ζ^2$ we see that this condition for $i=1$ implies that
\[
2 {\bf\dot{P}_0} b^1_1 - P_1 {\bf \dot{b}^1_0} + 3 s_0 P_1 b^1_1 = 0
\]
which fully determines $s_0$. Can we therefore simultaneously satisfy the condition for $i=2$? Note that \eqref{eq:EMPDiConformal} in the lowest degree says that
\[
\dot{P}_0 b^i_1 - 2P_1\dot{b}^i_0 = -3 P_1 c^i_0
\]
and \eqref{eq:QConformal} in the lowest degree yields
\begin{align*}
b^1_1 c^2_0 &= Q_0 P_1 + b^2_1 c^1_0 \\
b^1_1 \bra{{\bf\dot{P}_0} b^2_1 - 2P_1{\bf\dot{b}^2_0}} &= -3Q_0 (P_1)^2 + b^2_1 \bra{{\bf\dot{P}_0} b^1_1 - 2P_1{\bf\dot{b}^1_0}} \\
2b^1_1 {\bf\dot{b}^2_0} &= 3Q_0 P_1 + 2 b^2_1 {\bf\dot{b}^1_0} \\
~\\
b^1_1 \bra{2 {\bf\dot{P}_0} b^2_1 - P_1 {\bf \dot{b}^2_0} + 3 s_0 P_1 b^2_1}
&= 2 {\bf\dot{P}_0} b^1_1b^2_1 - P_1 b^1_1{\bf \dot{b}^2_0} + 3 s_0 P_1 b^1_1b^2_1 \\
&= 2 {\bf\dot{P}_0} b^1_1b^2_1 - P_1 \bra{\frac{3}{2}Q_0 P_1 + b^2_1 {\bf\dot{b}^1_0}} + 3 s_0 P_1 b^1_1b^2_1 \\
&= b^2_1\bra{2 {\bf\dot{P}_0} b^1_1 - P_1 {\bf\dot{b}^1_0} + 3 s_0 P_1 b^1_1 } - \frac{3}{2}Q_0 (P_1)^2 \\
&= - \frac{3}{2}Q_0 (P_1)^2
\end{align*}
So the second condition is also true if and only if $Q_0=0$. Hence we may only choose $Q$ to be of the form $Q = Q_1 ζ$ for some real scalar.

Having passed this check, there is still one free parameter, namely for any $Q_1$ and $r$, the corresponding tangent vectors are
\[
\{ ({\bf \dot P} + 2(s_0+\bar{s}_0ζ^2) \tilde{P} + 2s_1ζ \tilde{P}, {\bf \dot b^1} + (s_0+\bar{s}_0ζ^2) \tilde{b}^1 + s_1ζ \tilde{b}^1, {\bf \dot b^2} + (s_0+\bar{s}_0ζ^2) \tilde{b}^2 + s_1ζ \tilde{b}^2) \mid s_1 \in \R\}.
\]
But our free choice of $s_1\in\R$ is only adding multiples of $(2P,b^1,b^2)$, which as in the nonconformal case is simply a rescaling of the spectral curve.
\end{proof}
\end{lem}















\begin{lem}[Nonsigular, conformal, common zeroes]
The final nonsingular case to consider is to make a slight generalisation and assume again $\gcd(P,b^1)=ζF^1$ and $\gcd(P,b^2)=ζF^2$. Then for every pair of real numbers $(Q_1,r)$, there is a solution $(\dot P, \dot b^1, \dot b^2)$ to \eqref{eq:EMPDi}.

\begin{proof}
The proof of this is a hybrid of the previous two proofs. Necessarily $c^i = F^i \tilde{c}^i$, but then the argument from the conformal case applies to give a solution to \eqref{eq:Q} for every $Q = Q_1ζ$ and $r\in\R$. The \eqref{eq:EMPDiConformal} is massively underdetermined, however the argument from the nonconformal case applied here shows that $\dot P$ needs to have certain values at the nonzero roots of $P$, which there are $2p$ (not $2p+2$ as in the nonconformal case). Hence there is only a real quadratic freedom in the solution to that equation. Finally the argument about the residue condition applies to reduce this to just rescaling of spectral curve.
\end{proof}
\end{lem}












\subsection{Singular cases}
\label{sub:Singular cases}
These are incomplete. But there are no singular spectral curves in genus zero, one or two; hence it's a nice to have but not of consequence to later sections which focus on these genii (genuses?).


\begin{lem}[$b^1$ and $b^2$, on the unit circle]
Suppose that $b^1$ and $b^2$ share a single common root $β\in\S^1$ and that $P(β)\neq 0$. Then for either every real linear polynomial $\tilde Q$ or pair of real numbers $(r,\tilde Q)$ there is a solution $(\dot P, \dot b^1, \dot b^2)$ to \eqref{eq:EMPDi}.
\end{lem}

Proof.

Suppose first that $β$ is a simple root of at least one the polynomials. As before, begin by defining versions of the data with the common factor removed $b^i = \sqrt{-\bar{β}}(ζ-β) \tilde b^i$. Take any real linear polynomial $\tilde Q$ and endevour to solve
\[
\tilde b^1 c^2 - \tilde b^2 c^1 = \tilde Q P.
\]
Bezout's identity says that there is a unique solution to this equation $(c^1,c^2)$ where the polynomials are both of degree $g+1$. Multiplying the above equation by the factor $\sqrt{-\bar{β}}(ζ-β)$ gives a solution to \eqref{eq:Q}.

Suppose next that $β$ is a root of order 2 or more for both polynomials. Then we can see that it is only possible to satisfy the necessary condition of the Q-equation the $b$'s have at most an order 2 zero not in common with $P$, because $Q$ is quadratic and could not accomedate any higher order. Therefore consider $F=\bar{β}(ζ-β)^2$, $b^i = F \tilde b^i$ and the equation
\[
\tilde b^1 c^2 - \tilde b^2 c^1 = \tilde Q P.
\]
for a real scalar $\tilde Q$. Bezout's identity give fundemental solutions of degree $g+1$. The space of solutions to \eqref{eq:Q} are then
\[
F{\bf c^1} + r \tilde b^1, F{\bf c^2} + r \tilde b^2
\]

In either case, having found $c$'s that satisfy the necessary condition, the second half of the proof of the generic lemma applies to give a tangent vector.
\qed







\begin{lem}[$b^1$ and $b^2$, off the unit circle]
Suppose that $b^1$ and $b^2$ share a single pair of common roots $β,\bar{β}^{-1}\not\in\S^1$ and that $P(β)\neq 0$. Then for every pair of real numbers $(\tilde Q,r)$ there is a solution $(\dot P, \dot b^1, \dot b^2)$ to \eqref{eq:EMPDi}.
\end{lem}

Proof.

First observe, similiar to the previous case, that because $Q$ is quadratic and $P$ does not share the common root of the $b$'s, $β$ and $\cji{β}$ are first order zeroes. Write $b^i = (ζ-β)(1-\bar{β}ζ) \tilde b^i$ and consider the equation
\[
\tilde b^1 c^2 - \tilde b^2 c^1 = \tilde Q P.
\]
In this situation, Bezout's identity says that there is a family of solutions
\[
{\bf c^1} + r \tilde b^1, {\bf c^2} + r \tilde b^2
\]
where the $c$'s are degree $g+1$. $r$ is an affine coordinate for this space, but we may choose an origin for it such that ${\bf c^1}$ has degree $g$ for the sake of definiteness. Multiplying by the factor $(ζ-β)(1-\bar{β}ζ)$ gives a solution to \eqref{eq:Q} and a solution to \eqref{eq:EMPDi} follows easily. Different choices for $r$ lead to different $c$'s and therefore to different tangent vectors by the remark earlier that the relationship of tangent vectors to $c$'s is one-to-one.
\qed













\begin{lem}[$b^1$, $b^2$ and $P$] IN PROGRESS, NOT CERTAIN
Suppose that $b^1$, $b^2$ and $P$ share a single pair of common roots $α,\bar{α}^{-1}\not\in\S^1$ and $P$ has some constraint \todo{ find such constraint}. Then for every pair of real numbers $(\tilde Q,r)$ and real quadratic polynomial $s$ there is a solution $(\dot P, \dot b^1, \dot b^2)$ to \eqref{eq:EMPDi}.
\end{lem}

Proof.

Write $b^i = (ζ-α)(1-\bar{α}ζ) \tilde b^i$, $P = (ζ-α)(1-\bar{α}ζ) \tilde P$ and consider the equation
\[
\tilde b^1 \tilde c^2 - \tilde b^2 \tilde c^1 = \tilde Q \tilde P.
\]
In this situation, Bezout's identity says that there is a family of solutions
\[
{\bf \tilde c^1} + r \tilde b^1, {\bf \tilde c^2} + r \tilde b^2
\]
where the $c$'s are degree $g$. Use constraint to bump them down a degree. Let $c^1 = (ζ-α)(1-\bar{α}ζ) \tilde c^i$. Multiplying by the factor $(ζ-α)^2(1-\bar{α}ζ)^2$ gives a solution to \eqref{eq:Q}. Space of solutions is now
\[
{\bf c^1} + r \tilde b^1, {\bf c^2} + r \tilde b^2
\]
Because we forced the $c$'s to have the facotr, we can now remove the factor from every term in \eqref{eq:EMPDi}. Bezout says there is a space of solutions that look like
\[
{\bf\dot {\tilde b}^1} + q \tilde b^1, {\bf\dot {\tilde b}^2} + q \tilde b^2, {\bf\dot{\tilde P}} + 2q \tilde P,
\]
for real quadratic $q$. Choose $q$ such that the degrees are as low as possible, so then we may multiple the solutions to the reduced equation but the factor to get solutions to the nonreduced equation.


\subsection{Possibility of common factors}
Suppose that $(P,b^1,b^2)$ is a path in the space of spectral data, and that at $t=0$ we have the following common factors
\[
\gcd(P,b^1,b^2) = F,\;\; \gcd(P/F,b^i/F) = F^i,\;\; \gcd(b^1/FF^1, b^2/FF^2) = G,
\]
so that we may write
\[
P = F F^1 F^2 \tilde{P},\;\; b^i = F F^i G \tilde{b}^i.
\]
Inserting these into REF, we observe that
\[
\dot{P} F F^i G \tilde{b}^i = 2 F F^1 F^2 \tilde{P} (\dot{b}^i + \hat{c}^i - ζ\hat{c}^{i\prime}) + ζP' \hat{c}^i.
\]
Assuming that the spectral curve is nonsingular, $P'$ does not share any common factors with $P$. Hence we see that $FF^i$ divides $ζ\hat{c}^i$. If we assume that the spectral curve is nonconformal, $ζ$ is not a factor of $P$, so $FF^i$ divides $\hat{c}^i$. We write $\hat{c}^i = (ζ^2-1)FF^i\tilde{c}^i$. Putting these into the $Q$-equation
\begin{align*}
FF^1G\tilde{b}^1 FF^2\tilde{c}^2 - FF^2G\tilde{b}^2 FF^1\tilde{c}^1 &= Q FF^1F^2\tilde{P} \\
FG\bra{\tilde{b}^1 \tilde{c}^2 - \tilde{b}^2 \tilde{c}^1} &= Q \tilde{P}
\end{align*}
By definition, neither $F$ nor $G$ divide $\tilde{P}$ so they must divide $Q$. This provides a limit on the number of coincident roots that are allowed; $Q$ is quadratic so $FG$ degree two or less. Moreover, because all of $P,b^1,b^2$ are real, and $P$ has no roots on the unit circle, any common roots must come in conjugate inverse pairs. Hence at most one of $G$ and $F$ is not trivial. We write $Q = FG\tilde{Q}$.

In the conformal case, we know that $P_0, b^1_0$ and $b^2_0$ vanish at $t=0$. Thus $F$ is sure to include a factor of $ζ$. We write $F = ζ\tilde{F}$. The same reasoning as above then implies that $\tilde{F}G$ divides $Q$, with the same restriction that at most one of $F$ and $G$ can be nontrivial. Thus we have divided our analysis into six cases; where $F$, $G$ or neither is nontrivial, and conformal and nonconformal.
