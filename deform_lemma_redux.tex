%!TEX root = thesis.tex

\section{Deformation Theory}
\label{sec:Deformation Theory}
%
% \subsection{Bezout's Identity}
% \label{sub:Bezout's Identity}
% Suppose that $A$, $B$ and $C$ are single variable polynomials over the complex numbers of degrees $a,b,c$ respectively. A natural and long studied question is the solution to the linear equation
% \[
% AX - BY = C
% \labelthis{eqn:Bezout}
% \]
% in the unknowns $X$ and $Y$. Its complete solution for polynomials is due to \`Etienne B\'ezout REF. Let us recall the method of solution.
%
% Without loss of generality, we may assume that $\gcd(A,B,C) = 1$, otherwise we may divide out the common factor. Next, there cannot be a solution unless $\gcd(A',B') = 1$. If they have a common factor, then evaluation at a common root will cause the left hand side to vanish, but the right hand side will not. We will first solve $AX-BY = 1$ and then use this to give a complete solution. Evaluate the equation at the roots of $A$ to give values of $Y$, and at the roots of $B$ to learn similiarly about $X$. If either $A$ or $B$ has multiple roots, then differentiate the equation the appropriate number of times to generate further independent conditions. Hence we obtain a solution $(\tilde{X},\tilde{Y})$ of degrees at most $(b-1,a-1)$ by polynomial interpolation. $(C\tilde{X}, C\tilde{Y})$ is then a solution to $AX - BY = C$.
%
% If we have two solutions $(X,Y)$ and $(X',Y')$ to the equation, their difference satisfies $A(X-X') - B(Y-Y') = 0$. Again, evaluating at the roots of $A$ and $B$ show that the differences must be multiples of $B$ and $A$ respectively. Hence, given the solution $(\tilde{X},\tilde{Y})$ above, all other solutions are given by
% \[
% \Set {(C\tilde{X} + rB, C\tilde{Y} + rA) }{ r \text{ a polynomial} }.
% \]
%
% What can be said about the degree of these solutions? We can choose an $r$ in such a way that $X = CX' + rB$ is reduced to at most degree $b-1$. There are two regimes to consider for the corresponding $Y$ solution. If $c < a + \deg X$ then consideration of the leading terms of the equation forces $Y$ to be degree $a + \deg X - b$. Whereas if $c > a + \deg X$ the degree of $Y$ must stay high, $\deg Y = c-b$ and it is not possible to simultaneously lower the degrees of the solutions.
%
% Because of the reality condition that the spectral curve and differentials must satisfy, we are particularly interested in real solutions to Bezout equations. The real involution $ρ_k$ of the bundle $\mathcal{O}(k)$ over $\CP^1$ acts on sections $p(ζ)$ of this bundle by
% \[
% ρ_k^* p(ζ) = \bar{ζ}^{-k} \bar{p(ζ)}.
% \]
% We say that $p$ is real of degree $k$ if $ρ_k^* p (ζ)= p(\cji{ζ})$ and imaginary if $ρ_k^* p (ζ)= - p(\cji{ζ})$. Reality is multiplicative in the following sense. Suppose that $p$ is real of degree $k$ and $q$ is real of degree $l$. Then their product is real of degree $k+l$, since
% \[
% ρ_{k+l}^* pq(ζ) = \bar{ζ}^{-k-l} \bar{p(ζ)q(ζ)} = \bar{ζ}^{-k} \bar{p(ζ)} \bar{ζ}^{-l} \bar{q(ζ)} = \bra {ρ_{k}^* p(ζ)} \bra{ρ_{l}^* q(ζ)}.
% \]
% We say that a polynomial $p$ is real if it is real of degree $\deg p$, or if it has a factor of $ζ^k$ and it is real of degree $\deg p + k$. Suppose that $A,B,C$ are real polynomials of degrees $a,b,c$ and $(X,Y)$ is a solution to \eqref{eqn:Bezout}. Then applying $ρ_c$ and taking the sum, we see that
% \[
% A(X+ρ_{c-a}^*X) - B(Y+ρ_{c-b}^*Y) = 2C
% \]
% Hence the real parts of $(X,Y)$, namely $(\tfrac{1}{2}(X+ρ_{c-a}^*X), \tfrac{1}{2}(Y+ρ_{c-b}^*Y))$, are also a solution. All other real solutions can then be obtained by adding real multiples of $B$ and $A$ respectively. Explicitly, the space of real solutions to $AX-BY = C$ is
% \[
% \Set{
%     \bra { \frac{1}{2}C(\tilde{X}+ρ_{c-a}^*\tilde{X}) + rB, \frac{1}{2}C(\tilde{Y}+ρ_{c-b}^*\tilde{Y})) + rA) }
% }{
%     r \text{ a real polynomial}
% }.
% \]
%





\subsection{Set Up}
This chapter considers the infinitesimal deformations of the spectral data $(Σ,Θ^1,Θ^2)$, so called Whitham deformations. LIT REVIEW ABOUT WHITHAM DEFROMATIONS\todo{}. We have seen that the spectral data $(Σ,Θ^1,Θ^2)$ can be described more concretely as follows. The marked curve $Σ$ is described by $η^2 = P(ζ)$, for $P$ a real polynomial of degree $2g+2$, and each differential can be described by a real polynomial $b$ of degree $g+3$. Thus, we consider the moduli space of spectral data $\mathcal{M}$ as a subspace of the affine space of polynomials $\R^{4g+11} = \R^{2g+3}\times\R^{g+4}\times\R^{g+4}$.

A deformation of spectral data is a path $\ell:(-ε,ε) \to \mathcal{M}$, paramterised by $t$. An infinitesimal deformation is the tangent vector of such a curve at $t=0$. We shall use dashes to denote differentiation with respect to $\zeta$ and dots for differentiation with respect to $t$ evaluated at $t=0$. We may write $\ell(t) = (P(t),b^1(t),b^2(t))$ and $\ell(0) = (P,b^1,b^2)$.

We have seen that the differentials of the spectral data are the derivative of the logarithms of the eigenvalues $μ,\tilde{μ}$. With this in mind, introduce the notation $q^i$ for the functions $q^1 = \log μ$ and $q^2 = \log \tilde{μ}$, which are defined locally up to a constant and globally up to periods. For points along the curve $\ell$, by definition
\[
dq^i(t) = Θ^1(t) = \frac{1}{\zeta^2\eta}b^i(t,\zeta) d\zeta.
\]
Again, if we omit the parameter $t$ then we take this to mean the value at $t=0$.


The periods of the differentials of a triple of spectral data satify an integrality condition REF. Along a deformation, this forces that the periods have a fixed value. In particular then, their $t$-derivative is zero. Thus $\dot q^i$ is a well defined meromorphic function.

At a point of $Σ$ that is not a ramification point, a meromorphic function on $Σ(t)$ may be written as $ζ^k g(t,ζ)$. As $ζ$ is independent of $t$, the order is unchanged under differentiation by $t$. However, over a branch point $α(t)$, we must use a local coordinate $ξ(t)^2 = ζ - α(t)$, and a meromorphic function may be written as $ξ^k g(t,ξ)$. Differentiating with respect to $t$ yields
\[
\frac{d}{dt} ξ^k g(t,ξ) = -\frac{k}{2} ξ^{k-2} \dot{α}g - \frac{1}{2}ξ^{k-1}\dot{α}g' + ξ^k \dot{g}.
\]
As $dq^i$ has double poles without residues over $ζ=0,\infty$, it follows that $q^i$ has simple poles at those same points and is holomorphic at all other points. Applying the above calculation to $q^i$, we see that $\dot{q}^i$ may have simpe poles at the nonzero roots of $P$. If the curve is branched over $ζ=0$, then $\dot{q}^i$ may have a triple pole there. Otherwise $\dot{q}^i$ has at worst simple poles over $ζ=0$.

A consequence of this is that $ζη\dot{q}^i$ is holomorphic. This is invaraint under the holomorphic involution, so from REF MIRANDA, we can deduce that
\[
\dot{q}^i = \frac{1}{ζη}\hat{c}^i(ζ)
\]
for some degree $g+3$ imaginary polynomial $\hat{c}^i$. These polynomials encode the infinitesimal deformations of the differentials that preserve the periods.

Consider the closing condition in its integral form REF. For some consistent choice of $q^i$ along $γ_+$
\[
\int_{γ_+} dq^i = q^i(σ(ξ_1)) - q^i(ξ_1) \in 2π\iu \Z
\]
Hence, just like for the periods, the $t$-derivative of this integral is zero. Differentiating shows that $\dot q^i(σ(ξ_1)) = \dot q^i(ξ_1)$. But
\[
\dot{q}^i(σ(ξ_1)) = σ^* \dot{q}^i (ξ_1) = - \dot{q}^i(ξ_1).
\]
Thus $\hat c^i$ has a factor of $ζ-1$. The same reasoning applied to $γ_-$ leads to a factor of $ζ+1$. Let $\hat c^i(\zeta) = (\zeta^2 - 1) c^i(\zeta)$, for $c^i$ a real polynomial of degree $g+1$.

Naturally, there is a realtionship between the tangent vector to $\mathcal{M}$ at a point and these polynomials $c^i$. From the equality of mixed partial derivatives
\begin{align*}
\dot{(dq^i)} &= \frac{d\zeta}{\zeta^2\eta}\left( -\frac{1}{2}\frac{\dot P}{P}b^i + \dot b^i \right) \\
d\dot q^i & = \frac{1}{\zeta^2\eta}\left( -\hat c^i -\frac{1}{2}\frac{P'}{P}\zeta\hat c^i + \zeta\hat {c^i}'\right)d\zeta \\
\dot P b^i - 2P\dot b^i &= 2P\left( \hat c^i - \zeta\hat {c^i}'\right) + P'\zeta\hat c^i \labelthis{eqn:EMPDi}
\end{align*}
This yields equations linking $\dot{b}^i$ to $\hat{c}^i$ for each of $i=1,2$. The two equations are not however independent of one another, for they both contain $P$ and its derivatives. If we multiply the equations by $\hat c^2$ and $\hat c^1$ respectively and take the difference, we arrive at
\[
\dot P (b^1\hat c^2 - b^2\hat c^1) =  2P(\dot b^1\hat c^2 - \dot b^2\hat c^1 - \zeta\hat {c^1}'\hat c^2 + \zeta\hat {c^2}'\hat c^1).\labelthis{eqn:diffed}
\]
We aim to prove that $P$ divides $b^1\hat c^2 - b^2\hat c^1$ by showing that it vanishes at every root of $P$. If $α$ is a root of $P$ and not a root of $\dot{P}$, we see it is a root of $b^1\hat c^2 - b^2 \hat c^1$ immediately from \eqref{eqn:diffed}. Suppose then that $P$ and $\dot P$ have a common root $α$. If $α=0$, then we know from REF that $b^i_0=0$ and so $ζ$ divides $b^i$. If $α\neq 0$, from \eqref{eqn:EMPDi} we have that
\begin{align*}
\dot P(α) b^i(α) &= 2P(α)\left( \dot b^i(α) + \hat c^i(α) - α\hat {c^i}'(α)\right) +P'(α)α\hat c^i(α) \\
0 &= 0 + P'(α)α\hat c^i(α)
\end{align*}
But the assumption that the spectral curve is nonsingular forces $P'(α)\neq 0$. Thus we may conclude that $\hat{c}^i(α)=0$. Hence $P$ divides $b^1\hat c^2 - b^2 \hat c^1$ and there is some degree 4 polynomial $\hat{Q}$ such that
\[
b^1 \hat{c}^2 - b^2 \hat{c}^1 = \hat{Q} P
\]
As $\zeta^2-1$ is a factor of both $\hat{c}$'s, and $P$ has no zeroes on the unit circle, it must be a factor of $\hat{Q}$. Define $\hat Q = (\zeta^2-1)Q$ to give
\[
b^1 c^2 - b^2 c^1 = Q P \labelthis{eqn:Q}
\]
for some real quadratic polynomial $Q$. This is a necessary condition that the polynomials $c^i$ must satisfy if they correspond to a tangent vector.

The construction of the polynomials $c^i$ took into account the order of the poles, the real structure, the periods and the closing conditions of the differenitals, but we have not considered the residue free condition. For that, we need to preserve $P_1(t)b^i_0(t) - 2P_0(t)b^i_1(t) = 0$ at every point of the deformation $\ell$. Taking derivatives, we will need to show that
\[
\dot{P}_1 b^i_0 + P_1 \dot{b}^i_0 - 2 \dot{P}_0 b^i_1 - 2 P_0 \dot{b}^i_1 = 0 \labelthis{eqn:residueTangent}
\]
holds.

To close this section, we consider whether at a given point $(P,b^1,b^2)$ the polynomials $\hat{c}^i$ and $Q$ are uniquely determined by a tangent vector $(\dot P, \dot b^1, \dot b^2)$. Since the equations \eqref{EMPDi} are linear in the components of the tangent vector, we need only consider this question for the zero tangent vector. But then,
\[
0 = 2P\left[ (ζ^2-1) (c^i - ζc^{i\prime}) - 2ζ^2c^i \right] + ζ(ζ^2-1)P'c^i.
\]
The assumption of a nonsingular spectral curve requires that $P$ and $P'$ have no common factors. $P$ has at least $2g$ nonzero roots. Evaluation at any of these roots $α$ shows that $P'(α)α(α^2-1)c^i(α) = 0$, and hence $α$ is a root of $c^i$. And at the points $ζ = \pm 1$, $P(\pm 1)c^i(\pm 1) = 0$, which similarly shows that these two points are roots of $c^i$. Hence the degree $g+1$ polynomial vanishes at $2g+2$ distinct points and thus it is the zero polynomial. This immediately implies that $Q=0$ also. So every real quadratic $Q$ is associated to at most one tangent vector to the space of spectral data.













\subsection{Reconstructing a tangent vector}
Above we saw how to distill an infinitesimal deformation into a real quadratic polynomial $Q$. In this section we will show how to do the reverse; for a given $Q$, generate a tangent vector to $\mathcal{M}$. There are two steps to this process. Firstly, we construct polynomials $c^i$ solving \eqref{eqn:Q}. The second step is separate the information about the deformation contained in $c^i$ into $\dot{b}^i$ and $\dot{P}$ using the equations \eqref{eqn:EMPDi}. The essential method is to use Bezout's identity to produce solutions to each equations with the necessary properties. One must however do this in several different cases to handle the possibility of common factors, even though morally each case tells the same story.

Suppose that $\ell(t) = (P(t),b^1(t),b^2(t))$ is a path in the space of spectral data, and that at $t=0$ we have the following common factors
\[
\gcd(P,b^1,b^2) = F,\;\; \gcd(P/F,b^i/F) = F^i,\;\; \gcd(b^1/FF^1, b^2/FF^2) = G,
\labelthis{eqn:common factors}
\]
so that we may write
\[
P = F F^1 F^2 \tilde{P},\;\; b^i = F F^i G \tilde{b}^i.
\]
Inserting these into \eqref{eqn:EMPDi}, we observe that
\[
\dot{P} F F^i G \tilde{b}^i = 2 F F^1 F^2 \tilde{P} (\dot{b}^i + \hat{c}^i - ζ\hat{c}^{i\prime}) + ζP' \hat{c}^i.
\]
Assuming that the spectral curve is nonsingular, $P'$ does not share any common factors with $P$. Hence we see that $FF^i$ divides $ζ\hat{c}^i$. If we assume that the spectral curve is nonconformal, $ζ$ is not a factor of $P$, so $FF^i$ divides $\hat{c}^i$. We write $\hat{c}^i = (ζ^2-1)FF^i\tilde{c}^i$. We then then remove the common factors from the above equation,
\[
\dot{P} G \tilde{b}^i - 2 F^j \tilde{P} \dot{b}^i = 2 F^j \tilde{P} (\hat{c}^i - ζ\hat{c}^{i\prime}) + ζ(ζ^2-1)P' \tilde{c}^i,
\labelthis{eqn:EMPDi reduced}
\]
with $j\neq i$ (ie if $i=1$, then $j=2$). Inserting these expressions for $\hat{c}^i$ into the $Q$-equation \eqref{eqn:Q} on the other hand produces
\begin{align*}
FF^1G\tilde{b}^1 FF^2\tilde{c}^2 - FF^2G\tilde{b}^2 FF^1\tilde{c}^1 &= Q FF^1F^2\tilde{P} \\
FG\bra{\tilde{b}^1 \tilde{c}^2 - \tilde{b}^2 \tilde{c}^1} &= Q \tilde{P}
\end{align*}
By definition, neither $F$ nor $G$ divide $\tilde{P}$ so they must divide $Q$. This provides a limit on the number of coincident roots that are allowed; $Q$ is quadratic so $FG$ is degree two or less. Moreover, because all of $P,b^1,b^2$ are real, and $P$ has no roots on the unit circle, any common roots of the three polynomials must come in conjugate inverse pairs and so $F$ is even degree. Hence at most one of $G$ and $F$ is not trivial. We write $Q = FG\tilde{Q}$. With all these common factors removed, we have the equation
\[
\tilde{b}^1 \tilde{c}^2 - \tilde{b}^2 \tilde{c}^1 = \tilde{Q} \tilde{P}.
\labelthis{eqn:Q reduced}
\]

In the conformal case, we know that $P_0, b^1_0$ and $b^2_0$ vanish at $t=0$. Thus $F$ is sure to include a factor of $ζ$. We write $F = ζ\tilde{F}$. This time, we conclude that it is $\tilde{F}F^i$ that divides $\hat{c}^i$. The same reasoning as above then implies that $\tilde{F}G$ divides $Q$. However, the residue condition in this case simplifies in a way that forces another constraint on $Q$. At $t=0$,
\[
P_1 \dot{b}_0 - 2 \dot{P}_0 b_1 = 0.
\]
After removing the $ζ$ from every term, \eqref{eqn:EMPDi} in the lowest degree reads
\[
\dot P_0 b_1^i - 2P_1\dot b_0^i = 3P_1\hat{c}_0^i.
\]
Combining these two expressions shows that
\[
3P_1\hat{c}_0^i = \dot P_0 b_1^i - 4\dot{P}_0 b_1^i = -3\dot{P}_0 b_1^i.
\]
After removing the common factor of $ζ$, applying this to the lowest degree of \eqref{eqn:Q} we finally arrive at
\begin{align*}
Q_0 P_1 &= b^1_1 c^2_0 - b^2_1 c^1_0 \\
Q_0 (P_1)^2 &= b^1_1 (P_1 c^2_0) - b^2_1 (P_1 c^1_0) \\
&= b^1_1 (\dot{P}_0 b^2_1) - b^2_1 (\dot{P}_0 b^1_1) = 0.
\end{align*}
A spectral curve must be nonsingular at $ζ=0$, so if $P=0$ we can be sure that $P_1\neq 0$. Hence $Q_0$ must vanish at $t=0$. As $Q$ is a real quadratic polynomial, it must be of the form $Q=Q_1 ζ$ for some real number $Q_1$. Immediately then $\tilde{F} = G = 1$, as neither $P$ nor the polynomials $b^i$ are permitted to have multiple roots at $ζ=0$.

Thus we have divided our analysis into four cases; conformal or nonconformal, and within the nonconformal case whether $F$, $G$ or neither is nontrivial. The points where $F$ is nontrivial represent singular points of the vector field of infinitesimal deformations are are not considered further in this thesis.

In the cases where $F=G=1$, it will not be possible to solve for a tangent vector given an arbitrary quadratic polynomial $Q$. Consider \eqref{eqn:Q reduced} at the $k$ distinct roots of $\tilde{b}^2$, for $k \leq g + 3 - (d_F + d_2 + d_G)$. It is a linear system in the coefficients of $\tilde{c}^1$. If $\tilde{b}^2$ has higher order roots, then we may differentiate with respect to $ζ$ to obtain further relations. Let these roots be denoted $β_i$ and have multiplicities $r_i$. Suppose that $\tilde{c}^2$ were degree $n := g + 2 - (d_F + d_2 + d_G)$. The system of equations is
\[
\begin{bmatrix}
1 & β_1 & (β_1)^2 & \cdots & (β_1)^{r_1-1} & \cdots & (β_1)^{n} \\
0 & 1 & 2β_1 & \cdots & (r_1-1)(β_1)^{r_1-2} & \cdots & n(β_1)^{n-1} \\
\vdots & \vdots & \vdots & & \vdots & & \vdots \\
0 & 0 & 0 & \cdots & 1 & \cdots & \frac{n!}{(n+1 - r_1)!}(β_1)^{n+1-r_1} \\
1 & β_2 & (β_2)^2 & \cdots & (β_2)^{r_2-1} & \cdots & (β_2)^{n} \\
\vdots & \vdots & \vdots & & \vdots & & \vdots \\
0 & 0 & 0 & \cdots & 1 & \cdots & \frac{n!}{(n+1 - r_2)!}(β_2)^{n+1-r_2} \\
1 & β_3 & (β_3)^2 & \cdots & (β_3)^{r_2-1} & \cdots & (β_3)^{n} \\
\vdots & \vdots & \vdots & & \vdots & & \vdots \\
\vdots & \vdots & \vdots & & \vdots & & \vdots \\
0 & 0 & 0 & \cdots & 1 & \cdots & \frac{n!}{(n+1 - r_k)!}(β_k)^{n+1-r_k} \\
\end{bmatrix}
\begin{bmatrix}
\tilde{c}^2_0 \\~\\ \tilde{c}^2_1 \\~\\ \vdots \\~\\ \tilde{c}^2_n
\end{bmatrix}
=
\begin{bmatrix}
h(β_1) \\ h'(β_1) \\ \vdots \\ h^{(r_1-1)(β_1)} \\
h(β_2) \\ \vdots \\ h^{(r_2-1)(β_2)} \\
h(β_3) \\ \vdots \\ \vdots \\ h^{(r_k-1)(β_k)} \\
\end{bmatrix},
\]
where
\[
h(ζ) := \frac{\tilde{Q}(ζ)\tilde{P}(ζ)}{\tilde{b}^1(ζ)},
\]
which is holomorphic at every root $β_i$, as $\tilde{b}^1(β_i)\neq 0$.

The $(n+1)\times (n+1)$ matrix on the left is called the confluent Vandermonde matrix at the roots of $\tilde{b}^2$, and we shall denote it $V(\tilde{b}^2)$. Let the column vector on the right hand side be denoted $\mathbf{h}(P,b^1;b^2)$. A confluent Vandermonde matrix is always nonsingular, therefore there is always a solution to this equation. But we see that in general the solution $\tilde{c}^2$ is degree $n$, a degree larger than we require. Thus, we must introduce a constraint in our choice of $Q$ to ensure the solution is indeed degree $n-1 = g + 1 - (d_F + d_2 + d_G)$. From the highest order term of \eqref{eqn:Q reduced}, if $\tilde{c}^2$ is degree $g + 1 - (d_F + d_2 + d_G)$, then $\tilde{c}^1$ will be $g + 1 - (d_F + d_1 + d_G)$ without any further restrictions.

We define at any point $(P,b^1,b^2)$ of $\mathcal{M}$ and for any real quadratic polynomial $Q$, the function $R(P,b^1,b^2,Q)$ to be the dot product of the last row of $V(\tilde{b}^2)^{-1}$ with the vector $\mathbf{h}(P,b^1,b^2,Q)$. The condition that $R(Q) = 0$ is equivalent to the condition that $\tilde{c}^2_n=0$. This ensures that $\tilde{c}^2$ is the correct degree. It is important to note that $R$ is linear function in the coefficients of $Q$.

Further, $R$ satifies a reality type condition. We first note that $R$ is a continuous function on $\R^{d_{\tilde{P}}}\times\R^{d_{\tilde{b}^1}}\times\R^{d_{\tilde{b}^2}}$
\todo{describe this space better, roots of different polys are all disjoint, so open subset}.
This is obviously true at points where the roots of $\tilde{b}^2$ are distinct. Suppose that we are in a neighbourhood of a point where $\tilde{b^2}$ has a double root $β$. In this neighbourhood, there are two roots $β_1,β_2$ that collasce at $β$ to form the double root. The corresponding of the Vandermonde matrix is
\[
\begin{bmatrix}
1 & β_1 & (β_1)^2 & \cdots & (β_1)^{k} & \cdots & (β_1)^{n} \\
1 & β_2 & (β_2)^2 & \cdots & (β_2)^{k} & \cdots & (β_2)^{n} \\
&&&\vdots&&&
\end{bmatrix}
\begin{bmatrix}
\tilde{c}^2_0 \\~\\ \tilde{c}^2_1 \\~\\ \vdots \\~\\ \tilde{c}^2_n
\end{bmatrix}
=
\begin{bmatrix}
h(β_1) \\
h(β_2) \\
\vdots
\end{bmatrix}.
\]
Performing elementary row operations does not change the solution to this system, and so we may subtract the first row from the second and scale it by $(β_2-β_1)^{-1}$. This gives
\[
\begin{bmatrix}
1 & β_1 & (β_1)^2 & \cdots & (β_1)^{k} & \cdots & (β_1)^{n} \\
0 & 1 & β_2+β_1 & \cdots & \sum_{i=0}^{k-1}(β_1)^i(β_2)^{k-1-i} & \cdots & \sum_{i=0}^{n-1}(β_1)^i(β_2)^{n-1-i} \\
\vdots
\end{bmatrix}
\begin{bmatrix}
\tilde{c}^2_0 \\~\\ \tilde{c}^2_1 \\~\\ \vdots \\~\\ \tilde{c}^2_n
\end{bmatrix}
=
\begin{bmatrix}
h(β_1) \\
\frac{h(β_2) - h(β_1)}{β_2 - β_1} \\
\vdots
\end{bmatrix}.
\]
The limit of the above as $β_2 \to β_1$ is precisely the corresponding confluent Vandermonde matrix. The calculation for higher order roots is similar. This demonstrates that the coefficent matrix and the constant vector of the system of equations is continuous in the roots of the polynomial $b_2$. Combining this with the fact that inversion of a matrix is continuous and that the roots of a polynomial are continuous functions of coefficients (Viete map, track down reference in Whitney Complex Analytic Varieties \todo{ref}), this show that the solution is continuous. In particular, $R$ is the last component of the solution, and so is continuous.

This is useful because it allows one to compute the reality condition of $R$ on the open set where the roots of $\tilde{b}^2$ are distinct, and it will then follow that it holds everywhere on $SPACE$\todo{} by continuity. Let $\tilde{b}^2$ have $n+1$ distinct roots $β_i$. In this case, the form of the solution to the system of equations REF is elegant. Consider the Langrange polynomials at the roots of $\tilde{b}^2$,
\[
L_i (ζ) := \prod_{j \neq i} \frac{ζ-β_j}{β_i - β_j}.
\]
Each of these polynomials is degree $n$ and has the property that $L_i (β_j) = δ_{ij}$. The unique degree $n$ polynomial solving the system of equations is
\[
\tilde{c}^2(ζ) = \sum_{i = 1}^{n+1} h(β_i) L_i (ζ),
\]
and in particular the highest coefficent is
\[
R = \sum_{i = 1}^{n+1} h(β_i) \prod_{j \neq i} \bra{ β_i - β_j }^{-1}.
\]
Because $\tilde{b}^2$ is a real polynomial, its set of roots is invariant under $ζ \mapsto \cji{ζ}$. This creates an involution on the set of roots. Let $τ$ be such that $β_{τ(i)} = \cji{β}_i$. We now compute the conjugate of $R$.
\begin{align*}
\bar{R}
&= \sum_{i = 1}^{n+1} \bar{h(β_i)} \prod_{j \neq i} \bra{ \bar{β}_i - \bar{β}_j }^{-1} \\
&= \sum_{i = 1}^{n+1} \bar{\frac{Q(β_i)\tilde{P}(β_i)}{\tilde{b}^1(β_i)}}
 \prod_{j \neq i} \cji{β}_i \cji{β}_j \bra{ \cji{β}_j - \cji{β}_i }^{-1} \\
%%%%%%%%%%%%%%%%%%%%%%%%%%%%%%%%%%%%
&= \sum_{i = 1}^{n+1} \bar{β}_i^{g+1 - d_2} \frac{\tilde{Q}(\cji{β}_i)\tilde{P}(\cji{β}_i)}{\tilde{b}^1(\cji{β}_i)}
\prod_{j \neq i} \cji{β}_i β_{τ(j)} (-1)\bra{ β_{τ(i)} - β_{τ(j)} }^{-1} \\
%%%%%%%%%%%%%%%%%%%%%%%%%%%%%%%%%%%%
&= \sum_{i = 1}^{n+1} \bar{β}_i^{g+1 - d_2} h(β_{τ(i)})
\bra { \cji{β}_i }^{n} \bra{ \prod_{j \neq i}  β_{τ(j)}} (-1)^{n} \prod_{j \neq i}\bra{ β_{τ(i)} - β_{τ(j)} }^{-1} \\
%%%%%%%%%%%%%%%%%%%%%%%%%%%%%%%%%%%%
&= (-1)^{n}\bra{ \prod_{i=1}^{n+1}  β_i }  \sum_{i = 1}^{n+1} h(β_{τ(i)})
 \prod_{j \neq i}\bra{ β_{τ(i)} - β_{τ(j)} }^{-1} \\
%%%%%%%%%%%%%%%%%%%%%%%%%%%%%%%%%%%%
&= (-1)^{n}\bra{ \prod_{i=1}^{n+1}  β_i }  R. \\
\end{align*}
From this reality condition, it follows that $R(Q)=0$ only puts one real constraint on $Q$, and so at a point of SPACE\todo{} there is a two (real) dimensional plane of polynomials $Q$ that satisfy $R(Q) = 0$.

Let us return to the general situation now. The next obstacle to reconstrucitng a tangent vector from a given $Q$ is that the polynomials $c^i$ and $\dot{P},\dot{b}^i$ must be real. We establish that if a solution of the appropriate degree exists, then there will be a real solution of the same degree. Suppose that we have a polynomial equation $AX - BY = C$, for some real polynomials $A,B,C$ of degrees $a,b,c$ respectively. Further suppose that solutions $X,Y$ exist of degrees $c-a$ and $c-b$ respectively. Then observe
\begin{align*}
C(ζ) &= ζ^c \bar{ C(\cji{ζ})}
= ζ^{c}\bar{A(\cji{ζ})} \bar{X(\cji{ζ})} - ζ^{c}\bar{B(\cji{ζ})} \bar{Y(\cji{ζ})}
= A(ζ) ζ^{c-a}\bar{X(\cji{ζ})} - B(ζ) ζ^{c-b}\bar{Y(\cji{ζ})} \\
C(ζ)
&= A(ζ) \frac{1}{2}\bra{ X(ζ) + ζ^{c-a}\bar{X(\cji{ζ})}} - B(ζ) \frac{1}{2}\bra{ Y(ζ) + ζ^{c-b}\bar{Y(\cji{ζ})}} ,
\end{align*}
which demonstrates the existence of real solutions.

For \eqref{eqn:Q} it is obvious that it of this form. It is not obvious that the right hand side of \eqref{eqn:EMPDi} is real however. In particular, we must see how to take the real involution of a derivative. To do so, we compute the following, supposing $f$ is real of degree $k$
\begin{align*}
ζ^k \bar{f}(ζ^{-1}) &= f(ζ) \\
kζ^{k-1} \bar{f}(ζ^{-1}) - ζ^{k-2} \bar{f}'(ζ^{-1}) &= f'(ζ) \\
ρ^*(ζf') = ζ^{k-1}\bar{f}'(ζ^{-1}) &= k f(ζ) - ζf'(ζ)
\end{align*}
Thus we can compute the involution of the right hand side of \eqref{eqn:EMPDi}.
\begin{align*}
ρ^*\bra{ 2P\hat{c}^i - 2Pζ\hat{c}^{i\prime} + ζP'\hat{c}^i }
&= -2P\hat{c}^i - 2P\bra{-(g+3)\hat{c}^i + ζ\hat{c}^{i\prime}} - \bra{(2p+2)P - ζP'}\hat{c}^i \\
&= (-2+2g+6-2g-2)P\hat{c}^i - 2Pζ\hat{c}^{i\prime} + ζP'\hat{c}^i \\
&= 2P\hat{c}^i - 2Pζ\hat{c}^{i\prime} + ζP'\hat{c}^i
\end{align*}
Remarkably then, this is a real polynomial.

Finally, in some circumstances, we will need to describe the set of solutions of $AX-BY=C$. Consider the related equation $AX-BY = 0$. If $X$ is a solution, then by evaluating at the roots of $B$ we see that $B$ divides $X$. Return now to the equation $AX-BY=C$ and suppose then that $(X,Y)$. If $(X',Y')$ is any other solution, then $A(X-X') - B(Y-Y') = 0$. Therefore $X-X' = r B$ and $Y-Y' = rA$ for some polynomial $r$. Conversely, given any solution $(X,Y)$, it is clear that $(X+rB,Y+rA)$ is again a solution for every polynomial $r$.

Before we proceed to the results, it may be prudent to make a comment about the notational choices that have been made. We have attempted to use ligatures in a consistent way to indicate the factors that a polynomial does or does not have. For example, a hat indicates a factor of $ζ^2-1$, and a tilde indicates that common factors have been removed. Bold will signify a particular solution to an equation. To choice of this particular solution may or may not be unique. When a solution to an equation appears without bold, it signifies a solution from the set of potential solutions. We shall use $d$ with various subscripts to indicate the degrees of polynomials. Finally, we shall use $i$ and $j$ for indices ranging over $1$ and $2$, with the understanding that they are not equal. For example, if $i=1$, then we take $j=2$ and vice versa.




\begin{lem}[Nonconformal, $F=G=1$]
Take a triple of nonconformal spectral data $(P,b^1,b^2)$, with a nonsingular spectral curve given by $η^2 = P$ of genus $g$. Suppose that, with reference to \eqref{eqn:common factors}, $F=G=1$. For every $Q$ with $R(Q) = 0$, there exist unique real polynomials $c^i$ of degree $g+1$ that satisfy \eqref{eqn:Q} and, further, there is a unique tangent vector to the space of spectral data $(\dot P, \dot b^1, \dot b^2)$.

\begin{proof}
In order to solve \eqref{eqn:EMPDi} or \eqref{eqn:Q}, one must first solve their reduced counterparts \eqref{eqn:EMPDi reduced} and \eqref{eqn:Q reduced}.

If we consider \eqref{eqn:Q reduced} as a linear system on the coefficents of $\tilde{c}^i$, then it is overdetermined, but by evaluating it at the values at the roots of the polynomials $\tilde{b}^i$ it nicely seaprates into two independent subsystems. Consider \eqref{eqn:Q reduced} at the roots of $\tilde{b}^2$. If some of the roots are repeated, then consider the expansion near a $k$-th order root $β$ of $\tilde{b}^2$ and differentiate $(k-1)$ times. This yields a system of $g+3$ equations. The coefficent matrix in this system is a confluent Vandemonde matrix, and is always nonsingular. Likewise at the roots of $\tilde{b}^2$.

Thus there is a unique solution ${\bf \tilde{c}}^1, {\bf \tilde{c}}^2$ of degrees $g+2-d_1, g+2-d_2$, where $d_i = \deg F^i$. Explicitly, Hermite interpolation allows one to match the value of a function at $n+1$ points to order $m$ using a polynomial of degree $(n+1)(m+1)-1$. Take for $m$ the order of the highest order root of $\tilde{b}^1$ and consider the $(n+1)(m+1)$ polynomials such that\todo{don't use i j here, reserve them for indices.}
\[
H_{j,k}^{(l)}(β_i) = δ_{j,i}δ_{l,k}\;\;\text{ for }  l \leq m
\]

That is, $H_{j,k}$ is a polynomial that is zero and has zero derivatives to the first $m$ orders at every root $β_i$ of $\tilde{b}^1$, with the exception of its $k$-th derivative at $β_j$ which has a value of $1$. Each of these is degree $(n+1)(m+1)-1$. $g+3-d_1$ of these basis polynomials can be used to fit the data of the problem. The remaining $(n+1)(m+1)-(g+3-d_1)$ can be used to reduce degree of the solution by $1$ each, resulting in a solution of exactly degree $g+2-d_1$ as expected. The full formulae can be found in \cite{Spitzbart1960} for general confluent Vandermode matrices. We note only that ${\bf\tilde{c}}^1_{g+2-d_1}$ is $b^1_{g+3-d_1}R(Q)$, which by assumption is zero. By considering the leading order of \eqref{eqn:Q reduced}, if ${\bf\tilde{c}}^1_{g+2-d_1}$ vanishes, so too must ${\bf\tilde{c}}^2_{g+2-d_2}$.

Multiplying \eqref{eqn:Q reduced} through by $F^1F^2$, we arrive at unique solutions $c^i = F^i {\bf\tilde{c}}^i$ for \eqref{eqn:Q}. Both of these polynomials are degree $g+1$. We similiarly define $\hat{c}^i = (ζ^2 -1)c^i$.

Next we must solve \eqref{eqn:EMPDi reduced}, for which \eqref{eqn:Q reduced} was a necessary condition.
\[
\dot{P} \tilde{b}^i - 2 F^j \tilde{P} \dot{b}^i = 2 F^j \tilde{P} (\hat{c}^i - ζ\hat{c}^{i\prime}) + ζ(ζ^2-1)P' \tilde{c}^i.
\labelthis{eqn:EMPDi nc FG=1}
\]
Using Bezout's Identity, it has a solution because $\gcd(F^j\tilde{P},\tilde{b}^i) = 1$ The first concern is that the two equations for $i=1,2$ may give different solutions for $\dot P$, and indeed in general they do. However, for neither are the solutions unique, and we shall use the freedom in the space of solutions to find a common solution to both. Let a solution to each equation be $({\bf \dot{P}}^1, {\bf \dot{b}}^1)$ and $({\bf \dot{P}}^1, {\bf \dot{b}}^1)$. The sets of solutions are
\[
\Set { ({\bf \dot{P}}^1 + 2rF^2\tilde{P}, {\bf \dot{b}}^1 + r\tilde{b}^1 } )
{ r \text{ a real polynomial of degree } d_1 },
\]
and
\[
\Set { ({\bf \dot{P}}^2 + 2sF^1\tilde{P}, {\bf \dot{b}}^2 + r\tilde{b}^2 } )
{ s \text{ a real polynomial of degree } d_2 },
\]
respectively. First note that every element of both of these sets are in agreement at any root $α$ of $\tilde{P}$. From \eqref{eqn:EMPDi nc FG=1},
\[
\dot P^i(α) \tilde{b}^i(α) = P'(α)α(α^2 -1)\tilde{c}^i(α).
\]
By definition, roots of $\tilde{P}$ are roots of $P$ that are not common to either $\tilde{b}^1$ or $\tilde{b}^2$. If neither of them has a root at at $α$, then from \eqref{eqn:Q reduced} we see that
\[
\tilde{b}^1(α)\tilde{c}^2(α) - \tilde{b}^2(α)\tilde{c}^1(α) = Q(α)\tilde{P}(α) = 0,
\]
and
\[
\dot P^1(α)
= -P'(α)α (α^2 - 1)\frac{\tilde{c}^1(α)}{\tilde{b}^1(α)}
= -P'(α)α (α^2 - 1)\frac{\tilde{c}^2(α)}{\tilde{b}^2(α)}
= \dot P^2(α).
\]
At the $d_1$ roots of $F^1$, we see that every solution $\dot{P}^2$ takes the same value. Let $β$ be such a root, then
\[
\dot{P}^2(β)
= {\bf \dot{P}}^2(β) + 2sF^1(β)\tilde{P}(β)
= {\bf \dot{P}}^2(β)
= β (β^2-1) P'(β) \frac{\tilde{c}^2(β)}{\tilde{b}^2(β)},
\]
where we can be sure that $\tilde{b}^2(β) \neq 0$ because it cannot be a root of $b^2$ (if it were, $F\neq 1$). Also, $β$ is not a root of $\tilde{P}$ or $F^2$ by the assumption of nonsingularity. Thus this provides a constraint on the choice of $r$. Specifically we require that
\[
{\bf \dot{P}}^1(β) + 2r(β)F^2(β)\tilde{P}(β) = β (β^2-1) P'(β) \frac{\tilde{c}^2(β)}{\tilde{b}^2(β)}.
\]
Likewise, at the $d_2$ roots of $F^2$, we aquire constraints on the choice of $s$. It is always possible to satisfy such constraints, so we see that there is common solution $({\bf \dot{P}}, {\bf \dot{b}}^1, {\bf \dot{b}}^2)$ to \eqref{eqn:EMPDi reduced}. It is also a solution to \eqref{eqn:EMPDi}.

Still though, this solution is not unique. There remains one degree of freedom. For any real number $s$, we have solutions to \eqref{eqn:EMPDi} of the form
\begin{align*}
\dot P &= {\bf \dot{P}} + 2sP  \labelthis{eqn:P soln}\\
\dot b^i &= {\bf \dot{b}}^i + sb^i
\end{align*}
However, this freedom is simply the freedom to rescale $P$. We have chosen a preferred scaling of $P$, so our choice of $s$ is determined. Explicitly, if we were to allow other scalings, the formula for $P$ would be
\[
P = r(t) \prod_i (ζ-α_i)(1- \bar{α}_iζ),
\]
where $α_i$ are the roots inside the unit circle and $r(t)$ some real function. Then from any solution $\dot{P}$ (irrespective of the choice of $s$) we can determine how the roots $α_i$ are changing at $t=0$. Simply differentiate and evaluate at $α_i$
\begin{align*}
\dot{P} = \dot{r} \prod_i (ζ-α_i)(1- \bar{α}_iζ) + r(0) \sum_i (-\dot{α} &+ (\dot{α}\bar{α}+α\dot{\bar{α}})ζ - \dot{\bar{α}}ζ^2) \prod_{j\neq i} (ζ-α_j)(1- \bar{α}_jζ) \\
\dot{P}(α_i) &= -\dot{α}_i(1+α_i\bar{α}_i) \prod_{j\neq i} (α_i-α_j)(1- \bar{α}_j α_i).
\end{align*}
Thus we know the values of $\dot{α}_i$ independent of the scaling from our solution \eqref{eqn:P soln}. Alternatively, if we take the lowest order of this equation
\[
\dot{P}_0 = \dot {\bf P}_0 + 2sP_0 = \dot r P_0 + \sum_i (-\dot{α}_i)\prod_i (-α_j),
\]
so we may ensure that $r\equiv 1$ by choosing $s$ so that $\dot r = 0$. In summary, if we fix a scaling of the spectral curve, then there is a unique solution to \eqref{eqn:EMPDi}.

Finally then there is a second necessary condition that must be satisfied by our solution $(\dot{P},\dot{b}^1,\dot{b}^2)$. We must satisfy \eqref{eqn:residueTangent}, so that \eqref{eqn:residue} holds along the path. But this condition is satisfied already.
\begin{align*}
\dot{P}_1 b_0 + P_1 \dot{b}_0 - 2(\dot{P}_0 b_1 + P_0 \dot{b}_1)
&= P_1\dot{b}_0 - 2\dot{P}_0b_1 + 3P_1\hat{c}_0 - P_0\dot{b}_1 + 2P_1\dot{b}_0 \\
&= 3\bra{ P_1\dot{b}_0 - \dot{P}_0b_1 + P_1\hat{c}_0} \\
&= \frac{3}{P_0}\bra{ P_0P_1\dot{b}_0 - P_0\dot{P}_0b_1 + P_1\bra{ \frac{1}{2}\dot{P_0}b_0 - P_0\dot{b_0} }} \\
&= \frac{3\dot{P}_0}{P_0}\bra{ - P_0b_1 + \frac{1}{2}P_1b_0 }\\
&= 0
\end{align*}
The substitution in the first line comes from the $ζ^1$ terms of \eqref{eqn:Q}, the third line from the constant terms of that equation and the last line comes from the fact that the quantity in the bracket is exactly the residue at $ζ=0$, which is zero by the assumption that $(b^1,b^2,P)$ is spectal data.

Hence $(\dot{P},\dot{b}^1,\dot{b}^2)$ is a tangent vector to $\mathcal{M}$ at $(b^1,b^2,P)$.
\end{proof}
\end{lem}










\begin{lem}[Nonconformal, $G\neq1$]
Take a triple of spectral data $(P,b^1,b^2)$ associated with a nonconformal harmonic map, with a nonsingular spectral curve given by $η^2 = P$ of genus $g$. Suppose that, with reference to \eqref{eqn:common factors}, $G$ is a nonconstant real polynomial and $F=1$. If $G$ is linear, then for every real linear polynomial $\tilde{Q}$, or if $G$ is quadratic then for every pair of real numbers $(\tilde{Q},r)$ there is a unique tangent vector to the space of spectral data $(\dot P, \dot b^1, \dot b^2)$.

\begin{proof}
This case, and the several cases to follow are similar to the first proof. We proceed by first solving the reduced equations \eqref{eqn:EMPDi reduced} and \eqref{eqn:Q reduced} and using the solutions to those equations to establish solutions to \eqref{eqn:EMPDi} and \eqref{eqn:Q}. Regardless of the degree of $G$, which we shall denote $d_G$, we know that we must set $Q = G\tilde{Q}$. \eqref{eqn:Q reduced} reads
\[
\tilde{b}^1\tilde{c}^2 - \tilde{b}^2\tilde{c}^1 = \tilde{Q}\tilde{P}.
\]
There is a solution to this equation $({\bf \tilde{c}}^1,{\bf \tilde{c}}^2)$ of degree $g+2-d_1-d_G,g+2-d_2-d_G$. If we multiply these by $F^1$ and $F^2$ respectively, we have solutions to \eqref{eqn:Q} of degree $g+2-d_G$ each. If $G$ is linear therefore, this is the unique solution of degree $g+1$, but if $G$ is quadratic the space of solutions to \eqref{eqn:Q} is
\[
\Set { (F^1{\bf \tilde{c}}^1 + rF^1\tilde{b}^1, F^2{\bf \tilde{c}}^1 + rF^2\tilde{b}^2 }
{ r \text{ a real scalar} }.
\]
Hence for every $r\in\R$ there is a solution $c^1, c^2$. In either case, it was not necessary to have a condition similar to $R(Q)=0$, but corresponding the choice of $\tilde{Q}$ was more restricted.

Next we must solve \eqref{eqn:EMPDi reduced}, but the proof above applies again essentially without modification.
\[
\dot{P} G \tilde{b}^i - 2 F^j \tilde{P} \dot{b}^i = 2 F^j \tilde{P} (\hat{c}^i - ζ\hat{c}^{i\prime}) + ζ(ζ^2-1)P' \tilde{c}^i,
\labelthis{eqn:EMPDi nc G}
\]
Similar to last time, it has a solution because $\gcd(F^j\tilde{P},G\tilde{b}^i) = 1$. Analysis at the roots of $F^1F^2\tilde{P}$ and a choice of scaling for $P$ forces a unique common solution $({\bf \dot{P}}, {\bf \dot{b}}^1, {\bf \dot{b}}^2)$. This solution also satisfies \eqref{eqn:residueTangent}.
Hence it is a tangent vector to $\mathcal{M}$ at $(b^1,b^2,P)$.
\end{proof}
\end{lem}





\begin{lem}[Conformal]
Take a triple of spectral data $(P,b^1,b^2)$ associated with a conformal harmonic map, with a nonsingular spectral curve given by $η^2 = P$ of genus $g$. Then for every pair of real numbers $(Q_1,r)$, there is a unique tangent vector to the space of spectral data $(\dot P, \dot b^1, \dot b^2)$.

\begin{proof}
Recall that the condition for a triple of spectral data to be associated to a conformal harmonic map is that the spectral curve is branched over $ζ=0,\infty$. Thus $P$ is degree $2g+1$ and $P(0)=0$. From \eqref{eqn:Residue}, or directly by consideration of the order of the poles, $b^i_0 = 0$ also. We may write therefore that $P= ζF^1F^2\tilde{P}$ and $b^i = ζF^i \tilde{b}^i$, where $\tilde{P}$ is a real polynomial of degree $2g - d_1 - d_2$ and both $b^i$ are real polynomials of degree $g+1-d_i$.

Factoring these common roots, \eqref{eqn:Q reduced} is simply
\[
\tilde{b}^1 \tilde{c}^2 - \tilde{b}^2 \tilde{c}^1 = ζQ_1\tilde{P}.
\labelthis{eqn:Q conformal}
\]
In the way that the degrees of $\tilde{b}^i$ are reduced, this is similar to the nonconformal case where $G$ was quadratic. The space of solutions is
\[
\Set { ({\bf \tilde{c}}^1 + r \tilde{b}^1, {\bf \tilde{c}}^2 + r \tilde{b}^1) }{ r\in \R },
\]
where $({\bf \tilde{c}}^1,{\bf \tilde{c}}^2)$ is a solution of degree $(g+1-d_1, g+1-d_2)$. For every such solution, let $c^i = F^i\tilde{c}^i$ and consider the corresponding \eqref{eqn:EMPDi reduced}, namely
\[
\dot{P} \tilde{b}^i - 2 F^j \tilde{P} \dot{b}^i = 2 F^j \tilde{P} (\hat{c}^i - ζ\hat{c}^{i\prime}) + (ζ^2-1)P' \tilde{c}^i,
\labelthis{eqn:EMPDi conformal}
\]
In what is by now a familiar story, for the dotted quatities there is a common solution $({\bf \dot{P}}, {\bf \dot{b}}^1, {\bf \dot{b}}^2)$. The space of solutions is however
\[
\Set{
({\bf \dot P} + 2s F^1F^2\tilde{P}, {\bf \dot b}^1 + s F^1\tilde{b}^1, {\bf \dot b}^2 + s F^2\tilde{b}^2)
}{ s \text { a real quadratic polynomial} }.
\]
Thus there appears not to be a unique tangent vector corresponding to each choice $(Q,r)$, but \eqref{eqn:residueTangent} is not automatically true as it was in the nonconformal case. Let $s = s_0 + s_1ζ + \bar{s}_0 ζ^2$ we see that the condition for $i=1$ implies that
\[
2 {\bf\dot{P}}_0 b^1_1 - P_1 {\bf \dot{b}}^1_0 + 3 s_0 P_1 b^1_1 = 0
\]
which fully determines $s_0$. Can we therefore simultaneously satisfy the condition for $i=2$? Note that \eqref{eqn:EMPDiConformal} in the lowest degree implies that
\[
\dot{P}_0 b^i_1 - 2P_1\dot{b}^i_0 = -3 P_1 c^i_0
\]
and \eqref{eqn:Q conformal} in the lowest degree yields
\begin{align*}
b^1_1 c^2_0 &= b^2_1 c^1_0 \\
b^1_1 \bra{{\bf\dot{P}_0} b^2_1 - 2P_1{\bf\dot{b}^2_0}} &= b^2_1 \bra{{\bf\dot{P}_0} b^1_1 - 2P_1{\bf\dot{b}^1_0}} \\
2b^1_1 {\bf\dot{b}^2_0} &= 2 b^2_1 {\bf\dot{b}^1_0} \\
~\\
b^1_1 \bra{2 {\bf\dot{P}_0} b^2_1 - P_1 {\bf \dot{b}^2_0} + 3 s_0 P_1 b^2_1}
&= 2 {\bf\dot{P}_0} b^1_1b^2_1 - P_1 b^1_1{\bf \dot{b}^2_0} + 3 s_0 P_1 b^1_1b^2_1 \\
&= 2 {\bf\dot{P}_0} b^1_1b^2_1 + P_1 b^2_1 {\bf\dot{b}^1_0} + 3 s_0 P_1 b^1_1b^2_1 \\
&= b^2_1\bra{2 {\bf\dot{P}_0} b^1_1 - P_1 {\bf\dot{b}^1_0} + 3 s_0 P_1 b^1_1 } \\
&= 0
\end{align*}
So the condition holds for $i=2$ also. Having passed this check, there is still one free parameter, namely for any $Q_1$ and $r$, the corresponding tangent vectors are
\[
\Set {
({\bf \dot P} + 2(s_0+\bar{s}_0ζ^2) F^1F^2\tilde{P} + 2s_1ζ F^1F^2\tilde{P}, {\bf \dot b^1} + (s_0+\bar{s}_0ζ^2) F^1\tilde{b}^1 + s_1ζ F^1\tilde{b}^1, {\bf \dot b^2} + (s_0+\bar{s}_0ζ^2) F^2\tilde{b}^2 + s_1ζ F^2\tilde{b}^2)
}{ s_1 \in \R }.
\]
But our free choice of $s_1\in\R$ is only adding multiples of $(2P,b^1,b^2)$, which as in the nonconformal case is a rescaling of the spectral curve, and so also determined.
\end{proof}
\end{lem}




\begin{thm}
The open subset of $\mathcal{M}$ where $\gcd(P,b^1,b^2) = 1$ is a two dimensional manifold.

\begin{proof}
Consider the map defined on a small simply connected subset of SPACE
\begin{align*}
F(P,b^2,b^2)
= \left( \int_{A_1} Θ^1, \dots, \int_{A_g} Θ^1, \right.& \int_{B_1} Θ^1, \dots, \int_{B_g} Θ^1, \int_{A_1} Θ^2, \dots, \int_{A_g} Θ^2, \int_{B_1} Θ^2, \dots, \int_{B_g} Θ^2, \\
&\left. \int_{γ_+} Θ^1, \int_{γ_-} Θ^1, \int_{γ_+} Θ^2, \int_{γ_-} Θ^2,\, P_1b^1_0 - 2P_0b^1_1,\, P_1b^2_0 - 2P_0b^2_1,\, P_0 - \prod_{i}(-α_i) \right)
\end{align*}
where $A_i, B_i$ are the real and imaginary periods of the curve $η^2 = P(ζ)$, $γ_+,γ_-$ are the paths in the curve between the points of $ζ=1$ and $ζ=-1$, and $α_i$ are the roots of $P$ inside the unit circle. The choice of a small subset $U$ ensures that the choice of paths may be made smoothly. The differential of $F$ is a map from $\R^{4g+11}$ to $\R^{4g+9} = \R^{4g+4+2+2+1}$. A point of $\mathcal{M}$ where $\gcd(P,b^1,b^2) = 1$ falls into one of the above three cases. In each case we computed that the kernel of $dF$ is two dimensional. Therefore at every such point, by the implict function theorem, $\mathcal{M}$ is a manifold.
\end{proof}
\end{thm}

Let us give a geometrical interpretation to the polynomial $Q$. Recall that the conformal type of the domain of a harmonic map is given by the ratios of the principal parts of the differentials. Let the conformal type be denoted $τ$. For a nonconformal harmonic map we have that $b^2_0 = τ b^1_0$. Considering the constant terms of \eqref{eqn:EMPDi} we have that
\[
\dot{P}_0b^i_0 -2P_0 \dot{b}^i = 2P_0\hat{c}^i_0.
\]
Noting that $c^i_0 = - \hat{c}^i_0$, we substitute this into \eqref{eqn:Q} to arrive at
\[
Q_0 P_0 = b^1_0 c^2_0 - b^2_0 c^1_0 = b^1_0 \dot{b}^2_0 - \dot{b}^1_0 b^2_0.
\]
Differentiating the relationship $b^2_0 = τ b^1_0$ gives $\dot{b}^2_0 = \dot{τ} b^1_0 + τ \dot{b}^1_0$. Using this yields that
\[
Q_0 = \frac{\dot{τ}}{τ} \frac{b^1_0 b^2_0}{P_0}.
\]
We see therefore that $Q_0$ controls the change in the conformal type.
