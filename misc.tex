\chapter{Misc Stuff}

COMPUTATION OF T AT U OR V INFINITY


Returning to the specific form of the equation, one could draw a distinction between $u$ as a parameter, simply labeling a point, and $u$ as a coordinate, an assignment of a value to a point. $u$ is a parameter when it is $\infty$ but not a coordinate. A coordinate for this point would be $u' := 1/u$. To give a complete definition of $T$, it is necessary to give a formula in neighbourhoods of both $u=\infty$ and $v=\infty$. But because of the periods, it is impossible to give one that agrees with the formula above everywhere they are both defined. Instead, we shall define functions $T'$ and $T''$ that differ locally from $T$ by a multiple of the periods

To see explicitly how to do make this definition, let us consider the limit of $T$ as $v$ goes to infinity.
\begin{align}
\lim_{v\to+\infty} 2π\iu T
&= 4pE \lim_{v\to\infty} \tilde{F}(\iu v) - 4pK \lim_{v\to\infty} \bra{ \tilde{E}(\iu v) + \iu \frac{w(\iu v)}{u-v} }
- 4 \left[ E \tilde{F}(\iu u) - K \tilde{E}(\iu u) \right] - \lim_{v\to\infty} 4\iu K \frac{ w(\iu u)}{u-v} \\
&= 4\iu pEK' - 4pK \lim_{v\to\infty} \bra{ \tilde{E}(\iu v) - \iu k v + \iu k v + \iu \frac{w(\iu v)}{u-v} }
- 4\left[ E \tilde{F}(\iu u) - K \tilde{E}(\iu u) \right]\\
&= 4\iu pEK' -4\iu pK(K' - E') - 4pK \lim_{v\to\infty} \bra{\iu k v + \iu \frac{w(\iu v)}{u-v} }
- 4\left[ E \tilde{F}(\iu u) - K \tilde{E}(\iu u) \right]\\
&= 2π\iu p - 4\iu pK \lim_{v\to\infty} \frac{kuv - kv^2 + w(\iu v)}{u-v}
- 4\left[ E \tilde{F}(\iu u) - K \tilde{E}(\iu u) \right]\\
&= 2π\iu p + 4\iu pK ku - 4\left[ E \tilde{F}(\iu u) - K \tilde{E}(\iu u) \right]
\end{align}
And from the other side of $v=\infty$,
\begin{align}
\lim_{v\to-\infty} 2π\iu T
&= -2π\iu p + 4\iu pK ku - 4 \left[ E \tilde{F}(\iu u) - K \tilde{E}(\iu u) \right]
\end{align}
We see that the two limits of $T$ differ by a value of $2 p$, twice the period $p$. On this neighbourhood of $v=\infty$, let $v' = 1/v$ be a coordinate on $\RInf\setminus \{0\}$. Therefore define
\[
T'(k,u,v') =
\begin{cases}
T(k,u, v) & \text{ for } v' > 0 \\
\lim_{v\to +\infty} T(k,u,v) & \text{ for } v' = 0 \\
2p + T(k,u, v) & \text{ for } v' < 0
\end{cases}.
\]
Looking specifically at this last formula, where $v' < 0$, we can rearrange it in the following way.
\begin{align}
2π\iu& (2p + T)\\
&= 4p \left[ E \tilde{F}(\iu v) - K \tilde{E}(\iu v) \right] - 4 \left[ E \tilde{F}(\iu u) - K \tilde{E}(\iu u) \right] - 4\iu K \frac{p w(\iu v) + w(\iu u)}{u-v} + 8\iu p \bra{EK' + K(E' - K')} \\
&= 4p \left[ E (2\iu K' + \tilde{F}(\iu v)) - K (2\iu (K'-E') + \tilde{E}(\iu v)) \right] - 4 \left[ E \tilde{F}(\iu u) - K \tilde{E}(\iu u) \right] - 4\iu K \frac{p w(\iu v) + w(\iu u)}{u-v} \\
&= 4p \left[ E \left\{2\iu K' + \tilde{F}(\iu v)\right\} - K \left\{ 2\iu (K'-E') + \tilde{E}(\iu v)) - ikv\right\} \right] - 4 \left[ E \tilde{F}(\iu u) - K \tilde{E}(\iu u) \right] - 4\iu K \frac{p w(\iu v) + w(\iu u)}{u-v} - 4p\iu k K v
\end{align}
Compare this to $T'$ for $v' = 0$
\begin{align}
2π\iu T'
&= 2π\iu p + 4\iu pK ku - 4\left[ E \tilde{F}(\iu u) - K \tilde{E}(\iu u) \right] \\
&= 4p \left[ E \left\{\iu K'\right\} - K \left\{ \iu (K'-E')\right\} \right] - 4 \left[ E \tilde{F}(\iu u) - K \tilde{E}(\iu u) \right] + 4\iu pK ku
\end{align}
and for $v' > 0$
\[
2π\iu T' = 4p \left[ E \left\{\tilde{F}(\iu v)\right\} - K \left\{ \tilde{E}(\iu v)) - ikv\right\} \right] - 4 \left[ E \tilde{F}(\iu u) - K \tilde{E}(\iu u) \right] - 4\iu K \frac{p w(\iu v) + w(\iu u)}{u-v} - 4p\iu k K v
\]
In APPENDIX \todo{add ref} it was shown that the functions appearing in the braces of these equations defined above piecewise on each of $v' >0$, $v' = 0$ and $v' < 0$ were in fact  analytic in $v'$ when taken together. The second term is independent of $v'$ and so passes muster as well. It therefore remains to show that the leftovers form an analytic function. More precisely, that the following is an analytic function of $v'$.
\[
g :=
\begin{cases}
\frac{p w(\iu v) + w(\iu u)}{u-v} + p k v & \text{ for } v' \neq 0 \\
pku & \text{ for } v' = 0 \\
\end{cases}.
\]

\begin{align}
\frac{p w(\iu v) + w(\iu u)}{u-v} + p k v
&= \frac{pw(\iu v) + w(\iu u) + pkuv - pkv^2}{u-v} \\
&= p \frac{w(\iu v) - kv^2}{u-v} + p \frac{kuv}{u-v} + \frac{w(\iu u)}{u-v} \\
&= p \frac{1}{v'}\frac{\sqrt{(1+(v')^2)(k^2 + (v')^2)} - k }{uv'-1} + p \frac{ku}{uv'-1} + \frac{v' w(\iu u)}{uv'-1},
\end{align}
which is analytic as the difference $\sqrt{(1+(v')^2)(k^2 + (v')^2)} - k$ vanishes quadratically in $v'$ near $v'=0$, and so too does the third term. In summary we have shown that $T'$ is an analytic function on the part of the parameter space with $v=\infty$ (in particular a neighbourhood of this point), and by its definition it differs from $T$ locally up to a constant where they are both defined ($v' \neq 0$). We may also define $T''$ on a neighbourhood of $u = \infty$ with local coordinate $u' = 1/u$ in an alike manner.
\[
T''(k,u',v) =
\begin{cases}
T(k,u, v) & \text{ for } u' > 0 \\
\lim_{u\to +\infty} T(k,u,v) & \text{ for } u' = 0 \\
-2 + T(k,u, v) & \text{ for } v' < 0
\end{cases}.
\]
Due to the symmetry between $u$ and $v$ inherent in the definition of $T$, the the same argument applies to this case too. There is no need to consider the possibility of both $u=v=\infty$ because on the parameter space $u\neq v$. In practical terms, it does not really matter which version of the function we use, as for our purposes they will be mostly equivalent. For example, their derivatives are identical. One needs only to be careful when talking about a particular level set that one adjusts for the different values.















\section{The Cut space}
THIS PART COMPUTES T AS k TO 0 AND 1

Fix a value of $p \leq 1$.

Consider the parameter space $P$ for this fixed value of $p$. Topologically it is a fattened annulus. Suppose we make a cut along the plane $u=\infty$. Now we have a box $B = \{(k,u,v)\mid 0 < k <1, u \in \R , v \in\RInf, u \neq v\}$. Our aim will be to examine the moduli space within this box, given as the level surfaces $T=q$ for rational $q$, by sweeping through it with a height function $u$. To apply Morse theory, we require the height function to be proper map and so we will need to compactify this space. The preimage of a compact range $V$ of $u$ is just $(0,1)\times \R\cup\{\infty\}\times V \setminus \{u=v\} $, so we need to close in the $k$ and $v$ directions but not as $u\to\pm\infty$. Define $\overline{B} = \bigcup_{u\in\R} \{(k,u,v)\mid 0 \leq k \leq 1, v \in [u^+, \infty] \cup [\infty, u^-] \}$ where the range of $v$ is constructed by the following process: take $\R\cup\{\infty\}$, cut out $u$ to make an open interval, and then take the two point compactification where we add $u^+$ and $u^-$ on the ends. Let us now examine the behaviour of the function as we approach the boundary of the box.
\todo{diagram this weird ass thing}

\todo{talk about how we can do level sets of a function only locally defined. talk about how because the box is simply connected, we can specify a value at a point and its level set is a well defined object even if we have to switch formulas for $T$ and `change' the value as we move around}

For $k\in(0,1)$ as $v$ approaches $u$ from either above or below, the $1/(u-v)$ factor causes $T \to \pm\infty$. If $k \to 0$ then the period terms disappear and $T$ limits to an elementary function, namely
\[
T(0,u,v) = - 2π\iu \frac{p \sqrt{1+v^2} + \sqrt{1+u^2}}{u-v}.
\]
That leaves only the limit $k \to 1$. This one is more interesting. We arrange the terms into the $p$ and non-$p$ parts.
\[
2π\iu T
= p \left\{ 4E \tilde{F}(\iu v) - 4 K \left[ \tilde{E}(\iu v) + \iu\frac{ w(\iu v)}{u-v} \right] \right\}  - \left\{ 4 E \tilde{F}(\iu u) - 4 K \left[ \tilde{E}(\iu u) + \iu\frac{ w(\iu u)}{v-u} \right] \right\}.
\]
The braced expressions only differ by interchanging $u$ and $v$, so we may treat just one. The limit of $4E \tilde{F}(\iu v)$ is $4 \times 1 \times i\atan v$, but the presence of the $K$ makes the other part difficult, as $K$ tends to $\infty$ as $k\to 1$. We divide it in the following way to compute the limit.
\begin{align}
4 K \left[ \tilde{E}(\iu v) + \iu\frac{ w(\iu v)}{u-v} \right]
&= 4 K \left[ \tilde{E}(\iu v) - \iu k v + \iu k v - \iu v + \iu v +\iu \frac{1+v^2}{u-v} - \iu \frac{1+v^2}{u-v} + \iu\frac{ w(\iu v)}{u-v} \right] \\
&= 4 K \left[ \tilde{E}(\iu v) - \iu k v \right] + 4 \iu v K (k - 1) + 4\iu K \left[\frac{uv + 1}{u-v} \right] - 4\iu K \left[ \frac{1+v^2 - w(\iu v)}{u-v} \right]
\end{align}
We shall treat each term in turn. $K (-\iu \tilde{E}(\iu v) - kv)$ tends to $0$, independent of $v$. This comes from the inequality \eqref{eqn:tildeEatInf} derived in the first appendix:
\[
-(K'-E') \leq -\iu \tilde{E}(\iu u) - ku \leq K' - E'.
\]
If we multiply this by $K$, then the upper bound is $K(K' - E')$ and using Legendre's \todo{name?} relation
\[
\lim_{k\to 1} K (K'-E')
= \lim_{k\to 1} KK'- KE' -K'E + K'E
= \lim_{k\to 1} -\frac{π}{2} + K'E
= -\frac{π}{2} + K(0)E(1) = 0.
\]
The same calculation shows the lower bound tends to zero as well. The next term, $K (k - 1)$, is also zero by Lemma \ref{lem:lim(1-k)K}. We shall leave the third term momentarily, and proceed to the forth.
\begin{align*}
K \left[\frac{1+v^2 - w(\iu v)}{u-v} \right]
&= K \frac{1+v^2 - \sqrt{1 + (1+k^2)v^2 + k^2v^4}}{u-v} \\
&= K \frac{(1+v^2)^2 - (1 + (1+k^2)v^2 + k^2v^4)}{(u-v)\bra{1+v^2 + \sqrt{1 + (1+k^2)v^2 + k^2v^4}}} \\
&= K \frac{(1-k^2) v^2 (1+v^2)}{(u-v)\bra{1+v^2 + \sqrt{1 + (1+k^2)v^2 + k^2v^4}}} \\
&= (1-k)K \frac{(1+k) v^2 (1+v^2)}{(u-v)\bra{1+v^2 + \sqrt{1 + (1+k^2)v^2 + k^2v^4}}},
\end{align*}
which also goes to zero by Lemma \ref{lem:lim(1-k)K}. Hence the limit of $T$ is
\begin{align*}
\lim_{k\to 1} 2π\iu T
&= p \left\{ 4\iu \atan v - \lim_{k\to 1} 4\iu K \frac{uv+1}{u-v} \right\}
- \left\{ 4 \iu \atan u - \lim_{k\to 1} 4\iu K \frac{uv+1}{v-u} \right\} \\
&= 4\iu p \atan v - 4 \iu \atan u - 4\iu(1+p) \lim_{k\to 1} K \frac{uv+1}{u-v}.
\end{align*}


In the limit, $K$ has a logarithmic singularity. This could be canceled out if simultaneously $uv \to -1$, but otherwise it causes $T$ to tend to either positive or negative infinity. This line $\{k=1, v = -1/u\}$ where the limit is not well defined is therefore of particular interest. We shall change coordinates to examine this limit closely. Note at $u=0$ that $v$ is not a coordinate on a neighbourhood of this line. Let us change to $v'$ then and as above consider $T'$ which analytic in this coordinate.
\begin{align}
\lim_{k\to1} T' =
2π\iu p - 4\iu p \atan v' - 4 \iu \atan u - \lim_{k\to 1} 4\iu K (1+p)\frac{u+v'}{uv'-1}.
\end{align}
Consider for small positive $ε$ the family of surfaces indexed by a real number $λ$ defined parametrically in $(ε,ψ)$ by
\[
k = 1 - \exp \bra{-1/ε},\; u = ψ,\; v' = \frac{ψ+ελ}{εψλ - 1}.
\]
Along such a surface, $ε\to 0$ as $k\to 1$ and the limiting value of $T'$ is
\[
T' \to
2π\iu p + 4\iu (p-1) \atan ψ - 2\iu (1+p)λ.
\]
So by adjusting the direction of approach by varying $λ$, $T'$ in a neighbourhood of this line attains every value. It is clear from the above that $B$ is not suitable on its own to be the ambient space when doing Morse theory because $T$ is badly behaved on the boundary. To properly understand the behaviour of level sets of $T$, we should `blowup' the line $\{k=1, v' = -u\}$ so these surfaces do not have a common intersection and so that $T$ has a well defined limit on this boundary.  Instead, we will think of $B$ as one coordinate patch of a larger manifold $M$ on which $T$ is well defined. A second coordinate patch shall be given in terms of $(ε,ψ,λ)$ for $ε\in (-1,1)$, $ψ \in \R$ and $λ\in \R$. For $ε>0$,
\begin{align}
k &= 1 - \exp\bra{-\frac{1}{ε}}
    & ε &= -1/ \ln (1-k) \\
u &= ψ
    & ψ &= u \\
v' &= \frac{ψ+ελ}{εψλ - 1}
    & λ &= -\ln (1-k) \frac{u+v'}{uv'-1}
\end{align}
describes the transition between the two patches. We will work towards giving $T'$ a smooth extension to the manifold $M$. The theorem we intend to use is due to Seeley \cite{Seeley1964}: a smooth function in a half space where it and all its derivatives have continuous limits to the boundary can be extended to the whole space. Results in Appendix \ref{app:Extension} demonstrated a large class of functions that are extendable, so we can now apply those to the real thing. The calculations of the derivatives of $T$ are relatively mechanical though fairly long. There is only one point of that calculation that is worth elaborating upon. In the formula for $T$ one can conceptualise it into three groupings. There are two groupings, dubbed the `period terms' that arise from the parts of the differentials that generate the periods, one group coming from the integration over $γ_+$ and the other from $γ_-$. Each group is a function $k$ and one of either $u$ or $v$. The third group is an elementary term, and is made of leftovers that give the differentials in the correct symmetries. The period terms have an interesting derivative with respect to $k$, because there is a lot of cancellation that completely remove incomplete integrals.

Let $f(k,z) = E(k)\tilde{F}(z; k) - K(k)\tilde{E}(z; k)$. Then
\begin{align*}
\Partial{f}{k}
&= E_k \tilde{F} - K_k \tilde{E} + E\tilde{F}_k - K\tilde{E}_k \\
&= \frac{1}{k}E \tilde{F} - \frac{1}{k}K \tilde{F} - \frac{1}{k(1-k^2)}E \tilde{E} + \frac{1}{k}K \tilde{E} + \frac{1}{k(1-k^2)}E \tilde{E} - \frac{1}{k}E \tilde{F} - \frac{kz}{1-k^2}\sqrt{\frac{1-z^2}{1-k^2z^2}}E - \frac{1}{k}K\tilde{E} + \frac{1}{k}K\tilde{F} \\
&= - \frac{kz}{1-k^2}\sqrt{\frac{1-z^2}{1-k^2z^2}}E
\end{align*}

In the interest of having manageable formulae, we recall the definition of $w(z)^2 = (1-z^2)(1-k^2 z^2)$. Using primes to indicate coordinates at infinity, ie $z' = 1/z$, not derivatives, we introduce $w'(z)^2 = (1- z^2)(k^2 - z^2)$. As $\tilde F(z;k)$ and $\tilde E(z;k)$ are parameter integrals in $z$, we have that
\[
\Partial{}{u}\tilde F(\iu u; k) = \frac{\iu}{w(\iu u)},\;\;\;
\Partial{}{u}\tilde E(\iu u; k) = \iu\frac{1+k^2 u^2}{w(\iu u)},
\]
and
\[
\Partial{}{u} w(\iu u)
= \Partial{}{u} \sqrt{1+u^2}\sqrt{1+k^2 u^2}
= \frac{(1+k^2)u + 2k^2 u^3}{w(\iu u)}.
\]
The other derivatives of elliptic integrals are calculated in appendix \ref{chp:Elliptic Integrals}.
\begin{align*}\label{dTdk}
\frac{π}{2}\Partial{T}{k}
&= \frac{1}{k(1-k^2)}\frac{1}{u-v} \left[ p \sqrt { \frac{1+v^2}{1+k^2v^2}} + \sqrt { \frac{1+u^2}{1+k^2u^2} } \right] \left[ -(1+k^2uv) E + (1-k^2)K \right]
\end{align*}
\begin{equation}\label{dTds}
\frac{π}{2}\Partial{T}{u}
= -\frac{E}{w(\iu u)} + \frac{pK w(\iu v)}{(u-v)^2} + \frac{K}{w(\iu u)(u-v)^2}\left[1 + u^2 - uv + v^2 + k^2 uv + k^2 u^2v^2 \right]
\end{equation}
\begin{equation}\label{dTdt}
\frac{π}{2}\Partial{T}{v}
= \frac{pE}{w(\iu v)} - \frac{K w(\iu u)}{(u-v)^2} - \frac{pK}{w(\iu v)(u-v)^2}\left[1 + u^2 - uv + v^2 + k^2 uv + k^2 u^2v^2 \right]
\end{equation}
\begin{equation}\label{dTdt'}
\frac{π}{2}\Partial{T}{v'}
= -\frac{pE}{w'(\iu v')} + \frac{K w(\iu u)}{(uv'-1)^2} + \frac{pK}{w'(\iu v')(uv'-1)^2}\left[1 + k^2u^2 - uv' + k^2uv' + (v')^2 + u^2(v')^2 \right]
\end{equation}

Let us focus on the polynomials in the square brackets. We reuse the idea of a dominant and remainder for $k^2$. Define $\rem(k^2) = k^2 - 1$. Unlike for $\rem(K)$, we can be very explicit about $\rem(k^2)$, namely it is $(2-\exp(-1/ε))\exp(-1/ε)$, and so because of the $\exp(-1/ε)$ factor both $ε^{-n}\rem(k^2)$ and $ε^{-n}K\rem(k^2)$ are in $\mathcal{F}$ (a set of extendable functions, see Appendix \ref{chp:Extension Calculations}). \todo{FIX}

Plugging in the $(k,u,v')$-derivatives to the chain rule and ripping apart and rearranging pieces as necessary to get into a well behaved terms, we arrive at
\begin{align*}
\Partial{T'}{ε}
&=
C \bra{p\sqrt{\frac{1+(v')^2}{k^2 + (v')^2}} + \sqrt{\frac{1+s^2}{1 + k^2s^2}}}\bra{\frac{1}{ε}\frac{E}{k(1+k)} - K}
+ Cp\frac{s^2+1}{(Cεs-1)^2}\frac{1}{w'(\iu v')}E
\\&
- \frac{1}{k}\frac{v'}{uv'-1}\bra{p\sqrt{\frac{1+(v')^2}{k^2 + (v')^2}} + \sqrt{\frac{1+s^2}{1 + k^2s^2}}}\frac{1}{ε^2}K\exp(-1/ε)
\\&
+ \frac{1}{k(1+k)}\frac{1}{uv'-1}\bra{p\sqrt{\frac{1+(v')^2}{k^2 + (v')^2}} + \sqrt{\frac{1+s^2}{1 + k^2s^2}}}uEε^{-2}\rem(k^2)
\\&
- Cp \frac{s^2+1}{(Cεs-1)^2}\frac{1}{w'(\iu v')}\frac{uv'+u^2}{uv'-1}K\rem(k^2)
- C \frac{s^2}{w(\iu u)}K\rem(k^2)
\end{align*}

\begin{align*}
\Partial{T'}{s}
&=
-\frac{1}{w(\iu u)} E + p \frac{1+C^2ε^2}{(Cεs-1)^2}\frac{1}{w'(\iu v')}E
\\&
- p\sqrt{\frac{1+(v')^2}{k^2 + (v')^2}} \frac{Cεs-1}{(1+s^2)^2}Cεs K\rem(k^2)
- p\sqrt{\frac{1+(v')^2}{k^2 + (v')^2}} \frac{(Cεs-1)^2}{(1+s^2)^2}(1+k) K \exp(-1/ε)
\\&
+ \sqrt{\frac{1+s^2}{1 + k^2s^2}} \frac{Cεs-1}{(1+s^2)^2}Cεs K\rem(k^2)
+ \sqrt{\frac{1+s^2}{1 + k^2s^2}} \frac{1+C^2ε^2}{(1+s^2)^2}s^2(1+k) K \exp(-1/ε)
\end{align*}

\begin{align*}
\Partial{T'}{C}
&=
\frac{s^2+1}{(Cεs-1)^2}\left[ \frac{p}{w'(\iu v')}εE - \frac{w(\iu u)}{(uv'-1)^2}εK - p\frac{1+k^2u^2 -uv' + k^2uv' + (v')^2 + u^2(v')^2}{w'(\iu v')(uv'-1)^2}εK \right]
\end{align*}

We can see that each of these derivatives is a combination of extendable functions. Thus $T'$ meets the conditions and so has a smooth extension to $ε<0$. Having extended past this plane of $ε=0$, we now wish to chop it off. We check that level sets of $T'$ are transverse to the plane. Once we have established this, then we can apply the sequence of lemmas culminating in \ref{lem:stratified_level_set} to conclude that the level sets are a Whitney stratified space. We can determine whether or not the level surfaces are transverse by checking that the $v'$ derivative of $T'$ is non-zero. And what have we just done if not calculated the limits of the derivatives of $T'$?

In particular, look at the limit of the $C$ derivative as $ε\to 0^+$.
\[
\lim_{ε\to 0^+}\Partial{T'}{C}
=
\frac{s^2+1}{1}\left[ \frac{p}{1+s^2}\times 0 - \frac{1+s^2}{(1+s^2)^2}\times \frac{1}{2} - p\frac{(1+s^2)^2}{(1+s^2)^3}\times\frac{1}{2} \right] = -\frac{1}{2}(p+1)
\]
which is never zero and so the level sets are always transverse to the plane $ε=0$. This derivative also matches up with the $C$ derivative of the limit of $T'$, as we would expect, providing a check of our calculations. \todo{tell Emma and John that I found a mistake via this check!} By continuity there is a neighbourhood $U$ of $ε=0$ such that the derivative is non-zero. And so this also provides that every level set of $T'$ in this neighbourhood is a manifold by the implicit function theorem. The space $ε\geq 0$ is a closed half space of $U$. We are now in the situation of Lemma \ref{lem:stratified_level_set} and so we can conclude that for any value of $q$, $T^{-1}(q) \cap \{ε \geq 0\}$ is a Whitney stratified space.

Having dealt with the poorly behaved $k=1$, we now turn to the opposite side of the box $B$. Looking at the formulas of the derivatives of $T$, only the $k$ is not clearly defined at $k=0$. This time we have the advantage that both $K$ and $E$ are analytic functions of $k$ near 0, so we may simply look at the relevant factors and do a series expansion.
\begin{align*}
\frac{1}{k}\bra{ - (1+k^2 uv) E + (1-k^2)K }
&= \frac{1}{k}\frac{π}{2}\bra{ - (1+k^2 uv)\bra{ 1 - \frac{1}{4}k^2 + O(k^4)}  + (1-k^2)\bra{ 1+\frac{1}{4}k^2 + O(k^4) } } \\
&= \frac{1}{k}\frac{π}{2}\bra{ - k^2(1+uv) + O(k^4) }
\end{align*}
All the formulas are well defined for $k<0$, so extension beyond $B$ is easy in this direction. We need to check that the level sets are transverse to this face also. Looking at the $v$ derivative
\begin{align*}
\left. \frac{1}{4\iu}\Partial{T}{v} \right|_{k=0}
&= \frac{pπ/2}{\sqrt{1+v^2}} - \frac{π/2 \sqrt{1+u^2}}{(u-v)^2} - \frac{pπ/2}{\sqrt{1+v^2}(u-v)^2}\left[1 + u^2 - uv + v^2\right] \\
&= - \frac{π}{2}\frac{1}{\sqrt{1+v^2}(u-v)^2}\bra{\sqrt{1+u^2}\sqrt{1+v^2} + p\left[1 + uv \right] } \\
\end{align*}
If $1+uv$ is positive, then this is negative. It could only be zero if $1+uv$ were negative and cancelled the square root terms. However, recalling that $p \leq 1$, we have the following estimates
\begin{align*}
\left. \frac{1}{4\iu}\Partial{T}{v} \right|_{k=0}
&\leq - \frac{π}{2}\frac{1}{\sqrt{1+v^2}(u-v)^2}\bra{\sqrt{1+u^2}\sqrt{1+v^2} + \left[1 + uv \right] } \\
&\leq - \frac{π}{2}\frac{1}{\sqrt{1+v^2}(u-v)^2}\bra{\abs{u}\abs{v} + 1 + uv } \\
&\leq - \frac{π}{2}\frac{1}{\sqrt{1+v^2}(u-v)^2}
\end{align*}
So this derivative is always negative. Similarly, checking at $v=\infty$
\begin{align}
\left. \frac{1}{4\iu}\Partial{T}{v'} \right|_{k=0,v'=0}
= \frac{π}{2} \sqrt{1+u^2}
\end{align}
So the level surfaces are transverse here as well. Hence we have checked the entire $k=0$ plane, and so entirely analogously to the $ε=0$ end, we can say in some neighbourhood of this plane that the level surfaces are genuine submanifolds by the implicit function theorem and we can apply Lemma \ref{lem:stratified_level_set} to say the truncated level surface for $k \geq 0$ is a Whitney stratified space.












BOUNDS ON T NEAR BOUNDARY
Next we shall show that no where else does a level set approach the boundary of $B$. Let us consider the boundary where $0 < u - v < δ$. On this boundary, the dominant term is $1/(u-v)$ so we expect $T$ to be large and negative and that is exactly what we will show. In particular, let us consider a small neighbourhood of a strip $u_0 < u < S_1$, $k\in (0,1)$, $v=u^-$.
\begin{align*}
T
& \leq 2πp v - 4pK\asinh v - 4 \atan u + 4K u - 4K \frac{1+p}{δ} \\
& \leq 2πp v + 4K u - 4K \frac{1+p}{δ} \\
& \leq 4K \bra{ p u_1 + u_1 - \frac{1+p}{δ}}.
\end{align*}
So if $δ$ is chosen small enough, then the bracketed term is negative. One could choose it so as to make the bracket more negative than $- \abs{q} / 2π$. Thus no part of the level set $T^{-1}(q)$ lies in the volume $\Set{ (k,u,v) }{ k \in (0,1), u \in (u_0,u_1), u-δ < v < u }$. As $u_0$ and $u_1$ were arbitrary, for any given level set there is a neighbourhood of the face $v = u^-$ of $\overline{B}$ that excludes that level set, namely this neighbourhood is the union of this process over different ranges of $u$. An identical argument for $v=u^+$ show that in a neighbourhood of that face $T$ is arbitrarily large and any level set is likewise excluded on that side. This leaves only the face $k=1$ away from the exceptional line $v = -1/u$. And we have already computed the limit on this face. At every point on the face away from this line $T$ blowups. So every point has a neighborhood that on which $T$ takes value above (or below) any level. So these points too cannot be in the closure of any given level set. Taken together, these results shows what the closure of a level set is composed of. It is the level set in $B$ together with the intersection of the level set with the planes $ε=0$ and $k=0$. Moreover, subject to showing that the level set on the inside of $B$ is a submanifold, we have shown it to be a Whitney stratified space and a closed space in $\overline{B}$.




























LIMIT OF DIFFERENTIALS FOR BRANCH POINTS ON THE UNIT CIRCLE
\section{Boundary Limit}
Consider the behaviour of the differentials in the limit as $β$ tends to a point on the unit circle that is not $ζ=1,-1$. Such a point is on the boundary of the space of branch points $(α,β)$. Firstly note that
\begin{align}
A &\to (α-β)\abs{α-β} \\
B &\to -β(\bar{α}-\bar{β})\abs{α-β} \\
k &\to \frac{\abs{α-β} - \abs{α-β}}{\abs{α-β} + \abs{α-β}} = 0
\end{align}
so we expect the elliptic integrals present to reduce to circular integrals. To see this, note that for $β$ close to the unit circle $f(1)$ and $f(-1)$ are finite. Let us focus on the limit for the $ζ=1$ marked point, as the $ζ=-1$ point will be entirely similiar. Let $\lim f(1) = \iu σ$ and $t=\iu s$. Assume $σ\geq 0$, the case of $σ<0$ simply requires a sign change in the following argument. For $β$ sufficently close to the unit circle, $\abs{\lim f(1) - f(1)} < ε$ for any small $ε$ (in particular, we wish for $ε< σ$). Then
\begin{align}
\abs{\int_{\lim f(1)}^{f(1)} \tilde{ω} }
&= \abs{ \int_σ^{σ-ε} \frac{ds}{\sqrt{(1+s^2)(1+k^2s^2)}} } \\
&\leq \int_{σ-ε}^σ \frac{ds}{\sqrt{(1+(σ-ε)^2)(1+k^2(σ-ε)^2)}} \\
&= \frac{ε}{\sqrt{(1+(σ-ε)^2)(1+k^2(σ-ε)^2)}} \\
&\to 0
\end{align}
so this integral is neglible in the limit. Also consider
\begin{align}
\abs{ \int_0^{\lim f(1)} \tilde{ω}}
&= \int_0^σ \frac{ds}{\sqrt{(1+s^2)(1+k^2s^2)}} \\
&\leq \int_0^σ ds = σ
\end{align}
So this integral is dominated by $1$. Hence by the dominated convergence thereom
\[
\lim \int_0^{f(1)} \tilde{ω} = \lim \left[ \int_0^{\lim f(1)} + \int_{\lim f(1)}^{f(1)} \tilde{ω} \right] = \int_0^{\lim f(1)} \lim \tilde{ω}
\]
The analysis of the limit of $\int e$ is much the same. The dominating function this time is $\sqrt{1+k^2s^2} \leq \sqrt{1+s^2}$, as $k<1$. The upshot of this is that we can, for the purposes of the closing condition, interchange limits and integrals. We will use this later to consider the limit of genus one differentials as genus zero differentials. As $k\to 0$ both $K$ and $E$ tend to $π/4$. $M$ clearly trends to $-2\abs{α-β}$ and
\[
N \to -\left[ \abs{α}^2 - 2\bar{α}β + 1 \right]
\]
Hence
\begin{align}
Θ_L^2 &:= \lim Θ_1^2 \\
&= \frac{π}{2}\tilde{ω}-\frac{π}{2}\tilde{ω} - \frac{2}{M}\frac{π}{4}\left[ d\bra{\frac{η}{ζ}} - \frac{M}{\abs{α-β}}d\bra{\frac{η}{ζ-β}} \right] \\
&= \frac{1}{\abs{α-β}}\frac{π}{4}d\bra{\frac{η}{ζ}} - \frac{1}{\abs{α-β}}\frac{π}{2}d\bra{\frac{η}{ζ-β}}\\
&= \frac{1}{\abs{α-β}}\frac{π}{4} \frac{dζ}{ζ^2η}\left[ -αβ + \frac{1}{2}(2α + β(1+\abs{α}^2))ζ + (1+\abs{α}^2)ζ^2 + \frac{1}{2}(2\bar{α} + \bar{β}(1+\abs{α}^2))ζ^3 - \bar{α}\bar{β}ζ^4 \right] \\
&= \frac{1}{\abs{α-β}}\frac{π}{4}(ζ-β) \frac{dζ}{ζ^2η}\left[ α - \frac{1}{2}(1+\abs{α}^2)ζ + \frac{1}{2}\bar{β}(1+\abs{α}^2)ζ^2 - \bar{α}\bar{β}ζ^3 \right]
\end{align}
It is easier to see that
\[
Θ_L^1 := \iu\frac{dζ}{ζ^2η}(ζ-β)\left[ α - \frac{1}{2}(1+\abs{α}^2)ζ - \frac{1}{2}\bar{β}(1+\abs{α}^2)ζ^2 + \bar{α}\bar{β}ζ^3 \right]\\
\]
The two limit differentials $Θ_L^i$ both have zeroes at the double point $β$. The significance of this will shortly become apparent. Let $β=e^{iφ}$ and consider the normalisation map
\begin{align}
π : Σ_0(α) &\to Σ_1(α,β) \\
π : (ζ,η) &\to (ζ, s\iu e^{-\iu φ/2}(ζ-β)η)
\end{align}
where $s=\pm 1$ is a sign coming from the square root to be determined. This equation comes simply from the extra factor that the equation for $Σ_1(α,β)$ has compared to $Σ_0(β)$. Because the spectral curves have involutions, composition with the involution gives `another' normalisation map, which differs only in the sign of the $η$ fibre. But clearly we want to make the positive values of $η$ map to positive values. In particular, above $ζ=1$ if we consider the case that $β=i$, then we have
\begin{align}
η_0(1) &= \pm\abs{1-α} \\
η_1(1) &= \pm\abs{1-\iu}\abs{1-α} = \pm\sqrt{2}\abs{1-α} \\
s\iu e^{-\iu π/4}(1-\iu)η_0(1)
&= \pm s \iu \frac{1-\iu}{\sqrt{2}}(1-\iu)\abs{1-α} \\
&= \pm s \iu (-\iu\sqrt{2})\abs{1-α} \\
&= \pm s \sqrt{2}\abs{1-α}
\end{align}
From this we can see that we should chose $s=1$ to be the correct sign for the square root. The pull back of the tautilogical section is then by definition $π^* η = \iu e^{-\iu φ/2}(ζ-β)η$. And the limit differentials are
\begin{align}
\pi^* Θ_L^1 &= \frac{dζ}{ζ^2η} e^{\iu φ/2} \left[ α - \frac{1}{2}(1+\abs{α}^2)ζ - \frac{1}{2}\bar{β}(1+\abs{a}^2)ζ^2 + \bar{α}\bar{β}ζ^3 \right]\\
\pi^* Θ_L^2 &= \iu \frac{π}{4} \frac{1}{\abs{α-β}} \frac{dζ}{ζ^2η} e^{\iu φ/2} \left[ α - \frac{1}{2}(1+\abs{α}^2)ζ + \frac{1}{2}(1+\abs{α}^2)\bar{β}ζ^2 - \bar{α}\bar{β}ζ^3 \right]
\end{align}
These lie squarely in the plane of genus 0 differentials, as we would hope. We can now see the significance of the development of a zero at the double point. When we pull back regular differentials on a curve with a double point, the result is differentials with simple poles at the two preimages of the double point, with opposite residues at those points. But by having a zero at the double point, the pole is neutralised and we again have regular differentials.






















SHIT. AN ATTEMPT TO DO DEFORMATIONS DIRECTLY
\section{Deformations}
TO FIX \todo{this}

The full problem of deformations is hard, but consider the case of just $Θ^S$.
\begin{align*}
\int_{\gamma_1}aΘ^1 &= \left . a\iu\frac{η}{ζ}\right|_{(1, η(1)^-)}^{(1, η(1)^+)} = 2\iu a \abs{1-α}\abs{1-β} = 2π\iu Γ^S_+ \in 2π\iu\Z \\
a &= \frac{πΓ^S_+}{\abs{1-α}\abs{1-β}}
\end{align*}

Because it is exact, $q^1$ is well-defined and $\dot q^1$ is computable. Indeed
\[
\dot q^1 = \frac{d}{dt}\frac{a\iu}{ζ}\dot{η} = \frac{i}{ζη}\bra{\iu \dot{a}P +\frac{1}{2}\iu a\dot{P}}.
\]
A deformation preserves the half period condition precisely when $\dot q(1)=\dot q(-1) = 0$. The first is automatic by the choice of $a$. The second translates to
\begin{align*}
\dot{a} &= a\left[ \Re\bra{\frac{\dot{α}}{1-α}} + \Re\bra{\frac{\dot{β}}{1-β}} \right] \\
% \dot{P}(-1) &= 2P(-1) \left\[\Re\bra{\frac{\dot{α}}{1+α}} + \Re\bra{\frac{\dot{β}}{1+β}} \right\] \\
% 0 &= \dot{a}P(1) + \frac{1}{2}a\dot{P}(1) =0\\
% 0 &= 2aP(1) \left\[\Re\bra{\frac{\dot{α}}{1-α}} + \Re\bra{\frac{\dot{β}}{1-β}} \right\] \\
% 0 &= \dot{a}P(-1) + \frac{1}{2}a\dot{P}(-1)\\
\end{align*}
respectively. There are four degrees of freedom in choosing $\dot{α}  $ and $\dot{β}$, but here are two conditions. This gives the expected two degrees of freedom. Notice that if the deformation preserves the symmetric case, then we have $\dot{β} = - \dot{α}  $, as well as $β=- α  $, so the above two conditions collapse to the single condition
\[
0 = \Re \frac{2 α  \dot{α}  }{1- α  ^2}
\]
If we take the interpretation of quotienting out by the symmetry, so we have a rational curve over the coordinate $z= α  ^2$, then this condition becomes
\[
0 = \Re \frac{\dot z}{1-z}
\]
which can be interpretted geometrically for $z$ in the unit disc as saying that $z$ must move perpendicularly to $1-z$, ie along circles centered at 1. So it seems as if there is one direction of deformation that preserves the symmetry, and one direction that breaks it. One should note however that the above reasoning is entirely huersitic, as these a neccessary but not sufficent conditions on the deformations. It remains to be shown that there are no further restrictions arising.


















\section{Matching the limit differentials to the lattice}
\label{chp:Matching the limit differentials to the lattice}

INSERT JUSTIFICATION FOR INTERCHANGING LIMITS AND INTEGRALS AS JOHN POINTED OUT.\todo{this}

In general it is difficult to be explicit about the space $\mathcal{X}$, because the closing conditions are transcendental. However, over the boundary of $\mathcal{A}$ the conditions degenerate to simple trigonometry and can be solved. The first step is to treat each piece of $\mathcal{X}$ seperately, that is, fix four integers $m,n,\tilde m, \tilde n$. Then we can identify $\mathcal{X}(n,m,\tilde n,\tilde m)$ as lying within $\mathcal{A}$. With this view point adopted, the question of finding the moduli space is reduced to finding conditions on $(α,β)$ such that the closing conditions can be satisfied.


The half-periods of the limit can then be computed via the pull back to the normalisation, which as a genus zero curve the differentials are exact. If we denote the normalisation map as $π$ and the limits of the differentials as $Θ^1_L, Θ^2_L$ then
\begin{align*}
\int_{γ_+} Θ^1_L = \int_{π^{-1}(γ_+)} \pi^* Θ^1_L
&= 4\iu \abs{1-β} \sin \frac{φ}{2} \\
\int_{π^{-1}(γ_-)} \pi^* Θ^1_L &= -4\iu \abs{1+β} \cos \frac{φ}{2} \\
\int_{π^{-1}(γ_+)} \pi^* Θ^2_L &= 2π\iu \frac{\abs{1-β}}{\abs{α-β}} \cos \frac{φ}{2} \\
\int_{π^{-1}(γ_-)} \pi^* Θ^2_L &= 2π\iu \frac{\abs{1+β}}{\abs{α-β}} \sin \frac{φ}{2} \\
\end{align*}
where $α = \exp \iu φ$. \todo{double check signs here vis-à-vis correct square root sign choice in $η(\pm 1)$} From this we can computer the previous incomputable coefficents $a$ and $b$; they are simply ratios of the above. This gives us the half-periods in the better frame
\begin{align*}
2π\iu Γ^S_- := \int_{γ_-} Θ^S &= -2π\iu \frac{\abs{1+β}}{\abs{1-β}}\cot\frac{φ}{2} \\
2π\iu Γ^P_- := \int_{γ_-} Θ^P &= 2π\iu \frac{\abs{1+β}}{\abs{α-β}}\csc\frac{φ}{2}
\end{align*}
The condition $Γ_+, \tilde{Γ}_+$, the half-periods of $Θ, \tilde{Θ}$ respectively, is assured by this choice of basis, but we still require that $Γ_-, \tilde{Γ}_- \in\Z$. Indeed, based on our selection to focus on $\mathcal{X}(m,n,\tilde m, \tilde n)$ (a choice of two elements of the lattice $Λ$) we require that there be integers $k,\tilde k$ such that
\begin{align*}
k = Γ_- &= m Γ^S_- + n Γ^P_- \\
\tilde{k} = Γ_- &= \tilde{m} \tilde{Γ}^S_- + \tilde{n} \tilde{Γ}^P_-
\end{align*}
Elimination of the left hand sides of both equations leads to
\[
\frac{\abs{α-β}}{\abs{1-β}} = -\left( \frac{n\tilde{k} - \tilde{n}k}{m\tilde{k}-\tilde{m}k} \right) \sec\frac{φ}{2} =: q \sec\frac{φ}{2}
\]
whereas elimination of the rightmost term leads to
\[
\frac{\abs{1+β}}{\abs{1-β}} = \left( \frac{k\tilde{n} - \tilde{k}n}{\tilde{m}n - m \tilde{n}} \right) \tan\frac{φ}{2} =: p \tan\frac{φ}{2}
\]

For rational numbers $p,q$. If we have a solution for a particular $p$ and $q$, then as these two equations have three real degrees of freedom ($α\in\S^1$ and $β\in \C$), and the Jacobian of this as a map $F : (α,β) \mapsto (p,q)$ \todo{prove this} is generically maximal rank, we expect that there is a path of solutions. Also note the general fact that if $μ,\tilde{μ} \in \S^1$, $R\in\R$, and
\[
\frac{\abs{μ-β}}{\abs{\tilde{μ} - β}} = R,
\]
then also
\[
\frac{\abs{μ-\bar{β}^{-1}}}{\abs{\tilde{μ}-\bar{β}^{-1}}}
= \frac{\abs{μ}\abs{\bar{β}^{-1}}\abs{\bar{β} - μ^{-1}}}{\abs{\tilde{μ}}\abs{\bar{β}^{-1}}\abs{\bar{β} - \tilde{μ}^{-1}}} = \frac{\abs{μ-β}}{\abs{\tilde{μ} - β}} = R.
\]
The two equations above are both of this form, so if $(α,β)$ is a solution, so too is $(α,\bar{β}^{-1})$. This is expected, as there is nothing really distinguishing the root inside the unit circle from the one outside. Also note that the left hand sides are both positive numbers, so if $p>0$ then $ν$ is confined to the upper half circle, and if $p<0$ it is confined to the lower half.

The general solution to these equations is ugly and unilluminating. The projection of the solution space onto the $β$-plane is a real curve of degree 8 in the real and imaginary parts of $β$. There is however a nice solution to the question of when are both $ν$ and $β$ on the unit circle. Square both equations and make the t-substituion $t = \tan φ/2$. Also let $β= x + \iu y$. Then the equations are
\begin{align}
(x+1)^2 + y^2 &= p^2t^2 \left((x-1)^2 + y^2\right) \\
\left(x- \frac{1-t^2}{1+t^2} \right)^2 + \left(y - \frac{2t}{1+t^2}\right)^2 &= q^2(1+t^2) \left((x-1)^2 + y^2\right) \\
x^2 + y^2 &= 1.
\end{align}
Using the third equation to eliminate the $x^2 + y^2$ from both sides of the first equation, and rearranging to solve for $x$ gives
\[
x = \frac{p^2t^2 - 1}{p^2t^2 + 1}.
\]
Applying the same trick to make the second equation linear in $y$ gives
\[
y = \frac{2pt}{p^2t^2 + 1}.
\]
Finally then can we solve for $t$ in terms of $p$ and $q$ alone to give the neat formula
\[
t^2 = \operatorname{sgn} p \frac{1\pm q}{p \mp q}
\]
From the constraint that the left hand side is a nonnegative number, we know that $q$ must lie in the range $\abs{q}< \min\{\abs{p},1\}$.




OFF TOPIC: If we take the alternative definition of $\mathcal{A}$ that includes $\C^2$ outside $D\times D$ also, then there is a symmetry of this arc under inversion in the unit circle. That makes it a nice egg shape. Is there anything to be gained by doing this, for example can I show that it must lie within the sector formed of the origin, the two points corresponding to $β\in\S^1$ and the unit circle?

The two points are given by
\begin{align*}
t^2 &= (\sign p) \frac{1 \pm q}{p \mp q} \\
β &= \frac{p^2t^2 - 1}{p^2t^2 + 1} + \iu \frac{2pt}{p^2t^2+1}
\end{align*}
So you can see that there are two solutions (or a double solution) only when $\abs{q} < 1, \abs{p}$ \todo{What forbids only one solution?}. The condition that it be less than 1 forces the $k/\tilde{k}$ to lie in a certain range ....

INVESTIGATE how many $(k,\tilde{k})$ there are that satisfy the inequalities on $q,p$ for given $m,n,\tilde m, \tilde n$?

Since there is nothing particularly special about $α\to\S^1$ compared to $β\to\S^1$, and simliar analysis yields that the intersection of $\mathcal{X}(m,n,\tilde{m},\tilde{n})$ with $D\times\S^1$ is also a colection of arcs. These arcs join along the points where both $α,β$ lie in the unit circle to form closed cycles. UNPROVEN CLAIM \todo{verify this claim}. These are the boundary components of $\mathcal{X}(m,n,\tilde{m},\tilde{n})$.

BROAD STROKES PICTURE. There is some 2d surface in this 4-space $\mathcal{A}$. The exterior boundary is a three space topologically $S^3$. To see this, note that it is $S^1 \times D \cup D \times \S^1$. The intersection of the two pieces is $\S^1\times\S^1$, a torus. The two components of the union are the inside and outside of the torus. Putting it another way, in $\C^2$ the intersection is given by $\abs{α}=\abs{β}=1$, a scaled up version of the Clifford torus and you can project $\{(α,β) \mid α\in\S^1, β\in D\}$ onto the 3-sphere of radius $\sqrt{2}$ by rescaling by $\sqrt{2}/\sqrt{1+\abs{β}^2}$ and ditto for the other component.

The forbidden diagonal bit, ie where $α=β$ is a complex plane inside $\C^2$ that intersect this boundary in a real curve, the line on the torus where the toroidal angle is equal to the poloidal angle.

The moduli space $\mathcal{X}$ is also a surface, so we would expect it to intersect the boundary in arcs. Further, since we know that it has this symmetry that means we can consider its mirror on the of the boundary, we expect those arcs to be closed loops. The only way they could be otherwise (I think) is if the moduli space was tangent to the boundary somewhere.

Away from the boundary I don't know what the moduli space is doing. I don't know if separate boundary arc correspond to separate components of the moduli space, or if it's connected, or in between. I don't know if there are handles sitting inside. I don't know if several components come together on the forbidden diagonal.
