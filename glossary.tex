%!TEX root = thesis.tex
\section{Glossary}

\begin{description}[align=right]

\item[$f(ζ)$] The M\'obius transformation that takes one from the $ζ$ definition of an elliptic spectral curve to the Lengendre standard form.

\item[$P(ζ)$] The equation of the spectral curve is $η^2 = P(ζ)$. In standard form, where the roots are given by conjugate inverse pairs $α_j, \cji{α}_j$,
\[
    P(ζ) = \prod (ζ - α_j)(1 - \bar{α}_jζ).
\]

\item[$w(z)$] This is the equation of an elliptic curve in Legendre form. $w(z) = \sqrt{(1-z^2)(1-k^2z^2)}$.

\item[$w'(z')$] Variant of the equation of an elliptic curve in Legendre form, to deal with values at infinity. $w'(z') = \sqrt{(1-(z')^2)(k^2 - (z')^2)}$.

\item[$η$] The fibre coordinate of a bundle over $\CP^1$. The spectral curve is defined by $η^2 = P(ζ)$. With a subscript, it indicates the genus of the spectral curve.

\item[$η(ζ)^+$] The value of the fibre coordinate over $ζ$, taken as the positive square root of $P(ζ)$. For a point on the unit circle $ζ = e^{\iu θ}$ and a spectral curve in the standard form of genus $g$,
\[
    η(e^{\iu θ}) = e^{\iu θ \frac{g+1}{2}} \prod_j \abs{e^{\iu θ} - α_j}.
\]

\item[$ζ$] A coordinate on $\CP^1$, over which the spectral curve is defined.

\end{description}
