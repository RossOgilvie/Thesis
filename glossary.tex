%!TEX root = thesis.tex
\chapter{Glossary}
% Below is a list of notation, with a short description and a reference to their definition.
% They are arranged alphabetically, first Latin then Greek.

\begin{description}[align=right]

\item[$A$] The standard real period of a elliptic marked curve. See Figure~\ref{fig:standard periods}.

\item[$\mathcal{A}_g$] A parameter space that double covers the space $\mathcal{C}_g$ of marked curves of genus $g$. It is defined by~\eqref{eqn:def Ag} to be the space of ordered tuples of distinct branch points inside the unit disc.

\item[$b^i$] Every differential on a marked curve of genus $g$ that satisfies~\ref{P:poles}--\ref{P:imaginary periods} may be written in the form
\[
Θ = b(ζ) \frac{dζ}{ζ^2 η},
\]
for a polynomial $b(ζ) \in \mathcal{P}^{g+3}_\R$. The superscript refers to one of the pair of differentials in the spectral data $(Σ,Θ^1,Θ^2)$.

\item[$\dot{b}^i$] The derivative of $b^i$ with respect to $t$ along a path $\ell(t)$ in $\mathcal{M}_g$.

\item[$\tilde{b}^i$] A factor of $b^i$. $b^1 = F F^1 G \tilde{b}^1$.

\item[$B$] The standard imaginary period of a elliptic marked curve. See Figure~\ref{fig:standard periods}.

\item[$c^i$] A real polynomial of degree $g+1$ defined by factoring $\hat{c}^i$. $c^i = (ζ^2-1) \hat{c}^i$.

\item[$\hat{c}^i$] An infinitesimal deformation of differential $Θ^i = dq$ satisfying~\ref{P:poles}--\ref{P:periods} may be encoded as a meromorphic function with certain poles,
\[
\dot{q}^i = \frac{1}{ζη}\hat{c}^i(ζ),
\]
for a polynomial $\hat{c}^i(ζ) \in \mathcal{P}^{g+3}_\R$.

\item[$\mathcal{C}_g$] The space of marked curves of genus $g$.

\item[$\mathcal{\tilde{C}}_1$] The universal cover of $\mathcal{C}_1$.

\item[$\mathcal{D}$] The interior boundary of $\mathcal{A}_1$, defined to be $\{ (α,α) \in D \times D\}$.

\item[$e$] The Jacobi differential of the second kind.
\[
e = (1-k^2z^2) \frac{dz}{w}.
\]

\item[$E(k)$] The complete elliptic integral of the second kind. $E(k) = E(1;k)$.

\item[$E(z;k)$] The incomplete elliptic integral of the second kind. See Section~\ref{sec:Elliptic Defs}.
\[
E(z;k) = \int_0^z \sqrt{\frac{1-k^2 t^2}{1-t^2}} \;dt.
\]

\item[$E_0(x;k)$] A variant of $E(z;k)$ focused on imaginary arguments and which is well-behaved at infinity. $E_0 (x ; k) := \Imag E(\iu x; k) - kx$

\item[$f(ζ)$] In genus one, the Möbius transformation that takes the standard form of the spectral curve~\eqref{eqn:genus one curve} to the Legendre form. See~\eqref{eqn:f}.

\item[$f_s(ζ)$] In genus one, another the Möbius transformation that takes the standard form of the spectral curve~\eqref{eqn:genus one curve} to the Legendre form. See page~\pageref{para:def f_s}.

\item[$F$] The common factor of $P$, $b^1$ and $b^2$.
\item[$F^i$] The common factor of $P/F$ and $b^i/F$.

\item[$F(z;k)$] The incomplete elliptic integral of the first kind. See Section~\ref{sec:Elliptic Defs}.
\[
F(z;k) = \int_0^z \frac{dt}{\sqrt{(1-t^2)(1-k^2 t^2)}}.
\]

\item[$F_0(x;k)$] A variant of $F(z;k)$ focused on imaginary arguments. $F_0 (x ; k) := \Imag F(\iu x; k)$

\item[$G$] The common factor of $b^1/FF^1$ and $b^2/FF^2$.

\item[$\mathcal{G}$] The group of covering transformations of $\mathcal{\tilde{C}}_1$ over $\mathcal{C}_1$.

\item[$k$] The elliptic modulus. The elliptic modulus of a genus one marked curve is given by~\eqref{eqn:def_k}.

\item[$K(k)$] The complete elliptic integral of the first kind. $K(k) = F(1;k)$.

\item[$l$] The imaginary period of the differential $Ψ^P$ is $2π\iu l$, for some positive integer $l \in \Z^+$.

\item[$\ell$] A path in the moduli space $\mathcal{M}_g$ of spectral data.

\item[$L$] A function related to the derivatives of $T$. The subject of Lemma~\ref{lem:deriv no zeroes}.

\item[$L'$] A function derived from $L$ related to the derivatives of $T$ over the points where $u=\infty$. The subject of Lemma~\ref{lem:deriv no zeroes}.

\item[$\mathcal{M}_g$] The moduli space of spectral data $(Σ, Θ^1, Θ^2)$ consisting of a marked curve $Σ$ and two differentials satisfying conditions~\ref{P:poles}--\ref{P:closing}.

\item[$\Mat_2^* \Z$] The set of two-by-two integer matrices with nonzero determinant.

\item[$N_{(α,ν)}$] The normalisation map of a singular marked curve $Σ(α,ν)$ for $ν\in \S^1$.

\item[$p$] Represents a value of the function $S$. A coordinate on $\mathcal{A}_1$.

\item[$P(ζ)$] The equation of the spectral curve is $η^2 = P(ζ)$. In standard form~\eqref{eqn:def P}, where the roots are given by conjugate inverse pairs $α_j, \cji{α}_j$,
\[
P(ζ) = \prod (ζ - α_j)(1 - \bar{α}_jζ).
\]

\item[$P_k$] The coefficient of $ζ^k$ in $P(ζ)$.

\item[$\dot{P}$] The derivative of $P$ with respect to $t$ along a path $\ell(t)$ in $\mathcal{M}_g$.

\item[$\mathcal{P}^k$] The space of polynomials of degree at most $k$. See Definition~\ref{def:mathcal P}.

\item[$\mathcal{P}^k_\R$] The space of polynomials of degree at most $k$ that a real with respect to $ρ$. See Definition~\ref{def:mathcal P}.

\item[$q$] Represents a value of the function $\tilde{T}$. A coordinate on $\mathcal{\tilde{C}}_1$. See Lemma~\ref{lem:T_graph}.

\item[$Q$] A real quadratic polynomial that describes an infinitesimal deformation of the moduli space of spectral data. See~\eqref{eqn:Q}.

\item[$\hat{Q}$] An imaginary quartic polynomial that describes an infinitesimal deformation of a pair of differentials that satisfy~\ref{P:poles}--\ref{P:periods}. See~\eqref{eqn:Q hat}.

\item[$R$] A function $R : \mathcal{U}^{(i)} \times \R^3 \to \C$ given by Definition~\ref{def:def R}. This function must be zero at a point of $\mathcal{M}_g \cap \mathcal{U}^{(i)}$ in order for deformations to exist.

\item[$S$] A positive function on $\mathcal{A}_1$ such that a genus one marked curve $Σ(α,β)$ admits an exact differential meeting the closing conditions~\ref{P:closing} only if $S(α,β)$ is rational. See~\eqref{eqn:def_S}.

\item[$\mathcal{S}_g$] The moduli space of spectral curves of genus $g$. These are marked curves for which there are a pair of differentials which satisfy~\ref{P:poles}--\ref{P:closing}.

\item[$\mathcal{\tilde{S}}_1$] The universal cover of $\mathcal{S}_1$. It is a identified with a subspace of $\mathcal{\tilde{C}}_1$. See~\eqref{eqn:def tilde S}.

\item[$\su_2$] The Lie algebra of $\SU_2$.

\item[$\SU_2$] The group of two-by-two special unitary matrices, $A \bar{A}^T = I$.

\item[$T$] A multi-valued function on $\mathcal{A}_1$. such that a genus one marked curve $Σ(α,β)$ admits differentials meeting the closing conditions~\ref{P:closing} only if $T(α,β)$ is rational. See~\eqref{eqn:def_T}.

\item[$T_0$] A principal branch cut of $T$. See Definition~\ref{def:def_T_0}.

\item[$\tilde{T}$] A lift of $T_0$ from $\mathcal{A}_1$ to the universal cover $\mathcal{\tilde{C}}_1$. It may be computed from $T_0$ via the formula~\eqref{eqn:tilde T computable}.

\item[$\mathbb{T}^2$] The torus.

\item[$u$] A coordinate on $\mathcal{A}_1$, defined by $\iu u = f(1)$. See Lemma~\ref{lem:change of parameters}.

\item[$u'$] A coordinate on $\mathcal{A}_1$, defined by $u' = u^{-1}$. See Lemma~\ref{lem:change of parameters}.

\item[$\tilde{u}$] A coordinate on $\mathcal{\tilde{C}}_1$. See Lemma~\ref{lem:mathcal tilde C}.

\item[$U$] A coordinate on $\mathcal{A}_1$ on which $λ$ acts by a rotation by $π$ radians. Defined by $U=\sqrt{k} u$.

\item[$\tilde{U}$] A coordinate on $\mathcal{\tilde{C}}_1$ defined in relation to $U$ in the same way that $\tilde{u}$ is related to $u$ (via a half-$\tan$ covering map).

\item[$\mathcal{U}$] An open subset of the affine space $\mathcal{P}^{2g+2}_\R\times \mathcal{P}^{g+3}_\R\times\mathcal{P}^{g+3}_\R$ on which the moduli space $\mathcal{M}_g$ of spectral data is embedded via the reduction of a triple of spectral data $(Σ,Θ^1,Θ^2)$ to polynomials $(P,b^1,b^2)$. See Definition~\ref{def:def U}.

\item[$\mathcal{U}'$] An open subset of $\mathcal{U}$. See Definition~\ref{def:subsets U}.

\item[$\mathcal{U}''$] A subset of $\mathcal{U}$ corresponding to conformal harmonic maps. See Definition~\ref{def:subsets U}

\item[$\mathcal{U}^{(i)}$] An open dense subset of $\mathcal{U}'$ where the roots of the polynomials $P$, $b^1$ and $b^2$ associated to spectral data have distinct roots. See Definition~\ref{def:subsets U}

\item[$v$] A coordinate on $\mathcal{A}_1$, defined by $\iu v = f(-1)$.

\item[$v'$] A coordinate on $\mathcal{A}_1$, defined by $v' = v^{-1}$.

\item[$\tilde{v}$] A coordinate on $\mathcal{\tilde{C}}_1$. See Lemma~\ref{lem:mathcal tilde C}.

\item[$V$] A coordinate on $\mathcal{A}_1$ on which $λ$ acts by a rotation by $π$ radians. Defined by $V=\sqrt{k} v$.

\item[$\tilde{V}$] A coordinate on $\mathcal{\tilde{C}}_1$ defined in relation to $V$ in the same way that $\tilde{v}$ is related to $v$ (via a half-$\tan$ covering map).

\item[$w(z)$] The equation of an elliptic curve in Legendre form. $w(z)^2 = (1-z^2)(1-k^2z^2)$.

\item[$w'(z')$] Variant of $w(z)$ used to concisely write equations in the coordinates `at infinity', such as $u'$ and $v'$.
\[
w'(z')^2 = (z')^2\,w\bra{(z')^{-1}} =  (1-(z')^2)(k^2 - (z')^2).
\]


\item[$W$] The vector $3$-space of differentials on a marked curve satisfying conditions~\ref{P:poles}--\ref{P:reality}. It has bases $\{ ω, Θ^E, ε \}$ and $\{ ω, Θ^E, Θ^P \}$. See Section~\ref{sec:Differentials}.

\item[$x$] A parameter that determines the conformal type of the domain of a harmonic map with a genus zero spectral curve. See~\eqref{eqn:conformal type}.

\item[$X$] A vector in $\su_2$. A harmonic map $g$ corresponding to a genus zero spectral curve may be brought into the form $g(w_R + \iu w_I) = \exp (-4w_R X) \exp (4w_I Y)$, as in~\eqref{eqn:genus zero simple map}.

\item[$Y$] A vector in $\su_2$. See the glossary entry for $X$.

\item[$z$] A coordinate on $\CP^1$, over which the Legendre elliptic curve is written.

\item[$z_0$] The image of $0$ under $f$. It is crucial to the construction of the map $f^{-1}$ in terms of the parameters $(p,k,u,v)$, and given by~\eqref{eqn:formula xy}.






%%%%%%%%%%%%%%%%%%%%%%%%%%%%%%%%%%%%%%%%%%%%%%%%%%%%%%%%%%%%%%%%%%%%%%%%%%%%%%%%
%%%%%%%%%%%%%%%%%%%%%%%%%%%%%%%%%%%%%%%%%%%%%%%%%%%%%%%%%%%%%%%%%%%%%%%%%%%%%%%%
%%%%%%%%%%%%%%%%%%%%%%%%%%%%%%%%%%%%%%%%%%%%%%%%%%%%%%%%%%%%%%%%%%%%%%%%%%%%%%%%
%%%%%%%%%%%%%%%%%%%%%%%%%%%%%%%%%%%%%%%%%%%%%%%%%%%%%%%%%%%%%%%%%%%%%%%%%%%%%%%%


\item[$α$] A branch point of a marked curve, in the unit disc.

\item[$β$] Another branch point of a marked curve, in the unit disc.

\item[$γ_+$] A path on a marked curve $Σ$ between the two points lying over $ζ=1$.

\item[$\symbf{γ}_+$] A principal choice of the path $γ_+$. See page~\pageref{para:principal paths}.

\item[$γ_-$] A path on a marked curve $Σ$ between the two points lying over $ζ=-1$.

\item[$\symbf{γ}_-$] A principal choice of the path $γ_-$. See page~\pageref{para:principal paths}.

\item[$δ$] A parameter which determines the image of a harmonic map corresponding to a spectral curve of genus zero. It is defined in~\eqref{eqn:def delta} to be the angle between $X$ and $Y$.

\item[$ε$] A differential on a genus one marked curve used to construct a basis of $W$. Defined by~\eqref{eqn:def ε}.

\item[$ζ$] A coordinate on $\CP^1$, over which the marked curve is defined.

\item[$η$] A marked curve is given in standard form by $η^2 = P(ζ)$, for $P$ given by~\eqref{eqn:def P}.

\item[$η^+(ζ)$] The value of the fibre coordinate over $ζ$. For a point $ζ = e^{\iu θ}$ on the unit circle and a spectral curve in standard form, for odd genus $g$,
\[
η^+(e^{\iu θ}) = e^{\iu θ \frac{g+1}{2}} \prod_j \abs{e^{\iu θ} - α_j}.
\]

\item[$Θ^E$] An exact differential on a genus one marked curve which satisfies conditions~\ref{P:poles}--\ref{P:periods}.
\[
Θ^E = \iu\, d \bra{\frac{η}{ζ}}.
\]

\item[$Θ^P$] A differential on a genus one marked curve which satisfies conditions~\ref{P:poles}--\ref{P:periods} and has an imaginary period of $2π\iu$. See~\eqref{eqn:thetaP}.

\item[$κ$] A parameter for translation invariant solutions to~\eqref{eqn:Hit1.7} that determines the angle of the lattice of periods to the $z$-axis. See~\eqref{eqn:genus zero map}.

\item[$λ$] The involution on $\mathcal{A}_1$ that swaps the labelling of the branch points. See~\eqref{eqn:def_lambda}.

\item[$\tilde{λ}$] The translation on $\mathcal{\tilde{C}}_1$ that generates the group of covering transformation of $\mathcal{\tilde{C}}_1$ over $\mathcal{C}_1$. See~\eqref{eqn:tilde lambda action}.

\item[$μ$] In the genus one case, the branch points lie on a circle. $μ$ is the intersection of the branch circle with the unit circle that lies between $α$ and $\cji{α}$. See Figure~\ref{fig:zeta plane}.

\item[$ν$] The intersection of the branch circle with the unit circle that lies between $β$ and $\cji{β}$. See Figure~\ref{fig:zeta plane}.

\item[$ξ$] A point on a marked curve $Σ$.

\item[$\tilde{π}$] The projection of $\mathcal{\tilde{C}}_1$ to $\mathcal{A}_1$. See Lemma~\ref{lem:mathcal tilde C}.

\item[$ρ$] The real involution on a marked curve. See Definition~\ref{def:marked curve}.

\item[$\tilde{ρ}$] The real involution $ρ$ when written in the $z$-coordinate. $\tilde{ρ}(z) = - \bar{z}$.

\item[$σ$] The hyperelliptic involution. See Definition~\ref{def:marked curve}.

\item[$σ_1, σ_2, σ_3$] The standard basis of $\su_2$. See~\eqref{eqn:su2 basis}.

\item[$Σ$] A marked curve. See Definition~\ref{def:marked curve}.

\item[$Σ(α,β)$] The genus one marked curve with branch points $\{α,\cji{α},β,\cji{β}\}$.

\item[$τ$] The conformal parameter of the torus. See~\eqref{eqn:Q0 change conformal} or~\eqref{eqn:conformal type}.

\item[$Φ$] The Higgs field, a $(1,0)$ section of the vector bundle $\ad P$ for principal $G$-bundle over $M$. Every harmonic map corresponds to a pair $(A,Φ)$ satisfying~\eqref{eqn:Hit1.7}.

\item[$χ$] An involution on $\mathcal{A}_g$ that take branch points to their negatives. See~\eqref{eqn:def chi}.

\item[$Ψ^E$] An exact differential on a genus one marked curve that satisfies conditions~\ref{P:poles}--\ref{P:closing}. See Lemma~\ref{lem:closing_conds}.

\item[$Ψ^P$] A differential on a genus one marked curve that satisfies conditions~\ref{P:poles}--\ref{P:closing} and has imaginary period $2π\iu$. See Lemma~\ref{lem:closing_conds}.

\item[$ω$] A holomorphic differential on a genus one marked curve given by $dz/w$.



\end{description}
