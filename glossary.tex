%!TEX root = thesis.tex
\section{Glossary}
Below is a list of notation and terms, with a short description and a reference to their definition. They are arranged alphabetically, first Latin then Greek.

\begin{description}[align=right]

\item[$f(ζ)$] In genus one, the M\"obius transformation that takes one from the standard form of the spectral curve REF to the Jacobi form.

\item[$P(ζ)$] The equation of the spectral curve is $η^2 = P(ζ)$. In standard form, where the roots are given by conjugate inverse pairs $α_j, \cji{α}_j$,
\[
    P(ζ) = \prod (ζ - α_j)(1 - \bar{α}_jζ).
\]

\item[$w(z)$] The equation of an elliptic curve in Jacobi form. $w(z)^2 = (1-z^2)(1-k^2z^2)$.

\item[$w'(z')$] Variant of $w(z)$ used to concisely write equations in $z'$. $w'(z')^2 = (1-(z')^2)(k^2 - (z')^2)$.

\item[$z$] A coordinate on $\CP^1$, over which the Jacobi elliptic curve is written.

\item[$z'$] A coordinate on $\CP^1$, with $z' = z^{-1}$.

\item[$ζ$] A coordinate on $\CP^1$, over which the spectral curve is defined.

\item[$η$] The spectral curve is defined by $η^2 = P(ζ)$.

\item[$η(ζ)^+$] The value of the fibre coordinate over $ζ$, taken so that $η(1)$ is positive. For a point on the unit circle $ζ = e^{\iu θ}$ and a spectral curve in standard form, for odd genus $p$,
\[
    η(e^{\iu θ})^+ = e^{\iu θ \frac{p+1}{2}} \prod_j \abs{e^{\iu θ} - α_j}.
\]

\end{description}
