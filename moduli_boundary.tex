%!TEX root = thesis.tex

\chapter{The Boundary of the Moduli Space}
\label{chp:Moduli Boundary}

The boundary of $\mathcal{A} = \{ (α,β) \in D^2 \mid α \neq β\}$, can be thought of as composed of two pieces. On the outside we have the places where one or both of $α,β$ are in the unit circle. Fix a value of $p$. This is a `ball' in the space $(α,β)\in D^2$, as for any particular value of
\[
t = \frac{1}{\sqrt{p}} \frac{\abs{1-α}}{\abs{1+α}},
\]
$α$ ranges over an arc of a circle, and likewise does $β$ as constrained by the equation
\[
\frac{\abs{1-β}}{\abs{1+β}} = \frac{\sqrt{p}}{u},
\]
where the product of these two equations is just the definition of $p$. Hence for any $t$ we have the cross section given as the product of two arcs, with the length of the arcs diminishing as $u$ approaches $0$ or $\infty$. As $t$ approaches $0$, this forces the branch points to $S_0 := (1,-1)$, and likewise at $t \to \infty$, $S_\infty := (-1,1)$. In the reverse direction, $α = \pm 1$ if and only if $β = \mp 1$. Hence, these two points divide the unit circle into disconnected upper and lower halves: $\S^1_+$ and $\S^1_-$ respectively.

This ball of constant $p$ is an American football shape; an elongated ball with sharp points at either end, joined by four seams (`\emph{edges}') where both $α$ and $β$ lie in one of $\S^1_\pm$ and four `\emph{faces}' where only one does.

For other piece of the boundary, on the interior, we have the plane $α=β$. If again we fix $p$, then the intersection of this plane and the ball is $(α,α)$, where $α$ is constrained to
\[
p = \frac{\abs{1-α}^2}{\abs{1+α}^2},
\]
an arc. Though it reveals a counterintuitive aspect of our naming convention, the interior and exterior boundaries intersect at two points, $(α,α)$ and $(\bar{α},\bar{α})$ for some point $α\in\S^1_+$. We denote these intersection points as $P_+, P_-$ respectively. If one is visualising the ball as a solid $3$-ball and the exterior boundary as its surface sphere, then one could imagine $P_\pm$ as polar points and the interior boundary as an axis that run between them.




\section{Interior Boundary}
\label{sec:Interior}

We shall examine first the moduli space on the interior boundary, which we shall call $Δ$. It was already mentioned, but recall the definition of $k$ (eqn \eqref{eqn:def_k}),
\[
k = \frac{\abs{1-\bar{α}β}-\abs{α-β}}{\abs{1-\bar{α}β}+\abs{α-β}}.
\]
As $α\to β$, this tends to $1$. In the coordinates $(p,k,u,v)$, we have fixed $p$ and $k=1$, so $Δ$ is covered by the two coordinates $u,v$. However, this space is one-dimensional, so these coordinates are providing a natural blowup of this space. To see why, consider that as $α\to β$, the branch point circle straightens to a line. The double point is therefore the intersection of the arc $Δ$ with the line joining $μ,ν$. There are many such lines that give the same intersection, representing the different directions that $α$ may approach $β$. The changing the positions of $μ,ν$ on the unit circle changes the relative positions of $\iu u,\iu v$ on the imaginary axis, so the locations of $u,v$ are recording the direction of approach in this limit. \todo{this is truly terrible}

Let us compute the limits of the integrals (eqns \eqref{eqn:gamma_plus2}, \eqref{eqn:gamma_minus2}) as $k \to 1$. This is not trivial, because as $k\to 1$, $K(k) \to \infty$. We shall see however that the integrals remain finite, atleast in some places. Recall the inequality \eqref{eqn:K_bound}
\[
\ln 4 \leq K + \frac{1}{2}\ln (1-k) + \frac{1}{2}\ln (1+k) \leq \frac{π}{2}.
\]
This shows that at $k=1$, $K$ has a logarthimic singularity. We now compute
\begin{align*}
\lim_{k\to 1} \int_{γ_+} Θ^2
&= \lim_{k\to 1} 4 E(k) F(\iu u;k) - \lim_{k\to 1} 4 \iu K(k) \left[ \Imag E(\iu u;k) - \frac{w(\iu u)}{u-v}\right] \\
&= \lim_{k\to 1} 4 E(k) F(\iu u;k) - \lim_{k\to 1} 4 \iu K(k) \left[ \bra{\Imag E(\iu u;k) - ku} + (ku - u) + \bra{u - \frac{1+u^2}{u-v}} + \frac{1+u^2 - w(\iu u)}{u-v}\right].
\end{align*}
We have divided the limit calculation into five terms. We deal with them one at a time. First
\[
\lim_{k\to 1} 4 E(k) F(\iu u;k) = 4 \times 1 \times \iu \atan u
\]
by lemma \ref{lem:klimit}. Next, we recognise $\Imag E(\iu u;k) - ku$ as the function $E_0(u; k)$, which is bounded in magnitude by $K'-E'$. So
\[
\lim_{k \to 1} \abs{K(k) E_0(u;k)}
\leq \lim_{k \to 1} K(K'-E')
= \lim_{k \to 1} -\frac{π}{2} + K'E
= -\frac{π}{2} + \frac{π}{2}\times 1 = 0,
\]
where we used Legendre's relation in the second step. The next term is likewise zero, coming from the inequality on $K$
\begin{align*}
\ln 4 u(k-1) &\leq (K + \frac{1}{2}\ln (1-k) + \frac{1}{2}\ln (1+k))u(k-1) \leq \frac{π}{2} u(k-1) \\
0 &\leq \lim_{k\to 1} K u(k-1) + 0 \leq 0.
\end{align*}
We will skipping to the last term
\begin{align*}
\lim_{k\to 1} 4 \iu K \frac{1+u^2 - w(\iu u)}{u-v}
&= \lim_{k\to 1} 4 \iu K \frac{(1+u^2)\bra{ (1+u^2) - (1+k^2u^2)} }{(u-v)(1+u^2 + w(\iu u))} \\
&= \lim_{k\to 1} 4 \iu K (1 - k^2) \frac{(1+u^2)u^2}{(u-v)(1+u^2 + w(\iu u))} \\
&=0.
\end{align*}
What remains is
\[
\lim_{k\to 1} \int_{γ_+} Θ^2
= 4 \iu \atan u + 4\iu \lim_{k\to 1} K \frac{1+uv}{u-v} .
\]
At this point we have played our hand in terms of using expressions of $k$ to cancel the singularity of $K$. Indeed, for most values of $u,v$ this limit will diverge. However, if that fraction we to tend to zero as $k$ went to one, then we could still arive at a finite value. Suppose that is tending to one very rapidly, say $k = 1 - \exp (-1/ε)$ for $ε \to 0$. Then if $u$ and $v$ are functions of $ε$ chosen such that for some constant $λ$
\[
\frac{1+uv}{u-v} = λε,
\]
then
\[
lim_{k \to 1} K \frac{1+uv}{u-v}
= 2λ lim_{ε \to 0} Kε
= \frac{λ}{2}.
\]
So
\[
\lim_{k\to 1} \int_{γ_+} Θ^2
= 4 \iu \atan u + 2\iu λ.
\]
The same approach works for the $γ_-$ integral,
\[
\lim_{k\to 1} \int_{γ_-} Θ^2
= 4 \iu \atan v - 2\iu λ.
\]
Note that the choice of $u,v$ forces $1+uv \to 0$, so that in the limit we have that $v = -u^{-1}$. So together we have, for $u>0$,
\[
π T_0 = -πp + 2(p-1) \atan u + (p+1)λ.
\]
Things are particularly clear if we move the the universal cover, where $u = \tan (\tilde{u}/2)$
\[
π \tilde{T} = (p-1) \tilde{u} + (p+1)λ.
\]
In the universal cover, the realtion $uv=-1$ becomes $\tilde{v} = π + \tilde{u}$. The level surfaces of constant $\tilde{T} = q$ are clearly straight lines of $\tilde{u}$ and $λ$,
\[
λ = \frac{1-p}{1+p}\tilde{u} + π \frac{q}{1+p}.
\]
One interpretation of this is that as $\tilde{u}$ increases by $2π$, we have circled the line $k=1$, and the steepness at which the moduli space approaches this point, governed by $λ$, has increased. So as we repeated wrap around the interior boundary, the moduli space swirls around, accumulating at a single point. This justifies our earlier assertion that the moduli space is similiar to a helicoid.

Where is this point? Looking back at the equations that determine $z_0$, we have eqn \eqref{eqn:z_0_circle}
\[
x^2 + y^2 = y(u+v) - uv.
\]
Knowing that $uv \to -1$, means that $\pm 1$ lie on this circle, the circle through $\iu u, \iu v, z_0, -\bar{z}_0$. Transforming by $f^{-1}$, that in turn means that $α$ lies on the real axis. Using the condition for $p$, we have explicitly that
\[
α = \frac{1 - \sqrt{p}}{1+\sqrt{p}},
\]
which tends to $1$ as $p\to 0$, and $-1$ as $p \to \infty$. This is somewhat suprising. Hueristically, $\mathcal{S}(p)$ is a dense collection of surfaces in the $3$-space $\mathcal{C}(p)$, and it would make sense for the intersection to therefore be a dense collection of points in the arc $Δ$. Instead, there is only a single point that all the surfaces gather upon.






\section{Exterior Boundary}
\label{sec:Exterior}

We bgein with two calculations that will shows that, as $β$ moves onto the unit circle, there is a correspondence between the limit of genus one spectral data and genus zero spectral data. First we show that the spectral curves and differentials are related by a pullback operation. Then we will show that the closing conditions are preserved.

Recall that a marked curves of genus one may be written as $Σ(α,β) = \{ (ζ,η) \mid η^2 = P(ζ) = (ζ-α)(1-\bar{α}ζ)(ζ-β)(1-\bar{β}ζ) \}$ for $α,β$ distinct points inside the unit disc. On every such curve the plane of differentials satisfying conditions REF is spanned by the differentials $Θ^1$ and $Θ^2$. Let use take the limit as $β \to \S^1$. We note that $ν\in\S^1$ was defined to be the point of intersection between the unit circle and the circle on which $α,\cji{α},β$ and $\cji{β}$ lie, more specifically the point of intersection that lies between $β$ and $\cji{β}$. Thus, we can also express the limit $β \to \S^1$ as $β \to ν$.

In this limit as $β \to ν$, the curve $Σ(α,β)$ becomes a nodal curve
\[
Σ(α,ν) = \{ (ζ,η) \mid η^2 = P(ζ) = (-\bar{ν})(ζ-ν)^2 (ζ-α)(1-\bar{α}ζ) \},
\]
which has a genus zero normalisation
\[
Σ(α) = \{ (ζ,η) \mid η^2 = P(ζ) = (ζ-α)(1-\bar{α}ζ) \},
\]
where the normalisation map is given by
\begin{align*}
N : Σ(α) &\to Σ(α,ν) \\
(ζ,η)  &\mapsto (ζ,\sqrt{-\bar{ν}}(ζ-ν) η).
\end{align*}
The presence of the square root represents an unavoidable sign issue inherent in this normalisation map. On both curves, $η$ takes real values at the two points of lying over $ζ=1$. By the hyperelliptic involution $σ$, these two values necessarily have opposite signs. One may try to choose the sign of $\sqrt{-\bar{ν}}$ in such a way that the normalisation map sends the point over $ζ=1$ on $Σ(α)$ where $η$ is positive to the point on $Σ(α,ν)$ where $η$ is positive. However, there is no way to smoothly make this choice as $ν$ is allowed to vary, as after $ν$ traverses the unit circle once the sign of $\sqrt{-\bar{ν}}$ will have been reversed. Further, this kind of identification does not even make sense when $ν=1$, as there is only one point over $ζ=1$ on $Σ(α,1)$. The consequence of this is that we will only be able to identify certain differentials associated to each curve (and so also their integrals) up to a sign.

Under this map $N$, the exact differential $Θ^1$ on $Σ(α,ν)$ pulls back to
\[
N^* Θ^1 = N^* \iu\; d \bra{\frac{η}{ζ}} = \iu\; d \bra{ ζ^{-1}η\sqrt{-\bar{ν}}(ζ-ν) } ,
\]
which is of the form \eqref{eqn:genus zero differential}, so it is a genus zero differential that meets conditions REF. To compute the limit and pullback of $Θ^2$ is more involved. First note that
\[
k
= \frac{\abs{1-\bar{α}β}-\abs{α-β}}{\abs{1-\bar{α}β}+\abs{α-β}}
\to \frac{\abs{ν}\abs{\bar{ν}-\bar{α}}-\abs{α-ν}}{\abs{1-\bar{α}ν}+\abs{α-ν}} = 0.
\]
This enables us to compute the limit of the first two terms of \eqref{eqn:thetaP}. In particular, recall \eqref{eqn:def_e}, the definition of the differential $e$
\[
e := (1-k^2 z^2) \frac{dz}{w} \to \frac{dz}{w} = ω,
\]
where $ω$ is a holomorphic differential. Also recall that the complete elliptic integrals $K(k)$ and $E(k)$ take the value $π/2$ when $k=0$ REF. Thus
\begin{align*}
\lim_{β \to ν} Θ^2
&= \lim_{β\to ν} \left[ 2E ω - 2Ke - 2K d\left[ \frac{w(z-\iu\,\Imag z_0)}{(z-z_0)(z + \bar{z}_0)} \right] \right] \\
&= 2\frac{π}{2} ω - 2\frac{π}{2} \bra{\lim_{k\to 0} e} - 2\frac{π}{2} \lim_{β\to ν} d\left[ \frac{w(z-\iu\,\Imag z_0)}{(z-z_0)(z + \bar{z}_0)} \right] \\
&= - π\lim_{β\to ν} d\left[ \frac{w(z-\iu\,\Imag z_0)}{(z-z_0)(z + \bar{z}_0)} \right],
\end{align*}
which disposes of the transcendental part of $Θ^2$. To convert the remaining term into an expression in terms of $ζ$ and $η$, so that we may normalise as above, recall that $η/ζ$ and
\[
\frac{w}{(z- z_0)(z+\bar{z}_0)},
\]
are real scalar multiples of one another. By direct substitution of \eqref{eqn:f},
\[
z - \iu\,\Imag z_0
= \frac{1}{ζ-ν} \bra{-ζ(\bar{z}_0 + \iu\,\Imag z_0) + (μ\bar{z}_0 + \iu ν\,\Imag z_0)}
= -\frac{\Real z_0}{ζ-ν} \bra{ζ - ν\frac{\iu \,\Imag z_0+ μν^{-1}\bar{z}_0}{\iu\,\Imag z_0 + \bar{z}_0} }.
\]
From the definition $z_0 = f(0)$ and \eqref{eqn:f} we see that $μν^{-1}\bar{z}_0 = -z_0$. Then from \eqref{eqn:f_inv}, we recognise the constant inside the bracket as $f^{-1}(\iu\,\Imag z_0)$. This may be evaluated by means of a geometric argument. Consider the $z$-plane, and the horizontal line that passes through $-\bar{z}_0$, $\iu \,\Imag z_0$, $z_0$ and $\infty$. Under $f^{-1}$ these points are respectively mapped to $\infty$, $f^{-1}(\iu \,\Imag z_0)$, $0$ and $ν$. So the line is mapped to a line in the $ζ$-plane that passes through the origin and $ν$. $\iu \,\Imag z_0$ also lies on the imaginary axis of the $z$-plane, which is taken to the unit circle. Therefore we see that $f^{-1}(\iu \,\Imag z_0) = -ν$.

PCITURE OF THIS.\todo{}

Using this, we may simplify the factor to
\[
z - \iu\,\Imag z_0
= -\frac{\Real z_0}{ζ-ν} \bra{ζ + ν }.
\]
For some real number $r$,
\[
\lim_{β \to ν} Θ^2
= r π(\Real z_0)\, d\left[ \frac{η}{ζ}\frac{ζ + ν }{ζ-ν} \right].
\]
The pull back of this by the normalisation map $N$ is then
\[
N^* \lim_{β \to ν} Θ^2
= r π(\Real z_0)\, d\left[ ζ^{-1}η \sqrt{-\bar{ν}}(ζ + ν) \right],
\]
which also lies in the plane of differentials on the genus zero curve $Σ(α)$ meeting conditions REF. What we have shown is that if we move along a path in $\mathcal{M}_1$, the space of spectral data where the spectral curve has genus one, then the limit of this data as one of the branch points $β$ tends to the unit circles, after normalising the double point that develops on $Σ(α,ν)$, is potentially spectral data on the genus zero marked curve $Σ(α)$, because it meets conditions REF. There are no periods on a curve of genus zero, so REF is automatic. The only ocnditions that we have not yet shown to be satisfied in the limit is REF, the closing conditions.

The closing conditions are that certain integrals of the differentials must lie in $2π\iu \Z$. In particular, locally in $\mathcal{M}_1$ they are constant and so are unchanged under the limit. What we will show then is the more general proposition that if $Ψ$ is a differential satisfying REFs \todo{not the closing conditions}, then
\[
\int_{γ_+(α)} N^* \lim_{β\to ν} Ψ = \pm \lim_{β\to ν} \int_{γ_+(α,β)} Ψ,
\]
where $γ_+(α)$ is a path between the points over $ζ=1$ on $Σ(α)$ and $γ_+(α,β)$ is the path between $ζ=1$ on $Σ(α,β)$. The ambiguity in sign is exactly the ambiguity in the choice of sign of the square root discussed above. We will show a similar result for the other closing condition, over the path $γ_-$. The method of proof will be to show this for $Θ^1$ and $Θ^2$ and then because any such $Ψ$ may be expressed as a smooth combination of these, the general result will follow.

For exact differentials, such as $Θ^1$, one may compute both sides explicitly and compare. Consider the exact differential $d\left[ g(α,β)(ζ) + η h(α,β)(ζ) \right]$ on $Σ(α,β)$. On one hand we have
\begin{align*}
\int_{γ_+(α)} N^* \lim_{β\to ν} d\left[ g(α,β)(ζ) + η h(α,β)(ζ) \right]
&= \int_{γ_+(α)} N^* d\left[ g(α,ν)(ζ) + η h(α,ν)(ζ)\right] \\
&= \int_{γ_+(α)} d\left[ g(α,ν)(ζ) + η \sqrt{-\bar{ν}}(ζ-ν) h(α,ν)(ζ)\right] \\
&= 2 η(1) \sqrt{-\bar{ν}}(1-ν) h(α,ν)(1), \\
&= 2 \abs{1-α} \sqrt{-\bar{ν}}(1-ν) h(α,ν)(1),
\end{align*}
and on the other hand
\begin{align*}
\lim_{β\to ν} \int_{γ_+(α,β)} d\left[ g(α,β)(ζ) + η h(α,β)(ζ) \right]
&= \lim_{β\to ν} 2 η(1) h(α,β)(1) \\
&= \lim_{β\to ν} 2 \abs{1-α}\abs{1-β} h(α,β)(1) \\
&= 2 \abs{1-α}\abs{1-ν} h(α,β)(1),
\end{align*}
which are the same up to a sign as $\sqrt{-\bar{ν}}(1-ν) = \pm \abs{1-ν}$. A similar calculation holds for $γ_-$. This establishes the relationship for $Θ^1$. For $Θ^2$, we must deal with its elliptic integral terms. But from \eqref{eqn:gamma_plus} we have that
\[
\int_{γ_+} 2E\tilde {ω} - 2K e = 4 E(k) F(f(1);k)- 4K(k) \tilde E(f(1);k).
\]
We refer to Lemma \ref{lem:klimit}, which computes the limits of incomplete elliptic integrals as $k\to 0$. That lemma states that so long as $f(1)$ is contained in a compact set,
\[
\lim_{k\to 0} \int_{γ_+} 2E\tilde {ω} - 2K e = 2π \left[ \asinh (-\iu f(1)) - \asinh (-\iu f(1)) \right] = 0.
\]
For the range of $f(1) \in \iu\R$ to be bounded, we must forbid $f(1) \to \infty$. Translating this into the $ζ$-plane by applying $f^{-1}$, we have that $ν$ must not approach $1$. From \eqref{eqn:gamma_minus}, by the same reasoning,
\[
\lim_{k\to 0} \int_{γ_-} 2E\tilde {ω} - 2K e = 2π \left[ \asinh (-\iu f(-1)) - \asinh (-\iu f(-1)) \right] = 0,
\]
as long as $ν$ does not approach $-1$. Combining these two facts, for $ν \not\to \pm 1$,
\begin{align*}
\int_{γ_+(α)} N^* \lim_{β\to ν} Θ^2
&= \int_{γ_+(α)} 0 - N^* \lim_{β\to ν} 2K  d\left[ \frac{w(z-\iu\,\Imag z_0)}{(z-z_0)(z + \bar{z}_0)} \right] \\
&= \pm \lim_{β\to ν} \int_{γ_+(α,β)} 0 - 2K d\left[ \frac{w(z-\iu\,\Imag z_0)}{(z-z_0)(z + \bar{z}_0)} \right] \\
&= \pm \lim_{β\to ν} \int_{γ_+(α,β)} 2E ω - 2Ke - 2K d\left[ \frac{w(z-\iu\,\Imag z_0)}{(z-z_0)(z + \bar{z}_0)} \right] \\
&= \pm \lim_{β\to ν} \int_{γ_+(α,β)} Θ^2.
\end{align*}
Thus the relationship holds for $Θ^2$ as well, and by linear extension all differentials satisfying conditions REF.

Therefore, the limit of spectral data in $\mathcal{M}_1$ is spectral data in $\mathcal{M}_0$. It is natural to ask if we can give an give a geometric construction that connects the two spaces. To do this, we will take the partial closure of $\mathcal{M}_1$ and identify some points in the boundary with some points in $\mathcal{M}_0$. Recall that $\mathcal{M}_1$ was described by first describing the space of spectral curves $\mathcal{S}$, that is the space of marked curves that admit differentials satisfying the period and closing conditions REF, as well as the usual conditions REF which may be satisfied on any marked curve. This space was disjointly decomposed into pieces $\mathcal{S}(p)$, for $p\in\Q$ on which the function
\[
p(α,β) = \frac{\abs{1-α}\abs{1-β}}{\abs{1+α}\abs{1+β}}
\]
was constant. Each piece $\mathcal{S}(p)$ was considered as a subspace of the space $\mathcal{C}(p)$ of marked curves. However, to describe marked curves, we introduced a parameter space $\mathcal{A}_1(p)$ where the branch points inside the unit circle were considered as an ordered pair,
\[
\mathcal{A}_1 = \{ (α, β) \in D^2 \mid α \neq β \}.
\]
This parameter space is (by definition) a two-to-one cover of the space $\mathcal{C}(p)$ of marked curves $Σ(α,β)$ for which $p(α,β)$ is fixed at $p$. It is clear  to take the partial closure of this space within $\C^2$ to say that the space where one of the branch points lies on the unit circle is $D\times\S^1 \coprod \S^1\times D$. By Lemma \ref{lem:change of parameters}, we know that we may equivalently describe $\mathcal{A}_1(p)$ as the space
\[
\mathcal{A}'(p) = \Set{ (k,u,v) \in (0,1) \times (\RInf) \times (\RInf) }{ u \neq v}.
\]
Unfortunately however, this correspondence between pairs $(α,β)$ and triples $(k,u,v)$ does not extend to the limit where one branch point lies in the unit circle. To understand why, recall that the coordinates $(k,u,v)$ derive from the map $f$ that sends $α$ to $1$ and $β$ to $k^{-1}$. As a branch point moves towards the unit circle, so does its conjugate-inverse partner. However, in the image under $f$ there is no way to express $α$ and $\cji{α}$ moving together because their corresponding points are $1$ and $-1$ and are fixed. We consider only $β\to\S^1$, but note that this is not a loss of generality, because the labelling of branch points as $α$ and $β$ is arbitrary and is removed when we push back down to $\mathcal{S}(p)$.

We shall then identify curves $Σ(α,ν)$ with points $(α,ν)\in D\times \S^1$ with points $(u,v) \in \R^2 \setminus \{u=v\}$, which we view as boundary points of $\mathcal{A}'(p)$ where $k$ is zero. First, how can one describe the space of $(α,ν)\in D\times \S^1$ where $p$ is fixed? Observe that for any given $ν$,
\[
p = \frac{\abs{1-α}\abs{1-β}}{\abs{1+α}\abs{1+β}} \Leftrightarrow \frac{\abs{1-α}}{\abs{1+α}} = p \frac{\abs{1+ν}}{\abs{1-ν}}.
\]
As we have seen several times now, curves in the unit disc where the ratio of the distances to $1$ and $-1$ are fixed are circles centered on the real axis and perpendicular to the unit circle. The collection of such circles foliate the disc as the ratio changes value between $0$ and infinity. For $ν = \pm 1$, there are no such $α$ in the unit disc. As $ν$ moves from $1$ to $-1$ along the unit circle, the value of $\abs{1+ν} / \abs{1-ν}$ varies from $\infty$ to $0$. In this way, the parameter space is two discs, one where $ν$ is in the upper half of the unit circle and one where it is in the lower half.

\makefigure{A diagram of the space $D \times S^1$ and the subspace of fixed $p$. Points on the inner circle are values of $ν\in\S^1$. To each of these values there is a disc of values $α\in D$, of which a few have been given. Within each disc is an arc, on which $(α,ν)$ gives a fixed value of $p$.}{thesis_graphics_temp/boundary_wheel_disc.png}

The argument of Lemma \ref{lem:change of parameters} is still valid in the limit $k=0$ when $u,v \neq \infty$. However, if $u=\infty$ then that would force $β = f^{-1}(k^{-1}) = f^{-1}(\infty) = f^{-1}(\iu u) = 1$ and we have already seen that there is no corresponding $α$ in $\mathcal{A}_1(p)$. Likewise for $v=\infty$.

PICTURE OF TRIANGLES \todo{}

Recall that the universal cover of $\mathcal{S}(p)$ is $\mathcal{\tilde{S}}(p)$ and may be described as a collection of level sets of a function $\tilde{T}$ on the universal cover $\mathcal{\tilde{C}}(p)$ of $\mathcal{A}(p)$, where
\[
\mathcal{\tilde{C}}(p) =
\{(k,\tilde{u},\tilde{v}) \in (0,1)\times\R\times\R \mid  \tilde{u} < \tilde{v} < \tilde{u} + 2π \},
\]
with the projection map $\mathcal{\tilde{C}}(p) \to \mathcal{A}(p)$ given by
\[
k = k, \quad
u = \tan \frac{\tilde{u}}{2}, \quad
v = \tan \frac{\tilde{v}}{2}.
\]
From the $(u,v)$ description of the boundary space, we can immediately see that the boundary of $\mathcal{\tilde{C}}(p)$ where $k=0$ and neither of $\tilde{u}$ or $\tilde{v}$ lie in $π + 2π\Z$ corresponds to marked curves where $β$ lies in $\S^1 \setminus \{\pm 1\}$. We may extend $\tilde{T}$ to these points by taking the limit as $k\to 0$ to see where the boundary of the level sets lie, and therefore which parts of $\mathcal{M}_1$ have limits as spectral data on a genus zero curve. By \eqref{eqn:tilde T computable}, $\tilde{T}$ may be computed as the sum of a constant and the function $T_0$. But
\[
T_0 = p\int_{γ_-} Θ^2 - \int_{γ_+} Θ^2,
\]
and we have already seen how to take the limit of these integrals. In particular, these integrals have well defined limits when $f(1)$ and $f(-1)$ are remain finite, but these values are eaxctly $\iu u$ and $\iu v$ respectively, which are both in $\iu \R$ on these faces. Recalling equation \eqref{eqn:Teqn}, and the above limits
\begin{align*}
2π\iu \lim_{k\to 0} T_0
&= \lim_{k\to 0} \left\{ 4p \left[ E F(\iu v;k) - K E(\iu v;k) \right] - 4\left[ E F(\iu u;k) - K E(\iu u;k) \right] - 4\iu K \frac{p w(\iu v) + w(\iu u)}{u-v} \right\} \\
&= 0 - 0 - 4\iu \frac{π}{2} \frac{p \sqrt{1+v^2} + \sqrt{1+u^2}}{u-v} \\
\lim_{k\to 0} T_0
&= - \frac{p \sqrt{1+v^2} + \sqrt{1+u^2}}{u-v}.
\end{align*}
It is fruitful to understanding the level sets to understand the limit of this function at the points where $u$ or $v$ is infinite.
\[
\lim_{u \to \pm \infty} T_0(p, 0,u,v) = - \lim_{u \to \pm \infty} \frac{\abs{u}}{u} = - \sign u,
\]
and similarly
\[
\lim_{v \to \pm \infty} T_0(p, 0,u,v) = - \lim_{v \to \pm \infty} \frac{p\abs{u}}{-v} = p\sign v.
\]
And obviously as $u$ and $v$ near one another, the value of $T_0$ becomes unbounded. Thus the boundary of face is given by three values,

\makefigure{A diagram of $\mathcal{\tilde{C}}(p)$ for $k=0$. Values of $\tilde{T}$ indicated with blue.}{thesis_graphics_temp/boundary_tilde_T_values.png}

Note however that If both $u$ and $v$ approach an infinite value simultaneously, which corresponds to a corner point of the above diagram, then the value of $T_0$ depends on the limit of the ratio of $u$ to $v$. In fact, this dependence is monotonic. Let $u = t \sign u$ and $v = rt\sign v$ for $r$ a positive constant and $t$ a positive parameter. Then
\[
\lim_{t \to +\infty} T_0(p, 0,u,v)
= - \lim_{t \to +\infty} \frac{p \sqrt{1+r^2 t^2} + \sqrt{1+t^2}}{t(\sign u - r \sign v)}
= - \frac{p r + 1}{\sign u - r \sign v},
\]
which is monotonic, except for at its vertical asymptote. The vertical asymptote only occurs when $r = \sign u / \sign v = 1$, ie $u=v$. But this is excluded from the domain. Thus the values of $T_0$ in the corners covers the range of values between the values on the edges. Together with the fact that $T_0$ has no extrema on the faces, we can use this observation as a hueristic to describe the level sets on each face.

\makefigure{The level sets of $\tilde{T}$ on part of $\mathcal{\tilde{C}}(p)$ for $k=0$ }{thesis_graphics_temp/boundary_level_sets.png}
\todo{only $p>1$ version shown}

In particular, there are no closed level sets and the arcs foliate the face. Therefore, when we translate this in terms of the pair of discs $α\in D$ and $ν$ in the upper or lowever unit circle, the arcs must start and end on the boundary of the disc and foliate the disc. Thus the corners must represent $α$ close to the unit disc. To deduce where, we make the following observation. Let $ν = \exp \iu b$, for $b \in (-π, π)$, and $α = (1-ε) \exp \iu a$, for $ε$ small and $a \in (-π, π]$. Then
\[
\frac{\abs{1-α}} {\abs{1+α}} = \frac{\abs{1-e^{\iu a}}} {\abs{1+e^{\iu a}}} + O(ε) = \abs{\tan \frac{a}{2}} + O(ε),
\]
so that
\[
p = \abs{\tan \frac{a}{2}}\abs{\tan \frac{b}{2}} + O(ε).
\]
Let $b_0$ be the special angle for which $\tan(b_0/2) = \sqrt{p}$. When $\abs{b} = b_0$, then $\abs{a} \approx b_0$ also. Also, when $\abs{b} > b_0$, then necessarily for sufficently small $ε$ we must have $\abs{a} < b_0$. This divides the unit circle into four segments, divided by the points $1$, $-1$, $e^{\iu b_0}$ and $e^{-\iu b_0}$. We label these segements as in the following diagram

\makefigure{The four segments are label, anticlockwise from $1$, $\S^1_{UR}$, $\S^1_{UL}$, $\S^1_{LL}$ and $\S^1_{LR}$}{thesis_graphics_temp/boundary_circle_segments.png}

using the mnenomic that for the first letter $U$ stands for the upper unit circle, $L$ for the lower unit circle, and for the second letter $L$ stands for points left of $e^{\iu b_0}$ and $R$ for points to its right. Observe that $μ = e^{\iu a} + O(ε)$ also. Using the geometric properties of the function $f$, that it sends the unit circle to the imaginary axis (specifically $μ$ to $0$, $ν$ to $\infty$, $1$ to $\iu u$ and $-1$ to $\iu v$) and it send the unit disc to the right half plane, one may compute the following correspondences between $μ$ and $ν$ in the four segements of the unit circle and $(u,v)$ in $\R^2$, for $ε$ close to $0$.

\includegraphics[width=0.9\textwidth]{thesis_graphics_temp/boundary_corner_table.png}

From this table we may translate the level set diagram from the faces to the discs

DIAGRAM.

Note the swirliness around the special points $α = β = e^{\iu b_0}$ and $α = β = e^{-\iu b_0}$. These points correspond to the intersection of the diagonal from the previous section and the exterior boundary of this section. In particular, the spirals that were observed in that other case are also present here, as they must be.

TODO\todo{}
Argument about why two discs are sufficent.
Show that the limit of $Ψ,\tilde{Ψ}$ are well defined because the sign ambiguity cancels? It does for $Ψ$ it seems.
