%!TEX root = thesis.tex

\section*{Abstract}
In this thesis we investigate the topology of the moduli space of spectral data of harmonic maps from the torus into the 3-sphere.
Harmonic tori in the 3-sphere are in bijective correspondence with their spectral data, which consists of an algebraic curve (called a spectral curve), a pair of differentials, and a line bundle. Deformations of the spectral data correspond to deformations of the tori themselves. There are two classes of deformations; isospectral deformations vary only the line bundle, whereas non-isospectral deformations change the spectral curve itself.
This thesis explores the latter.
We use the theory of Whitham deformations to show that the moduli space of spectral data is a surface. For spectral curves of genus zero and one, the global topology of the moduli space is treated through explicit parameterisation. We enumerate the path connected components and show them to be simply connected, and prove that the moduli space of these adjacent spectral genera connect to one another in an appropriate limit.
