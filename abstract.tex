%!TEX root = thesis.tex

\section*{Abstract}
% TODO
Harmonic maps of the torus into the 3-sphere can be understood by means of their spectral curves, a pair of differentials and a line bundle. Deformations of these spectral data correspond to deformations of the maps themselves. Isospectral deformations vary only the line bundle whereas non-isospectral deformations change the spectral curve itself. The parameters describing these spectral data provide an ambient space to discuss variations, and within this parameter space the moduli space of harmonic tori exists as a subspace. We use the theory of Whitham deformations to show that the moduli space is a surface. For spectral curves of genus zero and one, the global topology of the moduli space is investigated directly. We enumerate the path connected components and show them to be simply connected surfaces. We prove that the moduli space of these adjacent spectral genera may be seen to connect to one another after taking an appropriate limit and normalising the resulting singular curve.
