%!TEX root = thesis.tex

\section*{Abstract}
% TODO
Harmonic maps of the torus into the 3-sphere can be understood by means of their spectral curves, a pair of differentials and a line bundle. Deformations of these spectral data correspond to deformations of the maps themselves. Isospectral deformations vary only the line bundle whereas non-isospectral deformations change the spectral curve itself. This thesis explores the latter. We use the theory of Whitham deformations to show that the moduli space of spectral data is a surface. For spectral curves of genus zero and one, the global topology of the moduli space is investigated directly. We enumerate the path connected components and show them to be simply connected. We prove that the moduli space of these adjacent spectral genera may be seen to connect to one another in an appropriate limit.
