%!TEX root = thesis.tex

\chapter*{Statement of Originality}

This is to certify that to the best of my knowledge, the content of this thesis is my own work. This thesis has not been submitted for any degree or other purposes.
\\
\includegraphics[width=4cm]{thesis_graphics/signature.png}
\\
Ross Ogilvie

\cleardoublepage

\chapter*{Abstract}
In this thesis we investigate the topology of the moduli space of spectral data of harmonic maps from the torus into the 3-sphere.
Harmonic tori in the 3-sphere are in bijective correspondence with their spectral data, which consists of an algebraic curve (called a spectral curve), a pair of differentials, and a line bundle. Deformations of the spectral data correspond to deformations of the tori themselves. There are two classes of deformations; isospectral deformations vary only the line bundle, whereas non-isospectral deformations change the spectral curve itself.
This thesis explores the latter.
We use the theory of Whitham deformations to show that the moduli space of spectral data is a surface. For spectral curves of genus zero and one, the global topology of the moduli space is treated through explicit parametrisation. We enumerate the path connected components and show them to be simply connected, and prove that the moduli space of these adjacent spectral genera connect to one another in an appropriate limit.

\cleardoublepage

\chapter*{Acknowledgments}

This thesis owes its existence to the efforts of many people. First and foremost, the advice and expertise of my supervisor Dr Emma Carberry has been exceptional.
Too many times I have come to Emma stuck on a problem that I can barely articulate, only to have her elucidate and resolve it in several different ways.
Her love of geometry is inspirational to anyone who has the good fortune to hear her speak of it, as I did in her differential geometry course. Without Emma's example I would not have started this project, and without Emma's patience and guidance I would not have able to finish it.

A great debt is owed to Prof John Rice for time that he gave so generously to me. I cannot count the number of times I would leave one of our meetings absolutely convinced that John was wrong, only to discover a day or two later that not only was his solution correct, but that it swept aside issues I had not even yet seen.

The School of Mathematics and Statistics at the University of Sydney has been my home for almost a decade. Its faculty have been mentors to me, kind, wise and encouraging. I would like to thank in particular Dr Zhou Zhang and Dr Daniel Daners for their support of projects throughout my undergraduate degree.
My fellow students have been a source sanity; a problem shared is a problem halved.
To this end, thank you to my office-mates whose note-passing on the whiteboard brought a smile to my face daily, and to Louis Bhim, who I would ask kindly to stop subtly rearranging my desk.
I would also like to thank the School for its generosity that allowed me to attend conferences near and far.

I thank my non-mathematical friends for their kind indulgence in listening to me ramble on about my work.

And finally, to my parents, sister, and wider family, for the unwavering support I have received I am deeply grateful. Your love has sustained me through it all.
