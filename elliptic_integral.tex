%!TEX root = thesis_single.tex

\section{Elliptic Integrals}
\label{sec:Elliptic Integrals}

\subsection{Definitions and Periods}
Elliptic integrals are ones that arise as the integrals of differentials on elliptic curves (curves of genus one). There are several sets of standard elliptic integrals, with the idea being that every elliptic integral can be reduced to a combination of the standard ones in a particular set. The most prevalent set, the Jacobi elliptic integrals, is used here. Note however that there are several variations in notation about these integrals; the main difference is the use of the modulus $k$ versus the parameter $m = k^2$. We will us the former. There are three standard integrals, and they come in {\it incomplete} and {\it complete} varieties. The incomplete and complete elliptic integral of second kind are traditionally both denoted $E$ to much confusion. I have therefore elected to denote incomplete elliptic integral with a tilde: $\tilde E$ compared to $E$. They are defined as
\[
\tilde E = \tilde E(z;k) = \int_0^z \sqrt{\frac{1-k^2 t^2}{1-t^2}} \;dt = \int_0^z \frac{1-k^2 t^2}{\sqrt{(1-t^2)(1-k^2 t^2)}}\;dt,
\qquad E(k) = \tilde E(1;k)
\]
The incomplete and complete elliptic integrals of first kind shall be denoted $\tilde F$ and $K$ respectively and are defined as
\[
\tilde F = \tilde F(z;k) = \int_0^z \frac{dt}{\sqrt{(1-t^2)(1-k^2 t^2)}},
\qquad K = K(k) = \tilde F(1;k).
\]
Note that a tilde has also been added to $\tilde F$ for consistency and solidarity, even though there is not a naming conflict in the usual notation of integrals of the first kind. We will not have need of integrals of the third kind. Elliptic integrals of the first kind integrals are of holomorphic forms on an elliptic curve (a torus), so one can see from the form above that the curve is given by the equation
\[
y^2 = (1-t^2)(1-k^2 t^2).
\]
This highly symmetric form is known as {\it Legendre form}. The branch cuts are usually made along $[1,k^{-1}]$ and $[-1,-k^{-1}]$. By M\"obius transformation (or by Landen's transformations), when $k$ is real one can arrange for $0< k < 1$, where the extremes of $k=0$ and $k=1$ reduces things to regular circle trigonomtry. The period of this curve around the branch points $-1$ and $1$ is obviously related to the definitions above. For example, this period of the holomorphic differential is precisely $4K$, which earns $K$ the nickname of `quarter-period'.

\begin{center}
\includegraphics{{torus_lagrange.sk}.pdf}
\end{center}
\todo{enable picture here}

In the picture, the red cycle is the real $A$-cycle and has the clockwise orientation as shown, whereas the blue cycle is the imaginary $B$-cycle and has orientation from left to right as shown. $B$ is divided into two pieces, $B^+$ on which $y$ is in $i\R$ (the occluded piece) and $B^-$ on which $y$ is in $-i\R$ (the front piece).

But what about the other period, the one about $1$ and $k^{-1}$? It's computation involves a trick called {\it Jacobi's imaginary transformation}, and makes use of a quadratic substitution \cite{Whittaker2000}. Let $t^2 = (1-k'^2 s^2)^{-1}$, where $k'^2 = 1 - k^2$. Then
\begin{align*}
dt &= \frac{k'^2 s}{(1-k'^2 s^2)^{3/2}}\;ds, \\
\sqrt{t^2 - 1} &= \frac{k's}{\sqrt{1-k'^2 s^2}} \\
\sqrt{1 - k^2 t^2} &= \frac{k' \sqrt{1-s^2}}{\sqrt{1-k'^2 s^2}} \\
y &= \pm \frac{i k'^2 s \sqrt {1-s^2}}{1-k'^2 s^2}
\end{align*}
so that
\begin{align*}
\int_B \frac{dt}{y}
&= \int_{B^+} \frac{dt}{y^+} + \int_{B^-} \frac{dt}{y^-} \\
&= \int_1^0 (-\iu) \frac{ds}{\sqrt{(1-s^2)(1-k'^2 s^2)}} + \int_0^1 \iu \frac{ds}{\sqrt{(1-s^2)(1-k'^2 s^2)}}\\
&= 2\iu K(k')
\end{align*}
Often $K(k')$ is abreviated to simply $K'$, and this should not be confused with the derivative of $K$. $k'$ is called the complementary modulus. A similar substitution is needed for $E$. Let this time $k^2 t^2 = 1-k'^2 s^2$, then
\begin{align*}
dt &= -\frac{k'^2}{k} \frac{s}{\sqrt{1-k'^2 s^2}}\;ds,\\
\sqrt{t^2 - 1} &= \frac{k'}{k}\sqrt{1-s^2} \\
\sqrt{1 - k^2 t^2} &= k's\\
y &= \pm \iu \frac{k'^2}{k}s\sqrt{1-s^2}
\end{align*}
\begin{align*}
\int_B \frac{dt}{y}(1-k^2t^2)
&= 2\iu \int_0^1 \frac{k'^2 s^2}{\sqrt{(1-s^2)(1-k'^2 s^2)}}\;ds \\
&= 2\iu \int_0^1 \frac{1-(1-k'^2 s^2)}{\sqrt{(1-s^2)(1-k'^2 s^2)}}\;ds \\
&= 2\iu \int_0^1 \frac{ds}{\sqrt{(1-s^2)(1-k'^2 s^2)}} - 2\iu \int_0^1 \sqrt{\frac{1-k'^2 s^2}{1-s^2}}\;ds\\
&= 2iK(k') - 2iE(k').
\end{align*}
Again, we introduce the notation that $E' = E(k')$.

\subsection{Derivatives}
\label{sub:Derivatives}
The derivatives of complete elliptic integrals are well known. The derivatives of the incomplete integrals with respect to the variable $z$ are trivial because they are simply parameter integrals. The derivatives of the incomplete integrals with respect to $k$ are calculated below, guided by the knowledge of the derivatives of the complete integrals.
\begin{align}
\frac{d}{dk}K &= \frac{1}{k(1-k^2)}E - \frac{1}{k}K, \\
\frac{d}{dk}E &= \frac{1}{k}E - \frac{1}{k} K.
\end{align}
\begin{align}
\Partial{}{z}\tilde F &= \frac{1}{\sqrt{(1-z^2)(1-k^2 z^2)}}, \\
\Partial{}{z}\tilde E &= \sqrt{\frac{1-k^2 z^2}{1-z^2}}.
\end{align}
\begin{align}
g &:= t\sqrt{\frac{1-t^2}{1-k^2t^2}} \\
g_t &= \frac{1-2t^2 + k^2 t^4}{\sqrt{1-t^2}\sqrt{1-k^2t^2}^3} \\
\Partial{}{k}\bra{\frac{1}{\sqrt{(1-t^2)(1-k^2 t^2)}}} -& \frac{1}{k(1-k^2)}\sqrt{\frac{1-k^2 t^2}{1-t^2}} + \frac{1}{k}\frac{1}{\sqrt{(1-t^2)(1-k^2 t^2)}} \\
&= \frac{1}{\sqrt{1-t^2}\sqrt{1-k^2t^2}^3}\frac{1}{k(1-k^2)}\bra{ k^2t^2(1-k^2) - (1-k^2t^2)^2 + (1-k^2)(1-k^2t^2) } \\
&= \frac{1}{\sqrt{1-t^2}\sqrt{1-k^2t^2}^3}\frac{1}{k(1-k^2)}\bra{ -k^2 + 2k^2t^2 -k^4t^4} \\
&= \frac{-k}{1-k^2}g_t \\
\Partial{\tilde F}{k} - \frac{1}{k(1-k^2)}\tilde E + \frac{1}{k} &= \frac{-k}{1-k^2}\int_0^z g_t\; dt \\
\Partial{\tilde F}{k} &= \frac{1}{k(1-k^2)}\tilde E - \frac{1}{k}\tilde F - \frac{k}{1-k^2}z\sqrt{\frac{1-z^2}{1-k^2z^2}}.
\end{align}
\begin{align}
k\Partial{}{k}\bra{\sqrt{\frac{1-k^2t^2}{1-t^2}}} -& \sqrt{\frac{1-k^2t^2}{1-t^2}} + \frac{1}{\sqrt{(1-t^2)(1-k^2 t^2)}} \\
&= \frac{1}{\sqrt{(1-t^2)(1-k^2 t^2)}}\bra{-k^2t^2 - (1-k^2t^2) + 1} \\
&= 0\\
\Partial{}{k}\tilde E &= \frac{1}{k}\tilde E - \frac{1}{k}\tilde F
\end{align}

\subsection{Inequalities and Limits}
\label{sub:Inequalities}
The inequalities we use are found in \cite{Anderson}. Their inequality (2) contains only $K$ and confines its behaviour to within a strip-like region of known width.
\[
\ln 4 \leq K + \frac{1}{2}\ln (1-k^2) \leq \frac{π}{2}.
\labelthis{K_bound}
\]
In particular, as $k \to 1$, the complete elliptic integral of the first kind has a logarithmic singularity. A stronger and more precise statement is
\[
\lim_{k \to 1} \left\{ K + \frac{1}{2}\ln(1-k) \right\} = \frac{3}{2}\ln 2.
\]
A similar result, (1), ties the integral of the second kind to that of the first type:
\[
\frac{π}{4}k^2 \leq E - (1-k^2)K \leq k^2.
\labelthis{E_bound}
\]
Cruder bounds that are sometimes useful to clear out complicated expression come from noting that $K$ is an increasing function and $E$ is a decreasing one. The exact values
\[
K(0) = E(0) = \frac{π}{2}, \;\; E(1) = 1
\]
give a lower bound for $K$ and a narrow range of values for $E$.

For the incomplete integrals, we need only investigate their properties on the imaginary axis. Both $\tilde{F}$ and $\tilde{E}$ take purely imaginary values on the imaginary axis, and we shall see that $\tilde{E}$ has a pole at infinity, so let us consider the functions $-\iu \tilde{F}(\iu x; k)$ and $-\iu \tilde{E}(\iu x; k) - k x$, both real valued functions of $x \in \R$ and $k\in (0,1)$.

First note that these are both increasing functions of $x$. In the case of
\[
-\iu \tilde{F}(\iu x;k) = \int_0^x \frac{dt}{\sqrt{(1+t^2)(1+k^2t^2)}}
\]
this is obvious. For the second function,
\begin{align*}
k <& 1 \\
k^2 + k^2 t^2 <& 1 + k^2t^2 \\
k \sqrt{1 + t^2} <& \sqrt{1 + k^2t^2},
\end{align*}
from which is follows that
\begin{align*}
-\iu \tilde{E}(\iu x; k) - k x
&= \int_0^x \sqrt{\frac{1+k^2t^2}{1+t^2}} - k \;\;dt\\
&= \int_0^x \frac{\sqrt{1+k^2t^2} - k\sqrt{1+t^2}}{\sqrt{1+t^2}}\;dt
\end{align*}
is increasing as well.

Naturally, one next asks to what value do these functions increase. We shall compute this by applying the standard technique of complex analysis of using a closed semicircular contour. Let the contour be composed of an interval along the imaginary axis from $-\iu R$ to $\iu R$, and let $C_R$ be the semicircular arc from $\iu R$ to $R$ back down to $-\iu R$. This contour is homologous to a standard period around $[1,k^{-1}]$ and so is a fixed quantity. The contribution coming from the semicircular arc is neglible in the limit, as can be seen below.
\begin{align*}
\abs{\int_{C_R}\frac{dz}{\sqrt{(1-z^2) (1-k^2z^2)}}}
&= \abs{\int_{π/2}^{-π/2} \frac{Re^{\iu θ} dθ}{\sqrt{(1-R^2e^{2\iu θ}) (1-k^2R^2e^{2\iu θ})}}} \\
&\leq \int_{-π/2}^{π/2} \frac{R}{\sqrt{(1-R^2) (1-k^2R^2)}} \;dθ \\
&= \frac{π R}{\sqrt{(1-R^2) (1-k^2R^2)}} \\
&\to 0.
\end{align*}
And since $\tilde{F}(z;k)$ is an odd function of $z$, we can conclude that
\begin{align*}
2K'
&= -\iu\; \lim_{R\to\infty} \oint \frac{dz}{\sqrt{(1-z^2) (1-k^2z^2)}} \\
&= -\iu\; \lim_{R\to\infty} \bra{\int_{-\iu R}^{\iu R} + \int_{C_R}}  \frac{dz}{\sqrt{(1-z^2) (1-k^2z^2)}} \\
&= -2\iu\; \lim_{R\to\infty}  \tilde{F}(\iu R;k),
\end{align*}
so
\[
\lim_{x\to\infty} -\iu \tilde{F}(\iu x;k) = K' ,
\qquad \lim_{x\to -\infty} -\iu \tilde F(\iu x; k) = - K' .
\]
The analysis of the other function proceeds much the same, however there is some delicate work to make the poles cancel. Using the same contour as before, we again show that the contribution from the semicircular arc is vanishing.
\begin{align*}
\abs{\int_{C_R}\sqrt{\frac{1-k^2z^2}{1-z^2}} - k \;\;dz}
&= \abs{\int_{C_R}\frac{ \sqrt{1-k^2z^2} - k \sqrt{1-z^2}}{\sqrt{1-z^2}} \;dz} \\
&= \abs{\int_{π/2}^{-π/2} \frac{ \sqrt{1-k^2R^2e^{2\iu θ}} - k \sqrt{1-R^2e^{2\iu θ}}}{\sqrt{1-R^2e^{2\iu θ}}} \;Re^{\iu θ}\;dθ} \\
&\leq \int_{-π/2}^{π/2} \frac{ \sqrt{1+k^2R^2} - k \sqrt{R^2-1}}{\sqrt{R^2-1}} \;R\;dθ \\
&= π R\frac{ \sqrt{1+k^2R^2} - k \sqrt{R^2-1}}{\sqrt{R^2-1}} \\
&= π R\frac{ (1+k^2R^2) - k^2 (R^2-1)}{\sqrt{R^2-1}(\sqrt{1+k^2R^2} + k \sqrt{R^2-1})} \\
&= π R\frac{ 1 + k^2 }{\sqrt{R^2-1}(\sqrt{1+k^2R^2} + k \sqrt{R^2-1})} \\
&\to 0,
\end{align*}
as $R \to \infty$. As before, we are dealing with an odd function of $x$ and hence
\begin{align}
\lim_{x\to+\infty} [-\iu \tilde E(\iu x; k) - k x] = K' - E',
\qquad \lim_{x\to-\infty} [-\iu \tilde E(\iu x; k) - k x ]= - \bra{K' - E'}.
\end{align}

A consequence of these limits is that we have bound the functions independently of $x$.
\begin{align}
-K' \leq -\iu &\tilde{F}(\iu x;k) \leq K' \\
- (K' - E') \leq -\iu &\tilde E(\iu x; k) - k x \leq K' - E'. \label{eqn:tildeEatInf}
\end{align}
It can also be useful to bound their growth independently of $k$. As the integrands are monotone functions of $k$, assuming $x>0$ one has
\[
\atan {x} = \int_0 ^x \frac{dt}{1+t^2} \leq -\iu \tilde{F}(\iu x;k) \leq \int_0 ^x \frac{dt}{\sqrt{1+t^2}} = \asinh x
\]
and
\[
0 \leq -\iu \tilde E(\iu x; k) - k x \leq
\int_0^x \frac{(1+kt) - kt}{\sqrt{1+t^2}}\;dt = \asinh x.
\]
It may seem disappointing that we only have a lower bound of $0$ for the second function, but by considering that function at $k=1$ we see that it is identically zero and so no better bound, independent of $k$, is possible. If one is willing to consider non uniform bounds, then the observation that both integrands (and so both functions) are decreasing functions of $k$ allows one to bound a range of $k$ by the value at infimum of the range.

These inequalities also serve to show that the integrals are dominated by functions independent of $k$, so one can compute the limits as $k \to 1$ (for any value of $x$) and $k \to 0$ (for finite values of $x$, as $\asinh x$ is not finite for $x=\infty$). Explicitly,
\begin{align}
\lim_{k\to 1} -\iu \tilde{F}(\iu x; k) &= \atan {x} \\
\lim_{k\to 1} -\iu \tilde{E}(\iu x; k) &= x \\
\lim_{k\to 0} -\iu \tilde{F}(\iu x; k) &= \asinh {x} \\
\lim_{k\to 0} -\iu \tilde{E}(\iu x; k) &= \asinh {x}.
\end{align}

\subsection{Analyticity}
\label{sub:Analyticity}

RELEVANCE ON NEXT BIT IF I GO WITH A UNIVERSAL COVER APPROACH?\todo{this} Perhaps should rewrite to be about universal cover instead.

It is a standard fact that the elliptic integrals are analytic functions on the plane away from the branch points. The relevant theorem is the Cauchy–Kowalevski theorem (\cite{Evans1998}, 4.6.3), which in generality provides the analytic solutions of PDEs, but in our situation yields that parameter integrals of analytic functions are analytic. However, because of the periods of integrals, it is not possible to extend $\tilde F, \tilde E$ to the whole complex plane. Indeed we have seen that the two limits to infinity along the imaginary axis are different. Consider then the following function
\[
f(y, k) :=
\begin{cases}
\tilde F(\iu y^{-1}; k)             & \text{ for } y > 0 \\
\iu K'                              & \text{ for } y = 0 \\
2\iu K' + \tilde F(\iu y^{-1}; k)   & \text{ for } y < 0
\end{cases}
\]
We claim that it is analytic on $(y,k) \in \R\times (0,1)$. To see this, first transform the integrals so that their base point is at $y=0$ rather than $y=\infty$. For $y>0$
\begin{align}
\tilde F(\iu y^{-1}; k)
&= \iu \int_0^{y^{-1}} \frac{dt}{\sqrt{(1+t^2)(1+k^2t^2)}} \\
&= -\iu \int_{+\infty}^{y} \frac{ds}{\sqrt{(1+s^2)(s^2+k^2)}} \\
&= \iu \left(\int_0^{+\infty} - \int_0^y \right) \frac{ds}{\sqrt{(1+s^2)(s^2+k^2)}} \\
&= \iu K' - \int_0^y \frac{ds}{\sqrt{(1+s^2)(s^2+k^2)}},
\end{align}
and for $y < 0$
\begin{align}
2\iu K' + \tilde F(\iu y^{-1}; k)
&= 2\iu K' + \iu \int_y^{-\infty} \frac{ds}{\sqrt{(1+s^2)(s^2+k^2)}} \\
&= \iu K' - \int_0^y \frac{ds}{\sqrt{(1+s^2)(s^2+k^2)}}.
\end{align}
So we have arived at a single formula for $f$, namely
\[
f = \iu K' - \int_0^y \frac{ds}{\sqrt{(1+s^2)(s^2+k^2)}},
\]
and since the integrand is analytic, the integral is too and we are done. In an entirely similar way, define the function
\[
g(y;k) :=
\begin{cases}
\tilde E(\iu y^{-1}; k) - \iu ky^{-1}             & \text{ for } y > 0 \\
\iu (K'-E')                                       & \text{ for } y = 0 \\
2\iu (K'-E') + \tilde E(\iu y^{-1}; k)  - \iu ky^{-1}  & \text{ for } y < 0
\end{cases}
\]
After transforming the integrals in the same way, we find ourselves faced with the formula, for all $y$,
\[
g = \iu (K'-E') - \iu \int_0^y s^{-2}\bra{ \sqrt{\frac{s^2 + k^2}{1 + s^2}} - k }\;ds.
\]
The part of the integrand in the bracket is analytic, but because of the $s^{-2}$, the integrand as a whole may not be analytic at $s=0$. Fortunately though, it is. To see this, use the square root estimate
\[
1 \leq \sqrt {1 + x^2} \leq 1 + \frac{1}{2}x^2,
\]
To obtain the bound
\[
-\frac{1}{2}k s^2 \leq \sqrt{k^2 + s^2} - k \sqrt{1+s^2} \leq \frac{1}{2}\frac{1}{k} s^2,
\]
which shows that the part in the bracket vanishes to order $s^2$ and so the integrand, and therefore $g$, is analytic.

\todo{remove me}
