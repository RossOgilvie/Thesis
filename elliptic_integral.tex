%!TEX root = thesis.tex

\chapter{Elliptic Integrals}
\label{chp:Elliptic Integrals}

%%%%%%%%% I don't want these subsections in the table of contents, because there are too many of them and they're not really important. There is a command at the bottom that turns the ToC back on again
\stoptocentries

The purpose of this appendix is to provide background information about elliptic integrals. It contains their definitions and basic properties, as well as some results that are used in the body of the thesis. It is not a comprehensive treatment, specifically elliptic integrals of the third kind are not treated, and incomplete integrals are only treated for an imaginary argument.

\section{Definitions and Periods}
\label{sec:Elliptic Defs}
Elliptic integrals are the integrals of differentials on elliptic curves (curves of genus one). The descriptor `elliptic' comes from the problem of finding the arc length of an ellipse, and elliptic integrals are in turn the origin of the term elliptic curve. This topic has a long history, at least two hundred years, and consequently there are several competing conventions. In this thesis we shall use the Legendre elliptic integrals, sometimes also called Jacobi elliptic integrals, of a modulus $k$. A history of the development of elliptic integral, first by Legendre and taken up by Jacobi and Abel, may be found in \cite{Bottazzini2013}. Every elliptic integral may be reduced to a combination of Legendre elliptic integrals \cite{armitage2006elliptic,Hancock1910}.

\begin{defn}
The incomplete elliptic integral of the first kind is defined to be
\[
F(z;k) = \int_0^z \frac{dt}{\sqrt{(1-t^2)(1-k^2 t^2)}}.
\]
\end{defn}

Of particular interest are \emph{complete} elliptic integrals, where $z=1$. The first complete integral is donated denoted $K(k)$. Often the modulus is understood and it is written simply as $K$. The complete integrals are used to compute the periods of the elliptic curve
\[
w^2 = (1-z^2)(1-k^2 z^2).
\]
Every elliptic curve may be brought to this highly symmetric form, known as \emph{Legendre's form}. If the curve carries a reality structure (if it admits an antiholomorphic involution) then $k$ must be real and by a further Möbius transformation one can arrange for $0 < k < 1$. We may choose to take branch cuts along $[1,k^{-1}]$ and $[-1,-k^{-1}]$.

% \makefigure{The $z$-plane with branch cuts along $[1,k^{-1}]$ and $[-1,-k^{-1}]$. The $A$ period is in red and $B$ period is in blue.}{thesis_graphics_temp/branch_cuts.png}
\makefigure{The $z$-plane with branch cuts along $[1,k^{-1}]$ and $[-1,-k^{-1}]$. The $A$ period is in red and $B$ period is in blue. \label{fig:standard periods}}{tikz/branch_cuts}

Consider the $A$-period, the anticlockwise loop around the branch points $-1$ and $1$. By the symmetry $w \mapsto -w$, the integral of the holomorphic differential $dz / w $ around this loop is twice the integral from $-1$ to $1$, which itself is twice the integral from $0$ to $1$. Thus this period of the holomorphic differential is $4K$, which earns $K$ the nickname of `quarter-period'.

\makefigure{The torus $w^2 = (1-z^2)(1-k^2 z^2)$ with $A$-period in red (in the centre) and $B$-period in blue.  The upper and lower halves of the torus correspond to the two sheets of $\C$.}{{torus_lagrange.sk}.pdf}

To complete a basis of homology, choose the other period, $B$, to be a clockwise loop around $1$ and $k^{-1}$. To compute the integral around this loop, one employs a trick called \emph{Jacobi's imaginary transformation} that makes the following quadratic substitution \cite{Whittaker2000}. Let $z^{-2} = 1-k'^2 s^2$, where $k'^2 = 1 - k^2$. Then
\begin{align*}
dz &= \frac{k'^2 s}{(1-k'^2 s^2)^{3/2}}\;ds, \\
w &= \pm \frac{i k'^2 s \sqrt {1-s^2}}{1-k'^2 s^2}
\end{align*}
so that
\begin{align*}
\int_B \frac{dz}{w}
&= \int_1^0 \frac{ds}{(+\iu)\sqrt{(1-s^2)(1-k'^2 s^2)}} + \int_0^1 \frac{ds}{(-\iu)\sqrt{(1-s^2)(1-k'^2 s^2)}}\\
&= 2\iu K(k')
\end{align*}
Often $K(k')$ is abbreviated to simply $K'$, and this should not be confused with the derivative of $K$. This auxillary parameter $k'$ is called the complementary modulus.


We need also in this thesis to consider periods of differentials of the second kind. The standard differential of the second kind is characterised by having a double pole at infinity with no residue. Analogously to the integrals of the first kind,

\begin{defn}
The incomplete elliptic integral of the second kind is
\[
E(z;k) = \int_0^z \sqrt{\frac{1-k^2 t^2}{1-t^2}} \;dt = \int_0^z \frac{1-k^2 t^2}{\sqrt{(1-t^2)(1-k^2 t^2)}}\;dt,
\]
and $E(k) = E(1;k)$ is the complete integral.
\end{defn}

Unfortunately, it is standard usage to denote both complete and incomplete integrals by $E$, so to avoid confusion we will always show incomplete integrals with two arguments, while we may omit the argument for complete integrals.

The $A$-period of this differential is $4E$. To compute the $B$-period requires another clever substitution. Let this time $k^2 z^2 = 1-k'^2 s^2$. Then
\begin{align*}
dz &= -\frac{k'^2}{k} \frac{s}{\sqrt{1-k'^2 s^2}}\;ds,\\
w &= \pm \iu \frac{k'^2}{k}s\sqrt{1-s^2}
\end{align*}
so that
\begin{align*}
\int_B (1-k^2z^2) \frac{dz}{w}
&= 2\iu \int_0^1 \frac{k'^2 s^2}{\sqrt{(1-s^2)(1-k'^2 s^2)}}\;ds \\
&= 2\iu \int_0^1 \frac{ds}{\sqrt{(1-s^2)(1-k'^2 s^2)}} - 2\iu \int_0^1 \sqrt{\frac{1-k'^2 s^2}{1-s^2}}\;ds\\
&= 2i \left[ K(k') - 2iE(k') \right].
\end{align*}
Again, we introduce the notation that $E' = E(k')$.





















\section{Inequalities and Limits}
\label{sec:Inequalities}
Throughout this thesis, we use inequalities to bound the behaviour of elliptic integrals. One can obtain a crude bound from the fact that $K$ is an increasing function and $E$ is a decreasing one. The exact values
\[
K(0) = E(0) = \frac{π}{2}, \;\; E(1) = 1,
\labelthis{eqn:K_E_values}
\]
therefore give a lower bound for $K$ and a narrow range of values for $E$. More precise inequalities may be found in \cite{Anderson}. Their inequality (2) contains only $K$ and confines its behaviour to within a strip-like region of known width.
\[
\ln 4 \leq K + \frac{1}{2}\ln (1-k^2) \leq \frac{π}{2}.
\labelthis{eqn:K_bound}
\]

\begin{SCfigure}
\includegraphics[width=0.6\textwidth]{thesis_graphics/K_inequality.png}
\caption{
\leavevmode\\\begin{minipage}{\linewidth}
Plot of $K$ (black) and its upper and lower bounding approximates. The blue lower bound is
\[
\ln 4 - \frac{1}{2}\ln (1-k^2),
\]
and the red upper bound is
\[
\frac{π}{2} - \frac{1}{2}\ln (1-k^2).
\]
\end{minipage}
}
\end{SCfigure}


In particular, as $k \to 1$, the complete elliptic integral of the first kind has a logarithmic singularity. A stronger and more precise statement is
\[
\lim_{k \to 1} \left[ K + \frac{1}{2}\ln(1-k) \right] = \frac{3}{2}\ln 2.
\]
A similar result, inequality (1) in \cite{Anderson}, ties the integral of the second kind to that of the first kind,
\[
\frac{π}{4}k^2 \leq E - (1-k^2)K \leq k^2.
\labelthis{eqn:E_bound}
\]

\begin{SCfigure}
\includegraphics[width=0.6\textwidth]{thesis_graphics/E_inequality.png}
\caption{
\leavevmode\\\begin{minipage}{\linewidth}
Plot of $E$ (black) and its upper and lower bounding approximates. The blue lower bound is
\[
\frac{π}{4} k^2 + (1-k^2)K(k),
\]
and the red upper bound is
\[
k^2 + (1-k^2)K(k).
\]
\end{minipage}
}
\end{SCfigure}



For the incomplete integrals, in this thesis we need only investigate their properties on the imaginary axis. Both $F(z;k)$ and $E(z;k)$ take purely imaginary values on the imaginary axis. We shall see that $E(z;k)$ has a pole at infinity, so we shall concentrate instead on $E(z;k) - kz$ which is well behaved everywhere. We make the following definitions.

\begin{defn}
\label{defn:F0 and E0}
Let
\[
F_0(x) = F_0 (x ; k) := \Imag F(\iu x; k)
= \int_0^x \frac{dt}{\sqrt{(1+t^2)(1+k^2 t^2)}},
\]
and
\[
E_0(x) = E_0 (x ; k) := \Imag E(\iu x; k) - kx
= \int_0^x \sqrt{\frac{1+k^2t^2}{1+t^2}} - k \;dt,
\]
where the $k$ is implicit if it is omitted.
\end{defn}

Both are real valued functions of $x \in \R$ and $k\in (0,1)$. We begin with the simple observation that both $F_0$ and $E_0$ are odd functions of $x$. Next note that both functions are increasing functions of $x$. In the case of $F_0$ this is obvious as the integrand is positive. For the other function, $k \sqrt{1 + t^2} < \sqrt{1 + k^2t^2}$ from which it follows that
\[
E_0(x)
= \int_0^x \frac{\sqrt{1+k^2t^2} - k\sqrt{1+t^2}}{\sqrt{1+t^2}}\;dt
\]
is increasing as well. For increasing functions it is natural to wonder whether they increase towards a limit. In our case they do, and we shall compute the value of the limit by a standard technique of complex analysis: integration around a closed semicircular contour.

\begin{lem}
\label{lem:F0 and E0 bounds}
The functions $F_0$ and $E_0$ are increasing functions, with the following limits as $x$ tends to infinity.
\[
\lim_{x\to+\infty} F_0(x; k) = K',
\qquad
\lim_{x\to+\infty} E_0(x; k) = K' - E'.
\]

\begin{proof}
Consider once again the standard incomplete elliptic integrals. Let the aforementioned semicircular contour be composed of an interval along the imaginary axis from $-\iu R$ to $\iu R$, and let $C_R$ be the semicircular arc in the right half plane from $\iu R$ back down to $-\iu R$. The contour is homologous to a standard $B$ period around $[1,k^{-1}]$.

% \makefigure{}{thesis_graphics_temp/contour.png}
\makefigure{The contour $C_R$}{tikz/contour}

We treat first the integral of the first kind. The contribution coming from the semicircular arc is negligible as $R$ tends to infinity, as can be seen below.
\begin{align*}
\abs{\int_{C_R}\frac{dz}{\sqrt{(1-z^2) (1-k^2z^2)}}}
&= \abs{\int_{π/2}^{-π/2} \frac{Re^{\iu θ} dθ}{\sqrt{(1-R^2e^{2\iu θ}) (1-k^2R^2e^{2\iu θ})}}} \\
&\leq \frac{R}{\sqrt{(R^2 - 1) (k^2R^2 - 1)}} \times π \\
&\to 0.
\end{align*}
And since $F_0$ is an odd function of $x$, we can conclude that
\begin{align*}
2\iu K'
&= \lim_{R\to\infty} \bra{\int_{-\iu R}^{\iu R} + \int_{C_R}}  \frac{dz}{\sqrt{(1-z^2) (1-k^2z^2)}} \\
&= 2\iu \lim_{R\to\infty} \int_{0}^{R} \frac{dt}{\sqrt{(1+t^2) (1+k^2t^2)}} + 0 \\
&= 2\iu \lim_{R\to\infty} F_0(R; k),
\end{align*}
which establishes the first half of the result.

The analysis of the second limit proceeds in the same manner, however there is an extra step arising from the cancellation of the pole at infinity. Using the same contour as before, we again show that the contribution from the semicircular arc is vanishing.
\begin{align*}
\abs{\int_{C_R}\sqrt{\frac{1-k^2z^2}{1-z^2}} - k \;\;dz}
&= \abs{\int_{π/2}^{-π/2} \frac{ \sqrt{1-k^2R^2e^{2\iu θ}} - k \sqrt{1-R^2e^{2\iu θ}}}{\sqrt{1-R^2e^{2\iu θ}}} \;Re^{\iu θ}\;dθ} \\
&\leq \frac{ \sqrt{1+k^2R^2} - k \sqrt{R^2-1}}{\sqrt{R^2-1}}R \times π \\
&= π \frac{R}{\sqrt{R^2-1}} \frac{ 1 + k^2 }{\sqrt{1+k^2R^2} + k \sqrt{R^2-1}} \\
&\to 0.
\end{align*}
As before, we are dealing with an odd function of $x$ and hence
\begin{align*}
\lim_{x\to \infty} E_0(x; k) = K' - E'.
\end{align*}
\end{proof}
\end{lem}

It immediately follows from this lemma that we have the following bounds independent of $x$.
\begin{align*}
-K' \leq &F_0(x;k) \leq K' \\
- (K' - E') \leq &E_0(x; k) \leq K' - E'.
\labelthis{eqn:F0_E0_uniform_bounds}
\end{align*}
It can also be useful to bound the growth of the two functions independently of $k$. As the integrands are monotonically decreasing functions of $k$, for $x>0$ so too are $F_0$ and $E_0$. In the extreme cases that $k$ is $0$ or $1$, the following lemma applies.

\begin{lem} \label{lem:klimit}
The elliptic integrals degenerate to the following elementary functions as $k$ approaches $0$ or $1$. Uniformly in $x$,
\begin{align*}
\lim_{k\to 1} F_0(x; k) &= \atan {x} \\
\lim_{k\to 1} E_0(x; k) &= 0.
\end{align*}
And uniformly for $x$ in a compact set,
\begin{align*}
\lim_{k\to 0} F_0(x; k) &= \asinh {x} \\
\lim_{k\to 0} E_0(x; k) &= \asinh {x}.
\end{align*}

\begin{proof}
As these are decreasing functions, for all $k > k_0$ we have
\begin{align*}
\abs{ F_0(x;k) } <& \abs{ F_0(x; k_0) } < K'(k_0), \\
\abs{ E_0(x;k) } <& \abs{ E_0(x; k_0) } < K'(k_0) - E'(k_0),
\end{align*}
using the previous lemma. This shows that the integrals are dominated, and we may therefore pass the limit $k\to1$ under the integral sign.
\begin{align*}
\lim_{k\to 1} F_0(x; k) &= \int_0 ^x \frac{dt}{1+t^2} = \atan {x}, \\
\lim_{k\to 1} E_0(x; k) &= \int_0^x \sqrt{\frac{1+ t^2}{1+t^2}} - 1 \;\;dt = 0.
\end{align*}

For the limit as $k \to 0$, observe that for positive $x$
\begin{align*}
F_0(x;k) &\leq \int_0 ^x \frac{dt}{\sqrt{1+t^2}} = \asinh x, \\
E_0(x; k) &\leq \int_0^x \frac{1}{\sqrt{1+t^2}} - 0 \;\;dt = \asinh x.
\end{align*}
These inequalities serve to show that the integrals are dominated by functions independent of $k$, but not uniformly in $x$ as $\asinh$ is an unbounded function. Thus for only values of $x$ in a compact set does the dominated convergence theorem hold and give the result.
\end{proof}
\end{lem}

\makefigure{Plot of $F_0(x;0.5)$ (black) and its upper and lower bounds. For each value of $k$ the function $F_0(x;k)$ is bounded, but as $k\to 0$ it tends to the unbounded $\asinh x$.}{thesis_graphics/elliptic_bounds_F0.png}
\makefigure{Plot of $E_0(x;0.5)$ (black) and its upper and lower bounds. Like $F_0$, it is bounded for each value of $k$, but unbounded in the limit $k \to 0$.}{thesis_graphics/elliptic_bounds_E0.png}














\section{Derivatives}
\label{sec:Derivatives}
The purpose of this section is to compute the derivatives of the elliptic integrals. The derivatives of the incomplete integrals with respect to the variable $z$ are trivial because they are simply parameter integrals
\begin{align}
    \Partial{}{z} F(z;k) &= \frac{1}{\sqrt{(1-z^2)(1-k^2 z^2)}}, \label{eqn:dFdz} \\
    \Partial{}{z} E(z;k) &= \sqrt{\frac{1-k^2 z^2}{1-z^2}}. \label{eqn:dEdz}
\end{align}
The derivatives of elliptic integrals with respect to the modulus are again elliptic integrals. In the interest of being concise, the correct combinations are presented ex nihilo. One may do the computation from scratch by differentiating the integrand and then subtracting terms to cancel off the poles and zeroes until only an exact differential remains. Differentiation under the integral sign is permitted because the integrals are dominated, as established in Lemma \ref{lem:klimit}. We compute the difference between the $k$ derivative of $dz/w$ and a certain combination of elliptic integrand terms.
\begin{align*}
\Partial{}{k}\bra{\frac{1}{\sqrt{(1-z^2)(1-k^2 z^2)}}}
&- \frac{1}{k(1-k^2)}\sqrt{\frac{1-k^2 z^2}{1-z^2}}
+ \frac{1}{k}\frac{1}{\sqrt{(1-z^2)(1-k^2 z^2)}} \\
&= \frac{-k}{1-k^2} \Partial{}{z} z\sqrt{\frac{1-z^2}{1 - k^2z^2}}.
\end{align*}
Integrating and rearranging gives
\begin{align*}
\Partial{}{k} F(z;k)
= \frac{1}{k(1-k^2)} E(z;k) - \frac{1}{k} F(z;k) - \frac{k}{1-k^2}z\sqrt{\frac{1-z^2}{1-k^2z^2}}. \labelthis{eqn:dtildeFdk}
\end{align*}
In a similar way, consider that
\begin{align*}
k\Partial{}{k}\bra{\sqrt{\frac{1-k^2t^2}{1-t^2}}} -& \sqrt{\frac{1-k^2t^2}{1-t^2}} + \frac{1}{\sqrt{(1-t^2)(1-k^2 t^2)}}
= 0.
\end{align*}
Thus we can integrate to obtain
\[
\Partial{}{k}E(z;k) = \frac{1}{k}E(z;k) - \frac{1}{k}F(z;k). \labelthis{eqn:dtildeEdk}
\]
We can recover the well known formulae for the derivatives of the complete elliptic integrals by making the substitution $z=1$.
\begin{align}
\frac{d}{dk}K &= \frac{1}{k(1-k^2)}E - \frac{1}{k}K, \label{eqn:dKdk}\\
\frac{d}{dk}E &= \frac{1}{k}E - \frac{1}{k} K. \label{eqn:dEdk}
\end{align}














\section{Legendre's Relation}
\label{sec:Legendre's Relation}
There is a useful relation between the complete elliptic integrals. It can be used to construct a basis of differentials with normalised periods. It also links the behaviour of the elliptic integrals at $k=0$ and $k=1$. In the computation of certain limits one could directly apply bounds, but it can often be more expedient to use Legendre's relation to transform the limiting term into a well behaved function.
The standard proof of Legendre's relation is reproduced below.

\begin{lem}[Legendre's relation]
\label{lem:Legendres relation}
\[
KE' + K'E - KK' = \frac{π}{2},
\]

\begin{proof}
We shall prove Legendre's relation in two stages. First we shall differentiate to show that it is constant. Then we will compute the constant by taking the limit as $k$ tends to $0$. Recall that the primes refer to the complementary modulus $k' = \sqrt{1-k^2}$. Its derivative is
\[
\frac{dk'}{dk} = -\frac{k}{k'},
\]
so
\begin{align*}
    \frac{d}{dk}&\bra{ KE' + K'E - KK'}\\
    &= \bra{\frac{1}{k(1-k^2)}E - \frac{1}{k}K}E' -\frac{k}{k'}K\bra{\frac{1}{k'}E' - \frac{1}{k'} K'} \\
    &\quad -\frac{k}{k'}\bra{\frac{1}{k'(1-k'^2)}E' - \frac{1}{k'}K'}E + K'\bra{\frac{1}{k}E - \frac{1}{k} K} \\
    &\quad - \bra{\frac{1}{k(1-k^2)}E - \frac{1}{k}K}K' +  \frac{k}{k'}K\bra{\frac{1}{k'(1-k'^2)}E' - \frac{1}{k'}K'} \\
    %%%%%%%%%%%%%%%%%%%%%%
    % &= \frac{1}{k k'^2}EE' - \frac{1}{k}KE' - \frac{k}{k'^2}KE' + \frac{k}{k'^2} KK' \\
    % &-\frac{1}{k'^2 k}EE' + \frac{k}{k'^2}K'E + \frac{1}{k}K'E - \frac{1}{k} KK' \\
    % & -\frac{1}{k k'^2}K'E + \frac{1}{k}KK' +  \frac{1}{k'^2 k}KE' - \frac{k}{k'^2}KK' \\
    % %%%%%%%%%%%%%%%%%%%%%%
    &= 0.
\end{align*}
Thus we have shown that the relation is constant. Determining the value of the constant is somewhat delicate. One could naïvely attempt to set $k=0$, but then $k'=1$ and $K'$ is infinite. And conversely, if we were to set $k=1$, then $K$ would be infinite. Instead, let us take the limit as $k \to 0$,
\[
\lim_{k \to 0} KE' + K'E - KK' = \frac{π}{2} + \lim_{k \to 0} (E - K) K'.
\]
It remains to show this latter limit is zero. We will show this using the inequalities for $K$ and $E$. From \ref{eqn:E_bound} we have
\[
\bra{ \frac{π}{4} - K} k^2 \leq E - K \leq \bra{1-K} k^2.
\]
And substituting the complementary modulus, \ref{eqn:K_bound} becomes
\[
\ln 4 \leq K' + \ln k \leq \frac{π}{2}.
\]
Since $k^2 \ln k$ goes to $0$ as $k$ does, the limit is established. Hence the constant in Legendre's relation is $π/2$.

\end{proof}
\end{lem}










\section{Analytic Continuation}
\label{sec:EllipticContinuation}

Above, we computed the limits of $F_0(x;k)$ and $E_0(x;k)$ as $x \to \infty$, but because both functions are odd, there is no way to extend them to single valued functions on $\RInf$. Instead we must extend these functions analytically. It is a standard fact that the elliptic integrals are analytic functions on the plane. To extend, we will use the fact that parameter integrals of analytic functions are analytic. Let $\tilde{x}\in\R$ and consider the covering map $\R \to \RInf$ given by
\[
\tilde{x} \mapsto \tan\frac{\tilde{x}}{2}.
\]
We identify $x\in\R$ with $\tilde{x} \in (-π,π)$.

Let $\tilde{F}$ and $\tilde{E}$ be the analytic continuations of $F_0$ and $E_0$ respectively. We will show that these continuations exist for all $\tilde{x}$ by explicit construction. Take $\tilde{x}$ in the range $(-π,π)$ and pull back the covering map
\begin{align*}
\tilde{F}(\tilde{x};k)
&= \int_0^{2 \atan \tilde{x}} \frac{dt}{\sqrt{(1+t^2)(1+k^2t^2)}} \\
&= \int_0^{\tilde{x}} \frac{1}{\sqrt{(1+(\tan s/2)^2)(1+k^2(\tan s/2)^2)}} \times \frac{1}{2}\sec^2\frac{s}{2}\;ds \\
&= \frac{1}{2} \int_0^{\tilde{x}} \frac{1}{\sqrt{\cos^2 s/2 + k^2 \sin^2 s/2}} \;ds.
\end{align*}
The integrand is analytic, and so $\tilde{F}$ is an analytic function for all values of $\tilde{x}\in\R$. The same approach works for $\tilde{E}$, but requires a little more work.
\begin{align*}
\tilde{E}(\tilde{x}; k)
&= \int_0^{2\atan \tilde{x}} \frac{\sqrt{1+k^2t^2} - k\sqrt{1+t^2}}{\sqrt{1+t^2}}\;dt \\
&= \frac{1}{2} \int_0^{\tilde{x}} \frac{\sqrt{\cos^2 s/2 + k^2\sin^2 s/2} - k}{ \cos^2 s/2 } \;ds \\
&= \frac{1}{2} \int_0^{\tilde{x}} \frac{\bra{\cos^2 s/2 + k^2\sin^2 s/2} - k^2}{ \cos^2 s/2 } \frac{1}{\sqrt{\cos^2 s/2 + k^2\sin^2 s/2} + k}\;ds \\
&= \frac{1}{2} (1-k^2) \int_0^{\tilde{x}} \frac{1}{\sqrt{\cos^2 s/2 + k^2\sin^2 s/2} + k}\;ds.
\end{align*}
Manipulating the integrand into a form that is plainly analytic demonstrates that $\tilde{E}$ is as well. One could reasonably ask how to compute the values of these extended functions. This is answered by noticing the integrands are $2π\iu$ periodic. Write $\tilde{x} = 2πn + y$ for an integer $n$ and $y\in(-π,π)$. Then
\begin{align*}
\tilde{F}(\tilde{x};k)
&= \frac{1}{2} \bra{ \int_0^{2πn} + \int_{2πn}^{2πn +y}} \frac{1}{\sqrt{\cos^2 s/2 + k^2 \sin^2 s/2}} \;ds \\
&= \frac{1}{2} \bra{ n\int_0^{2π} + \int_{0}^{y}} \frac{1}{\sqrt{\cos^2 s/2 + k^2 \sin^2 s/2}} \;ds \\
&= \bra{ n\oint_{\RInf} + \int_{0}^{x}} \frac{dt}{\sqrt{(1+t^2)(1+k^2t^2)}} \\
&= 2nK' + F_0(x;k),
\labelthis{eqn:tildeF_period}
\end{align*}
where $x = \tan \tilde{x}/2 = \tan y/2$ and recalling the period of $F_0$ from earlier. Likewise
\[
\tilde{E}(\tilde{x}; k) = 2n(K'-E') + E_0(x;k).
\labelthis{eqn:tildeE_period}
\]





%%%%%% Start allowing entries in the table of contents again.
\starttocentries
