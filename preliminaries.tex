%!TEX root = thesis.tex

\section{Preliminaries}
\label{sec:Preliminaries}

\subsection{Rambling}
Things that have to be covered
- intro what is a harmonic map
- recap of construction of spectral curve.
- recap Hitchin theorems
- reformulate into forms we use
- define spaces
- make argument about imaginary periods

HIGGS FIELD
Suppose that $M$ is a compact Riemann surface, and $G$ is a Lie group with a bi-invariant metric. Given any smooth map $f: M\to G$, we can pull back the metric connection on $TG$ to $d_A$ on $f^*(TG)$. Using the the orientation of the complex structure of $M$, we have the Hodge star operator $* : Ω^1(M) \to Ω^1(M)$. $f$ is harmonic if $d_A^* df$ is zero. Using the Hodge star to write the adjoint, this is equivalent to
\[
d_A (* df) = 0.
\]
Pulling back the Maurer-Cartan form of $G$ to $2(Φ - Φ^*)$, with $Φ$ the $(1,0)$ part of the form, harmonicity of $f$ implies
\[
d_A'' Φ = 0,\, F_A = [{Φ},{Φ}^*]. \labelthis{eqn:Hit1.7}
\]
(Hitchin eqn 1.7). If a pair $(A,Φ)$ satisfies \ref{eqn:Hit1.7} and further the connections
\[
\nabla_A + Φ - Φ^*, \nabla_A - Φ + Φ^*
\]
are trivial, then it is possible to recover a harmonic map from $M$ to $G$. For the case of $M$ a torus and $G=SU(2)$, Hitchin's theorem 8.1 \cite{Hitchin1990} provides a correspondence between solutions of \ref{eqn:Hit1.7} and so-callled spectral data. Theorem 8.20 gives an addtional condition on spectral data to ensure that the two connections are trivial, and so the spectral data corresponds to a harmonic map from the torus into $\S^3$. The conditions are reproduced here and labelled. Let $ρ$ be the real structure induced from $ζ\mapsto \cji{ζ}$ on $\CP^1$ and $σ$ the hyperelliptic involution given by $η\mapsto -η$. Given a hyperelliptic curve $Σ$ in the total space of $π: \mathcal{O}(p+1)\to \CP^1$ defined by the equation $η^2 = P(ζ)$ and meromorphic differentials $Θ, \tilde{Θ}$ on $Σ$ the following properties must be satisfied.
\begin{enumerate}
\item Real spectral curve: $P(ζ)$ is a real section of $\mathcal{O}$ with respect to $ρ$.
\item No real zeroes: $P(ζ)$ has no zeroes on the unit circle $\S^1\subset\CP^1$.
\item Simple zeroes: $P(ζ)$ has at most simple zeroes at $ζ=0,\infty$.
\item Poles: The differentials have double poles with no residues at $π^{-1}\{0,\infty\}$ where $π$ is the projection $π: Σ \to \CP^1$, but are otherwise holomorphic.
\item Symmetry: The differentials both satisfy $σ^* Θ = - Θ$.
\item Real differentials: The differentials both satisfy $ρ^* Θ = - \bar{Θ}$
\item Periods: The periods of the differentials lie in $2π\iu\Z$
\item Linear independence: The principal parts of the two differentials are real linearly independent.
\item Closing conditions: $μ$ has value $1$ at $π^{-1}\{\pm 1\}$, where $Θ = d\log μ$, $μσ^*μ = 1$.
\end{enumerate}

Reformulate a bit. The holomorphic differential is $dζ/η$, so meromorphic differentials on $Σ$ are of the form $dζ/η (f(ζ) + η g(ζ))$ for $f,g$ meromorphic functions on $\CP^1$ (ie they are rational functions). The symmetry constraint requires that $g\equiv 0$. To have double poles above $ζ=0,\infty$, $f$ must be of the form $ζ^{-2}b(ζ)$ for a polynomial $b$ of degree $p+3$. Reality of the differential means that this polynomial is real, that is $\overline{b(ζ)}=\bar{ζ}^{p+3}b(\cji{ζ})$. To handle the residue condition, expand as a series at zero to get that the residue is
\[
b_1 - \frac{1}{2}\frac{P_1}{P_0}b_0,
\]
where subscripts denote coefficents of the polynomials. Thus is determines $b_1$ in terms of $b_0$. The period condition is difficult to calculate explicitly, but we can consider a linearization of it. So counting real degrees of freedom, we have $2\times(p+4)/2 - 2 = p+2$ in choosing $b$. We may choose a basis of homology such that the $A$ periods are all real, and the $B$ periods are purely imaginary. Then a necessary condition is that the $A$ periods must vanish: $p$ linear constraints. Thus there is a real 2-plane of differentials with purely imaginary periods that otherwise satisfy the conditions.

The closing condition can be reformulated as a period type constraint also. Consider if we had such a $μ$ for $Θ$. Then suppose that $γ_1$ was a path connecting the two points of $π^{-1}(1)$.
\[
\int_{γ_1} Θ = \log μ(1^+) - \log μ(1^-) \in 2π\iu\Z
\]
and likewise for the two points over $-1$. Conversely, suppose that
\[
\int_{γ_1} Θ, \int_{γ_{-1}} Θ \in 2π\iu\Z
\]
Then we could define $μ$ to be $\exp(\int θ)$. The integral of $Θ$ is defined up to a constant, periods and residues. The latter are zero, the constant of integration is fixed by the condition $μσ^*μ = 1$, and the periods are all in $2π\iu\Z$ so do not change $μ$. These integrals then imply that
\[
μ(1^+)/μ(1^-) = \exp \left(\int_{γ_1} Θ \right) = 1.
\]
Together with $μ(1^+)μ(1^-) = μ(1^+)σ^*μ(1^+) = 1$, this implies that $μ(1^\pm) = \pm 1$. Looking at the construction of solutions of \eqref{HME}, we are free to tensor with an element of $H^1(M,\Z_2)$, which we can use to fix the signs to be just $+1$ in a unique way. Thus for all $Θ$ that satisfy these `half-period' integral conditions, there is one $μ$ that satisfies Hitchin's closing condition.

Thus we have found a form of the differentials that satisfies most of the conditions, and the conditions not yet fulfilled are all the integrality of certain integrals.

Given a set of spectral data, we can detect certain features of the corresponding map directly. For example, if $P$ is an even polynomial, then the image of the map is a geodesic sphere. Of particular interest is the case where $P(0)=0$; in this case the map is conformal. Why this is of interest is because it changes the form of $θ$ given above. Because $P$ can have at most a simple zero at $ζ=0$, $P_1 \neq 0$. Let a local coordinate by $ξ =\sqrt{ζ}$ and expand
\[
θ = \frac{dζ}{ζ^2η}b(ζ) \sim \frac{2ξdξ}{ξ^4ξ} \frac{1}{\sqrt{P_1}} \bra{1-\frac{1}{2}\frac{P_2}{P_1}ξ^2+O(ξ^4)} \bra{b_0 + b_1 ξ^2 + O(ξ^4)} = \frac{2dξ}{ξ^4} \frac{1}{\sqrt{P_1}} \bra{b_0 + ξ^2 \bra{b_1-\frac{1}{2}\frac{P_2}{P_1}b_0} + O(ξ^4)}
\]
So to have a double pole requires that $b_0$ must also vanish for a conformal map, and note that this is automatically residue free. In light of this, we may rephrase the equation to be residue free in the nonconformal case as
\[
P_1b_0 - 2P_0b_1 = 0
\]
and observe the relation continues to hold in the conformal case, even though it now is needed to produce the desired cancellation as the two double poles `collide'.


SPACES
Spectral curves are hyperelliptic curves as above that meet conditions i,ii,iii. In this thesis we consider non-singular spectral curves. Something something genus. The equation of the spectral curve $η^2 = P(ζ)$ is determined up to real scalar by the roots of $P$. By the reality structure, if $α$ is a root, so is $\cji{α}$. Different real scalars define the same curve up to biholomorphism. We therefore choose the following cannonical form of $P$,
\[
P(ζ) = \prod_{i=0}^{g} (ζ-α_i)(1-\bar{α}_i ζ),
\]
where $α_i$ are the roots of $P$ inside the open unit disc $D$. Define
\[
\mathcal{A}_g = \{ (α_0, α_1, \ldots, α_g) \in D^{g+1} \mid α_i \neq α_j \}.
\]
Every point of $\mathcal{A}_g$ determines a non-singular spectral curve $Σ(α_0,\ldots,α_g)$ via a polynomial $P$ with those roots. However, different permutations of components of a point of $\mathcal{A}_g$ yeild the same spectral curve. Let $Sym(g+1)$ be the symmetric group on $g+1$ elements, and we have it act on $\mathcal{A}_g$ by permutation. We define $\mathcal{C}_g$ to be $\mathcal{A}_g / Sym(g+1)$ and we call this the space of spectral curves. By excluding singular curves, that is curves with multiple roots, the action of the symmetric group has no fixed points and so the quotient is also a smooth manifold.

SINGULARITIES AND THEIR EXCLUSION

In the case of spectral curves of genus two or lower, there are no differentials with real linearly independent principal parts on singular curves.

\begin{lem}
On a spectral curve of genus zero, differentials with linearly independent principal parts do not have common roots.
\begin{proof}
The conformal case is established in Hitchin p659. We repeat it here. Any differential has the form
\[
(b + \bar{b}ζ)\frac{dζ}{ζη}.
\]
If two such differentials have a common root, then they must be real scalars of one another.

In the nonconformal case, the form of the differentials has more terms. Let the spectral curve be given by
\[
η^2 = -α + (1+α\bar{α})ζ -\bar{α} ζ^2,
\]
for $α\neq 0$ in the unit disc, and let $a = -α^{-1}(1+α\bar{α})$. Then the differential is determined by a nonzero constant $b$,
\[
(b + ab ζ + \overline{ab}ζ^2 + \bar{b}ζ^3)\frac{dζ}{ζ^2η}.
\]
Take two differentials $Θ,\tilde{Θ}$ with real linearly independent principal parts. We shall compute their greatest common divisor. Observe
\[
\bar{b}\tilde{Θ} - \bar{c}Θ = (\bar{b}c - \bar{c}b)(aζ+1)\frac{dζ}{ζ^2η}.
\]
By assumption, $b/c$ is not real, so this is nonzero. Its only root is $-a^{-1}$, but at $ζ=-a^{-1}$
\[
b + ab (-a^{-1}) + \overline{ab}(-a^{-1})^2 + \bar{b}(-a^{-1})^3
= \bar{b}a^{-3} ( a\bar{a} - 1),
\]
which is only zero if $\abs{a} = 1$, which itself only occurs when $α=0$. But we are considering the nonconformal case, ie $α\neq 0$, so $Θ$ and $aζ+1$ have no common factors. Hence
\[
\gcd(Θ,\tilde{Θ})
= \gcd\bra{ Θ, \bar{b}\tilde{Θ} - \bar{c}Θ }
= \gcd( Θ, aζ+1) = 1
\]
\end{proof}
\end{lem}
