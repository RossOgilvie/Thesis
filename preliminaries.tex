%!TEX root = thesis.tex

\section{Preliminaries}
\label{sec:Preliminaries}

HIGGS FIELD

Suppose that $M$ is a compact Riemann surface, and $G$ is a Lie group with a bi-invariant metric. Given any smooth map $f: M\to G$, we can pull back the metric connection on $TG$ to $d_A$ on $f^*(TG)$. Using the the orientation of the complex structure of $M$, we have the Hodge star operator $* : Ω^1(M) \to Ω^1(M)$. $f$ is harmonic if $d_A^* df$ is zero. Using the Hodge star to express the adjoint, this is equivalent to
\[
d_A (* df) = 0.
\]
Pulling back the Maurer-Cartan form of $G$ to $2(Φ - Φ^*)$, with $Φ$ half of the $(1,0)$ part of the form, harmonicity of $f$ implies
\[
d_A'' Φ = 0,\, F_A = [{Φ},{Φ}^*]. \labelthis{eqn:Hit1.7}
\]
(Hitchin eqn 1.7). If a pair $(A,Φ)$ satisfies \ref{eqn:Hit1.7} and further the connections
\[
\nabla_A + Φ - Φ^*, \nabla_A - Φ + Φ^*
\]
are trivial, then it is possible to recover the harmonic map from $M$ to $G$.

Consider now the case $M$ a torus and $G=SU(2)$.

MARKED CURVE

Consider a hyperelliptic curve $Σ$ over $\CP^1$ with hyperelliptic involution $σ$ and projection $π:Σ\to\CP^1$. We call a tuple $(Σ,π^{-1}(0),π^{-1}(1),π^{-1}(-1))$ a \emph{marked curve} if
\begin{enumerate}
\item there is a real involution $ρ: Σ \to Σ$ such that $ρ\circ σ$ is without fixed points and $π(ρ(ξ)) = \cji{π(ξ)}$.
\item the curve is non-singular at $π^{-1}\{0\}$.
\end{enumerate}
Note that there may be one or two points of $π^{-1}(0)$, which correspond to the cases of conformal and nonconformal harmonic maps respectively. The condition that $ρ\circ σ$ is without fixed points ensures that the curve is not branched ($ξ = σ(ξ)$) at any point over the unit circle. Thus there are always two points of both $π^{-1}(1)$ and $π^{-1}(-1)$. We shall use $ρ$ to also denote the real involution $π(ζ) = \cji{ζ}$ on $\CP^1$.

We may build a model for marked curves, unique up to biholomorphism. Suppose that $Σ$ has genus $g$. Consider the bundle $π: \mathcal{O}(g+1)\to \CP^1$, and the tautological section $η$ of $π^* \mathcal{O}(g+1) \to \mathcal{O}(g+1)$. There is a reality structure on $\mathcal{O}(k)$ for any $k$ compatible with $ρ$, namely if $w$ is the fibre coordinate
\[
ρ(ζ,w) = \bra{ \cji{ζ}, \bar{ζ}^{-k} \bar{w}}.
\]

Take a section $P(ζ)$ of $\mathcal{O}(2g+2)$ and consider the curve $Σ'$ defined by $η^2 = P(ζ)$. It is hyperelliptic, with involution $σ(ζ,η) = (ζ,-η)$, and projection $π(ζ,η) = ζ$. Suppose that $P(ζ)$ has the following properties
\begin{enumerate}
\item Real spectral curve: $P(ζ)$ is a real section of $\mathcal{O}(2g+2)$ with respect to the real structure induced by $ρ$.
\item No real zeroes: $P(ζ)$ has no zeroes on the unit circle $\S^1\subset\CP^1$.
\item Simple zeroes: $P(ζ)$ has at most simple zeroes at $ζ=0,\infty$.
\end{enumerate}
Then $Σ'$ is a marked curved. The holomorphic involution $ρ$ is the restriction of $ρ$ on $\mathcal{O}(g+1)$. The fixed points of $σ$ are the roots of $P$, so ``no real zeroes" ensures that $ρ\circ σ$ is fixed point free and ``simple zeroes" provides that the curve is not singular over $0$. As hyperelliptic curves are determined by their branch points in $\CP^1$, up to automorphism of $\CP^1$ which we have fixed by fixing $0,1,-1$, all marked curves correspond to some section $P(ζ)$. The only freedom remaining is the choice of a real scaling of $P$, which we shall fix now.

Let $P(ζ) = A(ζ-β_1)\cdots(ζ-β_k)$ be a section of $\mathcal{O}(k)$. Applying the real involution
\[
\bar{ζ}^{-k} \overline{P(ζ)}
= \bar{ζ}^{-k} \bar{A}(\bar{ζ}-\bar{β}_1)\cdots(\bar{ζ}-\bar{β}_n)
= \bar{A}\bra{\prod_{β_i \neq 0} \bar{β}_i (\cji{β}_i-\cji{ζ})}
\bra{\prod_{β_i=0}\cji{ζ}}.
\]
If $P$ is real then this is equal to $P(\cji{ζ})$ and if $P$ has no real zeroes then for no $i$ is $β_i$ equal to $\cji{β}_i$. Thus the set of roots is fixed under this involution, and the nonzero roots must arrange into pairs. Any roots at zero, $β_i = 0$, are paired with roots `at infinity', which manifest as a reduction of degree; $P$ has a root at zero of order $k-n$.

Suppose we have a marked curve of genus $g$. Denote its branch points $α_0,\cji{α}_0,\ldots,α_g,\cji{α}_g$. We fix the following scaling of $P$
\[
P(ζ) = \prod_{i=0}^{g} (ζ - α_i)(1 -\bar{α}_i ζ).
\]
This scaling is well-behaved if one branch point is zero. It makes $P$ a real section.

DIFFERENTIALS

Symmetry conditions on the differentials
\begin{enumerate}
\item Poles: The differentials have double poles with no residues at $π^{-1}\{0,\infty\}$ where $π$ is the projection $π: Σ \to \CP^1$, but are otherwise holomorphic.
\item Symmetry: The differentials both satisfy $σ^* Θ = - Θ$.
\item Real differentials: The differentials both satisfy $ρ^* Θ = - \bar{Θ}$
\item Imaginary Periods: The differentials have purely imaginary periods
\end{enumerate}
argument about plane of imaginary period differentials. Call it $\mathcal{B}_Σ$.
Reformulate a bit.
A holomorphic differential is $dζ/η$, so meromorphic differentials on $Σ$ are of the form $(f(ζ) + η g(ζ))dζ/η$ for $f,g$ meromorphic functions on $\CP^1$ REF Miranda. All meromorphic functions on $\CP^1$ are rational functions. Applying the hyperelliptic involution gives
\[
σ^*\bra{ (f(ζ) + η g(ζ))\frac{dζ}{η} } = (-f(ζ) + η g(ζ))\frac{dζ}{η},
\]
so the symmetry condition forces $g\equiv 0$. Suppose first that $Σ$ does not have a branch point at $0$. Then $ζ$ is a local coordinate at each point of $π^{-1}(0)$, and for  the differential to have double poles above $ζ=0,\infty$, $f$ must be of the form $ζ^{-2}b(ζ)$ for a polynomial $b$ of degree $g+3$. To handle the residue condition, expand as a series at zero to get that the residue is
\[
b_1 - \frac{1}{2}\frac{P_1}{P_0}b_0,
\]
where subscripts denote coefficents of the polynomials. This must therefore vanish. Reality of the differential means that this polynomial is real, that is $\overline{b(ζ)}=\bar{ζ}^{g+3}b(\cji{ζ})$.

Now suppose that $P$ has a root at $ζ=0$. Because $P$ can have at most a simple zero at $ζ=0$, $P_1 \neq 0$. Let a local coordinate of a point of $π^{-1}(0)$ be $ξ =\sqrt{ζ}$ and expand
\[
θ = \frac{dζ}{η}f(ζ)
\sim \frac{2ξdξ}{ξ} \frac{1}{\sqrt{P_1}} \bra{1-\frac{1}{2}\frac{P_2}{P_1}ξ^2+O(ξ^4)} f(ξ^2)
\]
So to have double poles requires that $f(ζ) = ζ^{-1}a(ζ)$ for some polynomial $a(ζ)$ of degree $g+1$ and note that this is automatically residue free. Reality of the differential means that this polynomial is also real, that is $\overline{a(ζ)}=\bar{ζ}^{g+1}a(\cji{ζ})$. If we write $b(ζ) = ζa(ζ)$, we see that this is a special case of the form above, where if $P_0$ vanishes so too must $b_0$. In light of this, we may rephrase equation REF among the coefficients to be
\[
P_1b_0 - 2P_0b_1 = 0.
\]
In summary, every differential meeting conditions REF can be written as
\[
Θ = b(ζ)\frac{dζ}{ζ^2η},
\]
for some real degree $g+3$ polynomial $b(ζ)$, meeting the condition $P_1b_0 - 2P_0b_1 = 0$ on its coefficents.

Finally, having purely imginary periods imposes further linear relations on the coefficents of $b$. Counting real degrees of freedom, we have $2\times(g+4)/2 - 2 = g+2$ in choosing $b$. We may choose a basis of homology such that the $A$ periods are all real, and the $B$ periods are purely imaginary. Then a necessary condition is that the $A$ periods must vanish: $g$ linear constraints. Thus there is a real 2-plane of differentials with purely imaginary periods.

SPECTRAL DATA

More conditions
\begin{enumerate}
\item Linear independence: The principal parts of the two differentials are real linearly independent.
\item Periods: The periods of the differentials lie in $2π\iu\Z$
\end{enumerate}
Talk about line bundle $L$.
Hitchin's theorem 8.1 \cite{Hitchin1990} provides a correspondence between solutions of \ref{eqn:Hit1.7} and so-callled spectral data $(Σ,Θ,\tilde{Θ},L)$. $Σ$ is called a spectral curve if it admits spectral data.
Called isoperiodic set REF. Justify why we will now ignore $L$.


CLOSING CONDITIONS

Theorem 8.20 gives an addtional condition on spectral data to ensure that the two connections are trivial, and so the spectral data corresponds to a harmonic map from the torus into $\S^3$.
\begin{enumerate}
\item Closing conditions: $μ$ has value $1$ at $π^{-1}\{\pm 1\}$, where $Θ = d\log μ$, $μσ^*μ = 1$.
\end{enumerate}
The closing condition can be reformulated as a period type constraint also. Consider if we had such a $μ$ for $Θ$. Then suppose that $γ_1$ was a path connecting the two points of $π^{-1}(1)$.
\[
\int_{γ_1} Θ = \log μ(1^+) - \log μ(1^-) \in 2π\iu\Z
\]
and likewise for the two points over $-1$. Conversely, suppose that
\[
\int_{γ_1} Θ, \int_{γ_{-1}} Θ \in 2π\iu\Z
\]
Then we could define $μ$ to be $\exp(\int θ)$. The integral of $Θ$ is defined up to a constant, periods and residues. The latter are zero, the constant of integration is fixed by the condition $μσ^*μ = 1$, and the periods are all in $2π\iu\Z$ so do not change $μ$. These integrals then imply that
\[
μ(1^+)/μ(1^-) = \exp \left(\int_{γ_1} Θ \right) = 1.
\]
Together with $μ(1^+)μ(1^-) = μ(1^+)σ^*μ(1^+) = 1$, this implies that $μ(1^\pm) = \pm 1$. Looking at the construction of solutions of \eqref{eqn:Hit1.7}, we are free to tensor with an element of $H^1(M,\Z_2)$, which we can use to fix the signs to be $+1$ in a unique way. Thus for all $Θ$ that satisfy these closing conditions, there is one $μ$ that satisfies Hitchin's closing condition.


INTERPRETATION OF SPECTRAL DATA

Given a set of spectral data, we can detect certain features of the corresponding map directly. For example, if $P$ is an even polynomial, then the image of the map is a geodesic sphere. Of particular interest is the case where $P(0)=0$; in this case the map is conformal. Totally geodesic 2-sphere case.


SPACES

In this thesis we consider non-singular marked curves. We classify the curves by their genus. Recall that by the choice of scaling of $P$, the marked curve $Σ$ is determined uniquely by the roots of $P$. Let $D$ be the open unit disc and define
\[
\mathcal{A}_g = \{ (α_0, α_1, \ldots, α_g) \in D^{g+1} \mid α_i \neq α_j \}.
\]
Every point of $\mathcal{A}_g$ determines a non-singular spectral curve $Σ(α_0,\ldots,α_g)$ via a polynomial $P$ with roots $\{α_0,cji{α}_0,ldots,α_g,cji{α}_g}$. However, different permutations of components of a point of $\mathcal{A}_g$ yeild the same spectral curve. Let $Sym(g+1)$ be the symmetric group on $g+1$ elements, and have it act on $\mathcal{A}_g$ by permutation. We define $\mathcal{C}_g$ to be $\mathcal{A}_g / Sym(g+1)$ and we call this the space of marked curves. By excluding singular curves, that is curves with multiple roots, the action of the symmetric group has no fixed points and so the quotient is also a smooth manifold.

SINGULARITIES AND THEIR EXCLUSION

In the case of spectral curves of genus two or lower, there are no differentials with real linearly independent principal parts on singular curves.

\begin{lem}
On a spectral curve of genus zero, differentials with linearly independent principal parts do not have common roots.
\begin{proof}
The conformal case is established in Hitchin p659. We repeat it here. Any differential has the form
\[
(b + \bar{b}ζ)\frac{dζ}{ζη}.
\]
If two such differentials have a common root, then they must be real scalars of one another.

In the nonconformal case, the form of the differentials has more terms. Let the spectral curve be given by
\[
η^2 = -α + (1+α\bar{α})ζ -\bar{α} ζ^2,
\]
for $α\neq 0$ in the unit disc, and let $a = -α^{-1}(1+α\bar{α})$. Then the differential is determined by a nonzero constant $b$,
\[
(b + ab ζ + \overline{ab}ζ^2 + \bar{b}ζ^3)\frac{dζ}{ζ^2η}.
\]
Take two differentials $Θ,\tilde{Θ}$ with real linearly independent principal parts. We shall compute their greatest common divisor. Observe
\[
\bar{b}\tilde{Θ} - \bar{c}Θ = (\bar{b}c - \bar{c}b)(aζ+1)\frac{dζ}{ζ^2η}.
\]
By assumption, $b/c$ is not real, so this is nonzero. Its only root is $-a^{-1}$, but at $ζ=-a^{-1}$
\[
b + ab (-a^{-1}) + \overline{ab}(-a^{-1})^2 + \bar{b}(-a^{-1})^3
= \bar{b}a^{-3} ( a\bar{a} - 1),
\]
which is only zero if $\abs{a} = 1$, which itself only occurs when $α=0$. But we are considering the nonconformal case, ie $α\neq 0$, so $Θ$ and $aζ+1$ have no common factors. Hence
\[
\gcd(Θ,\tilde{Θ})
= \gcd\bra{ Θ, \bar{b}\tilde{Θ} - \bar{c}Θ }
= \gcd( Θ, aζ+1) = 1
\]
\end{proof}
\end{lem}

argument/explanation about why a common root of the differentials is a singular point of the curve.

MORE SPACES

Over $\mathcal{C}$, there is a bundle $\mathcal{B}$ of differentials with imaginary periods. As argued above it is a rank 2 vector bundle; the fibre over $Σ$ is $\mathcal{B}_Σ$. Bundle of differentials with integral periods is $\mathcal{B}(n_1,\ldots,n_g)$.
\[
\int_{B_k} Θ = 2π\iu n_k.
\]
Space of spectral data with closing conditions is $\mathcal{M}$, a subsapce of the total space of $\mathcal{B}\times\mathcal{B}$. Space of spectral curves with spectral data meeting closing conditions is $\mathcal{S}$, a subspace of $\mathcal{C}$.
