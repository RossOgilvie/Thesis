%!TEX root = thesis_single.tex

\chapter{Genus One Moduli Space}
\label{chp:Moduli Space}
% \epigraph{All analytic functions are alike; each non-analytic function is non-analytic in its own way.}

\section{Intro}
\label{sec:Intro}

An outline:
\begin{enumerate}
\item
Give the outline
\item
Reformulate in coordindates more suited to calculation.
\item
Talk about shape of this parameter space.
\item
Talk about $T$ and its multi-defined-ness. Universal cover
\item
Uniform coordinate
\item
Show derivative is nonzero
\item
Apply Implicit Function theorem in uniform coord and k
\item
Go to the Winchester, have a pint, and wait for this to blow over.
\end{enumerate}


In the previous section, we performed two important calculations. For any genus one spectral curve, we computed the differentials on it that meet the spectral data conditions. We further computed the closing conditions for these differentials. A spectral curve and pair of differentials is only periodic spectral data if the closing conditions are rational. This section furthers this line of investigation with a shift to consideration of the space of periodic spectral data as a whole. In the genus one case, every spectral curve admits spectral data but not every one admits periodic spectral data. The rationality of the closing conditions therefore can be viewed as defining equations for $\mathcal{M}_1$, the moduli space of spectral curves admiting periodic spectral data. Our aim is to investigate the topology of this space. In doing so we will actually parameterise all of $\tilde{\mathcal{M}}_1$, the bundle of periodic spectral data.




\section{Marked Point Coordinates}
\label{sec:Reformulate}

Recall the two rationality conditions that must be satisfied for a genus one spectral curve to admit differentials meeting the closing condition
\begin{align}
p &:= \frac{\abs{1-α}\abs{1-β}}{\abs{1+α}\abs{1+β}} \in \Q, \\
T &:=  \frac{1}{2π\iu} \bra{p\int_{γ_-} Θ^2 - \int_{γ_+} Θ^2} \in \Q,
\end{align}
Using the formula for the differentials, we can explicitly give an expression for $T$ in terms of the branch points $α$ and $β$
\[
T = \frac{2p}{π\iu} \left[ E F(f(-1);k) - K E(f(-1);k) \right] - \frac{2}{π \iu}\left[ E F(f(1);k) - K E(f(1);k) \right] + \frac{2}{π\iu}K \frac{\abs{1-α}\abs{1-β}}{\abs{α-β} + \abs{1-\bar{α}β}} \frac{4ν}{1-ν^2}, \labelthis{Teqn1}
\]
where $f$ is the map normalises the branch points of the spectral curve (it takes $α$ to $1$, $β$ to $k$, etc) and $ν$ is the point of intersection between the branch point circle and the unit circle. $T$ is a real valued function because $f(1)$ and $f(-1)$ are purely imaginary and $F, E$ take the imaginary axis to itself, and because $ν$ is on the unit circle, $ν/(1-ν^2)$ is purely imaginary as well.

It is our ultimate aim to examine the surfaces defined by the two conditions and our examination will result in coordinates for those surfaces. This involves applying the implicit function theorem and the necessary computation for that theorem requires differentiating the above expressions. It is therefore prudent to adopt a parameterisation of the space of spectral curves more suited to that task than $(α,β)$. An obvious first coordinate is $p$ itself, as then we can enforce the first condition simply by holding it constant. Elliptic integrals are the most difficult part of the above to differentiate, so to minimise our labour we choose the other three (real) coordinates to be $k$, $\iu u = f(1)$ and $\iu v = f(-1)$.

This motivation misses some of the subtleties of the situation however. For example, $f$ is a M\"obius transformation of $\CP^1$, so may take the value infinity. We therefore make the following definition.

\begin{defn}\label{defn:parameter space}
Consider the product space $\R_{>0} \times (0,1) \times \RP^1 \times \RP^1$. Let $p \in \R_{>0}$ and $k \in (0,1)$ be coordinates on the first two factors. Let $u, u' \in \R$, with $u' = u^{-1}$, be coordinates covering the third factor, and likewise let $v, v' \in\R$, with $v' = v^{-1}$ be coordinates covering the fourth factor. Together, the four coordinate patches $(p,k,u,v)$, $(p,k,u',v)$, $(p,k,u,v')$ and $(p,k,u',v')$ to $\R_{>0} \times (0,1) \times \R^2$ cover $\R_{>0} \times (0,1) \times \RP^1 \times \RP^1$.

Consider the hypersurface $H$ that is the closure of $\{u=v\}$. In each of the four coordinate patches it is given by the equations $u=v$, $(u')^{-1} = v$, $u = (v')^{-1}$, $u' = v'$. Define $P$ to be the complement of $H$.
\end{defn}

First we must show that $P$ does indeed parameterise the space of spectral curves. We proceed by determining a formula for the change of parameters from $(α,β)$ to $(p,k,u,v)$. The forward direction has already been achived. The map $f$ is written in terms of $(α,β)$ \todo{ref} and the new parameters are the images of points under this map (except for $p$, which has its own formula in terms of $α$ and $β$). As the new parameters all respresent point in the $z$-plane, one method to derive the reverse change in parameters is to express the inverse transformation $f^{-1}(z)$ in term of our new parameters. Then $α = f^{-1}(1)$ and $β = f^{-1}(k^{-1})$, entirely analogous to to forward direction. Denote $z_0 = f(0)$, the image of $ζ=0$ in the $z$-plane. The real structure of the $z$-plane then forces $f(\infty) = - \bar{z_0}$. As a M\"obius transformation is described up to a scalar by the points sent to $0$ and $\infty$, $f^{-1}$ is a scalar multiple of
\[
\frac{z-z_0}{z + \bar{z_0}},
\]

Note the following trick using cross ratios.
\[
\abs{\frac{α-1}{α+1}}
= \abs{\frac{α-1}{α+1}} \abs{\frac{0+1}{0-1}}
= \abs{ \cross{α}{0}{1}{-1} }
= \abs{ \cross{1}{z_0}{\iu u}{\iu v} }
= \abs{\frac{1-\iu u}{1 - \iu v}} \abs{\frac{z_0 - \iu v}{z_0 - \iu u}}
\]
The same trick gives a similar formula for $β$.
\[
\abs{\frac{α-1}{α+1}}
= \abs{\frac{1-k\iu u}{1 - k\iu v}} \abs{\frac{z_0 - \iu v}{z_0 - \iu u}}
\]
The terms on the left are the two halves of the expression for $p$. Multiplying and expanding shows that for fixed values of the parameters $(p,k,u,v)$, $z_0$ lies on a particular circle. Let $z_0 = x+\iu y$.
\begin{align*}
p &= \abs{\frac{1-\iu u}{1 - \iu v}}\abs{\frac{1 - k\iu u}{1 - k\iu v}} \abs{\frac{z_0 - \iu v}{z_0 - \iu u}}^2 \\
\abs{\frac{z_0 - \iu v}{z_0 - \iu u}} ^2
&= p \frac{\sqrt{1+v^2}}{\sqrt{1+u^2}}\frac{\sqrt{1+k^2v^2}}{\sqrt{1+k^2u^2}} \\
&= p \frac{w(v)}{w(u)} \\
\abs{z_0 - \iu v} ^2 &= p \frac{w(v)}{w(u)} \abs{z_0 - \iu u}^2 \\
x^2 + y^2 &+ 2y \frac{puw(v) - vw(u)}{w(u)-pw(v)} + \frac{v^2w(u) - pu^2w(v)}{w(u)-pw(v)} = 0.
\end{align*}

In the $ζ$-plane, the points $-1,0,1$ all lie on a straight line that is perpendicular to the unit circle at both $-1$ and $1$, and invariant under the real involution. Applying the M\"obius transformation $f$ we can therefore say that $\iu v, z_0, \iu u$ all lie on a circle that is perpendicular to the imaginary axis and symmetric under reflection in the imaginary axis. Then $z_0$ lies on the circle
\[
x^2 + \bra{ y - \frac{u+v}{2} }^2 = \frac{(u-v)^2}{4},
\]
which simplifies to the relation
\[
x^2 + y^2 = y(u+v) - uv.
\]
These two circles intersect in two points: $z_0$ and $-\bar{z_0}$. The precise formulas are
\[
x = \frac{\sqrt{pw(u)w(v)}}{pw(v) + w(u)} \abs{u-v},\; y = \frac{puw(v) + vw(u)}{pw(v) + w(u)}
\]
where the sign of $x$ is chosen to make $z_0$ lie in the right half of the $z$-plane. This choice amounts to choosing the branch points $α,β$ inside the unit circle, as opposed to choicing their conjugate-inverse partners. Having found $z_0$ in terms of $(p,k,u,v)$ it remains to find the correct scaling of $f^{-1}$. We use the fact that $f^{-1}(\iu u) = 1$.
\begin{align*}
f^{-1}(z) &= C \frac{z-z_0}{z + \bar{z_0}} \\
1 = f^{-1}(\iu u) &= C \frac{\iu u - z_0}{\iu u + \bar{z_0}} \\
f^{-1}(z) &=  \frac{\iu u + \bar{z_0}}{\iu u - z_0} \frac{z-z_0}{z + \bar{z_0}}
\end{align*}
As was previously presented, one can simply take $α = f^{-1}(1)$ and $β = f^{-1}(k^{-1})$ to give formula for the branch points in terms of the new parameters. This correspondence fails when the formula is not well defined, which could potentially occur if $z_0$ were to equal $\iu u$, $-1$ or $-k^{-1}$. The latter two are excluded by the choice of sign of $x$, whereas the first could only occur if $x=0$, which itself only occurs if $u=v$. The correspondence therefore excludes these values.

In the case that one of $f(1)$ or $f(-1)$ is infinite, one needs to transistion to the coordinates $u'$ or $v'$ to maintain valid formulae. Though standard, it is perhaps still of some interest to consider the above geometrical argument in this limit. Suppose that $u' = 0$, which is to say geometrically that $1$ is mapped to infinity by $f$. Then the transformation $f$ takes the line through $1,0,-1$ to a line perpendicular to the imaginary axis, cutting at $f(-1)$. This line is therefore horizontal and so $z_0$ and $\iu v$ have the same imaginary parts. This gives $y=v$ directly. Considering next $p$, the special case of the cross ratio gives that
\begin{align}
p
&= \frac{\abs{z_0 -\iu v}^2}{\abs{1-\iu v}\abs{k^{-1}-\iu v}} \\
&= \frac{x^2}{k^{-1}\sqrt{1+v^2}\sqrt{1+k^2 v^2}} \\
x &= \sqrt{\frac{p w(\iu v)}{k}}.
\end{align}
These are the same formulas for $x$ and $y$ that one arrives by substituting $u = (u')^{-1}$ in the previous formulas and taking a limit $u' \to 0$. Given this modification of $z_0$, the rest of the formula for $f^{-1}$ is as before. The geometry for $f(-1) = \infty$ is similar.

In summary, we have found a new parameter space for the space of spectral curves. Formerly we had considered $\{ (α,β) \in D\times D | α \neq β \}$, whereas the new space is $P$ (defined above \ref{defn:parameter space}). The moduli space $\mathcal{M}$ is a subspace of this parameter space defined implicitly by the two conditions that $p$ and $T$ are constant.

FORMULATE THIS RESULT AS A LEMMA ABOUT DIFFEOMORPHISM? OR IS THAT OVERWROUGHT \todo{this}







Having established formula for the inverse transformation from the new coordinates, it is time to rewrite $T$ in terms of $(p,k,u,v)$. Particularly, we must compute the factors in the third term of $T$. By direct computation
\begin{align*}
\abs{1-α} &= \abs{\frac{(z_0 + \bar{z_0}) (\iu u - 1) }{ (\iu u - z_0) (1 + z_0) }} \\
\abs{1-β} &= \abs{\frac{(z_0 + \bar{z_0}) (k \iu u - 1) }{ (\iu u - z_0) (1 + k z_0) }} \\
%%%%%%%%%%%%%%%%
\abs{α-β} &= \abs{\frac{\iu u + \bar{z_0}}{\iu u - z_0}} \abs{\frac{(z_0 + \bar{z_0}) (k - 1) }{ (1 + z_0) (1 + k z_0) }} \\
&= \abs{\frac{(z_0 + \bar{z_0}) (k - 1) }{ (1 + z_0) (1 + k z_0) }}
\\
%%%%%%%%%%%%%%%%
\abs{1-\bar{α}β} &= \abs{\frac{(z_0 + \bar{z_0}) (k + 1) }{ (1 + z_0) (1 + k z_0) }}.
\end{align*}
where the expressions were simplified using the observation that
\[
\abs{\frac{\iu u + \bar{z_0}}{\iu u - z_0}} = 1,
\]
because the denominator is the conjugate of the numerator. Next $ν = f^{-1}(\infty)$ and
so
\begin{align*}
ν &= \frac{\iu u + \bar{z_0}}{\iu u - z_0} \\
\frac{1+ν}{1-ν} &= -\iu \sqrt{\frac{w(u)}{pw(v)}}\sign{(u-v)} \\
\frac{1-ν}{1+ν} &= \iu \sqrt{\frac{pw(v)}{w(u)}}\sign{(u-v)} \\
\frac{4ν}{1-ν^2} &= \frac{1+ν}{1-ν} - \frac{1-ν}{1+ν} \\
&= -\iu \frac{pw(v) + w(u)}{\sqrt{pw(u)w(v)}} \sign{(u-v)}.
\end{align*}
These expressions combine to give
\begin{align*}
\frac{\abs{1-α}\abs{1-β}}{\abs{α-β} + \abs{1-\bar{α}β}} \frac{4ν}{1-ν^2}
&= \frac{1}{2}\abs{\frac{ (z_0 + \bar{z_0}) (1 - \iu u) (1-k\iu u) }{ (\iu u - z_0)^2 }}
    \times -\iu \frac{pw(v) + w(u)}{\sqrt{pw(u)w(v)}} \sign{(u-v)} \\
&= \frac{ \sqrt{pw(u)w(v)} }{ \abs{u - v}}
    \times -\iu \frac{pw(\iu v) + w(\iu u)}{\sqrt{pw(u)w(v)}} \sign{(u-v)} \\
&= -\iu \frac{ w(\iu u) }{ u - v} (1+R^2)  \\
&= -\iu \frac{ pw(\iu v) + w(\iu u) }{ u - v},
\end{align*}
and hence that
\begin{align*}
\labelthis{eqn:Teqn2}
T(p,k,u,v) = \frac{2p}{π\iu} \left[ E \tilde{F}(v) - K \tilde{E}(v) \right] - \frac{2}{π\iu}\left[ E \tilde{F}(u) - K \tilde{E}(u) \right] - \frac{2}{π} K \frac{p w(v) + w(u)}{u-v}.
\end{align*}


NEEDS SOMETHING TO CLOSE OUT THIS SECTION, MAYBE MOVE DISCUSSION FROM MIDDLE TO END. OR ADD DISCUSSION OF ANALYTICITY AT U=INFTY.\todo{this}





From now on we will only consider the parameter space for a fixed value of $p$. \todo{change the definition of $P$?} Topologically it is the product of an interval cross and an annulus, a feature not easily seen from the $α,β$ description. The interval component is obvious, it comes from $k\in (0,1)$. To see the annulus part, consider $\RP^1 \times \RP^1$. Topologically $\RP^1$ is just a circle, so this is a torus. The line $u=v$ can be represented as the line where the toroidal and poloidal angles are equal, and removing this line leaves an annulus.

A more instructive way of visualising $P$ is to think of it as a solid cylinder with a line along the central axis removed. One should think of the `radius' of point being given by $1-k$, so that the central axis is identified with the value $k=1$. To motivate this, consider the formula for $k$.
\[
k = \frac{\abs{1-\bar{α}β}-\abs{α-β}}{\abs{1-\bar{α}β}+\abs{α-β}}.
\]
In the limit as $α \to β$, this formula says that $k \to 1$. From the equation of $p$, for a fixed value of $p$, the subspace $α=β$ is an arc. In this visualisation we are imagining this arc as the central axis of the cylinder. In latter chapters, the interesting structure of the moduli space in this limit will be investigated.

Another way that this model is useful is it allows us to see the twisting in the space of spectral curve and the essential appearence of the period behaviour. Take a small loop around the central axis, which is to say a certain path in the space of spectral curves with $k$ close to $1$. This implies that $α$ and $β$ must be close together, and so the branch point circle is approximately a line.
As we move around the axis, the values of $f(1),f(-1)$ are moving through the values of $\RP^1$.
Another way to put this is that the points $μ,ν$ on the unit circle in the $ζ$-plane, which define the origin and infinity of the $z$-plane, are moving around the unit circle. Every time $ν$ crosses either $1$ or $-1$, we have moved across a branch cut, and an extra period term appears in our formulas.
We can infer from the rotation of $μ,ν$ that the branch point `line' is rotating about the point $(α, β)$, which in turn means that $α$ and $β$ are circling one another.
After one full rotation, $α$ and $β$ have returned to their original positions, but $T$ has been incremented by $1+p$. Though you have returned to your original spectral curve, you are now in a different point in the space of spectral data.







\section{Universal Cover of the Parameter Space}

To deal with the multivaluedness of $T$ on the parameter space $P$, we shall move to its universal cover $\tilde{P}$, on which $T$ can be realised as a single valued function $\tilde{T}$. Let the coordinates on $\tilde{P}$ be
\[
\{(k,\tilde{u},\tilde{v}) \in (0,1)\times\R\times\R \mid \tilde{u} < \tilde{v} < \tilde{u} + 2π \}.
\]
The projection map $p : \tilde{P} \to P$ is given by
\begin{align*}
u &= \tan \frac{\tilde{u}}{2},
\quad  u' = \tan \frac{π - \tilde{u}}{2} = \cot \frac{\tilde{u}}{2}, \\
v &= \tan \frac{\tilde{v}}{2},
\quad  v' = \tan \frac{π - \tilde{v}}{2} = \cot \frac{\tilde{v}}{2}, \\
k &= k.
\end{align*}
Define the winding number $N : \R \to \Z$ of a number $x$ to be the integer $N(x)$ such that $-π < x - 2πN(x) \leq π$.
\[
\tilde{T}(k,\tilde{u},\tilde{v}) = 2pN(\tilde{v}) - 2N(\tilde{u}) + T(k,u,v).\tidle
\]































\section{Derivaitves}

In the interest of having manageable formulae, we recall the definition of $w(y)^2 = (1+y^2)(1+k^2 y^2)$. Using primes to indicate coordinates at infinity (ie $y' = 1/y$), not derivatives, we introduce $w'(y')^2 = (1 + (y')^2)(k^2 + (y')^2)$. As $F(z;k)$ and $E(z;k)$ are parameter integrals in $z$, we have that
\[
\Partial{}{u} F(\iu u; k) = \frac{\iu}{w(u)},\;\;\;
\Partial{}{u} E(\iu u; k) = \iu\frac{1+k^2 u^2}{w(u)},
\]
and
\[
\Partial{}{u} w(u)
= \Partial{}{u} \sqrt{1+u^2}\sqrt{1+k^2 u^2}
= \frac{(1+k^2)u + 2k^2 u^3}{w(u)}.
\]
The other derivatives of elliptic integrals are calculated in appendix \ref{chp:Elliptic Integrals}. Equipped with these tools, the calculation of the derivatives of $T$ is elementary if tedious.
\begin{align*}\label{dTdk}
\frac{π}{2}\Partial{T}{k}
&= \frac{1}{k(1-k^2)}\frac{1}{u-v} \left[ p \sqrt { \frac{1+v^2}{1+k^2v^2}} + \sqrt { \frac{1+u^2}{1+k^2u^2} } \right] \left[ -(1+k^2uv) E + (1-k^2)K \right]
\end{align*}
\begin{equation}\label{dTdu}
\frac{π}{2}\Partial{T}{u}
= -\frac{E}{w(u)} + \frac{pK w(v)}{(u-v)^2} + \frac{K}{w(u)(u-v)^2}\left[1 + u^2 - uv + v^2 + k^2 uv + k^2 u^2v^2 \right]
\end{equation}
\begin{equation}\label{dTdv}
\frac{π}{2}\Partial{T}{v}
= \frac{pE}{w(v)} - \frac{K w(u)}{(u-v)^2} - \frac{pK}{w(v)(u-v)^2}\left[1 + u^2 - uv + v^2 + k^2 uv + k^2 u^2v^2 \right]
\end{equation}
\begin{equation}\label{dTdv'}
\frac{π}{2}\Partial{T}{v'}
= -\frac{pE}{w'(v')} + \frac{K w(u)}{(uv'-1)^2} + \frac{pK}{w'(v')(uv'-1)^2}\left[1 + k^2u^2 - uv' + k^2uv' + (v')^2 + u^2(v')^2 \right]
\end{equation}
While $T$ may be multivalued on $P$, its derivatives are not. The ambiguous constant is removed by differentiation.

\begin{lem}
    \label{lem:deriv no zeroes}
The functions
\[
U(x,k,u,v) := -(u-v)^2 E + xKw(u)w(v) + K\left[ 1 + u^2 - uv + k^2 uv + v^2 + k^2 u^2 v^2 \right]
\]
and
\[
V(x,k,u,v') := -(uv'-1)^2 E + xKw(u)w'(v') + K\left[ 1 + k^2u^2 - uv' + k^2 uv' + (v')^2 + u^2 (v')^2 \right]
\]
have no zeroes for $x \geq 1$, $k\in (0,1)$, $u,v,v' \in \R$.
\begin{proof}
We shall prove this statement by showing that the two functions are in fact always positive. The first step is to eliminate $E$. From \cite{Anderson}, we have the sharp inequality $E < k^2 + (1-k^2)K$. If we apply the crude estimate that $K>1$ (in fact $K > π/2$), we see that
\[
E < k^2 + (1-k^2)K = K + k^2 (1-K) < K.
\]
Also using the assumption that $x\geq 1$, one can then begin
\begin{align*}
U(x,k,u,v)
&= -(u-v)^2 E + xKw(\iu u)w(\iu v) + K\left[ 1 + u^2 - uv + k^2 uv + v^2 + k^2 u^2 v^2 \right] \\
&> -(u-v)^2 K + Kw(\iu u)w(\iu v) + K\left[ 1 + u^2 - uv + k^2 uv + v^2 + k^2 u^2 v^2 \right] \\
&= K \left[ w(\iu u)w(\iu v) + 1 + (1 + k^2) uv + k^2 u^2 v^2 \right]
\end{align*}
This formula is almost sufficent. The only term that could be negative is the one featuring $uv$. However, a lower bound for the square root terms is
\[
w(u) = \sqrt{1 + (1+k^2)u^2 + k^2u^4} > \sqrt{(1+k^2)u^2} = \sqrt{(1+k^2)}\abs{u},
\]
so
\begin{align*}
U(x,k,u,v)
&> K \left[ (1+k^2)\abs{uv} + 1 + (1 + k^2) uv + k^2 u^2 v^2 \right] \\
&\geq K \left[ 1 + k^2 u^2 v^2 \right].
\end{align*}
This is positive, so $U$ is without zeroes. Taking a similar approach for $V$
\[
V(x,k,u,v') > K \left[ w(u) w'(v') + (1+k^2)uv' + k^2u^2 + (v')^2\right].
\]
In addition to the above bound for $w(u)$, we also now require a lower bound for $w'(v')$:
\[
w(v') = \sqrt{k^2 + (1+k^2)(v')^2 + (v')^4} > \sqrt{(1+k^2)(v')^2} = \sqrt{(1+k^2)}\abs{v'},
\]
which can now be applied to show that
\[
V(x,k,u,v') > K \left[k^2u^2 + (v')^2\right].
\]
This establishes that $V$ has no roots either.
\end{proof}
\end{lem}

This result is of interest because it shows that the $v$ and $v'$ derivatives of $T$ are nonzero :
\begin{align*}
\frac{π}{2}\Partial{T}{v} &= \frac{-p}{w(v)(u-v)^2} U\bra{ \frac{1}{p},k,u,v }, \\
\frac{π}{2}\Partial{T}{v'} &= \frac{p}{w'(v')(uv'-1)^2} V\bra{ \frac{1}{p},k,u,v' }.
\end{align*}






\section{Coordinates on $\mathcal{M}$}
\begin{lem}
$(k, \tilde{u})$ are coordinates on $\mathcal{M}$.
\begin{proof}
We prove this by direct application of the implict function theorem. Consider the function $\tilde{T}(k,\tidle{u},\tilde{v})$ on $\tilde{P}$. Consider the two formula of the its derivatives
\[
\Partial{\tilde{T}}{\tilde{v}}
= \Partial{T}{v}\frac{dv}{d\tilde{v}}
= \frac{1}{2}\sec^2\bfrac{\tilde{v}}{2} \Partial{T}{v},
\]
for $\tilde{v} \not\in π + 2π\Z$ and
\[
\Partial{\tilde{T}}{\tilde{v}}
= \Partial{T}{v'}\frac{dv'}{d\tilde{v}}
= -\frac{1}{2}\csc^2\bfrac{\tilde{v}}{2} \Partial{T}{v'},
\]
for $\tilde{v} \not\in 2π\Z$. As witnessed in Lemma \ref{lem:deriv no zeroes}, neither of the partial derivatives of $T$ with respect to $v$ or $v'$ are every zero, and neither are $\sec$ or $\csc$. Hence $\parital \tilde{T} / \partial \tilde{v}$ is never zero. By the implicit function theorem therefore, there is a function $g$ such that the level set $\tilde{T}=q$ is a graph of the form $(k, \tilde{u}, g(k,\tilde{u}))$.
\end{proof}
\end{lem}

The topological implication of this is that the moduli space $\mathcal{M}$ is diffeomorphic to a product of $(0,1)$ and $\R$.
