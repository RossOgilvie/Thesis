%!TEX root = thesis_single.tex

\section{Genus One Moduli Space}
\label{sec:Moduli Space}
\epigraph{All analytic functions are alike; each non-analytic function is unhappy in its own way.}

\subsection{Intro}
\label{sub:Intro}

An outline:
Give the outline
Introduce the condition.
Reformulate in coordindates more suited to calculation.
Talk about shape of this parameter space. Deficiencies.
Cut open this space.
Extend past boundary.
Bound away from other edges.
Compactify.
Morse theory - interior and edges.
Swap to other piece.
Glue isotropies together.
Go to the Winchester, have a pint, and wait for this to blow over.

The strategy is to use Morse theory to exhibit the topology of the moduli space of spectra data. More explicitly, we will prove that the moduli surface retracts onto its boundary. We will take a height function and on the interior will show that there are no critical points. TODO: link in the relevant points of stratified Morse theory.









\subsection{The Conditions}

There are two conditions that must be satisfied by a spectral curve to admit differentials, and if a spectral curve admits differentials then it admits a $\Z^2$ lattice of them. The first condition is the same as before. The exact differential leads to the condition that
\[
p := \frac{\abs{1-α}\abs{1-β}}{\abs{1+α}\abs{1+β}} \in \Q.
\]
In other words that $p$ is rational. The other condition, concerning the differential with period $2π\iu g$, cannot be evaluated with elementary functions. If instead we keep the integrals in the equations, we arrive at the condition
\[
2π\iu q := 2π\iu \bra{\frac{p\tilde{Γ}^- - \tilde{Γ}}{g}} = p \int_{γ_-} Θ^2 - \int_{γ_+} Θ^2
\]
for some rational $q$. This however is only a locally defined expression because of the multivaluedness of the integrals. But as $q$ is rational, it must be locally constant and so serves as a constraint. Recall the expression for the integrals.
\begin{align*}
\int_{γ_+} Θ^2 &= 4 E(k) \tilde{F}(f(1);k) - 4 K(k) \tilde{E}(f(1);k) - 4K(k) \frac{\abs{1-α}\abs{1-β}}{\abs{α-β} + \abs{1-\conj{α}β}} \frac{1+ν}{1-ν} \\
\int_{γ_-} Θ^2 &= 4 E(k) \tilde{F}(f(-1);k) - 4 K(k) \tilde{E}(f(-1);k) - 4K(k) \frac{\abs{1+α}\abs{1+β}}{\abs{α-β} + \abs{1-\conj{α}β}} \frac{1-ν}{1+ν}.
\end{align*}
From this we define
\begin{align*}
T &:=  p \int_{γ_-} Θ^2 - \int_{γ_+} Θ^2 \\
&=  p \left[ 4 E \tilde{F}(f(-1)) - 4 K \tilde{E}(f(-1)) - 4K \frac{\abs{1+α}\abs{1+β}}{\abs{α-β} + \abs{1-\conj{α}β}} \frac{1-ν}{1+ν} \right] \\
&\qquad\qquad   - \left[ 4 E \tilde{F}(f(1)) - 4 K \tilde{E}(f(1)) - 4K \frac{\abs{1-α}\abs{1-β}}{\abs{α-β} + \abs{1-\conj{α}β}} \frac{1+ν}{1-ν} \right] \\
&=  p \left[ 4 E \tilde{F}(f(-1)) - 4 K \tilde{E}(f(-1)) \right] - \left[ 4 E \tilde{F}(f(1)) - 4 K \tilde{E}(f(1)) \right] \\
&\qquad\qquad    - 4K\left[ p\frac{\abs{1+α}\abs{1+β}}{\abs{α-β} + \abs{1-\conj{α}β}} \frac{1-ν}{1+ν} - \frac{\abs{1-α}\abs{1-β}}{\abs{α-β} + \abs{1-\conj{α}β}} \frac{1+ν}{1-ν} \right]  \\
&=  p \left[ 4 E \tilde{F}(f(-1)) - 4 K \tilde{E}(f(-1)) \right] - \left[ 4 E \tilde{F}(f(1)) - 4 K \tilde{E}(f(1)) \right] \\
&\qquad\qquad    - 4K \frac{\abs{1-α}\abs{1-β}}{\abs{α-β} + \abs{1-\conj{α}β}} \left[ \frac{1-ν}{1+ν} - \frac{1+ν}{1-ν} \right]  \\
&=  p \left[ 4 E \tilde{F}(f(-1)) - 4 K \tilde{E}(f(-1)) \right] - \left[ 4 E \tilde{F}(f(1)) + 4 K \tilde{E}(f(1)) \right] + 4K \frac{\abs{1-α}\abs{1-β}}{\abs{α-β} + \abs{1-\conj{α}β}} \frac{4ν}{1-ν^2} \labelthis{Teqn} \\
\end{align*}









\subsection{Marked Point Coordinates}
\label{sub:Reformulate}

Finding (potential) critical points involves essentially differentiating the above expressions. It is therefore prudent to adopt coordinates on the space of spectral curves more suited to the task than $(α,β)$. An obvious first coordinate is $p$ itself, as then we can enforce the first condition simply by holding it constant. Elliptic integrals are the most difficult part of the above to differentiate, so to minimise our labour we choose the other three (real) coordinates to be $k$, $\iu σ = f(1)$ and $\iu τ = f(-1)$ for $σ,τ \in \R^\infty$. First we must show these do indeed parameterise the space of spectral curves. We proceed by exhibiting $α,β$ as a function of these.

Denote $z_0 = f(0)$, the image of $ζ=0$ in the $z$-plane. This is an important point, because the new parameters all lie in the $z$-plane so we must construct the inverse transformation $f^{-1}$, which is a scalar multiple of
\[
\frac{z-z_0}{z + \conj{z_0}}
\]
using the real structure to locate $f(\infty) = - \conj{z_0}$. Note the following trick using cross ratios.
\[
\abs{\frac{α-1}{α+1}}
= \abs{\frac{α-1}{α+1}} \abs{\frac{0+1}{0-1}}
= \abs{ \cross{α}{0}{1}{-1} }
= \abs{ \cross{1}{z_0}{\iu σ}{\iu τ} }
= \abs{\frac{1-\iu σ}{1 - \iu τ}} \abs{\frac{z_0 - \iu τ}{z_0 - \iu σ}}
\]
Substituting $β$ for $α$ changes the $1$ to $k^{-1}$. Together these give the equation of a circle
\begin{align*}
p &= \abs{\frac{1-\iu σ}{1 - \iu τ}}\abs{\frac{1 - k\iu σ}{1 - k\iu τ}} \abs{\frac{z_0 - \iu τ}{z_0 - \iu σ}}^2 \\
\abs{\frac{z_0 - \iu τ}{z_0 - \iu σ}} ^2
&= p \frac{\sqrt{1+τ^2}}{\sqrt{1+σ^2}}\frac{\sqrt{1+k^2τ^2}}{\sqrt{1+k^2σ^2}} \\
&= p \frac{w(\iu τ)}{w(\iu σ)} \\
&=: R^2
\end{align*}
In the $ζ$-plane, the points $-1,0,1$ all lie on a straight line that is perpendicular to the unit circle at both $-1$ and $1$, and invariant under the real involution. Applying the M\"obius transformation $f$ we can therefore say that $\iu τ, z_0, \iu σ$ all lie on a circle that is perpendicular to the imaginary axis and symmetric under reflection in the imaginary axis. Let $z_0 = x+\iu y$. Then $z_0$ lies on the circle
\[
x^2 + \bra{ y - \frac{σ+τ}{2} }^2 = \frac{(σ-τ)^2}{4},
\]
which simplifies to the relation
\[
x^2 + y^2 = y(σ+τ) - στ.
\]
The circle previously computed that $z_0$ lies on gives the relation
\[
x^2 + y^2 + 2y \frac{R^2 σ - τ}{1-R^2} + \frac{τ^2 - R^2 σ^2}{1-R^2}.
\]
Together they have the solution
\[
x = \frac{R}{1 + R^2} \abs{σ-τ},\; y = \frac{R^2σ + τ}{1+R^2}
\]
where the sign of $x$ is chosen to make lie in the right half of the $z$-plane. Choosing the opposite sign corresponds to $f(\infty) = -\conj{z_0}$, which also lies on both circles, whereas this choice amounts to choosing the branch points $α,β$ \emph{inside} the unit circle. Having found $z_0$ in terms of $p,k,σ,τ$ it remains to find the correct scaling of $f^{-1}$. We use the fact that $f^{-1}(\iu σ) = 1$.
\begin{align*}
f^{-1}(z) &= C \frac{z-z_0}{z + \conj{z_0}} \\
1 = f^{-1}(\iu σ) &= C \frac{\iu σ - z_0}{\iu σ + \conj{z_0}} \\
f^{-1}(z) &=  \frac{\iu σ + \conj{z_0}}{\iu σ - z_0} \frac{z-z_0}{z + \conj{z_0}}
\end{align*}
Now, one can simply take $α = f^{-1}(1)$ and $β = f^{-1}(k^{-1})$. We can compute the factors in the third term of $T$. Particularly, $ν = f^{-1}(\infty)$ and
\[
\abs{\frac{\iu σ + \conj{z_0}}{\iu σ - z_0}} = 1
\]
so that
\begin{align*}
\labelthis{Teqn2}
T(p,k,σ,τ) = p \left[ 4 E \tilde{F}(\iu τ) - 4 K \tilde{E}(\iu τ) \right] - \left[ 4 E \tilde{F}(\iu σ) - 4 K \tilde{E}(\iu σ) \right] - 4\iu K \frac{p w(\iu τ) + w(\iu σ)}{σ-τ}.
\end{align*}
In the parameter space $P = \{(p,k,σ,τ)\mid p>0, 0 < k <1, σ,τ \in\RInf, σ \neq τ\}$, the moduli space is defined implicitly by the two conditions that $p$ is constant, as is $T$.












\subsection{The Cut space}
Consider the parameter space $P$, defined previously, for a fixed value of $p \leq 1$. Topologically it is a fatten annulus. Suppose we make a cut along the plane $σ=\infty$. Now we have a box $B = \{(k,σ,τ)\mid 0 < k <1, σ \in \R , τ \in\RInf, σ \neq τ\}$. We will soon introduce an auxillary coordinate to compensate for the fact that $τ$ is not a coordinate on the plane $τ=\infty$. Our aim will be to examine the moduli space within this box, given as the level surfaces $T=q$ for rational $q$, by sweeping through it with a height function $σ$. To apply Morse theory, we require a proper map and so we will need to compactify this space. First we shall examine the behaviour of the function as we approach the boundary of the box.

For $k\in(0,1)$ as $τ$ approaches $σ$ from either above or below, the $1/(σ-τ)$ factor causes $T \to \pm\infty$. If $k \to 0$ then the period terms dissapear and $T$ limits to an elementary function, namely
\[
T(0,σ,τ) = - 2π\iu \frac{p \sqrt{1+τ^2} + \sqrt{1+σ^2}}{σ-τ}.
\]
That leaves only the limit $k \to 1$ (we needn't consider $σ\to \infty$). This one is more interesting.
\begin{align}
T
&= 4p E \tilde{F}(\iu τ) - 4 E \tilde{F}(\iu σ) + 4 K \left[ - p\tilde{E}(\iu τ) + \tilde{E}(\iu σ) - \iu\frac{p w(\iu τ) + w(\iu σ)}{σ-τ} \right] \\
&\to 4\iu p \atan τ - 4 \iu \atan σ - \lim_{k\to 1} 4\iu K \frac{(1+p)(στ+1)}{σ-τ}.
\end{align}
In the limit, $K$ has a logarithmic singularity. This could be canceled out if simultaneously $στ \to -1$ (or if $σ \equiv 0$, then $τ \to \infty$), but otherwise it causes $T$ to tend to infinity. At this point, let us normalise the $τ$ coordinate thusly
\[
τ' = \frac{στ+1}{σ-τ},
\]
so that the surface $στ = -1$ is normalised to $τ'=0$ and the boundaries $τ \to σ^-$ and $τ \to σ^+$ become $τ' \to -\infty$ and $τ' \to \infty$ respectively.
Consider, for small positive $ε$ the family of surfaces defined parametrically in $(σ,ε)$ by
\[
τ' = Cε,\; k = 1 - \exp \bra{-1/ε}.
\]
Along such a surface, $ε\to 0$ as $k\to 1$ and the limiting value of $T$ is
\[
T \to 2 π \iu - 4\iu (1+p) \atan σ + 4\iu (1+p)C.
\]
Before we compactify the box $B$, we should `blowup' the line $\{k=0,τ'=0\}$ so these surfaces do not have a common intersection and so that $T$ has a well defined limit on the boundary. Focus on a particular value of $T$, say $q$. We aim to show that the moduli surface $\M = T^{-1}(q) \subset B$ compactifies in $\overline{B}$ to $\overline{\M} = \M \cup \{(k=1,σ,τ'=0)\} \cup \ell$, where $\ell$ is a curve in the plane $k=0$ that we will describe subsequently.










\subsection{Extension}
It is clear from the above that $B$ is not suitable on its own to be the ambient space when doing Morse theory because $T$ is badly behaved on the boundary. Instead, we will think of $B$ as one coordinate patch of a larger manifold $M$ on which $T$ is well defined. A second coordinate patch shall be given in terms of $(ε,σ',C)$ for $ε\in (-1,1)$, $σ' \in \R$ and $C\in \R$. For $ε>0$,
\begin{align}
k &= 1 - \exp\bra{-\frac{1}{ε}}
& ε &= -1/ \ln (1-k) \\
σ &= σ'
& σ' &= σ \\
τ' &= Cε
& C &= -\ln (1-k) τ'
\end{align}
describes the transition between the two patches. It is useful to note the how the derivative operators can be related between the two coordinate systems.
\begin{align}
\Partial{}{ε}
&= \Partial{k}{ε}\Partial{}{k} + \Partial{τ'}{ε}\Partial{}{τ'} \\
&= -[\ln(1-k)]^2 (1-k) \Partial{}{k} - \ln(1-k) τ' \Partial{}{τ'} \\
\Partial{}{σ'}
&= \Partial{σ}{σ'}\Partial{}{σ} \\
&= \Partial{}{σ} \\
\Partial{}{C}
&= \Partial{τ'}{C}\Partial{}{τ'} \\
&= - [\ln(1-k)]^{-1} \Partial{}{τ'}
\end{align}
We will work towards giving $T$ a smooth extension to the manifold $M$. The theorem we intend to use is due to Seeley \cite{Seeley1964}: a smooth function in a half space where it and all its derivatives have continuous limits to the boundary can be extended to the whole space. First we establish the limits of certain classes of functions.



The function $T$ is not of this form, but its limit was computed previously and indeed the coordinate system $(ε,σ',C)$ was purposely designed for that it would have a limit. We now check that its derivatives fall into this paradigm.
\begin{align}
\Partial{T}{σ'}
&= -\frac{E}{w(\iu σ)} + \frac{σ(1+k)}{w(\iu σ)(σ-τ)}K(1-k) + p \frac{E}{w(\iu τ)}\frac{1+τ^2}{1+σ^2} + p \frac{τ(1+τ^2)(1+k)}{w(\iu τ)(σ-τ)(1+σ^2)}K(1-k)
\Partial{T}{ε}
&= -C p \sqrt{\frac{1+τ^2}{1+k^2τ^2}}\frac{σCε(σCε-1)}{(C^2ε^2+1)(σ+Cε)}
\end{align}







\subsection{Interior}
\label{sub:Interior}

\begin{lem}
The function
\[
U(x,k,σ,τ) := -(σ-τ)^2 E + xKw(\iu σ)w(\iu τ) + K\left[ 1 + σ^2 - στ + k^2 στ + τ^2 + k^2 σ^2 τ^2 \right]
\]
has no zeroes for $x \geq 1$, $k\in (0,1)$, $σ,τ \in \R$, $σ\neq τ$.
\begin{proof}
From \cite{Anderson}, we have the sharp inequality $E < k^2 + (1-k^2)K$. If we apply the crude estimate that $K>1$ (in fact $K > π/2$), we see that $E < K$. Also using the assumption that $x\geq 1$, one can then begin
\begin{align*}
U(x,k,σ,τ)
&= -(σ-τ)^2 E + xKw(\iu σ)w(\iu τ) + K\left[ 1 + σ^2 - στ + k^2 στ + τ^2 + k^2 σ^2 τ^2 \right] \\
&\geq -(σ-τ)^2 K + Kw(\iu σ)w(\iu τ) + K\left[ 1 + σ^2 - στ + k^2 στ + τ^2 + k^2 σ^2 τ^2 \right] \\
&= K \left[ w(\iu σ)w(\iu τ) + 1 + (1 + k^2) στ + k^2 σ^2 τ^2 \right]
\end{align*}
A lower bound for the square root terms is
\[
w(\iu σ) = \sqrt{1 + (1+k^2)σ^2 + k^2σ^4} \geq \sqrt{(1+k^2)σ^2} = \sqrt{(1+k^2)}\abs{σ},
\]
so
\begin{align*}
U(x,k,σ,τ)
&\geq K \left[ (1+k^2)\abs{στ} + 1 + (1 + k^2) στ + k^2 σ^2 τ^2 \right] \\
&\geq K \left[ 1 + k^2 σ^2 τ^2 \right].
\end{align*}
This is positive, so we are done.
\end{proof}
\end{lem}

\begin{lem}
The function
\[
V(x,k,τ) := - E + xkKw(\iu τ) + K\left[ 1 + k^2 τ^2 \right]
\]
has no zeroes for $x \geq 0$, $k\in (0,1)$, $τ \in \R$.
\begin{proof}
The estimates are easier than before, as the potentially large and negative $στ$ terms have disappeared. Again, use $E < K$ to immediately arrive at
\[
V(x,k,τ) \geq xkKw(\iu τ) + K k^2 τ^2.
\]
Note that each term is positive.
\end{proof}
\end{lem}









We now compute the partial derivatives of $T$. As $\tilde F(z;k)$ and $\tilde E(z;k)$ are parameter integrals in $z$, we have that
\[
\Partial{}{σ}\tilde F(\iu σ; k) = \frac{\iu}{w(\iu σ)},\;\;\;
\Partial{}{σ}\tilde E(\iu σ; k) = \iu\frac{1+k^2 σ^2}{w(\iu σ)},
\]
and
\[
\Partial{}{σ} w(\iu σ)
= \Partial{}{σ} \sqrt{1+σ^2}\sqrt{1+k^2 σ^2}
= \frac{(1+k^2)σ + 2k^2 σ^3}{w(\iu σ)}.
\]
One then computes that
\begin{equation}\label{dTds}
\frac{1}{4\iu}\Partial{T}{σ}
= \frac{-E}{w(\iu σ)} + \frac{pK w(\iu τ)}{(σ-τ)^2} + \frac{K}{w(\iu σ)(σ-τ)^2}\left[1 + σ^2 - στ + τ^2 + k^2 στ + k^2 σ^2τ^2 \right] = \frac{1}{w(\iu σ) (σ-τ)^2} U (p,k,σ,τ).
\end{equation}
And in a similar manner
\begin{equation}\label{dTdt}
\frac{1}{4\iu}\Partial{T}{τ}
= \frac{pE}{w(\iu τ)} - \frac{K w(\iu σ)}{(σ-τ)^2} + \frac{pK}{w(\iu τ)(σ-τ)^2}\left[1 + σ^2 - στ + τ^2 + k^2 στ + k^2 σ^2τ^2 \right] = \frac{-p}{w(\iu τ) (σ-τ)^2} U (1/p,k,σ,τ).
\end{equation}
If we introduce coordinates at infinity for $σ,τ$, say $σ' = σ^{-1}$ and $τ' = τ^{-1}$, we get the expressions
\begin{align*}
\label{derivesInf}
\frac{1}{4\iu}\left.\Partial{T}{σ'} \right|_{σ' = 0} &= \frac{-1}{k} V (p,k,τ), &\qquad
\frac{1}{4\iu}\left.\Partial{T}{τ} \right|_{σ' = 0} &= 0, \\
\frac{1}{4\iu}\left.\Partial{T}{σ} \right|_{τ' = 0} &= 0, &\qquad
\frac{1}{4\iu}\left.\Partial{T}{τ'} \right|_{τ'=0} &= \frac{-p}{k} V (1/p,k,σ).
\end{align*}

As one of $p$ and $1/p$ must be greater than or equal to one, at least one of these derivatives is nonzero. For example, if $p \geq 1$, then the $σ$ derivatives are nonzero everywhere except the plane $τ=\infty$.

It remains to compute the $k$ partial derivative. We refer to the appendix \todo{add reference} for the derivatives of incomplete elliptic functions.
\[
\Partial{}{k}\tilde F(\iu σ; k) = \frac{\tilde E}{k(1-k^2)} - \frac{\tilde F}{k} - \frac{k}{1-k^2} \iu σ \sqrt { \frac{1+σ^2}{1+k^2σ^2} },\;\;\;
\Partial{}{k}\tilde E(\iu σ; k) = \frac{\tilde E}{k} - \frac{\tilde F}{k},
\]
and
\[
\Partial{}{k}w(\iu σ) = \frac{1}{w(\iu σ)}(1+σ^2)kσ^2.
\]
Somewhat remarkably, the period generating terms have a simple derivative (comparatively so; all the incomplete elliptic integrals cancel)
\[
\Partial{}{k} (E\tilde F - K\tilde E) = - \frac{k}{1-k^2} \iu σ \sqrt { \frac{1+σ^2}{1+k^2σ^2} } E
\]
The full partial derivative is therefore
\begin{align*}
\frac{1}{4\iu}\Partial{T}{k}
&= -\frac{pk}{1-k^2} τ \sqrt { \frac{1+τ^2}{1+k^2τ^2} } E + \frac{k}{1-k^2} σ \sqrt { \frac{1+σ^2}{1+k^2σ^2} } E\\
&\qquad - \frac{1}{σ-τ} \left\{ \frac{E}{k(1-k^2)}\left[ pw(\iu τ) + w(\iu σ) \right] - \frac{p(1+τ^2)}{w(\iu τ)}\frac{K}{k} - \frac{1+σ^2}{w(\iu σ)}\frac{K}{k} \right\} \\
&= \frac{1}{k(1-k^2)}\frac{1}{σ-τ} \left[ p \sqrt { \frac{1+τ^2}{1+k^2τ^2}} + \sqrt { \frac{1+σ^2}{1+k^2σ^2} } \right] \left[ -(1+k^2στ) E + (1-k^2)K \right] \labelthis{dTdk}
\end{align*}






\subsection{Slice}
\label{sub:Slice}
We now wish to sweep most of the interior with the height function $h=σ$. So suppose that $p\leq1$, which from the previous section means that $\partial T / \partial τ$ is nonzero everywhere off $σ=\infty$. Let us begin by analysing the slice $σ = 0$. $T$ restricted to this slice becomes
\[
\tilde T := \frac{1}{4\iu}T|_{σ=0} = -\iu p (E\tilde F(\iu τ) - K \tilde E(\iu τ))
+ K \frac{pw(\iu τ) + 1}{τ}.
\]
To begin, we note the limit of this function on the boundaries of our slice, when $τ\to0^+,0^-$ or $k\to0,1$.
\begin{align*}
\tilde T|_{k \to 0} &= \frac{π}{2}\frac{p\sqrt{1+τ^2} + 1}{τ} \\
\tilde T|_{k \to 1} &\to (\sign τ) \infty \text{ if } τ \neq \infty \\
\tilde T|_{τ \to 0^+} &\to \infty \\
\tilde T|_{τ \to 0^-} &\to -\infty \\
\end{align*}
There are some other peculiarities that are worth noting. If both $k \to 1$ and $τ \to \infty$, then the value of $\tilde T$ depends on the path of approach. We will address this explicitly shortly. Also, as $τ \to \infty$  for values of $k\in(0,1)$, $\tilde T \to (\sign τ) \frac{π}{2}p$. This is not a genuine discontinuity in $T$, but instead we are seeing that $τ=\infty$ is the branch cut we have made.

Using $k$ as a height function for this rectangle, we can see that level sets with a endpoint on the edge $k=0$ (and these points can be found explicitly for any particular value of $T$ you'd like to choose), we can see that these lines have no critical points, and sense the other edges are mostly infinite, which provides a `barrier' that level set cannot approach, these lines converge to the point $τ=\infty, k=1$ on the opposite edge. Further, for precisely the reason that $\tilde T$ is infinite at all other places, these can be the only endpoints for levelsets, and there can be no closed-loop levelsets inside the slice. So the levelsets are a family of curves spanning between these two sides.

Let this box be called $B$. $B = \Set { (k,σ,τ) }{ k \in (0,1), σ \in \R, τ \in \RInf, τ\neq σ\} }$. Its partial closure is $\overline B = \Set { (k,σ,τ) }{ k \in [0,1], σ \in \R, τ \in \RInf, τ\neq σ\} }$. Fix a value $c$ of $T$. We will compute the level set $M := T^{-1}(c)$ within the cut space $σ\neq\infty$. Let $Z$ be the closure of $M$ in $\overline B$. Practically speaking, it is $T^{-1}(c)$ with the $k=0$ and $k=1$ boundary limits included. We will show that it is a Whitney stratified space.




\subsection{Boundary}
\label{sub:Boundary}
We have two coordinate systems for the parameter space of spectral curves. And having proved that it has no handles, its topology is entirely determined by the behaviour on the boundary. However, `the boundary' is not a particularly well defined thing. Just as an open disc can be seen to be the interior of a closed disc, it can equally be seen as the complement of a point on the sphere. And the two coordinate systems we have available to us given different pictures for the shape of the boundary. In essence, the different coordinates suggest different compactifications of the parameter space, and unfortunately neither is entirely satisfactory. In some ways though, the $α,β$ version is easier to understand and nicer, so we begin there.

Fix a value of $p$. This is a `ball' in the space $(α,β)\in D^2$, as for any particular value of
\[
u = \frac{1}{\sqrt{p}} \frac{\abs{1-α}}{\abs{1+α}},
\]
$α$ ranges over an arc of a circle, and likewise does $β$ as constrained by the equation
\[
\frac{\abs{1-β}}{\abs{1+β}} = \frac{\sqrt{p}}{u},
\]
where the product of these two equations is just the definition of $p$. Hence for any $u$ we have the cross section given as the product of two arcs, with the length of the arcs diminishing as $u$ approaches $0$ or $\infty$. As $u$ approaches $0$, this forces the branch points to $S_0 := (1,-1)$, and likewise at $u \to \infty$, $S_\infty := (-1,1)$. In the reverse direction, $α = \pm 1$ if and only if $β = \mp 1$. Hence, these two points divide the unit circle into disconnected upper and lower halves: $\S^1_+$ and $\S^1_-$ respectively.

This ball is an (Australian) football shape; an elongated ball with sharp points at either end, joined by four `\emph{edges}' where both $α$ and $β$ lie in one of $\S^1_\pm$ and four `\emph{faces}' where only one does. The four edges are labelled by two signs, for example $\edge{+-}$ where $α\in \S^1_+$ and $β \in \S^1_-$. The faces are labelled by a letter and a sign, for example $\face{α+}$ denotes $α\in\S^1_+$ and $β\in D$. $\face{α+}$ is bounded by $S_0, S_\infty$ as all the faces are, and by $\edge{++}$ and $\edge{+-}$.

\todo{picture}

This split also shows that the diagonal $Δ=\{(α,α)\}$ intersects each ball in an line, with $u=1$ and both $α$ and $β$ at the same point in their respective arcs. It runs from the middle of $\edge{++}$ to the middle of the opposite edge $\edge{--}$. We will denote the two `polar' points where the diagonal intersects the edges as $P_+$ and $P_-$.

We now turn to how the boundary looks for the $στ$ coordinates. We may proceed more directly. $(σ,τ)$ lie in $\RP^1\times\RP^1\setminus \{σ=τ\}$, which is a torus minus a $(1,1)$-loop. This results in a cylinder. The parameter $k$ may vary in $(0,1)$, and $p$ is fixed so we end up with a fattened cylinder. To make this line up with the previous picture, imagine the cylinder with $k=0$ on the outside, and $k=1$ on the inside. The tube $k=1$ is a blowup of the diagonal.

\todo{picture}

The two ends of the cylinder are the two sides of the line $σ=τ$, for different values of $k$. They correspond to blowups of the two poles $P_\pm$. To see this, let $τ=σ+ε$. Then
\begin{align*}
z_0 &\sim \frac{\sqrt{p}\abs{ε}}{p+1} + \iu \left(σ + \frac{ε}{p+1}\right) \\
α,β &\sim \frac{\iu σ + \conj{z_0}}{\iu σ - z_0} \sim  \frac{1-p}{1+p} + \iu \frac{2\sqrt{p}}{1+p} \sign{ε}
\end{align*}
which are exactly the two polar points.

To understand the exterior boundary is a little more difficult. First note that the $στ$ coordinates cannot express that $α\in\S^1$, for the map $f$ takes $α,\cji{α}$ to $\pm 1$. So there is no sequence of $στ$ parameters that have these two points coming together. The effect is that the faces $\face{α+}$ and $\face{α-}$ and all the edges are crushed in this version of the boundary. The lines $σ=\infty$ and $τ=\infty$ correspond to the points $S_\infty$ and $S_0$ respectively. If we view $(0, σ,τ) \in \RP^1\times\RP^1\setminus \{σ=τ\}$ as a parallelogram, it is cut into two triangles, one with $σ<τ$ and one with $σ>τ$. The two triangles correspond to the two remaining faces $\face{β+}$ and $\face{β-}$ respectively.

\todo{picture}





Now turn to how $k$ varies across the parameter space. On the outside, we have one of $α$ or $β$ in $\S^1$, so $k = 0$. On the diagonal, $α=β$, so $k=1$. How then does $k$ behave at the poles where the diagonal and the exterior intersect? To answer this, let the point of intersection be $(μ,μ)$, and for small $ε$ let
\[
α = μ(1-ε), \qquad β = μ(1-mε)
\]
where $m$ is to be thought of as the angle of approach. For illustrative purposes, take $1>m>0$ so we may ignore absolute value signs.
\begin{align*}
\abs{α-β} &= ε(1-m), \\
\abs{1-\conj{α}β} &\sim ε (m+1), \\
k = \frac{\abs{1-\conj{α}β} - \abs{α-β}}{\abs{1-\conj{α}β} + \abs{α-β}} &\sim m
\end{align*}
The cases for other $m$ and even when $α,β$ do not lie on a ray are similar. So depending on the angle of approach, at the poles $k$ may assume any value. What we conclude then is that surfaces of constant $k$ are cigar shapes with ends at the poles. They surround the diagonal, which is the limit as $k\to 1$.

TODO: explain about $\tilde{f}$ the map which is life $f$ but fixes $β$ instead.












\subsection{Faces and Edges}
\label{sub:Faces and Edges}

TODO: Cleanup this section into a cleaner narative.

We wish to compute the boundary of the moduli space. To do this we must extend the function $T$ to the boundary. Suppose that $β\to\S^1_\pm$. The period terms of $T$ go to zero, leaving only the rational term. \todo{justify/ insert calc}
\[
T|_{\face{β\pm}} = - 2\pi\iu \frac{\abs{1-α}\abs{1-β}}{\abs{α-β}} \frac{2β}{1-β^2}.
\]

\todo{clean up this cut}

Together with the equation for $p$, this defines a curve on the faces. These two conditions are somewhat messy to solve exactly for $α$ as a function of $β$. However, we can get a handle on what is going on by considering the border case that $α$ is also in the unit circle. Let $α=e^{\iu ψ}$, $β=e^{\iu φ}$. Also, we can rationalise everything by setting $s= \tan ψ/2$ and $t= \tan φ/2$. The $p$-condition then becomes
\[
\abs{st} = p
\]
whereas the other can be expanded to
\[
\abs{1-α}^2\abs{1-β}^2 = q^2 \abs{α-β}^2 \sin^2 φ
\]
Solving
\begin{align*}
\bra{2-2\frac{1-s^2}{1+s^2}} \bra{2-2\frac{1-t^2}{1+t^2}}
&= q^2 \frac{4t^2}{(1+t^2)^2}\bra{2-2\frac{1-s^2}{1+s^2}\frac{1-t^2}{1+t^2} - 2\frac{2s}{1+s^2}\frac{2t}{1+t^2}} \\
\bra{(1+s^2)-(1-s^2)} \bra{(1+t^2)-(1-t^2)}
&= q^2 \frac{t^2}{(1+t^2)^2}\bra{2(1+s^2)(1+t^2)-2(1-s^2)(1-t^2) - 2(2s)(2t)} \\
4s^2 t^2
&= q^2 \frac{t^2}{(1+t^2)^2}2\bra{2(s^2+t^2) - 4st} \\
&= q^2 \frac{t^2}{(1+t^2)^2}4(s-t)^2 \\
s^2 (1+t^2)^2 &= q^2 (s-t)^2
\end{align*}
To eliminate $s$ requires using $\abs{st}=p$ which itself requires a choice of sign. Let $ε=\pm 1$ and $s = εp/t$.
\begin{align*}
\frac{p^2}{t^2}(1+t^2)^2 &= q^2 \bra{ε\frac{p}{t}-t}^2 \\
p^2(1+t^2)^2 &= q^2 (εp-t^2)^2 \\
0 &= t^4 (p^2-q^2) + 2t^2 p(p+εq^2) + p^2(1-q^2)\\
t^2 &= \frac{-(p^2+εpq^2) \pm pq(p+ε)}{p^2-q^2} \\
&= p \frac{q-1}{εq+p},\; p \frac{q+1}{εq-p} \\
s^2 &= p \frac{εq+p}{q-1},\; p \frac{εq-p}{q+1}
\end{align*}
One could transform the solution back to the unit circle, but instead visualise the unit circle by its stereographic projection. This projected line is precisely parameterised by $s$ (or $t$), with $α=1$ at the origin and $α=-1$ at infinity, so we consider the solutions as lying the $st$-plane. To extend the visualisation further, the exterior boundary is composed of two pieces $D\times\S^1 \cup \S^1\times D$, which intersect in a torus $\S^1\times\S^1$. The boundary therefore is topologically $\S^3$, with the intersection as an (untwisted) torus and the two pieces being respectively the inside and outsides of that torus. In our planar visualisation, the torus has been stretched out to a plane, so the two pieces of the boundary would correspond to the two sides of the plane in $3$-space (with some points missing). The solutions, which are arcs, with $β\in\S^1$ would be on one side, and the arcs with $α\in\S^1$ would be on the other.

What are the endpoints of the arcs for a fixed value of $p$ or $q$? When we take the square root of $s^2,t^2$, we must use $σ$ to determine the correct pairing of signs. This gives eight possibilities. However, $q\sin φ$ is positive, so $q$ and $t$ have the same sign. These two conditions, and the fact we cannot take the square root of a negative in this situation, mean that there are either two solutions or none. So the arcs have two endpoints on the dividing plane, and what we have found is the coordinates of endpoints of arcs with $β\in\S^1$. The coordinates of the endpoints of arcs from the other side can be found by interchanging the role of $α, β$, which practically means the swapping of $s$ and $t$.

Consider first the case when $p>1$. We write a solution $(s,t)$. For $q>p$ the two points are
\[
\bra{\sqrt{p \frac{q+p}{q-1}}, \sqrt{p \frac{q-1}{q+p}}},\;\;
\bra{\sqrt{p \frac{q-p}{q+1}}, \sqrt{p \frac{q+1}{q-p}}}
\]
For $p>q>1$
\[
\bra{\sqrt{p \frac{q+p}{q-1}}, \sqrt{p \frac{q-1}{q+p}}},\;\;
\bra{-\sqrt{p \frac{-q+p}{q-1}}, \sqrt{p \frac{q-1}{-q+p}}}
\]
For $1>q>-1$ there are no solutions. For $-1>q>-p$
\[
\bra{-\sqrt{p \frac{q-p}{q+1}}, -\sqrt{p \frac{q+1}{q-p}}},\;\;
\bra{\sqrt{p \frac{-q-p}{q+1}}, -\sqrt{p \frac{q+1}{-q-p}}}
\]
And for $-p>q$
\[
\bra{-\sqrt{p \frac{q-p}{q+1}}, -\sqrt{p \frac{q+1}{q-p}}},\;\;
\bra{-\sqrt{p \frac{q+p}{q-1}}, -\sqrt{p \frac{q-1}{q+p}}}
\]
Note that as $p>1$ then by the previous discussion the region $1>q>-1$ is precisely the region that does not occur. These arcs all join up; an arc with $β\in\S^1$ connects on either end to an arc with $α\in\S^1$ and this alternating continues. Adjacent arcs have the same $p$ but differ in $q$ by $p-1$, except the arc that `jumps' the forbidden middle, where the $q$ value differs by $p+1$. Thus, the arcs link up to form a spiral type shape, wrapping around the diagonal and converging as $q \to \infty$ to $(\sqrt{p},\sqrt{p})$ and $(-\sqrt{p},-\sqrt{p})$ at the two ends respectively.
\begin{center}
\includegraphics{thesis_graphics/extboundary.png}
^^ Make this a proper caption: The orange line is a boundary link, the red dots are the limit points. The scale is actually radians of arguments of $α,β$.
\end{center}

In the other case that $p<1$, an entirely similar analysis gives another set of spirals that again converge to $(\sqrt{p},\sqrt{p})$ and $(-\sqrt{p},-\sqrt{p})$. This is the typical situation. There are exceptional cases however, a pair for each $p\neq 1$ and separately when $p=1$. The pair comes from when subtracting $p-1$ reduces $q$ to $\max{1,p}$ or when adding $p-1$ increases $q$ to $-\max{1,p}$. In these cases, there is no arc that spans over the excluded region of $q$ from positive to negative values, instead it trails off to $α=1,β=-1$ or vice versa. \todo{TODO: WRONG!} It then connects to the exceptional arc with `$p$ value' $1/p$ and $q$ of the same sign. That is, if you start on an exceptional link with $p>1$, $q>0$, then $q$ decreases by $p-1$ until $q=p$. Then the next link goes to marked points, where it meets a link coming from $1/p$ and $q$ also positive. Likewise for $p$ and $1/p$ with $q<0$.
\begin{center}
\includegraphics{thesis_graphics/extboundary_sym.png}
^^ Make this a proper caption
\end{center}

The final case is when $p=1$. In this case, there are no spiral links. Instead $p-1$ is $0$ so the $q$ parameter is not changed, meaning that the arcs close up into disjoint loops.
\begin{center}
\includegraphics{thesis_graphics/extboundary_loop.png}
^^ Make this a proper caption
\end{center}



Not to take the computer's word for it, can we prove that these are arcs which begin and end on the edges? It proceeds much like for finding critical points in the interior. We remember that $σ,τ$, with $k=0$, parameterise two of the faces. In these coordinates, $T$ is given by
\[
\tilde{T} := \frac{1}{- 2\pi\iu} T|_{\face{β\pm}} = \frac{p\sqrt{1+τ^2} + \sqrt{1+σ^2}}{σ-τ}
\]
from which we compute the two partial derivatives
\begin{align*}
\Partial{\tilde{T}}{σ} &= \frac{-1}{\sqrt{1+σ^2}(σ-τ)^2} \bra{1 + στ + p\sqrt{1+σ^2}\sqrt{1+τ^2}} \\
\Partial{\tilde{T}}{τ} &= \frac{p}{\sqrt{1+σ^2}(σ-τ)^2} \bra{1 + στ + \frac{1}{p}\sqrt{1+τ^2}\sqrt{1+τ^2}}
\end{align*}
If $x \geq 1$, we can apply the following familiar estimate
\[
1 + στ + x\sqrt{1+σ^2}\sqrt{1+τ^2} \geq 1 + στ + x\abs{στ} \geq 1 + στ + \abs{στ} > 0.
\]
And since one of $p$ or $1/p$ is greater than or equal to $1$, one of these partial derivatives is always non-zero on the face. Hence that the height function to be the other coordinate and we have shown that there are no critical points with respect to that coordinate. This is also sufficient to establish that the level sets are smooth arcs.
\todo{go full stratified morse theory on its arse to finish off the proof. Probably have to swap back to $αβ$ coords to get the `critical points' on the edges are good/isolated/simple normal data.}




\subsection{Marked Points}
\label{sub:Marked Points}

\subsection{Polar Points}
\label{sub:Polar Points}

\subsection{Diagonal}
\label{sub:Diagonal}

TODO: Not strictly necessary for the proof of topology, but include for completeness?
