%!TEX root = thesis.tex

\section{Genus Zero}
\label{sec:Genus Zero}

The moduli of spectral data where the spectral curve is genus zero is an instructive example. We shall denote this space $\mathcal{M}_0$.
We have seen in Lemma \ref{lem:no singularities} that on a spectral curve of genus zero the pair of differentials may not have a common root. It follows from Theorem \ref{thm:moduli manifold} therefore that $\mathcal{M}_0$ is a two-dimensional manifold. Because the all differentials on $\CP^1$ are exact, and all meromomorphic functions are rational, it is possible to describe this space explicitly using only elementary functions.
We shall describe all translation invariant solutions to \eqref{eqn:Hit1.7} that arise from a harmonic map, which we will see are precisely the harmonic maps that correspond to genus zero spectral data.

This example can be divided into two parts. In the first part, we start with \eqref{eqn:Hit1.7} and find all corresponding harmonic maps. In the second part we take these harmonic maps and work through the steps of Section \ref{sub:construction} to produce the associated spectral data. The approach for the first part will be to find all translation invariant solutions to \eqref{eqn:Hit1.7}, for each solution calculate the formula of associated gauge transformation and manipulate this formula into a standard form. Finally, we may then determine which formulae correspond to maps from a torus.

Let $z\in\C$ be a uniformising coordinate on the torus $\C/\Z\langle 1,τ\rangle$. Following \cite[(9.11)]{Hitchin1990}, we write the Higgs field as $Φ = F \,dz$ and the connection as $d''_A = d'' + G \,d\bar{z}$ with respect to the trivialisation arising from the translation action, for constant traceless matrices $F$ and $G$.
With these definitions, equations \eqref{eqn:Hit1.7} become
\[
[F,G] = 0\;\; \text{and}\;\; [G,G^*] = [F,F^*].
\]
If $F$ commutes with is conjugate-transpose then so too does $G$, and they are simultaneously diagonalisable by an $SU(2)$ matrix. This case corresponds to a map from the torus to a circle, which does not produce a spectral curve. Assume therefore that $[F,F^*] \neq 0$. The first equation implies that $F$ and $G$ commute, so we may write $G = λF$. Then $(\abs{λ}^2 - 1)[F,F^*] = 0$ implies that $\abs{λ}=1$. Let $κ\in\S^1$ be such that $\bar{κ}^2=λ$.

Recall that for a solution for \eqref{eqn:Hit1.7}, there is an $\S^1$ family of flat unitary connections \eqref{eqn:flat connections}, which are given in this case by
\[
d_ζ := d_A + ζ^{-1}Φ - ζΦ^*
= d + G\,d\bar{z} - G^*\,dz + ζ^{-1}F\,dz - ζF^*\,d\bar{z}.
\labelthis{eqn:flat connection translation}
\]
As the matrices $F$ and $G$ are constant, for each $ζ \in \C^\times$ one can solve the parallel transport equation $d_ζ X = 0$ explicitly for a vector field $X$ on $\C$.
One may then recover the associated harmonic map as the change of gauge between the two parallel vector fields for the connections $d_1$ and $d_{-1}$, as in \eqref{eqn:gauge change}.
We may therefore write the harmonic map as
\begin{align*}
f(z) = \exp( &(κz + \bar{κ}\bar{z})(-\bar{κ}F + κF^*) ) \\
&\cdot \exp( (κz - \bar{κ}\bar{z})(-\bar{κ}F - κF^*) ).
\end{align*}
We may simplify this formula by making the following substitutions. On the torus, we make a change of coordinates to $w = κz$ and write $w$ as the sum of its real and imaginary parts $w_R + \iu w_I$. We note that $sl_2\C$ can be decomposed as a direct sum $su_2 \oplus \iu su_2$. Write $\bar{κ}F = X + \iu Y$ for $X,Y\in su_2$ and we recognise the two expressions $-\bar{κ}F + κF^*$ and $-\bar{κ}F - κF^*$ as taking the real and imaginary parts of $κF$. With these substitutions, the above formula becomes
\[
f(w) = \exp( -4w_R X ) \cdot \exp( 4 w_I Y  ).
\labelthis{eqn:genus zero simple map}
\]
The matrix exponential of a $su_2$ matrix is of course an $SU(2)$ matrix, and a line through the origin of $su_2$, such as $\Set{-4w_R X }{w_R \in \R}$, is mapped to a one parameter subgroup of $SU(2)$, a circle. Immediately therefore we can see \eqref{eqn:genus zero simple map} as the product of two circles. As we vary $w_R$ the image of $\exp (-4w_R X)$ is a circle and so for a fixed value of $w_I$, the image of $f$ is the right translation of this circle by $\exp (4w_I Y)$. The same is true if we fix $w_R$ and vary $w_I$.

This is a map from the plane $w\in\C$ to $SU(2)$. We would like to know when it descends to a map from the torus. Thus we must compute the periodicity of this map, and in particular we need to know how to compute explicitly the matrix exponential of an $su_2$ matrix $Z$. The determinant of an $su_2$ matrix is always positive, so let the norm of $Z$ be given by the square root of its determinant. We may define the unit matrix $\hat{Z}$ in direction $Z$ to be $Z$ divided by its norm. If we note that $Z^2 = -\abs{Z}^2 I$, the matrix exponential may be written concisely as
\[
\exp Z = I \cos \abs{Z} + \hat{Z} \sin \abs{Z}.
\]
Given this explicit formula for the matrix exponential, we now compute the periods of the map $f$. For an $su_2$ matrix $Z$, $\exp Z = I$ when $\abs{Z} \in 2π\Z$. We may identify $su_2$ with the imaginary quaternions $\R\langle i,j,k\rangle$ via the standard basis of $su_2$
\[
σ_1 = \begin{pmatrix}
\iu & 0 \\ 0 & -\iu
\end{pmatrix} \mapsto i, \;
σ_2 = \begin{pmatrix}
0 & 1 \\ -1 & 0
\end{pmatrix}\mapsto j, \;
σ_3 = \begin{pmatrix}
0 & \iu \\ \iu & 0
\end{pmatrix} \mapsto k.
\]
Under this identification, matrix multiplication agrees with quaternionic multiplication. In particular, if $X$ and $Y$ are linearly independent matrices, then the set $\{X,Y,YX, I\}$ is linearly independent. An element $w\in\C$ is a period of $f$ exactly when
\begin{align*}
f(w) = I \cos(4w_R\abs{X})&\cos(4w_I\abs{Y})
- \hat{X}\sin(4w_R\abs{X})\cos(4w_I\abs{Y}) \\
&+ \hat{Y}\cos(4w_R\abs{X})\sin(4w_I\abs{Y})
- \hat{X}\hat{Y}\sin(4w_R\abs{X})\sin(4w_I\abs{Y}) = I
\end{align*}
Squaring and adding the coefficients of $\hat{Y}$ and $\hat{Y}\hat{X}$ together shows that $\sin^2(4w_I\abs{Y}) = 0$, and likewise for $\hat{X}$ and $\hat{Y}\hat{X}$ that $\sin^2(4w_R\abs{X}) = 0$. Let $4w_R\abs{X} = πk$ and $4w_I\abs{Y} = πl$. Then the coefficient of $I$ requires that
\[
1 = \cos(4w_I\abs{Y})\cos(4w_R\abs{X}) = (-1)^k (-1)^l = (-1)^{k+l}.
\]
Thus the lattice of periods, expressed in the variable $w$, is generated by
\[
κ_0 := \frac{π}{4}\bra{\frac{1}{\abs{X}} - \iu\frac{1}{\abs{Y}}},
\;\;\;\text{and}\;\;\;
κ_1 := -\frac{π}{4}\bra{\frac{1}{\abs{X}} + \iu\frac{1}{\abs{Y}}}.
\]

\makefigure{The lattice of periods of $f$.}{thesis_graphics_temp/genus_zero_lattice.png}

The roles of $κ$, $\abs{X}$, and $\abs{Y}$ are now apparent; they determine the conformal type $τ$ and the number of windings of the torus $\C/\Z\langle1,τ\rangle$ onto its image. Only when three parameters are chosen so that $z=1,τ$ are points of this lattice of periods does $f$ descend to a map from the torus. That is, there must be integers $n,m,\tilde{n},\tilde{m}$
\[
κ\cdot 1 = n κ_0 + m κ_1,\;\;
κ\cdot τ = \tilde{n} κ_0 + \tilde{m} κ_1
\labelthis{eqn:def kappa}
\]
Solving for $τ$ yields
\[
τ
= \frac{(\tilde{n} + \tilde{m}) + \iu x(\tilde{n} - \tilde{m})}
{(n+m) +\iu x(n-m)},
\labelthis{eqn:conformal type}
\]
for $x = \frac{\abs{Y}}{\abs{X}}$. This shows that the conformal type of domain of the map $f$ depends only on $x$, the ratio of $\abs{X}$ to $\abs{Y}$, and some integers.

Before proceeding to the second part of this example, where we compute the spectral data associated to the map $f$, we may further simplify \eqref{eqn:genus zero simple map}. The correspondence between spectral data and harmonic maps does not distinguish between maps that differ by a rotation of $\S^3 = SU(2)$. We may use our freedom of rotation in $SU(2)$ to align $X$ in the direction of $σ_2$ and have $Y$ lie in the plane spanned by $σ_2$ and $σ_3$. After this rotation, $X = \abs{X}σ_2$ and for some $δ\in (0,π)$,
\[
Y = \abs{Y} \begin{pmatrix}
0 & e^{\iu δ} \\ -e^{-\iu δ} & 0
\end{pmatrix}.
\]
Using the norm on $su_2$, we may consider $δ$ as the angle between $X$ and $Y$. This transformation of $f$ into a standard form shows that the image of $f$ is determined up to rotation of $\S^3$ by the angle $δ\in (0,π)$. When $δ = 0$ of $π$, the image collapses to be the circle
\[
\Set{ I \cos r + σ_2 \sin r }{ r \in \R },
\]
but otherwise the image is a torus.

PICTURES OF THE MAPS FOR DIFFERENT DELTAS. Perhaps note that when $δ = π/2$, the image may be rotated to $\abs{z_1}=\abs{z_2} = 1/\sqrt{2}$, the (image of the) Clifford torus.


% And from \eqref{eqn:conformal type}, we know that the domain is determined by the integers $n,m,\tilde{n},\tilde{m}$ and $x\in \R_{>0}$. Thus this space of harmonic maps depends on two continuous and four discrete parameters.

We now turn to computing the spectral data associated to one of these harmonic maps. We may unwind the substitutions to express $F$ as
\[
F = κ\abs{X} \begin{pmatrix}
0 & 1 + \iu xe^{\iu δ} \\ -1-\iu xe^{-\iu δ} & 0
\end{pmatrix},
\]
and likewise $G = \bar{κ}^2 F$. Recall the family of flat connections \eqref{eqn:flat connection translation}. Again, because we can explicitly solve the parallel transport equations, we can easily compute the holonomy matrices. For a connection $d_ζ$, the holomony matrix for the loop from $z=0$ to $1$ is
\[
H(ζ) = \exp (- G + G^* - ζ^{-1}F + ζ F^*),
\]
and for the loop from $z=0$ to $τ$ it is
\begin{align*}
\tilde{H}(ζ)
&= \exp (- G\bar{τ} + G^*τ - ζ^{-1}Fτ + ζ F^*\bar{τ}) \\
&= \exp \left\{ ζ^{-1}(κτ + \bar{κτ}ζ)\abs{X} \begin{pmatrix}
0 & -\left[ (1 + \iu xe^{\iu δ}) - (-1 + \iu x e^{\iu δ})ζ \right] \\
\left[ (1+\iu xe^{-\iu δ}) + (1-\iu xe^{-\iu δ})ζ  \right] & 0
\end{pmatrix}\right\}.
\end{align*}
We note that setting $τ=1$ in the above formula gives the formula for $H(ζ)$, so we shall do all our computations with $\tilde{H}$ without loss of generality.

To compute the spectral curve, we must determine for which values of $ζ$ do the two eigenlines of $H(ζ)$ coincide, and to what order. We know by REF HITCHIN, that the eigenlines of $H(ζ)$ coincide exactly when the eigenlines of $\tilde{H}(ζ)$ do. If $\tilde{B}(ζ)$ is defined by $\tilde{H}(ζ) = \exp \tilde{B}(ζ)$, then the eigenvectors of $\tilde{H}(ζ)$ and $\tilde{B}(ζ)$ are the same. The matrix $\tilde{B}(ζ)$ is off-diagonal, so $w(ζ) = (w_1(ζ)\; w_2(ζ))^T$ is an eigenvector if and only if
\begin{align*}
-ζ^{-1}(κτ + \bar{κτ}ζ)\abs{X}(1 - \iu x e^{\iu δ})(ζ-α) w_2(ζ)^2
&= ζ^{-1}(κτ + \bar{κτ}ζ)\abs{X}(1+\iu xe^{-\iu δ}) (1-\bar{α}ζ)  w_1(ζ)^2 \\
-(1 - \iu x e^{\iu δ})(ζ-α) w_2(ζ)^2
&= (1+\iu xe^{-\iu δ}) (1-\bar{α}ζ)  w_1(ζ)^2,
\end{align*}
where
\[
α = \frac{1+\iu x e^{\iu δ}}{-1+\iu x e^{\iu δ}}
= \frac{x e^{\iu δ} - \iu}{x e^{\iu δ} +\iu},
\labelthis{eqn:def branch point genus zero}
\]
is a point that is always inside the unit circle because we have chosen $δ \in (0,π)$ and $x$ to be positive.
From this we can see where and to what order any eigenspaces coincide. As $\iu x e^{\iu δ}$ is always in the left half of the complex plane, $1 - \iu x e^{\iu δ}$ and its conjugate $1 + \iu x e^{-\iu δ}$ never vanish. Hence the eigenlines coincide only at $α$ and $\cji{α}$, and only to first order. The spectral curve is therefore $η^2 = (ζ-α)(1-\bar{α}ζ)$, a genus one curve without singularities.




The pair of differentials $Θ,\tilde{Θ}$ arise as the derivatives of the logarithms $μ,\tilde{μ}$ of the eigenvalues of $H,\tilde{H}$ respectively. The two eigenvalues of $\tilde{H}$ are $\tilde{μ}^{\pm 1} = \exp \pm \tilde{ν}$, so in this example $\tilde{Θ} = d\log \tilde{μ} = d\tilde{ν}$. Therefore we must compute the eigenvalues of $\tilde{B}(ζ)$. As it is a traceless matrix,
\begin{align*}
\tilde{ν}^2
&= -\det (- G\bar{τ} + G^*τ - ζ^{-1}Fτ + ζ F^*\bar{τ}) \\
&= -ζ^{-2}(κτ + \bar{κτ}ζ)^2 \abs{X}^2 \abs{1- \iu xe^{\iu δ}}^2 (ζ-α)(1-\bar{α}ζ).
\labelthis{eqn:eigenvalue}
\end{align*}
Let us pause for a moment to make a small calculation to simplify the coefficients appearing in \eqref{eqn:eigenvalue}. First note that
\[
\abs{1-α}
= \frac{2}{\abs{1 - \iu xe^{-\iu δ}}},
\text{ and }\;
\abs{1+α}
= \frac{2x}{\abs{1 - \iu xe^{-\iu δ}}}.
\]
It follows then
\[
\iu κ_0 \abs{X}\abs{1-\iu x e} = \frac{π}{2}\bra{ \frac{1}{\abs{1+α}} + \iu \frac{1}{\abs{1-α}} },
\text{ and }\;
\iu κ_1 \abs{X}\abs{1-\iu x e} = \frac{π}{2}\bra{ \frac{1}{\abs{1+α}} - \iu \frac{1}{\abs{1-α}} }.
\]
Finally therefore, the differential corresponding the the eigenvalues of $\tilde{H}(ζ)$ is
\[
\tilde{Θ} = d\,\log \tilde{μ} = d\, \Big[ ζ^{-1}(κτ + \bar{κτ}ζ) \iu \abs{1 - \iu xe^{\iu δ}} η \Big].
\labelthis{eqn:genus zero differential}
\]
Observe that this is real linear in $κ$, so that all the differentials lie in a lattice spanned by the basis
\[
d\, \Big[ ζ^{-1}(κ_0 + \bar{κ_0}ζ) \iu\abs{X} \abs{1 - \iu xe^{\iu δ}} η \Big],
\;\;\;
d\, \Big[ ζ^{-1}(κ_1 + \bar{κ_1}ζ) \iu\abs{X} \abs{1 - \iu xe^{\iu δ}} η \Big].
\]
This identifies the differentials with the lattice of periods of the map $f$, demonstrating the interpretation of the pair of differentials as selecting the degree of winding of the torus onto its image. This same interpretation holds for the general construction of a harmonic map from spectral data, where the domain of the map is constructed as the parallogram spanned by the differentials.

We can also see how the two continuous parameters $δ$ and $x$ have been incorporated into the definition of $α$. In fact, if we treat $xe^{\iu δ}$ as a point in the upper half plane, \eqref{eqn:def branch point genus zero} is a M\"obius transformation. One can write the inverse transformation as
\[
x e^{\iu δ} = \iu \frac{1+α}{1-α}.
\]
Taking the magnitude of both sides shows that $x$ is constant along arcs such that
\[
\abs{\frac{1+α}{1-α}},
\]
is fixed. These are arcs of circles centered on the real axis with radii such that the circle is perpendular to the unit circle. If $x$ is constant, so is $τ$, and so along these arc the corresponding family of harmonic maps have the same domain but images changing as in REF TO PICTURES. For example, if $x=1$ then $α$ lies in the imaginary axis, and as $δ$ moves from $0$ to $π/2$ to $π$, $α$ moves from $-\iu$ to $0$ to $\iu$ correspondingly.

\makefigure{Constant parameter lines.}{thesis_graphics_temp/genus_zero_parameters.png}

Conversely if $δ=π/2$ is fixed, then $α$ is given by
\[
α = \frac{x-1}{x+1},
\]
and takes values along the real axis. The two extremes, when $α=-1,1$ correspond to $x=0,\infty$. In this limit, the domain of the harmonic map is being stretched into a cylinder. When we compute the energy of these maps, we will see that as we aproach the two points $α=\pm 1$ the torus tends to having infinite area.

Having now explored the geometric meanings of the various parameters, let us describe the moduli space of triples $(Σ,Θ,\tilde{Θ})$ as a whole. We have seen that the marked curve $Σ$ is completely determined by its sole branch point $α$ in the unit disc $D$. And for every $α$ we may choose $Θ$ and $\tilde{Θ}$ from a rank two lattice. However, they must be real linearly independent. If we represent our choice of lattice points in terms of an integer combination of some basis, then linear independence is equivalent to the integer matrix
\begin{align*}
\begin{pmatrix}
n & m \\
\tilde{n} & \tilde{m} \\
\end{pmatrix},
\end{align*}
having non-zero determinant. This provides a concise way to refer to the choice of differentials, as a matrix $M$ from $\Mat_2^*\Z = \Set{ M \in \Mat_2\Z }{\det M \neq 0}$. The moduli space may be expressed as the product $D \times \Mat_2^*\Z$.

This matrix formulation is also well suited to talk about changes of the conformal parameter. Recall that the conformal parameters may be computed by taking the ratio of the principal parts of the two differentials. Observe that if $(Σ,Θ,\tilde{Θ})$ is a triple of spectral data with conformal paramters, then so too is $(Σ,nΘ + m\tilde{Θ}, \tilde{n}Θ +\tilde{m}\tilde{Θ})$, and that the new conformal parameter is
\[
\frac{\tilde{n}+\tilde{m}τ}{n + mτ}.
\]
But this is just the action of $\Mat_2^*\Z$ on the upper half plane by M\"obius transformations. The conformal type of the basis given above is
\[
τ = \frac{1-x^2 - 2\iu x}{1+x^2},
\]
so as $α$ moves in the unit disc, $τ$ sweeps out the upper half unit circle. The range of the conformal parameter for any other genus zero harmonic map is the image of this semicircle under some element of $\Mat_2^*\Z$, so the possible ranges are semicircles centered on the real axis and vertical lines, all with rational endpoints.

The previous chapter developed a method for computing the deformations of spectral data. In the genus zero case, it is clear that given a triple of spectral data one can move the sole branch point in the unit disc as one pleases, and this uniquely determines the deformation. An infinitesimal therefore is determined by the value of the derviative of the branch point, $\dot{α}$. Let us see how that manifests in the functions $\dot{q}^i$ and $Q$.

Starting from equation \eqref{eqn:genus zero differential}, because the spectral curve is simply connected, the differentials are exact and may be integrated to give the functions $q^i$.
\begin{align*}
q^i &= (r^i ζ^{-1} - \bar{r}^i)η + C \\
r^i &= \frac{π}{2}\left( \frac{n^i}{\abs{1+α}} + i \frac{m^i}{\abs{1-α}} \right).
\end{align*}
Differentiating with respect to a deformation parameter $t$,
\[
\dot q^i = \frac{1}{ζη}(ζ^2-1)\left[ (-α\dot {\bar r}^i - \frac{1}{2} \bar r^i \dot{α}) + ζ(-\bar{α}\dot{r}^i - \frac{1}{2} r^i \dot{\bar{α}}) \right].
\]
We note the factor $ζ^2 - 1$, which is a consequence of the closing conditions. Also, the other factor, $c^i(ζ)$, is a linear imaginary polynomial as expected. Next we compute the derivative in \eqref{eqn:genus zero differential} to find the polynomials $b^i(ζ)$ and subtitute into \eqref{eqn:Q}. Dividing that by $P$ leaves $Q$, which is
\[
Q = \frac{\pi^2}{4}\frac{1}{\abs{1-α^2}}(n^1m^2-n^2m^1) \left[ - 2\iu α \Re \bra{\frac{\dot{α}}{1-α^2}} + (1-α\bar{α})\Im \bra{\frac{\dot{α}}{1-α^2}} ζ + 2\iu \bar{α}\bra{\Re \frac{\dot{α}}{1-α^2}}ζ^2\right].
\]
At a fixed $α$, we see from the form of $Q$ that $Q_0$ can only take values on a real line. This is in accordance with the remark that $Q_0$ determines the change in the conformal parameter, and observation above that during a deformation the conformal parameter $\tau$ moves along an arc in the upper half plane. When $α\neq 0$ we are in the situation of Lemma \ref{lem:tangent generic}, we see also that $Q_0$ and $Q_1$ determine the real and imaginary parts of $\dot{α} (1-α)^{-2}$ respectively. In this case, the deformation is completely specified by $Q$, which exactly encodes a value for $\dot{α}$.

In the conformal case, when $α=0$, we must refer to Lemma \ref{lem:tangent conformal}. Here we see that $Q = Q_1 ζ$. Now, the deformation is not determined solely by $Q$, as $Q_1 = \Im \dot{α}$. However, the infinitesimal deformation is still determined by the polynomials $c^i$,
\[
c^i(ζ) = \frac{1}{2} \bar r^i \dot{α} -\frac{1}{2} r^i \dot{\bar{α}}ζ.
\]
In that lemma, given a solution to \eqref{eqn:Q reduced}, we are free to add a real multiple of $\tilde{b}^i$, which we know from \eqref{eqn:genus zero differential} is
\[
\tilde{b}^i(ζ) = \frac{1}{2}r^i + \frac{1}{2}\bar{r}^i ζ.
\]
This is exactly the freedom to choose a value of $\Re \dot{α}$.

Finally, there is a formula to compute the energy of harmonic map from its spectral data, given in \cite[Thm 12.17]{Hitchin1990}. In the non-conformal case if the differential $Θ$ is expanded as
\[
Θ = \frac{dζ}{ζ^2}(θ_{-2} + θ_0 ζ^2 + \dots),
\]
and similiarly for $\tilde{Θ}$ then the energy is given as $E = 4\iu(θ_0 \tilde{θ}_{-2} - \tilde{θ}_0 θ_{-2})$. However, we may express this in terms of the coefficients of the polynomial $b(ζ)$,
\[
Θ = \frac{dζ}{ζ^2}\frac{1}{\sqrt{P_0}}
\left[
b_0
+ ζ\left( b_1 - \frac{1}{2}\frac{P_1}{P_0}b_0 \right)
+ ζ^2\left( b_2 - \frac{1}{2}\frac{P_1}{P_0}b_1 + \frac{3P_1^2 - 4P_0P_2}{8P_0^2}b_0 \right)
+ \dots\right].
\]
from which we see that $θ_{-2} = b_0 / \sqrt P_0$ and that
\[
θ_0 = \frac{1}{\sqrt{P_0}}\left[b_2 - \frac{1}{2}\frac{P_1}{P_0}b_1 + \frac{3P_1^2 - 4P_0P_2}{8P_0^2}b_0\right] = \frac{1}{\sqrt{P_0}}\left[b_2 - A b_0\right],
\]
for some constant $A$. We eliminated $b_1$ from the above formula using \eqref{eqn:residue condition}. Putting this into the energy formula, we arrive at the following version
\[
E = \frac{4i}{P_0} (b_2 \tilde b_0 - \tilde b_2 b_0).
\]
Now we specialise to the genus zero case. The coefficients of the polynomials $b,\tilde{b}$ are entirely determined by the choice of four integers $n,m,\tilde n, \tilde m$ and a branch point $α$ in the unit disc as in \eqref{eqn:genus zero differential}. Thus
\[
E = \pi^2(1-α\bar{α})\frac{m\tilde n - n\tilde m}{\abs{1-α^2}}.
\]
One can interpret the fraction as giving the area of the domain of the map, this is the `area' of the parallelogram spanned by the differentials. The factor $1-α\bar{α}$ may be seen as a measure of how close the map is to being conformal. If we compute the derivative of this expression, we observe that
\[
\dot E = 0 \Rightarrow \left(\Re α\right)\left( \Re \frac{\dot{α}}{1-α^2} \right) = 0.
\]
The left factor corresponds to the imaginary axis. The second factor corresponds to a circle centered on the real axis that cuts the unit circle perpedicularly. In other words, $\dot E$ is zero precisely when $α$ moves on an arc that preserves $τ$. This is because, for a given conformal class, harmonic maps are minimisers for the energy.

\begin{center}
\begin{figure}
\begin{tikzpicture}
\begin{axis}[
    title={$E(α)$},
    xlabel=$x$, ylabel=$y$,
]
\addplot3[
	surf,
	domain=-0.9:0.9,
	domain y=-1:1,
]
	{(1+x^2 + y^2)/(sqrt((1-x^2+y^2)^2 + 4*x^2 *y^2))};
\end{axis}
\end{tikzpicture}
\caption{
A plot of the energy as a function over $α = x + i y$. There are singularities at $α=1,-1$, where the domain becomes a cylinder.}
\end{figure}
\end{center}
