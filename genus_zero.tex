%!TEX root = thesis.tex

\section{Genus Zero}
\label{sec:Genus Zero}

The moduli of spectral data where the marked curve is genus zero is an instructive example. We shall denote this space $\mathcal{M}_0$
We have seen in \ref{lem:no singularities} that on a spectral curve of genus zero the pair of differentials may not have a common root. It follows from \ref{thm:moduli manifold} therefore that $\mathcal{M}_0$ is a two-dimensional manifold. Because the all differentials on $\CP^1$ are exact, and all meromomorphic functions are rational, it is possible to describe this space explicitly.

Let $z\in\C$ be a uniformising coordinate on the torus $\C/\Z\langle 1,τ\rangle$. We consider all translation invariant solutions to \eqref{eqn:Hit1.7}. We will shows these to be precisely the harmonic maps that correspond to genus zero spectral data. First though, we find all such maps and manipulate them into a standard form. Following \cite[(9.11)]{Hitchin1990}, write the Higgs field as $Φ = F \,dz$ and the connection as $d''_A = d'' + G \,d\bar{z}$ with respect to the trivialisation arising from the translation action, for constant traceless matrices $F$ and $G$. The family of flat connections \eqref{eqn:flat connections} is
\[
d_ζ := d_A + ζ^{-1}Φ - ζΦ^*
= d + G\,d\bar{z} - G^*\,dz + ζ^{-1}F\,dz - ζF^*\,d\bar{z},
\]
and equations \eqref{eqn:Hit1.7} become
\[
[F,G] = 0\;\; \text{and}\;\; [G,G^*] = [F,F^*].
\]
If $F$ commutes with is conjugate-transpose then so too does $G$, and they are simultaneously diagonalisable by an $SU(2)$ matrix. This case corresponds to a map from the torus to a circle, which does not produce a spectral curve, though we will see how degenerations of harmonic with a genus zero spectral curves may be identified with such maps. Assume therefore that $[F,F^*] \neq 0$. The first equation implies that $F$ and $G$ commute, so we may write $G = λF$. Then $(\abs{λ}^2 - 1)[F,F^*] = 0$ implies that $\abs{λ}=1$. Let $κ\in\S^1$ be such that $\bar{κ}^2=λ$.

As the matrices $F$ and $G$ are constant, one can solve the parallel transport equation $d_ζ X = 0$ for a vector field $X$ explicitly for each $ζ \in \C^\times$. Recall that the harmonic map $f$ may be recovered as a gauge transformation between constant sections of $d_{-1}$ and $d_1$. We may therefore write the harmonic map as
\begin{align*}
f(z) = \exp( &(κz + \bar{κ}\bar{z})(-\bar{κ}F + κF^*) ) \\
&\cdot \exp( (κz - \bar{κ}\bar{z})(-\bar{κ}F - κF^*) ).
\end{align*}
On the torus, we make a change of coordinates to $w = κz$. We write the real and imaginary parts of $w$ as $w_R + \iu w_I$. Recall that $sl_2\C$ can be decomposed as a direct sum $su_2 \oplus \iu su_2$. Write $\bar{κ}F = X + \iu Y$ for $X,Y\in su_2$ and we recognise the two expressions $-\bar{κ}F + κF^*$ and $-\bar{κ}F - κF^*$ as taking the real and imaginary parts of $κF$. The above formula becomes
\[
f(w) = \exp( -4w_R X ) \cdot \exp( 4 w_I Y  ).
\labelthis{eqn:genus zero simple map}
\]

Let us see how to compute the matrix exponential of an $su_2$ matrix $Z$. The determinant of an $su_2$ matrix is always positive, so let the norm of $Z$ be given by the square root of its determinant. We may define the unit matrix $\hat{Z}$ in direction $Z$ to be $Z$ divided by its norm. If we note that $Z^2 = -\abs{Z}^2 I$, the matrix exponential may be written concisely as
\[
\exp Z = I \cos \abs{Z} + \hat{Z} \sin \abs{Z}.
\]
Immediately therefore we can see \eqref{eqn:genus zero simple map} as the product of two circles. As we vary $w_R$ the image of $\exp (-4w_R X)$ is a circle and so for a fixed value of $w_I$, the image of $f$ is the right translation of this circle by $\exp (4w_I Y)$. The same is true if we fix $w_R$ and vary $w_I$.

Given this explicit formula for the matrix exponential, we now compute the periods of the map $f$. For an $su_2$ matrix $Z$, $\exp Z = I$ when $\abs{Z} \in 2π\Z$. We may identify $su_2$ with the imaginary quaternions via the standard basis of $su_2$
\[
σ_1 = \begin{pmatrix}
\iu & 0 \\ 0 & -\iu
\end{pmatrix} \mapsto i, \;
σ_2 = \begin{pmatrix}
0 & 1 \\ -1 & 0
\end{pmatrix}\mapsto j, \;
σ_3 = \begin{pmatrix}
0 & \iu \\ \iu & 0
\end{pmatrix} \mapsto k.
\]
Under this identification, matrix multiplication agrees with quaternionic multiplication. In particular, if $X$ and $Y$ are linearly independent matrices, then the set $\{X,Y,YX, I\}$ is linearly independent. An element $w\in\C$ is a period of $f$ exactly when
\begin{align*}
I = I \cos(4w_R\abs{X})&\cos(4w_I\abs{Y})
- \hat{X}\sin(4w_R\abs{X})\cos(4w_I\abs{Y}) \\
&+ \hat{Y}\cos(4w_R\abs{X})\sin(4w_I\abs{Y})
- \hat{X}\hat{Y}\sin(4w_R\abs{X})\sin(4w_I\abs{Y})
\end{align*}
Squaring and adding the coefficients of $\hat{Y}$ and $\hat{Y}\hat{X}$ together shows that $\sin^2(4w_I\abs{Y}) = 0$, and likewise for $\hat{X}$ and $\hat{Y}\hat{X}$ that $\sin^2(4w_R\abs{X}) = 0$. Let $4w_R\abs{X} = πk$ and $4w_I\abs{Y} = πl$. Then the coefficient of $I$ requires that
\[
1 = \cos(4w_I\abs{Y})\cos(4w_R\abs{X}) = (-1)^k (-1)^l = (-1)^{k+l}.
\]
Thus the lattice of periods, expressed in the variable $w$, is generated by
\[
κ_0 := \frac{π}{4}\bra{\frac{1}{\abs{X}} - \iu\frac{1}{\abs{Y}}}.
\;\;\;\text{and}\;\;\;
κ_1 := -\frac{π}{4}\bra{\frac{1}{\abs{X}} + \iu\frac{1}{\abs{Y}}}
\]

PICTURE

The roles of $κ$, $\abs{X}$, and $\abs{Y}$ are now apparent; they determine the conformal type $τ$ and the number of windings of the torus $\C/\Z\langle1,τ\rangle$ onto its image. The three parameters must be chosen so that $z=1,τ$ are points of the period lattice in the $w$ plane. That is, there must be integers $n,m,\tilde{n},\tilde{m}$
\[
κ\cdot 1 = n κ_0 + m κ_1,\;\;
κ\cdot τ = \tilde{n} κ_0 + \tilde{m} κ_1
\labelthis{eqn:def kappa}
\]
Solving for $τ$ yields
\[
τ
= \frac{(\tilde{n} + \tilde{m}) + \iu x(\tilde{n} - \tilde{m})}
{(n+m) +\iu x(n-m)},
\labelthis{eqn:conformal type}
\]
for $x = \frac{\abs{Y}}{\abs{X}}$. This shows that the conformal type of domain of the map $f$ depends only on $x$, the ratio of $\abs{X}$ to $\abs{Y}$, and some integers.

The correspondence between spectral data and harmonic maps does not distinguish between maps that differ by a rotation of $SU(2)$. We may use our freedom of rotation in $SU(2)$ to align $X$ in the direction of $σ_2$ and have $Y$ lie in the plane spanned by $σ_2$ and $σ_3$. After this rotation, $X = \abs{X}σ_2$ and for some $δ\in (0,π)$,
\[
Y = \abs{Y} \begin{pmatrix}
0 & e^{\iu δ} \\ -e^{-\iu δ} & 0
\end{pmatrix}.
\]
Using the norm on $su_2$, we may consider $δ$ as the angle between $X$ and $Y$. When $δ = 0, π$, the image collapses to be the circle
\[
\Set{ I \cos r + σ_2 \sin r }{ r \in \R },
\]
but otherwise the image is a torus.

This transformation of $f$ into a standard form shows that the image of $f$ is determined up to rotation of $\S^3$ by the angle $δ\in (0,π)$. And from \eqref{eqn:conformal type}, we know that the domain is determined by the integers $n,m,\tilde{n},\tilde{m}$ and $x\in \R_{>0}$. Thus this space of harmonic maps depends on two continuous and four discrete parameters.

PICTURES OF THE MAPS FOR DIFFERENT DELTAS

We now turn to computing the spectral data associated to one of these harmonic maps. We may unwind the simplifications to express $F$ as
\[
F = κ\abs{X} \begin{pmatrix}
0 & 1 + \iu xe^{\iu δ} \\ -1-\iu xe^{-\iu δ} & 0
\end{pmatrix},
\]
and likewise $G = \bar{κ}^2 F$. Recall the family of flat connections \eqref{eqn:flat connections}. Again, because we can explicitly solve the parallel transport equations, we can easily compute the holonomy matrices. For a connection $d_ζ$, the holomony matrix for the loop from $z=0$ to $1$ is
\[
H(ζ) = \exp (- G + G^* - ζ^{-1}F + ζ F^*),
\]
and for the loop from $z=0$ to $τ$ it is
\begin{align*}
\tilde{H}(ζ)
&= \exp (- G\bar{τ} + G^*τ - ζ^{-1}Fτ + ζ F^*\bar{τ}) \\
&= \exp \left\{ ζ^{-1}(κτ + \bar{κτ}ζ)\abs{X} \begin{pmatrix}
0 & -\left[ (1 + \iu xe^{\iu δ}) - (-1 + \iu x e^{\iu δ})ζ \right] \\
\left[ (1+\iu xe^{-\iu δ}) + (1-\iu xe^{-\iu δ})ζ  \right] & 0
\end{pmatrix}\right\}.
\end{align*}
To find the hyperelliptic curve, we may compute the eigenvalues of these holonomy matrices. Let $\tilde{μ}(ζ)$ be an eigenvalue of $\tilde{H}(ζ)$. If $\tilde{B}(ζ)$ is the exponent of $\tilde{H}(ζ)$ and $\tilde{ν}(ζ)$ is an eigenvalue $\tilde{B}(ζ)$, then the two eigenvalues of $\tilde{H}$ are $\tilde{μ}^{\pm 1} = \exp \pm \tilde{ν}$. As the exponent is traceless, we compute
\begin{align*}
\tilde{ν}^2
&= -\det (- G\bar{τ} + G^*τ - ζ^{-1}Fτ + ζ F^*\bar{τ}) \\
&= -ζ^{-2}(κτ + \bar{κτ}ζ)^2 \abs{X}^2 \left[ (1 + \iu xe^{\iu δ}) - (-1 + \iu x e^{\iu δ})ζ \right] \left[ (1+\iu xe^{-\iu δ}) + (1-\iu xe^{-\iu δ})ζ  \right].
\end{align*}
Given that we have chosen $δ \in (0,π)$ and $x$ is positive, let
\[
α = \frac{1+\iu x e^{\iu δ}}{-1+\iu x e^{\iu δ}}
= \frac{x e^{\iu δ} - \iu}{x e^{\iu δ} +\iu},
\labelthis{eqn:def branch point genus zero}
\]
which is always inside the unit circle. Finishing the computation
\[
\tilde{ν}^2
= -ζ^{-2}(κτ + \bar{κτ}ζ)^2 \abs{X}^2 \abs{1- \iu xe^{\iu δ}}^2 (ζ-α)(1-\bar{α}ζ).
\labelthis{eqn:eigenvalue}
\]
The branch points of the hyperelliptic curve are the odd order zeroes of $(\tr \tilde{H})^2 - 4 = (e^{\tilde{ν}} - e^{-\tilde{ν}})^2$, which are the odd order roots of $\tilde{ν}^2$, namely $α$ and $\cji{α}$. Thus the equation of the hyperelliptic curve is $η^2 = (ζ-α)(1-\bar{α}ζ)$.

The eigenvectors of $\tilde{H}(ζ)$ and $\tilde{B}(ζ)$ are the same, so we may compute the eigenline bundle over the hyperelliptic curve. The matrix $\tilde{B}$ is off-diagonal, so $w(ζ) = (w_1(ζ)\; w_2(ζ))^T$ is an eigenvector if and only if
\begin{align*}
-ζ^{-1}(κτ + \bar{κτ}ζ)\abs{X}(1 - \iu x e^{\iu δ})(ζ-α) w_2(ζ)^2
&= ζ^{-1}(κτ + \bar{κτ}ζ)\abs{X}(1+\iu xe^{-\iu δ}) (1-\bar{α}ζ)  w_1(ζ)^2 \\
-(1 - \iu x e^{\iu δ})(ζ-α) w_2(ζ)^2
&= \abs{X}(1+\iu xe^{-\iu δ}) (1-\bar{α}ζ)  w_1(ζ)^2 \\
\end{align*}
From this we can see where and to what order any eigenspaces coincide. As $\iu x e^{\iu δ}$ is always in the right half of the complex plane, $1 - \iu x e^{\iu δ}$ and its conjugate $1 - \iu x e^{\iu δ}$ never vanish. Hence the eigenlines coincide only at the branch points of the hyperelliptic curve, and only to first order. The spectral curve is therefore the hyperelliptic curve; there is no need to add singularities.

Before proceeding, it is worth making a small calculation to simplify the coefficients appearing in \eqref{eqn:eigenvalue}. First note that
\[
\abs{1-α}
= \frac{2}{\abs{1 - \iu xe^{-\iu δ}}},
\text{ and }\;
\abs{1+α}
= \frac{2x}{\abs{1 - \iu xe^{-\iu δ}}}.
\]
It follows then
\[
\iu κ_0 \abs{X}\abs{1-\iu x e} = \frac{π}{2}\bra{ \frac{1}{\abs{1+α}} + \iu \frac{1}{\abs{1-α}} },
\text{ and }\;
\iu κ_1 \abs{X}\abs{1-\iu x e} = \frac{π}{2}\bra{ \frac{1}{\abs{1+α}} - \iu \frac{1}{\abs{1-α}} }.
\]

The pair of differentials $Θ,\tilde{Θ}$ arise as the derivatives of the logarithms of the eigenvalues. However, in this example $Θ = d\log μ = dν$. Therefore
\[
\tilde{Θ} = d\,\log \tilde{μ} = d\, \Big[ ζ^{-1}(κτ + \bar{κτ}ζ) \iu \abs{1 - \iu xe^{\iu δ}} η \Big].
\labelthis{eqn:genus zero differential}
\]
Observe that this is real linear in $κ$, so that all the differentials lie in a lattice spanned by the basis
\[
d\, \Big[ ζ^{-1}(κ_0 + \bar{κ_0}ζ) \iu \abs{1 - \iu xe^{\iu δ}} η \Big].
d\, \Big[ ζ^{-1}(κ_1 + \bar{κ_1}ζ) \iu \abs{1 - \iu xe^{\iu δ}} η \Big].
\]
This shows the interpretation of the lattice of differentials as selecting the degree of winding of the torus onto its image. This interpretation is supported by the general construction of a harmonic map from spectral data, where the domain of the map is constructed as the real span of the differentials. It also demonstrates the observation that one cannot deform the spectral data smoothly by altering the differentials alone. Hence the non-isospectral deformations arise only by moving the branch points of the spectral curve.

We can see how the two continuous parameters $δ$ and $x$ have been incorporated into the definition of $α$. In fact, if we treat $xe^{\iu δ}$ as a point in the upper half plane, \eqref{eqn:def branch point genus zero} is a M\"obius transformation. One can write the inverse transformation as
\[
x e^{\iu δ} = \iu \frac{1+α}{1-α}.
\]
Taking the magnitude of both sides shows that $x$ is constant along arcs such that
\[
\abs{\frac{1+α}{1-α}},
\]
is fixed. These are arcs of circles centered on the real axis with radii such that the circle is perpendular to the unit circle. If $x$ is constant, so is $τ$, and so along these arc the corresponding family of harmonic maps have the same domain but images changing as in REF TO PICTURES. For example, if $x=1$ then $α$ lies in the imaginary axis, and as $δ$ moves from $0$ to $π/2$ to $π$, $α$ moves from $-\iu$ to $0$ to $\iu$ correspondingly.

DIAGRAM \todo{put a diagram and cut most of the words. Lines of blue, constant $δ$, lines of red constant $x$, etc}

Conversely if $δ=π/2$ is fixed, then $α$ is given by
\[
α = \frac{x-1}{x+1},
\]
and takes values along the real axis. The two extremes, when $α=-1,1$ correspond to $x=0,\infty$. In this limit, the domain of the harmonic map is being stretched into a cylinder. We shall see this when we compute the energy of the maps; as  we aproach these two points $α=\pm 1$ the torus tends to having infinite area.

Having now explored the geometric meanings of the various parameters, let us describe the moduli space of triples $(Σ,Θ,\tilde{Θ})$ as a whole. We have seen that the marked curve $Σ$ is completely determined by its sole branch point $α$ in the unit disc $D$. And for every $α$ we may choose $Θ$ and $\tilde{Θ}$ from a rank two lattice. However, they must be real linearly independent. If we represent our choice of lattice points in terms of an integer combination of some basis, then linear independence is equivalent to the integer matrix
\begin{align*}
\begin{pmatrix}
n & m \\
\tilde{n} & \tilde{m} \\
\end{pmatrix},
\end{align*}
having non-zero determinant. This provides a concise way to refer to the choice of differentials, as a matrix $M$ from $\Mat_2^*\Z = \Set{ M \in \Mat_2\Z }{\det M \neq 0}$. The moduli space may be expressed as the product $D \times \Mat_2^*\Z$.

This matrix formulation is also well suited to talk about changes of the conformal parameter. Recall that the conformal parameters may be computed by taking the ratio of the principal parts of the two differentials. Observe that if $(Σ,Θ,\tilde{Θ})$ is a triple of spectral data with conformal paramters, then so too is $(Σ,nΘ + m\tilde{Θ}, \tilde{n}Θ +\tilde{m}\tilde{Θ})$, and that the new conformal parameter is
\[
\frac{\tilde{n}+\tilde{m}τ}{n + mτ}.
\]
But this is just the action of $\Mat_2^*\Z$ on the upper half plane by M\"obius transformations. The conformal type of the basis given above is
\[
τ = \frac{1-x^2 - 2\iu x}{1+x^2},
\]
so as $α$ moves in the unit disc, $τ$ sweeps out the upper half unit circle. The range of the conformal parameter for any other genus zero harmonic map is the image of this semicircle under some element of $\Mat_2^*\Z$, so the possible ranges are semicircles centered on the real axis and vertical lines, all with rational endpoints.




Stuff about $q$ and $Q$
\begin{align*}
q^i &= (a^i - \bar a^i ζ^{-1})η + C \\
c^i &= \frac{π}{2}\left( \frac{n^i}{\abs{1+α}} + i \frac{m^i}{\abs{1-α}} \right) \\
% r &= \sqrt{1 + α\barα + α + \barα} \\
% s &= \sqrt{1 + α\barα - α - \barα} \\
% dq^i &= \frac{dζ}{ζ^2η} \left( -a^i\barαζ^3 + \frac{1}{2}a^i(1+α\barα)ζ^2 + \frac{1}{2}\bar a^i(1+α\barα)ζ  - α\bar a^i\right) \\
\dot q^i &= \frac{1}{ζη}(ζ^2-1)\left[ ζ(-\bar{α}\dot{a}^i - \frac{1}{2} a^i \dot{\bar{α}}) + (-α\dot {\bar a}^i - \frac{1}{2} \bar a^i \dot{α}) \right]\\
Q &= i \frac{\pi^2}{4}\frac{1}{rs}(n^1m^2-n^2m^1) \left[ \bar{α} 2\Re \frac{\dot{α}}{1-α^2}ζ^2 - \frac{1}{2}(1-α\bar{α})2\iu\Im \frac{\dot{α}}{1-α^2} ζ - α 2\Re \frac{\dot{α}}{1-α^2}\right]
\end{align*}

At a fixed $α$, we see from the form of $Q$ that $Q_0$ can only take values on a real line, which comes from the fact $\tau$ is restricted to an arc.

\subsection{Energy in the genus zero case}
There is a formula to compute the energy of harmonic map from its spectral data, given in \cite[Thm 12.17]{Hitchin1990}. In the non-conformal case if the differential $Θ$ is expanded as
\[
Θ = \frac{dζ}{ζ^2}(θ_{-2} + θ_0 ζ^2 + \dots),
\]
and similiarly for $\tilde{Θ}$ then the energy is given as $E = 4\iu(θ_0 \tilde{θ}_{-2} - \tilde{θ}_0 θ_{-2})$. About $ζ=0$, we may expand $η$ so that we may write $θ_{-2}$ and $θ_0$ in terms of the coefficients of the polynomial $b(ζ)$.
\[
Θ \sim \frac{dζ}{ζ^2}\frac{1}{\sqrt{P_0}}
\left[
b_0
+ ζ\left( b_1 - \frac{1}{2}\frac{P_1}{P_0}b_0 \right)
+ ζ^2\left( b_2 - \frac{1}{2}\frac{P_1}{P_0}b_1 + \frac{3P_1^2 - 4P_0P_2}{8P_0^2}b_0 \right)
+ \dots\right].
\]
from which we see that $θ_{-2} = b_0 / \sqrt P_0$ and that
\[
θ_0 = \frac{1}{\sqrt{P_0}}\left[b_2 - \frac{1}{2}\frac{P_1}{P_0}b_1 + \frac{3P_1^2 - 4P_0P_2}{8P_0^2}b_0\right] = \frac{1}{\sqrt{P_0}}\left[b_2 - A b_0\right],
\]
for some constant $A$. We eliminated $b_1$ from the above formula using \eqref{eqn:residue condition}. Putting this into the energy formula, we arrive at the following version
\[
E = \frac{4i}{P_0} (b_2 \tilde b_0 - \tilde b_2 b_0).
\]
Now we specialise to the genus zero case. The coefficients of the polynomials $b,\tilde{b}$ are entirely determined by the choice of four integers $n,m,\tilde n, \tilde m$ and a branch point $α$ in the unit disc.  then we write
\begin{align*}
c &= \frac{\pi}{2}\left( \frac{n}{\abs{1+α}} + i\frac{m}{\abs{1-α}} \right) \\
\tilde c &= \frac{\pi}{2}\left( \frac{\tilde n}{\abs{1+α}} + i\frac{\tilde m}{\abs{1-α}} \right)\\
b_0 &= - α \bar c \\
\tilde b_0 &= - α \bar{\tilde c} \\
% & \\
E
% &= 4i \frac{1}{P_0}\frac{1}{2}\frac{P_1}{\overline P_0}\frac{\pi^2}{4}\frac{-2i}{rs}(m\tilde n - n\tilde m) \\
&= \pi^2\frac{P_1}{\abs{1-α^2}}(m\tilde n - n\tilde m)
\end{align*}
If we compute the derivative of this expression, we arrive at the fact that
\[
\dot E = 0 \Rightarrow \left(\Re α\right)\left( \Re \frac{\dot{α}}{1-α^2} \right) = 0.
\]
The left factor corresponds to the imaginary axis. The second factor corresponds to a circle centered on the real axis that cuts the unit circle perpedicularly. In other words, $\dot E$ is zero precisely when $α$ moves on an arc that preserves $τ$. This is because for a given conformal class, harmonic maps are minimisers for the energy.

\begin{center}
\begin{figure}
\begin{tikzpicture}
\begin{axis}[
    title={$E(α)$},
    xlabel=$x$, ylabel=$y$,
]
\addplot3[
	surf,
	domain=-0.9:0.9,
	domain y=-1:1,
]
	{(1+x^2 + y^2)/(sqrt((1-x^2+y^2)^2 + 4*x^2 *y^2))};
\end{axis}
\end{tikzpicture}
\caption{
A plot of the energy as a function over $α = x + i y$. There are singularities at $α=1,-1$, where the domain becomes a cylinder.}
\end{figure}
\end{center}
