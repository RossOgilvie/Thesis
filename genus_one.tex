%!TEX root = thesis.tex

\section{Genus One Moduli Space}
% \epigraph{All analytic functions are alike; each non-analytic function is non-analytic in its own way.}

\subsection{Intro}
\label{sub:Intro}

An outline:
\begin{enumerate}
\item
Give the outline
\item
Compute a basis of differentials with the correct symmetries
\item
Compute their periods and a basis of differentials
\item
Introduce the various moduli spaces we will consider
\item
Formulate the conditions
\item
Reformulate in coordindates more suited to calculation.
\item
Talk about shape of this parameter space.
\item
Talk about $T$ and its multi-defined-ness. Universal cover
\item
Uniform coordinate
\item
Show derivative is nonzero
\item
Apply Implicit Function theorem in uniform coord and k
\item
Project down. Compute the deck transformations and quotient by them to get the topology
\item
Give the basis of differentials over each point. Show every closing diff meets lies in this span. Triviality of the p=1 differentials
\item
Corollories:
The spirally nature, the density
Something about the k=1 limit. Is it a single point?
something about the modulus $τ$ of the domain?
the p, 1/p symmetry corresponding to a reflection, via swapping 1 and -1.
interchanging the diffs swaps the orientation.
Known examples, special cases
\item
Go to the Winchester, have a pint, and wait for this to blow over.
\end{enumerate}

The different moduli spaces we could be interested in

There are several interrelated moduli spaces of spectral data that we could be interested in. Spectral data is a triple $(Σ,Θ,\tilde{Θ})$ of a spectral curve $Σ$ and a pair of differentials on it. We know that every spectral curve has many differentials with with correct poles and symmetries. Spectral curves are parameterised by their branch points, and by symmetries we need only track branch points inside the unit circle. By counting degrees, we know there is a plane of such differentials with purely imaginary periods, but in general not every spectral curve will satisfy the further condition that there is a rank two lattice of differentials within this plane that have integral periods. Let the space of spectral curves that have \emph{integral} differentials be denoted $\mathcal{C}$. Within this space we could impose the further condition, the closing condition, on the differentials that requires their integrals over the marked points to also be integral. We shall call such differentials \emph{closing}, and the space of spectral curves with closing differentials $\mathcal{S}$.

One can note that the plane of differentials with imaginary periods is naturally thought of as a rank two vector bundle over the space of spectral curves. Thus the moduli of closing spectral data $\mathcal{M}$ can be thought of a bundle over $\mathcal{S}$. Our ultimate aim will be to describe this bundle.

If we are considering only spectral curves of a fixed genus, we will use a subscript on the moduli space to denote this. In this chapter we are considering only genus one, so we will omit subscripts.

In the genus one case, every spectral curve admits integral differentials but not every one admits closing differentials. The closing conditions therefore can be viewed as defining equations for $\mathcal{S}$ within $\mathcal{C}$. We will investigate the topology of this space and show it to be naturally divided into path connected components that are dense in $\mathcal{C}$. Most components will be simply connected, except for an exceptional family of annuli. In the course of our investigation we will develop an explicit parameterisation of this space. The moduli space $\mathcal{M}$ will be show to be a trivial bundle.

In the calculations that follow, we will actually work with a cover of the space of spectral curves. A spectral curve is determined by its branch points, but we will consider instead ordered pairs of branch points. Let $\mathcal{A}$ be the space $\{ (α,β) \in D^2 \mid α \neq β \}$, where $D$ is the open unit disc. If we consider the action of $\Z_2$ that interchanges the two points, then we note that it is a free and transitive action. The quotient space is smooth, indeed it can be identified with the space of spectral curves $\mathcal{C}$. We consider $\mathcal{A}$ as a two-to-one cover of $\mathcal{C}$. We will work primarily with the former, which we may consider as the space of spectral curves with a distinguished branch point $α$.

Vocab summary
\begin{enumerate}
\item
differentials = differentials with the correct poles and symmetries and imaginary periods
\item
integral differentials = differentials with the correct periods
\item
closing differentials = integral differentials meeting the closing conditions
\item
$\mathcal{C}$ = the space of spectral curves = $(D^2\setminus Δ) / \Z_2$
\item
$\mathcal{A}$ = the space of spectral curves with a distinguished branch point = $D^2\setminus Δ$
\item
principal path = a path that follows the recipe laid out for traversing from the two points over $1$ or $-1$. Do not exist on every spectral curve (convention, expand the convention?)
\item
distinguished path = like a principal path, but with possibly extra winding
\item
$\mathcal{P}$ = the universal cover of $\mathcal{A}$
\item
$\mathcal{S}$ = the space of spectral curves that have spectral data, $\subset \mathcal{C}$
\item
$\tilde{\mathcal{S}}$ = the subspace of $\mathcal{P}$ that meets the conditions $p\in\Q^+, T\in\Q$
\item
Any of these followed by $(p)$, eg $\mathcal{A}(p)$ = the subspace where $p(α,β)$ is a fixed value $p$
\item
Any of these followed by $(\Q^+)$, eg $\mathcal{A}(\Q^+)$ = the disjoint union of the subspaces where $p$ is a positive rational
\item
Any of these followed by $(p,q)$, eg $\mathcal{A}(p,q)$ = the subspace where $p(α,β)$ is a fixed value $p$, and the principal branch $T_0$ of $T$ is $q$ modulo blah. May be better to describe it as the image of $\mathcal{P}(p,q)$ where $\tilde{T}_0 = q$

\item
$\mathcal{B}$ = integral differentials
\item
$\mathcal{B}(n)$ = integral differentials with period $2π\iu n$
\item
$\mathcal{B}^2$ = pairs of integral differentials
\item
$\mathcal{B}^2(n,\tilde{n})$ = pairs of integral differentials with periods $2π\iu n$, $2π\iu \tilde{n}$ respectively
\item
$\mathcal{M},\mathcal{M}(n,\tilde{n})$ = space of spectral data, resp. with specified periods
\end{enumerate}

Once we understand which curves admit spectral data, then we can understand the structure of the differentials on those curves.

\[
\begin{diagram}
&& && && && \mathcal{B} \\
&& && && && \dOnto_{\R\times\Z} \\
%%%%%%%%%%%%%%%%%%%%%%%%%%%%%%%%%%%%%
&& && && \mathcal{A}    & \rOnto^{\Z_2}        & \mathcal{C} \\
&& && & \ruInto         &&      & \\
%%%%%%%%%%%%%%%%%%%%%%%%%%%%%%%%%%%%%
&& \mathcal{P}(\mathbf{Q}^+) = \coprod \mathcal{P}(p)  & \rOnto^{\Z}    & \mathcal{A}(\mathbf{Q}^+) = \coprod \mathcal{A}(p) && \ruInto && \\
& \ruInto & && && && \\
%%%%%%%%%%%%%%%%%%%%%%%%%%%%%%%%%%%%%
\tilde{\mathcal{S}} =  \coprod \mathcal{P}(p,q)     && \rOnto && \mathcal{S}   && &&
\end{diagram}
\]



\subsection{Differentials of the Elliptic Spectral Curve}
It will be necessary to have explicit formulae for the integral differentials of the spectral curve. To compute the periods of differentials, we must first reduce them into a standard form. Every differential of the second kind, that is, those that have double poles with no residues, may be written as the sum of standard elliptic integrals of the first and second kind, and exact differentials. Towards this end, we will transform from the form of the spectral curve prefered by Hitchin to the Jacobi form of an elliptic curve, on which the standard elliptic integrals are defined. Consider the genus one spectral curve
\[
η^2 = P(ζ) = (ζ-α)(1-\bar{α}ζ)(ζ-β)(1-\bar{β}ζ)
\]
with roots $α, \cji{α}, β$ and $\cji{β}$, and the Jacobi normal form
\[
w^2 = (1-z^2)(1-k^2z^2)
\]
where $k$ is the elliptic modulus. There is a M\"obius transformation between the two, as we shall demonstrate. For a given spectral curve, how is one to compute the modulus and therefore determine the appropriate Jacobi form to transform into? Consider the cross ratio of the roots
\[
[α,\cji{α};β,\cji{β}] = \frac{\abs{α-β}^2}{\abs{1-\bar{α}β}^2} \labelthis{eqn:roots_cross_ratio}
\]
This is a real quantity, and so the four roots lie on a circle (or a line). Thus the roots of the Jacobi form do also, which forces $k$ to be real. Any transformation between the curves must take branch points to branch points, so we must decide on a correspondence for the roots. There are twenty-four possible orderings that can be divided into six groups of four, based on the value of the cross ratio. By convention, $k \in (0,1)$, which eliminates four of the groups. Of the two groups that remain we choose a representative: either $(α,\cji{α},β,\cji{β}) \mapsto (1,-1,k^{-1},-k^{-1})$ or $(1,k^{-1},-1,-k^{-1})$. A M\"obius transformation preserves the cross ratio, so we observe that
\begin{align}
[1,-1;k^{-1},-k^{-1}] &= \bra{\frac{k-1}{k+1}}^2, \label{eqn:good_ordering}\\
[1,k^{-1};-1,-k^{-1}] &= \frac{4k}{(k+1)^2}. \label{eqn:bad_ordering}
\end{align}
To choose between the two, consider the degenerations of the curve where two pairs of roots come together, that is $α=β$ (and $\cji{α} = \cji{β}$). In the Jacobi form, this manifests as the extreme $k=1$. \ref{eqn:roots_cross_ratio} takes the value zero, as does \ref{eqn:good_ordering}, but \ref{eqn:bad_ordering} takes the value one. Thus the latter is eliminated from contention. Only one group remains, and the modulus is given by
\[
k = \frac{\abs{1-\bar{α}β}-\abs{α-β}}{\abs{1-\bar{α}β}+\abs{α-β}}.
\labelthis{eqn:def_k}
\]

Within this group of orderings, four possibilities still remain
\[
  \begin{array}{ l|l|l|l}
    1 & -1 & k^{-1} & -k^{-1} \\
    \hline\hline
    α & \cji{α} & β & \cji{β} \\
    \cji{α} & α & \cji{β} & β \\
    β & \cji{β} & α & \cji{α} \\
    \cji{β} & β & \cji{α} & α \\
  \end{array}
\]
The first two rows and the last two rows exchanged by swapping the labelling of $α$ and $β$. We shall choose to send $α$ to a fixed root of the Jacobi form. Secondly, the first two rows send the points inside the unit circle to either the left or right half $z$-plane. We shall choose to send them to the right half plane.

Having fixed a mapping of the roots, namely $(α,\cji{α},β,\cji{β}) \mapsto (1,-1,k^{-1},-k^{-1})$, the formula for the transformation can be found by rearranging the relation $[ζ,β;α,\cji{α}] = [f(ζ),k^{-1};1,-1]$ to get
\begin{align}
A &:= (α-β)\abs{1-\bar{α}β}, \label{eqn:A} \\
B &:= (1-\bar{α}β)\abs{α-β}, \label{eqn:B} \\
z = f(ζ) &= \frac{ζ(\bar{α}A + B) - (A + αB)}{ζ(\bar{α}A - B) - (A - αB)}, \label{eqn:f} \\
ζ = f^{-1}(z) &= \frac{-z(A-αB) + (A+αB)}{-z(\bar{α}A-B) + (\bar{α}A+B)}. \label{eqn:f_inv}
% den &:= ζ(\bar{α}A-B) - (A - αB)\\
% ned &:= -z(\bar{α}A-B) + (\bar{α}A+B)
\end{align}

Because of the holomorphic involution $η\to-η$, this formula almost but not quite specifies a relation between $η$ and $w$. There is a free sign choice to make. In $(ζ,η)$ there is a notion of the `upper' unit circle, the one on which $η$ is positive over $ζ=1$. More generally, on this copy of the unit circle in $Σ$, $η(ζ)=ζ\abs{ζ-α}\abs{ζ-β}$. Under the transformation the unit circle is mapped to the imaginary axis. On the imaginary axis, $w(iu) = \pm \sqrt{1+u^2}\sqrt{1+k^2u^2}$ so also has a notion of an upper sheet. We choose the transformation between elliptic curves to map the upper sheet to the upper sheet.
% The sign choice needed to make the upper unit circle map to the upper imaginary axis is
% \[
% η = \frac{AB\bra{1-\abs{α}^2}\bra{\abs{α-β}+ \abs{1-\bar{α}β}}}{(\bar{α}A-B)^2(z+\bar{z_0})^2}w
% \]

The unit circle in the $ζ$-plane plays an important role in the definition of a spectral curve, so it is natural to ask what its imagine under this transformation is.
The real involution $ρ(ζ)$ fixes the unit circle, and exchanges the pairs of branch points $α,\cji{α}$ and $β,\cji{β}$. The corresponding antiholomorphic involution in the $z$-plane is $z\mapsto -\bar{z}$. Its fixed point set is the imaginary axis, which therefore is the image of the unit circle.

As already mentioned, the four roots of the spectral curve are concyclic (or lie on a line), which we shall call the branch point circle. The intersection of this circle with the unit circle correspond to the points $z=0$ and $z=\infty$, which are the intersections of the real and imaginary axes. $z=0$ lies between $α$ and $\cji{α}$ and similiarly $z=\infty$ lies between $β$ and $\cji{β}$. We denote these points $μ = f^{-1}(0)$ and $ν = f^{-1}(\infty)$. Likewise we write $z_0 = f(0)$ and $-\bar{z_0} = f(\infty)$ (using the reality structure). These allow us to write alternate formulae for $f$ and $f^{-1}$
\begin{align}
z = f(ζ) &= -\bar{z}_0 \frac{ζ - μ}{ζ - ν},
\label{eqn:f_2} \\
ζ = f^{-1}(z) &= ν \frac{z - z_0}{z + \bar{z_0}}.
\label{eqn:f_inv_2}
\end{align}

\makefigure{The $ζ$-plane, with points marked in red. The black labels are their images under $f$}{thesis_graphics_temp/zeta_plane.png}
\makefigure{The $z$-plane, with points marked in black. The red labels are their images under $f^{-1}$}{thesis_graphics_temp/z_plane.png}

Having set-up the transformation, we construct the desired differentials in two stages. First, we narrow the space to a 3 dimensional one, of differentials with the correct poles and symmetries. Then we compute the periods of the basis of the space. This allows us to find the differentials inside that space with the integral periods.

If we write differentials in the form
\[
b(ζ)\frac{dζ}{ζ^2η},
\]
for some polynomial $b(ζ)$, the restriction that the differentials have double poles at $ζ=0,\infty$ means that the polynomial part must be degree four. The restiction that these poles have no residues means that the bottom and top terms determine the first and third order terms. And finally the reality condition $ρ^* Θ = -\bar{Θ}$ means that the middle term must be real and the top term is the conjugate of the bottom term. In effect, we only have a complex number of choice in the bottom term and a real scalar in the middle term. This is our real 3-space of differentials.

First note that the holomorphic differential
\[
ω := \frac{dz}{w}
\]
lies in this space. It makes for an obvious first basis vector. It spans the middle terms completely.
In genus one, there is an exact differential with the correct pole behaviour, namely
\[
Θ^1 := \iu\; d\left( \frac{η}{ζ} \right)
% = \iu\left[ -αβ + \frac{1}{2}\left(α(1+\abs{β}^2) + β(1+\abs{α}^2)\right)ζ - \frac{1}{2}\left(\bar{α}(1+\abs{β}^2) + β(1+\abs{α}^2)\right)ζ^3 + \bar{α}\bar{β}ζ^4 \right]\frac{dζ}{ζ^2η}.
\]
It is real, curtesy of the factor $\iu$, and so it we take it as the second basis vector. As has been mentioned, every differential can be written as the sum of an exact differential, one of the first kind and one of the second kind. Given that we already have taken the standard differential of the first kind as a basis vector, we seek to complete the basis with the sum of the standard differential of the second kind and an exact differential.

The standard differential of the second kind is defined to be
\[
e := (1-k^2 z^2) \frac{dz}{w}.
\]
It is real, but it has a pole at $z=\infty$ ($ζ=ν$), which not allowed. This pole can be moved to $ζ=\infty$ by adding an exact differential.
\[
e + d\left[ \frac{w}{z + \bar{z}_0} \right].
\]
We verify now that the pole at $z=\infty$ has actually been cancelled. Letting $z' = z^{-1}$, at infinity $e$ has the expansion
\begin{align*}
w &\sim kz'^{-2} \bra{1 - \frac{1}{2}\frac{1+k^2}{k^2} z'^2 + \dots} \\
e &\sim kz'^{-2} dz' \bra{1-k^{-2}z'^2}\bra{1 + \frac{1+k^2}{k^2} z'^2 + \ldots}
\end{align*}
whereas the exact differential has
\begin{align*}
d\left[ \frac{w}{z + \bar{z}_0} \right]
&\sim d\left[ k z'^{-1} \bra{1 - \frac{1}{2}\frac{1+k^2}{k^2} z'^2 + \dots} \bra{1 - \bar{z}_0z' + \bar{z}_0^2z'^2  + \ldots} \right]\\
&= -k z'^{-2}dz' \bra{1 + \bra{\bar{z}_0^2 - \frac{1}{2}\frac{1+k^2}{k^2}}z'^2 + \ldots}
\end{align*}
which shows that it does cancel the double pole of $e$. Unfortunately, this exact differential does not have the correct symmetries. The space of exact differentials with the correct poles is spanned by
\[
d\left[ \frac{w}{(z-z_0)(z + \bar{z}_0)} \right],
\]
so we add a differential of this form to restore the symmetry.
\[
d\left[ \frac{w}{z + \bar{z}_0} + C\frac{w}{(z-z_0)(z + \bar{z}_0)}\right]
= d\left[ \frac{w(z - z_0 + C)}{(z-z_0)(z + \bar{z}_0)}\right]
\]
is real when $z_0 - C$ is an imaginary number. Thus $C \in \Re z_0 + i\R$. This degree of freedom to choose any imaginary amount is exactly adding a scalar of $Θ^1$. There are two obvious choices; taking $z_0 - C$ to be zero, or taking $C$ to be purely real. Both are valid, but the latter choice ends up being superior as it makes the principal part perpendicular to the principal part of $Θ^1$, which introduces a symmetry into the formulae that we will use later. Hence we take as our third basis differential
\[
λ := e + d\left[ \frac{w}{z + \bar{z}_0} \right] + \Re z_0 d\left[ \frac{w}{(z-z_0)(z + \bar{z}_0)} \right]
= e + d\left[ \frac{w(z - \iu\,\Im z_0)}{(z-z_0)(z + \bar{z}_0)} \right],
\]
which has both the correct poles and the correct symmetries.

We are now in a position to compute the periods of our basis. In the $ζ$-plane, let $γ_R, γ_I$ denote the real and imaginary periods respectively. We will choose the basis in the following way. For $γ_R$, start on the upper unit circle at $μ$, traverse in and around $α$ anticlockwise, then cross the lower unit circle and continue anticlockwise around $\cji{α}$ before returning to the starting point. For $γ_I$, start at the same point we began $γ_R$ and follow the unit circle clockwise. The image of $γ_R$ under $f$ is the anticlockwise loop around $-1$ and $1$ with the left to right part of the path on the upper sheet. $f(γ_I)$ is simply a traversal of the imaginary axis from bottom to on the upper sheet. But this is homologous to the standard clockwise loop around $1$ and $k^{-1}$.

\makefigure{$γ_R$ in red and $γ_I$ in black}{thesis_graphics_temp/zeta_periods.png}

To summarise the standard elliptic integrals
\begin{align*}
\int_{f(γ_R)} ω &= 4K(k) \\
\int_{f(γ_I)} ω &= 2\iu K' \\
\int_{f(γ_R)} e &= 4E(k) \\
\int_{f(γ_I)} e &= 2\iu(K'-E')
\end{align*}
where $K$ and $E$ are the complete elliptic integrals of the first and second kind, and the prime denote not the derivative but instead the complement. By definition $k' = \sqrt{1-k^2}$ and $K'(k) = K(k')$ and ditto for $E'$. Let $Θ^2 = Ωω + Λλ$ and we wish for this differential to have a real period of zero and an imaginary period of $2π\iu$. That requires
\begin{align*}
4KΩ + 4EΛ &= 0 \\
2\iu K' Ω + 2\iu(K'-E')Λ &= 2π\iu
\end{align*}
From the first equation, we can write $Λ = - ΓK$ and $Ω = ΓE$ for some $Γ$. Substituting this into the second equation gives
\begin{align*}
π
&= K'ΓE - (K'-E')ΓK \\
&= Γ(K'E + KE' - KK') \labelthis{eqn:legendre_relation}\\
&= \frac{\pi}{2}Γ \\
Γ &= 2
\end{align*}
where \eqref{eqn:legendre_relation} uses Legendre's relation. Thus $Θ^2 = 2Eω - 2Kλ$, or if we unwind these definitions a little
\[
Θ^2 = 2E ω - 2Ke - 2K d\left[ \frac{w(z-\iu\,\Im z_0)}{(z-z_0)(z + \bar{z}_0)} \right]
\]
which shows a nice division, with the first two terms providing the desired periods and the last two terms giving the required pole behaviour. Though the various choices up to this point may seem contrived, they pay-off as we can characterise the differential $Θ^2$ in the following way.

\begin{lem}
    \label{lem:theta2_characterisation}
The differential $Θ^2$ is the unique differential on the spectral curve with the correct poles and symmetries, with periods $0,2π\iu$ and, with principal part over $ζ=0$
\[
\pp Θ^2 \in \iu \R \pp Θ^1.
\]

\begin{proof}
We first verify that $Θ^2$ has such properties, then verify uniqueness. The only property not yet demonstrated is the third one, concerning the principal part. By considering poles and symmetries, and noting that $ζ=0$ is $z=z_0$, for some real scalar $r$
\[
\pp Θ^1
= \iu \pp d\bra{\frac{η}{ζ}}
= \iu r \pp d \bra{ \frac{w}{(z-z_0)(z + \bar{z}_0)} }
= -\iu r \frac{w(z_0)}{z_0 + \bar{z}_0}\frac{dz}{(z-z_0)^2}.
\]
On the other hand
\begin{align*}
\pp Θ^2
&= - 2K \pp d\left[ \frac{w(z-\iu\,\Im z_0)}{(z-z_0)(z + \bar{z}_0)} \right] \\
&= + 2K \frac{w(z_0)(z_0-\iu\,\Im z_0)}{z_0 + \bar{z}_0} \frac{dz}{(z-z_0)^2} \\
&= 2K \frac{w(z_0)(\Re z_0)}{z_0 + \bar{z}_0} \frac{dz}{(z-z_0)^2}.
\end{align*}
To establish uniqueness, suppose that $Θ$ was another such differetial. Then $Θ-Θ^2$ would be exact and have the correct poles and symmetries. So for some real $s$
\[
Θ = Θ^2 + s Θ^1.
\]
As taking principal part is linear, this shows that $s=0$.
\end{proof}
\end{lem}


Having found this pair of differentials, any other integeral differential may be written in the form $a Θ^1 + n Θ^2$ for some $a\in\R$ and $n\in\Z$. If we think of the differentials as forming a $\Z\times\R$-bundle over $\mathcal{A}$, then this pair trivialises that bundle.

Recall that $\mathcal{A}$ double covers $\mathcal{C}$. The points $(α,β)$ and $(β,α)$ in $\mathcal{A}$ correspond to the same curve $Σ$ in $\mathcal{C}$. If we have a section of differentials $Θ : (α,β) \mapsto Ω^1(Σ(α,β))$, this raises the question on how $Θ(α,β)$ and $Θ(β,α)$, which are on the same curve, differ. We observe directly that $Θ^1 = \iu d (η/ζ)$ is invariant under the interchange of $α$ and $β$. Now we may use the characterisation of $Θ^2$ given by lemma \ref{lem:theta2_characterisation} to conclude the same for it, because
\[
\pp Θ^2(β,α) \in \iu \R \pp Θ^1(β,α) = \iu \R \pp Θ^1(α,β)
\]
and so uniqueness forces $Θ^2(β,α) = Θ^2(α,β)$.

The consequence of this is that $Θ^1,Θ^2$ push forward to a well-defined basis of the differentials over $\mathcal{C}$ as well, and just as for $\mathcal{A}$ they trivialise the bundle $\mathcal{B} \to \mathcal{C}$. Though we mainly concentrate on the subspace of $\mathcal{A}$ of spectral curves that admit spectral data, at the end of this chaper we will consider the space of spectral data within the total space of $\mathcal{B}^2$.









\subsection{The Closing Conditions}
Recall the closing conditions. These are the conditions that spectral data must meet in order that they correspond to a harmonic map from the torus, rather than a harmonic map of the plane (of finite type). If $(Σ,Θ,\tilde{Θ})$ is a triple of spectral data, $Θ$ satisfies the closing condition at $ζ=1$ if
\[
\int_{γ_{+}} Θ \in 2π\iu \Z,
\]
where $γ_+$ is a path that begins at $(1,η(1)^-)$ and ends at $(1,η(1)^+)$, the two points on the spectral curve lying over $ζ=1$. However, the value of the integral is dependent on the particular path chosen. To see that this condition is well defined, suppose that $γ$ and $γ'$ are two paths between the two points over $ζ=1$. Their difference $γ-γ'$ is a closed loop and homologous to an integral combination of periods $aγ_R + bγ_I$. The difference in the values of the integrals is therefore
\[
\int_{γ} Θ - \int_{γ'} Θ
= \int_{γ - γ'} Θ
= a\int_{γ_R} Θ + b\int_{γ_I} Θ
= 2π\iu nb,
\]
because by the period conditions, the real period of $Θ$ is zero and its imaginary period is a multiple of $2π\iu$. So although the value of the integral is dependent on the path, the condition that the value must lie in $2π\iu \Z$ is not. Likewise, the closing condition at $ζ=-1$ is defined by taking $γ_-$, a path from $(-1,η(-1)^-)$ to $(-1,η(-1)^+)$, and requiring the integral of $Θ$ over this path to lie in $2π\iu \Z$ also.

For an exact differential, the particular path of integration is irrelevant and the value of the integral is well defined.
\[
\int_{γ_{+}} Θ^1 = i \left. d\bra{\frac{η}{ζ}} \right|_{(1, η(1)^-)}^{(1, η(1)^+)} = 2i η(1)^+ = 2i \abs{1-α}\abs{1-β}
\]
And similiarly for the other marked point
\[
\int_{γ_{-}} Θ^1 = -2i η(-1)^+ = 2i \abs{1+α}\abs{1+β}.
\]
One can use these formulae to define an explicit condition for a spectral curve to admit an exact closing differential. An exact closing differential must be a real multiple of $Θ^1$. Let the scaling factor be $a$. The two closing conditions applied to $a Θ^1$ are then
\begin{align*}
2\iu η(1)^+ a &\in 2π\iu \Z, \\
-2\iu η(-1)^- a &\in 2π\iu \Z.
\end{align*}
Eliminating $a$ from the two equations, there is a common solution for $a$ if and only if
\[
p := \frac{2\iu η(1)^+}{-2\iu η(-1)^-} = \frac{\abs{1-α}\abs{1-β}}{\abs{1+α}\abs{1+β}} \in \Q^+.
\labelthis{eqn:def_p}
\]
In other words, if and only if $p(α,β)$ is a positive rational. This gives the flavour of what we are aiming to achieve. We would like to give conditions on a spectral curve that ensures it admits closing spectral data, and we would like to give such conditions explicitly so we can give results on the subspace of such spectral curves within the space of all spectral curves.

Before we plow ahead to differentials with periods, there is a simplification we can make. Suppose that a spectral curve admits a pair of closing differentials $Θ,\tilde{Θ}$ with periods $2π\iu n, 2π\iu \tilde{n}$ respectively. Let $g$ be the greatest common denominator of $n$ and $\tilde{n}$ (choosen to be a positive number), and by Bezout's identity let $x,y$ be the integers that satisfy
\[
xn + y\tilde{n} = g.
\]
Then consider the differentials $Ψ,\tilde{Ψ}$ defined by the following integer combination
\[
\vt{Ψ}{\tilde{Ψ}} =
\begin{pmatrix}
\tfrac{\tilde{n}}{g}    &   -\tfrac{\tilde{n}}{g} \\
x                       &   y
\end{pmatrix}
\vt{Θ}{\tilde{Θ}}
\]
The new pair of differentials are simpler in the sense that their imaginary periods are $0$ and $2π\iu g$ respectively. They are also closing, because they are an integer combination of closing differentials. And the integer matrix has determinant one, so is invertible over the integers. Further, the two differentials are linearly dependent exactly when $Ψ$ is zero. Hence,
\begin{lem}
A curve admits linearly independent closing differentials if and only if it admits nonzero closing differentials with imaginary periods $0$ and $2π\iu g$, for some positive integer $g$.
\end{lem}

Thus the condition above, $p\in\Q^+$, is a necessary conditon for a spectral curve to admit closing differentials. To find a second necessary condition, concerning the differential with period $2π\iu g$, we follow the same line of reasoning. For some real number $b$, we may write $\tilde{Ψ} = b Θ^1 + g Θ^2$. Fix two paths $γ_+, γ_-$. The two closing conditions applied to $\tilde{Ψ}$ are then
\begin{align*}
2η(1)^+ \iu b + g\int_{γ_+} Θ^2 &= 2π\iu Γ^+ \in 2π\iu \Z, \\
-2η(-1)^+ \iu b + g\int_{γ_-} Θ^2 &= 2π\iu Γ^- \in 2π\iu \Z.
\end{align*}
Again elimination of $b$ yields the condition for a common solution to exist. This equation can be written as
\[
2π\iu T := p \int_{γ_-} Θ^2 - \int_{γ_+} Θ^2 = 2π \iu \frac{p Γ^- - Γ^+}{g}.
\labelthis{eqn:def_T}
\]

\begin{lem}
\label{lem:closing_conds}
A spectral curve admits a pair of nonzero closing differentials, one exact and one inexact, if and only if $p\in\Q^+$ and $T\in\Q$ for any paths $γ_+, γ_-$.

\begin{proof}
From the above discussion, these are a necessary conditions.

For the converse suppose that the are both rational, say $p = n/m$ and $T = n'/m'$. The rationality of $p$ ensures the consistent solution of an $a$, and hence the existence of $Ψ = aΘ^1$ satisfying the closing conditions. Explicitly
\[
a = \frac{2π\iu n}{2\iu η(1)^+} = \frac{2π\iu n}{-2\iu η(-1)^+}\frac{-2\iu η(-1)^+}{2\iu η(1)^+} = \frac{2π\iu n}{-2\iu η(-1)^+} \frac{1}{p} = \frac{2π\iu m}{-2\iu η(-1)^+}.
\]

To find a $\tilde{Ψ}$, we must first solve
\[
T = \frac{n'}{m'} = \frac{n'm}{m'm} = \frac{n Γ^- - mΓ^+}{gm}.
\]
We can see that we may take $g$ to be $m'$. Let $x,y$ be integers such that $xm-ny = 1$. Then we may take
\[
Γ^+ = n'mx,\;\; Γ^- = n'my,
\]
to obtain equality of the numerator. Hence the condition for the existence of a consistent solution for $b$ is met. $b$ may be computed as
\[
b = \frac{1}{2\iu η(1)^+}\bra{ 2π\iu y n'm - m' \int_{γ_+} Θ^2 }.
\]
In summary, we have found $a,b$ and $g$ such that $Ψ = aΘ^1, \tilde{Ψ} = bΘ^1 + gΘ^2$ solve the closing conditions, and so we have demonstrated that the spectral curve admits closing differentials.
\end{proof}
\end{lem}

Thus the space of genus one spectral curves that admit closing spectral data is defined by the equations $p(α,β) \in \Q^+, T(α,β) \in \Q$ inside $\mathcal{A}$. In general the closing conditions are difficult to work with, harder than even the period conditions. In the genus one case that we are dealing with the integrals of $Θ^1$ lead to an algebraic expression, as we have just seen, but the integrals of $Θ^2$ will lead to a transcendental conditions involving incomplete elliptic integrals.

Note that although the condition $T\in\Q$ is well defined, $T$ is a multivalued function on $\mathcal{A}$. It it dependent on the paths of integration. As previously seen, different paths will change the integrals by multiples of $2π\iu$ and therefore $T$ is defined up to an element of $\Z\langle 1,p\rangle \subset \Q$. In practice, we will make some branch cuts on $\mathcal{A}$ and choose a principal branch of $T$.

Let us then make a principal choice of paths, and let the value of $T$ on these principal choices be denoted $T_0$. Consider the open dense subset $\mathcal{A}_0$ of $\mathcal{A}$ where $ν \neq \pm 1$. On any spectral curve of $\mathcal{A}_0$, let $γ_+$ be the path that begins at $(1,-η(1))$, traverses the unit circle to the point $μ$ without crossing $ν$, follows the branch point circle to $α$, circles this branch point anticlockwise, goes back along the arc (though on a different sheet now) to the unit circle, and back to $(1,+η(1))$. Likewise for $γ_-$ from $(-1,-η(-1))$ to $(-1,η(-1))$. In the case $ν=\pm 1$, it would be impossible to `avoid' $ν$, so this case had to be excluded.

\makefigure{$γ_+$ in red and $γ_-$ in black}{thesis_graphics_temp/gamma_pm_zeta.png}


In $(z,w)$ coordinates these paths are easy to describe by design. Start from the point $f(1)$ on the imaginary axis and go to the origin. Go out along the real axis, around $z=1$ (which corresponds to $ζ=α$) and back again to the origin. Return along the imaginary axis to $f(1)$. By taking the path that avoids $ν$, which corresponds to $z=\infty$, we keep to the realm of path integrals in the plane.

\makefigure{$f(γ_+)$ in red and $f(γ_-)$ in black}{thesis_graphics_temp/gamma_pm_z.png}

We can derive explicit formula for the value of the integrals on these particular paths.
\begin{align*}
\int_{γ_+} \tilde {ω}
&= \bra{2\int_0^{f(1)} - 2\int_0^1} \tilde {ω} \\
&= 2 F(f(1);k) - 2 K(k) \\
\int_{γ_+} e
&= 2 \tilde E(f(1);k) - 2 E(k) \\
\int_{γ_+} Θ^2
&= \int_{γ_+} (2Eω - 2Ke) - 2K \int_{γ_+} d\left[ \frac{w(z-\iu\,\Im z_0)}{(z-z_0)(z + \bar{z}_0)} \right] \\
&= 4 E(k) F(f(1);k) - 4 K(k) E(f(1);k) - 4K \frac{w(f(1))\,(f(1)-\iu\,\Im z_0)}{(f(1)-z_0)(f(1) + \bar{z}_0)}.\labelthis{eqn:gamma_plus}
\end{align*}

For the closing condition at $ζ=-1$
\begin{align*}
\int_{γ_-} Θ^2
&= 4 E(k) F(f(-1);k) - 4 K(k) E(f(-1);k) - 4K \frac{w(f(-1))\,(f(-1)-\iu\,\Im z_0)}{(f(-1)-z_0)(f(-1) + \bar{z}_0)}.\labelthis{eqn:gamma_minus}
\end{align*}

If one wishes, one could substitute these formulae into \ref{eqn:def_T} to get $T_0$, though we hold off doing that until later. Previously, the comment was made that the particular algorithm to chose a path is not valid when $ν = \pm 1$. Indeed, the result of this can be seen directly in the formulae we have derived. When $ν$ takes either of these values, then one of $f(1)$ or $f(-1)$ will be inifinite. $T_0$ is a real valued function because $f(1)$ and $f(-1)$ are purely imaginary, $F(z;k)$ and $E(z;k)$ take the imaginary axis to itself and,
\[
(f(1)-z_0)(f(1) + \bar{z}_0) = -(f(1)-z_0)(\overline{f(1) - z_0}) = - \abs{f(1) - z_0}^2.
\]

% Properly understood, $T$ is not a function on the space of spectral curves $\mathcal{A}$, but rather it is a function on the space of paths $(γ_+, γ_-)$, which we shall call $\mathcal{P}$. To be more precise, let $\mathcal{P}$ be the space $\{(γ_+, γ_-, Σ, α)\}$ where $α$ is one of the branch points inside the unit circle of some genus one spectral curve $Σ$ and $γ_+$ and $γ_-$ are paths $(0,1) \to Σ$ defined in the following way. $γ_+$ starts at $(1, η(1)^-)$, monotonically tranverses the unit circle to $μ$ (that is, it does not pause or reverse direction), goes along the branch point circle to $α$, encircles it, returns to $μ$ along the branch point circle and retraces its path to $(1, η(1)^+)$. In an analogous way, $γ_-$ starts at $(-1, η(-1)^-)$ and finishes at $(-1, η(-1)^+)$. We shall call such paths rigid. Every path on $Σ$ that connects the two points over $1$ or $-1$ is homologous to one of these rigid paths.
%
% $\mathcal{P}$ is naturally a $\Z^2$-bundle over the space $\mathcal{A}$. The projection map takes a tuple $(γ_+, γ_-, Σ, α)$ to the underlying spectral curve $(Σ,α)$. The fibre over any spectral curve $Σ$ is seen to be $\Z^2$, because if one fixes a point on the upper unit circle, not lying over $1,-1,μ$, the (signed) number of crossings of that point by a path uniquely determines that path. As an aside, the issue with points over $1,-1$ and $μ$ is that the paths may terminate there (in the obvious way for the first two, in the sense that the path may leave the unit circle in the latter case). One could probably introduce a convention to get around this, but it is not necessary to do so for our purposes.
%
% Viewed from this perspective, we consider $T$ to be a multivalued function on $\mathcal{A}$ defined locally up the addition of multiples of $p$ and $1$, and $T_0$ is a principal branch cut. We call $\tilde{T}$ the well defined version that is defined on $\mathcal{P}$. $\mathcal{P}$ is actually a cover of the universal cover of $\mathcal{A}$ (implying $\mathcal{P}$ is disconnected), a fact that will become apparent once we have adopted coordinates better adapted to the situation.







\subsection{Marked Point Coordinates}
\label{sub:Reformulate}

It is our ultimate aim to examine the surfaces defined by the two conditions and our examination will result in coordinates for those surfaces. This involves applying the implicit function theorem and the necessary computation for that theorem requires differentiating the above expressions. It is therefore prudent to adopt a parameterisation of the space of spectral curves more suited to that task than $(α,β)$. An obvious first coordinate is $p$ itself, as then we can enforce the first condition simply by holding this coordinate constant. Elliptic integrals are the most difficult part of the above to differentiate, so to minimise our labour we choose the other three (real) coordinates to be $k$, $\iu u = f(1)$ and $\iu v = f(-1)$.

\begin{defn}\label{def:parameter space}
Consider the product space $\R_{>0} \times (0,1) \times (\RInf) \times (\RInf)$. Let $p \in \R_{>0}$ and $k \in (0,1)$ be coordinates on the first two factors. Let $u, u' \in \R$, with $u' = u^{-1}$, be coordinates covering the third factor in the standard way, and likewise let $v, v' \in\R$, with $v' = v^{-1}$ be coordinates covering the fourth factor. Together, the four coordinate patches $(p,k,u,v)$, $(p,k,u',v)$, $(p,k,u,v')$ and $(p,k,u',v')$ to $\R_{>0} \times (0,1) \times \R^2$ cover $\R_{>0} \times (0,1) \times (\RInf)^2$.

Consider the hypersurface $H$ that is the closure of $\{u=v\}$. In each of the four coordinate patches it is given by the equations $u-v = 1-u'v = uv'- 1 = v' - u' = 0$. Define $\mathcal{A}'$ to be the complement of $H$.
\end{defn}

Note that $\mathcal{A}'$ does not include the points where $u'=v'=0$, so it is covered by three of the coordinate patches; $(p,k,u,v)$, $(p,k,u',v)$ and $(p,k,u,v')$ are sufficent for example. We must show that $\mathcal{A}'$ does indeed parameterise the space of spectral curves.

\begin{lem}
The given formulae for $p,k$ (eqns \ref{eqn:def_p}, \ref{eqn:def_k} respectively) and $\iu u = f(1)$, $\iu v = f(-1)$ extend to a diffeomorphism between the spaces $\mathcal{A}$ and $\mathcal{A}'$.

\begin{proof}
We proceed by determining a formula for the change of parameters from $(α,β)$ to $(p,k,u,v)$. The forward direction has already been achived. As stated, both $p$ and $k$ already have their own formulae in terms of $α$ and $β$ (eqns \ref{eqn:def_p}, \ref{eqn:def_k} respectively), which are combinations of the absolute values of various terms. These formulae could only fail to be smooth if those terms were zero. The terms in question are
\[
1-\bar{α}β, α-β, 1-α, 1+α, 1-β, 1+β.
\]
The last four are nonzero because the branch points are inside the unit circle. $α-β$ is nonzero because $α=β$ is excluded from $\mathcal{A}$. That leaves $1-\bar{α}β = \bar{α}(\cji{α} - β)$ which is zero iff $\cji{α}=β$. This is impossible because $β$ is inside the unit circle whereas $\cji{α}$ is outside it.

The map $f$ is written in terms of $(α,β)$, eqn \ref{eqn:f}, and its coefficents are smooth functions. The parameters $\iu u$ and $\iu v$ are the images of points $1$ and $-1$ respectively under this map. This motivates the extended nature of the definition above. The map $f$ is a M\"obius transformation, and so may take the value infinity. Indeed, we have seen that $f(ν) = \infty$, and so one of these parameters is infinite when $ν=\pm 1$ and hence the necessity of the coordinates $u'$ and $v'$ to cover this situation. $z_0 = f(0)$ is a smooth function of $α,β$ and is never zero or infinity, because $0 = f(μ), \infty = f(ν)$ and $μ,ν\in\S^1$. Using the form of eqn \ref{eqn:f_2}, on $ν\neq 1$
\[
\iu u = -\bar{z}_0 \frac{1-μ}{1-ν},
\]
and on $μ\neq 1$
\[
-\iu u' = -\frac{1}{\bar{z}_0} \frac{1-ν}{1-μ},
\]
demonstrating that these are smooth functions on their respective domains of definition. Likewise for $v,v'$ using $f(-1)$.

As the old parameters $α,β$ are points in the $ζ$-plane, one method to derive the reverse change in parameters is to express the inverse transformation $f^{-1}(z)$ in term of our new parameters. Then $α = f^{-1}(1)$ and $β = f^{-1}(k^{-1})$, entirely analogous to $\iu u, \iu v$ in the forward direction. As a M\"obius transformation is described up to a scalar by the points sent to $0$ and $\infty$, $f^{-1}$ is a scalar multiple of
\[
\frac{z-z_0}{z + \conj{z_0}},
\]
(compare to \ref{eqn:f_inv_2}). Thus the construction of $f^{-1}$ proceeds in two steps; first find $z_0$, then determine the correct scaling factor. Note the following trick using cross ratios.
\[
\abs{\frac{α-1}{α+1}}
= \abs{\frac{α-1}{α+1}} \abs{\frac{0+1}{0-1}}
= \abs{ \cross{α}{0}{1}{-1} }
= \abs{ \cross{1}{z_0}{\iu u}{\iu v} }
= \abs{\frac{1-\iu u}{1 - \iu v}} \abs{\frac{z_0 - \iu v}{z_0 - \iu u}}
\]
The same trick gives a similar formula for $β$.
\[
\abs{\frac{β-1}{β+1}}
= \abs{\frac{1-k\iu u}{1 - k\iu v}} \abs{\frac{z_0 - \iu v}{z_0 - \iu u}}
\]
The terms on the left are the two factors of the expression for $p$. Multiplying and expanding shows that for fixed values of the parameters $(p,k,u,v)$, $z_0$ lies on a particular circle. Let $z_0 = x+\iu y$.
\begin{align*}
p
&= \abs{\frac{α-1}{α+1}} \abs{\frac{β-1}{β+1}}
= \abs{\frac{1-\iu u}{1 - \iu v}}\abs{\frac{1 - k\iu u}{1 - k\iu v}} \abs{\frac{z_0 - \iu v}{z_0 - \iu u}}^2 \\
\abs{\frac{z_0 - \iu v}{z_0 - \iu u}} ^2
&= p \frac{\sqrt{1+v^2}}{\sqrt{1+u^2}}\frac{\sqrt{1+k^2v^2}}{\sqrt{1+k^2u^2}}
= p \frac{w(\iu v)}{w(\iu u)} \\
\abs{z_0 - \iu v} ^2 &= p \frac{w(\iu v)}{w(\iu u)} \abs{z_0 - \iu u}^2 \\
x^2 + y^2 &+ 2y \frac{puw(\iu v) - vw(\iu u)}{w(\iu u)-pw(\iu v)} + \frac{v^2w(\iu u) - pu^2w(\iu v)}{w(\iu u)-pw(\iu v)} = 0.
\end{align*}

In the $ζ$-plane, the points $-1,0,1$ all lie on a straight line that is perpendicular to the unit circle at both $-1$ and $1$, and invariant under the real involution. Applying the M\"obius transformation $f$ we can therefore say that $\iu v, z_0$ and $\iu u$ all lie on a circle that is perpendicular to the imaginary axis and symmetric under reflection in the imaginary axis. Therefore $z_0$ lies on the circle
\[
x^2 + \bra{ y - \frac{u+v}{2} }^2 = \frac{(u-v)^2}{4},
\]
which simplifies to the relation
\[
x^2 + y^2 = y(u+v) - uv. \labelthis{eqn:z_0_circle}
\]
These two circles intersect in two points: $z_0$ and $-\conj{z_0}$. The precise formulas are
\[
x = \frac{\sqrt{pw(\iu u)w(\iu v)}}{pw(\iu v) + w(\iu u)} \abs{u-v},\; y = \frac{puw(\iu v) + vw(\iu u)}{pw(\iu v) + w(\iu u)}
\]
where the sign of $x$ is chosen to make $z_0$ lie in the right half of the $z$-plane. This choice amounts to choosing the branch points $α,β$ inside the unit circle, as opposed to choicing their conjugate-inverse partners. Note that these are smooth functions of $(p,k,u,v)$. $w(\iu x)^2 = (1+x^2)(1+k^2x^2)$ is strictly positive, as is $p$. Thus the term under the square root and the denominators never vanish. $u-v$ is never zero as this occurs exactly on the hypersurface $H$, which was excluded from $\mathcal{A}'$.

Having found $z_0$ in terms of $(p,k,u,v)$ it remains to find the correct scaling of $f^{-1}$. We use the fact that $f^{-1}(\iu u) = 1$ and $f^{-1}(\iu v) = -1$.
\[
f^{-1}(z)
=  \frac{\iu u + \conj{z_0}}{\iu u - z_0} \frac{z-z_0}{z + \conj{z_0}}
=  -\frac{\iu v + \conj{z_0}}{\iu v - z_0} \frac{z-z_0}{z + \conj{z_0}}
\]
As was previously presented, one can simply take $α = f^{-1}(1)$ and $β = f^{-1}(k^{-1})$ to give formula for the branch points in terms of the new parameters. A problem could potentially occur if $z_0$ were to equal $\iu u$ or $\iu v$, in which case the scaling factor would be $0/0$. This could only occur if $\Re{z_0}=0$, which itself only occurs if $u=v$. But we have already noted that this hyperplane is excluded from $\mathcal{A}'$. Likewise the formula would be in trouble if $z_0 = -1, -k^{-1}$ (for then $α$ or $β$ would be infinite), but again this occurs only if $\Re{z_0} < 0$, which is impossible.

In the case that one of $f(1)$ or $f(-1)$ is infinite, one needs to transistion to the coordinates $u'$ or $v'$ to maintain valid formulae. Using the notation $w'(\iu a)^2 = (1+a^2)(k^2 + a^2)$ we have
\begin{align*}
x
&= \frac{\sqrt{pw(\iu u)w(\iu v)}}{pw(\iu v) + w(\iu u)} \abs{u-v}
= \frac{\sqrt{u^2 \times pw'(\iu u')w(\iu v)}}{pw(\iu v) + u^2 w'(\iu u')} \abs{u}\abs{1-u'v} \\
&= \frac{\sqrt{pw'(\iu u')w(\iu v)}}{pu'^2w(\iu v) + w'(\iu u')} \abs{1-u'v}.
\end{align*}
Likewise
\[
x
= \frac{\sqrt{pw(\iu u)w(\iu v')}}{pw'(\iu v') + v'^2w'(\iu u')} \abs{uv'-1},
\]
and
\[
y
= \frac{pu'w(\iu v) + vw'(\iu u')}{pu'^2w(\iu v) + w'(\iu u')}
= \frac{puw'(\iu v') + v'w(\iu u)}{pw'(\iu v') + v'^2w(\iu u)}. \labelthis{eqn:y_prime_version}
\]
These all lead to smooth formulae for $z_0$ on their respective coordinate patches. Having made the change in formula for $z_0$, the same formula for $f^{-1}$ applies. Hence we have found a way from $\mathcal{A}'$ back to $(α,β) \in \mathcal{A}$. These two maps are inverses by construction, and we have verified that they are smooth. Therefore the spaces are diffeomorphic.
\end{proof}
\end{lem}

Though standard, it is perhaps still of some interest to consider the above geometrical argument in the limit $u\to\infty$ to assure ourselves that nothing singular is happening. Suppose that $u' = 0$, which is to say geometrically that $1$ is mapped to infinity by $f$. Then the transformation $f$ takes the line through $1,0,-1$ to a line perpendicular to the imaginary axis, cutting at $f(-1)$. This line is therefore horizontal and so $z_0$ and $\iu v$ have the same imaginary parts. This gives $y=v$ directly, as can be observed by setting $u'=0$ in eqn \ref{eqn:y_prime_version}.

This shows that $\mathcal{A}$ and $\mathcal{A}'$ can be identified as spaces, and we will no longer use a prime to distinguish between them. Note when $ν=1$, $f(1) = \iu\infty$ and when $ν=-1$, $f(-1) = \iu\infty$, so the subset $\mathcal{A}_0$ is precisely covered by the coordinates $(p,k,u,v)$. Having established formula for the inverse transformation from the new coordinates, it is time to rewrite $T_0$ in terms of $(p,k,u,v)$. Particularly, we must compute the factors in the elementary terms of the half-periods. By direct computation
\begin{align*}
u-y &= \frac{w(\iu u)(u-v)}{pw(\iu v) + w(\iu u)} &
\abs{\iu u - z_0}^2 &= \frac{w(\iu u)(u-v)^2}{pw(\iu v) + w(\iu u)} \\
v-y &= - \frac{w(\iu v)(u-v)}{pw(\iu v) + w(\iu u)} &
\abs{\iu v - z_0}^2 &= \frac{w(\iu v)(u-v)^2}{pw(\iu v) + w(\iu u)}
\end{align*}
These expressions combine to give from \ref{eqn:gamma_plus} and \ref{eqn:gamma_minus}
\begin{align*}
- 4K \frac{w(f(1))\,(f(1)-\iu\,\Im z_0)}{(f(1)-z_0)(f(1) + \bar{z}_0)}
&= 4\iu K \frac{w(\iu u)\,(u-y)}{\abs{\iu u-z_0}^2}
= 4\iu K \frac{w(\iu u)}{u-v}, \\
\int_{γ_+} Θ^2
= 4 E(k) F(\iu u;k) &- 4 K(k) E(\iu u;k) + 4\iu K \frac{w(\iu u)}{u-v}.
\labelthis{eqn:gamma_plus2}\\
%%%%%%%%%%%%%%%%%%%%%%%%%%
- 4K \frac{w(f(-1))\,(f(-1)-\iu\,\Im z_0)}{(f(-1)-z_0)(f(-1) + \bar{z}_0)}
&= 4\iu K \frac{w(\iu v)\,(v-y)}{\abs{\iu v-z_0}^2}
= - 4\iu K \frac{w(\iu v)}{u-v},\\
\int_{γ_-} Θ^2
= 4 E(k) F(\iu v;k) &- 4 K(k) E(\iu v;k) - 4\iu K \frac{w(\iu v)}{u-v}.\labelthis{eqn:gamma_minus2}
\end{align*}
And hence that
\begin{align*}
2π\iu T_0(p,k,u,v)
&= p\left\{ 4 E F(\iu v;k) - 4K E(\iu v;k) - 4\iu K \frac{w(\iu v)}{u-v} \right\}
- \left\{ 4 E F(\iu u;k) - 4K E(\iu u;k) + 4\iu K \frac{w(\iu u)}{u-v} \right\} \\
&= p \left[ E F(\iu v;k) - K E(\iu v;k) \right] - 4\left[ E F(\iu u;k) - K E(\iu u;k) \right] - 4\iu K \frac{p w(\iu v) + w(\iu u)}{u-v}.
\labelthis{eqn:Teqn}
\end{align*}

Consider the parameter space $\mathcal{A}$ for a fixed value of $p$, denoted $\mathcal{A}(p)$. Topologically it is the product of an interval and an annulus, a feature not easily seen from the $α,β$ description. The interval component is obvious, it comes from $k\in (0,1)$. To see the annulus part, consider $\RInf \times \RInf$. Topologically $\RInf$ is just a circle, so this product is a torus. The line $u=v$ can be represented as the line where the toroidal and poloidal angles are equal, and removing this line leaves an annulus.

A more instructive way of visualising $\mathcal{A}$ is to think of it as a solid cylinder with a line along the central axis removed. One should think of the `radius' of point being given by $1-k$, so that the central axis is identified with the value $k=1$. To motivate this, consider formula \ref{eqn:def_k} for $k$.
\[
k = \frac{\abs{1-\bar{α}β}-\abs{α-β}}{\abs{1-\bar{α}β}+\abs{α-β}}.
\]
In the limit as $α \to β$, this formula says that $k \to 1$. From the equation of $p$, for a fixed value of $p$, the subspace $α=β$ is an arc. In this visualisation we are imagining this arc as the central axis of the cylinder. In later chapters, the interesting structure of the moduli space in this limit will be investigated.

% Another way that this model is useful is it allows us to see how the space of spectral curves is not simply connected and the essential appearence of the multivalued behaviour of $T$. Take a small loop around the central axis, which is to say a certain path in the space of spectral curves with $k$ close to $1$. This implies that $α$ and $β$ must be close together, and so the branch point circle is approximately a line. As we move around the axis, the points $μ,ν$ in the $ζ$-plane are moving around the unit circle. Every time $ν$ crosses either $1$ or $-1$, we have moved across a branch cut. After one full rotation, $α$ and $β$ have returned to their original positions, but $T$ has been incremented by $1+p$.
%
% Viewed in
%
% PERIOD BEHAVIOR HASN"T BEEN INTRODUCED YET \todo{reorder this}

The fact that the parameter space is not simply connected and that $T$ is not even a single valued function could obstruct our use of the implicit function theorem. The obvious way to correct this is to move the the unversal cover.

\begin{defn}
The universal cover of $\mathcal{A}$ is
\[
\mathcal{\tilde{A}} =
\{(p, k,\tilde{u},\tilde{v}) \in \R^+\times(0,1)\times\R\times\R \mid  \tilde{u} < \tilde{v} < \tilde{u} + 2π \},
\]
with the projection map $\mathcal{\tilde{A}} \to \mathcal{A}$ is given by
\begin{align*}
    p &= p, \\
    k &= k, \\
    u = \tan \frac{\tilde{u}}{2},       &\quad
        u' = \cot \frac{\tilde{u}}{2},  \\
    v = \tan \frac{\tilde{v}}{2},       &\quad
        v' = \cot \frac{\tilde{v}}{2}.
\end{align*}
\end{defn}

The justification of this definition of $\mathcal{\tilde{A}}$ precedes in two steps. First, the universal cover of $\R_{>0} \times (0,1) \times \RInf\times\RInf$ is $\R_{>0} \times (0,1) \times \R\times\R$, with the two circle components being covered by the line via a standard half-tan mapping as witnessed in the formulae above. The second step is to recall that $\mathcal{A}$ is the complement of the line $u-v = 0$. When pulled back to the universal cover, this becomes a collections of lines $u-v \in 2π\Z$. Only one stripe is needed to cover $\mathcal{A}$, and this is $\mathcal{\tilde{A}}$ above.

It is straightforward to lift $T_0$ to a single valued function $\tilde{T}$ on $\mathcal{\tilde{A}}$. Recall the defintions of $F_0$ and $E_0$ from Appendix \ref{sec:Elliptic Integrals}.
\[
F_0(x ;k) = \Im F(\iu x; k), \qquad
E_0(x ;k) = \Im F(\iu x; k) - kx.
\]
Using these, we rewrite $T_0$ in the follwoing way.
\[
2π T_0(p,k,u,v) =
4p \left[ E F_0(v) - K E_0(v) \right]
-4 \left[ E F_0(u) - K E_0(u) \right]
- 4 K \left[p\left\{\frac{w(\iu v)}{u-v} + kv \right\} + \left\{\frac{w(\iu u)}{u-v} - ku \right\} \right] .
\]

\begin{lem}
The function
\[
\frac{w(\iu u)}{u-v} - ku
\]
is an analytic function on $\mathcal{A}$.

\begin{proof}
As $u-v \neq 0$ on $\mathcal{A}$, this is an anlytic function of these coordinates. It remains to show that it is simiarly analytic in the other coordinates required to cover $\mathcal{A}$. Firstly, examining this function when using the coordinate $v'$ gives
\[
\frac{w(\iu u)}{u-v} - ku = \frac{w(\iu u)v'}{uv'-1} - ku,
\]
which is analytic. Next, when using the coordinate $u'$ we have
\begin{align}
\frac{w(\iu u)}{u-v} - k u
&= \frac{w(\iu u) - ku^2}{u-v} + \frac{kuv}{u-v} \\
&= \frac{1 + (1+k^2)u^2}{(u-v) (w(\iu u) + ku^2)} + \frac{kv}{1-u'v} \\
&= \frac{u'((u')^2 + (1+k^2)} {(1-u'v) (w'(\iu u') + k)} + \frac{kv}{1-u'v},
\end{align}
where we have introduced the shorthand $w'(z')^2 = (1-(z')^2)(k^2 - (z')^2)$ to deal with this function at infinity. This is now an analytic function of $u'$. As these three coordinate patches cover all of $\mathcal{A}$, we are done and there is no need to consider the case using both $u'$ and $v'$, though this case is easy to derive from the last expression, and also clearly analytic.
\end{proof}
\end{lem}

In Appendix \ref{sub:EllipticContinuation}, extensions of $F_0(x;k)$ and $E_0(x;k)$ to the universal covers of $(k,x)\in (0,1)\times(\RInf)$ are constructed. They are denoted respectively as $\tilde{F}$ and $\tilde{E}$. Thus the first two brackets of $T_0$ can be lifted to the universal cover by simply swapping to those ready-made functions. The previous lemma has established that the third bracket is analytic, and so lifts to the universal cover without the need for modifcation at all. Therefore we define
\[
2π \tilde{T}(p,k,\tilde{u},\tilde{v})
= 4p \left[ E \tilde{F}(\tilde{v}) - K \tilde{E}(\tilde{v}) \right]
- 4 \left[ E \tilde{F}(\tilde{u}) - K \tilde{E}(\tilde{u}) \right]
- 4 K \left[p\left\{\frac{w(\iu v)}{u-v} + kv \right\}
+ \left\{\frac{w(\iu u)}{u-v} - ku \right\} \right] .
\]
If we wish to relate this to a specific local expression, it is simply a matter of substituting the correct expression for $\tilde{F}$ or $\tilde{E}$, \'a la \ref{eqn:tildeF_period} and \ref{eqn:tildeE_period}. In particular, define the winding number $W : \R \to \Z$ of a number $x$ to be the integer $W(x)$ such that $-π < x - 2πW(x) < π$. Then
\begin{align*}
2π \tilde{T}(p,k,\tilde{u},\tilde{v})
%%%%%%%%%%%%%%%%%%%%%%%%
&= 4p \left[ E (2K'W(\tilde{v})+F_0(v)) - K (2(K'-E')W(\tilde{v})+E_0(v)) \right] \\
&\quad - 4 \left[ E (2K'W(\tilde{u})+F_0(u)) - K (2(K'-E')W(\tilde{u})+E_0(u)) \right]
- 4 K \left[p\left\{\frac{w(\iu v)}{u-v} + kv \right\}
+ \left\{\frac{w(\iu u)}{u-v} - ku \right\} \right] \\
%%%%%%%%%%%%%%%%%%%%%%%%
&= 4p \left[ 2EK' - 2K(K'-E') \right]W(\tilde{v})
- 4 \left[ 2EK' - 2K(K'-E') \right]W(\tilde{u}) + T_0(p,k,u,v)\\
%%%%%%%%%%%%%%%%%%%%%%%%
\tilde{T}(p,k,\tilde{u},\tilde{v})
&= 2\bra{ pW(\tilde{v}) - W(\tilde{u}) } + T_0(p,k,u,v) .
\end{align*}
Thus to do computations with the function $\tilde{T}$, one can continue to work with the function $T_0$ downstairs on $\mathcal{A}$ and make a note of the constants. In particular, $T \in \Q$ if and only if $\tilde{T} \in \Q$, and so we can seek to find moduli space as a subset of the universal cover.



























\subsection{The topology of the moduli space}

In the interest of having manageable formulae, we recall the definition of $w'(z')^2 = (1 - (z')^2)(k^2 - (z')^2)$. As $F(z;k)$ and $E(z;k)$ are parameter integrals in $z$, we have that
\[
\Partial{}{u} F(\iu u; k) = \frac{\iu}{w(\iu u)},\;\;\;
\Partial{}{u} E(\iu u; k) = \iu\frac{1+k^2 u^2}{w(\iu u)},
\]
and
\[
\Partial{}{u} w(\iu u)
= \Partial{}{u} \sqrt{1+u^2}\sqrt{1+k^2 u^2}
= \frac{(1+k^2)u + 2k^2 u^3}{w(\iu u)}.
\]
The other derivatives of elliptic integrals are calculated in appendix \ref{sec:Elliptic Integrals}. Equipped with these tools, the calculation of the derivatives of $T_0$ is mechanical if tedious.
% \begin{align*}\label{dTdk}
% \frac{π}{2}\Partial{T_0}{k}
% &= \frac{1}{k(1-k^2)}\frac{1}{u-v} \left[ p \sqrt { \frac{1+v^2}{1+k^2v^2}} + \sqrt { \frac{1+u^2}{1+k^2u^2} } \right] \left[ -(1+k^2uv) E + (1-k^2)K \right]
% \end{align*}
\begin{equation}\label{dTdu}
\frac{π}{2}\Partial{T_0}{u}
= -\frac{E}{w(\iu u)} + \frac{pK w(\iu v)}{(u-v)^2} + \frac{K}{w(\iu u)(u-v)^2}\left[1 + u^2 - uv + v^2 + k^2 uv + k^2 u^2v^2 \right]
\end{equation}
% \begin{equation}\label{dTdv}
% \frac{π}{2}\Partial{T_0}{v}
% = \frac{pE}{w(\iu v)} - \frac{K w(\iu u)}{(u-v)^2} - \frac{pK}{w(\iu v)(u-v)^2}\left[1 + u^2 - uv + v^2 + k^2 uv + k^2 u^2v^2 \right]
% \end{equation}
While $T$ may be multivalued on $\mathcal{A}'$, its derivatives are not: the ambiguity of being locally defined only up to a constant is removed by differentiation. Henceforth we can drop the subscript $0$ on derivatives.

% \begin{equation}\label{dTdu'}
% \frac{π}{2}\Partial{T}{u'}
% = \frac{E}{w'(u')} + \frac{pK w(\iu v)}{(1-u'v)^2} + \frac{K}{w'(u')(1-u'v)^2}\left[1 + k^2v^2 + - u'v + k^2u'v + (u')^2 + (u')^2v^2 \right]
% \end{equation}
% \begin{equation}\label{dTdv'}
% \frac{π}{2}\Partial{T}{v'}
% = -\frac{pE}{w'(v')} + \frac{K w(\iu u)}{(uv'-1)^2} + \frac{pK}{w'(v')(uv'-1)^2}\left[1 + k^2u^2 - uv' + k^2uv' + (v')^2 + u^2(v')^2 \right]
% \end{equation}

\begin{lem}
    \label{lem:deriv no zeroes}
The functions
\[
U(p,k,u,v) := -(u-v)^2 E + pKw(\iu u)w(\iu v) + K\left[ 1 + u^2 - uv + k^2 uv + v^2 + k^2 u^2 v^2 \right]
\]
and
\[
V(p,k,v) := -E + pkKw(\iu v) + K\left[ 1 + k^2v^2 \right]
\]
have no zeroes for $p \geq 1$, $k\in (0,1)$, $u,v \in \R$.
\begin{proof}
We shall prove this statement by showing that the two functions are in fact always positive. The first step is to eliminate $E$. We apply the crude estimate that $K>E$, \todo{reference the appendix}
and also the assumption that $p\geq 1$, to simplify
\begin{align*}
U(p,k,u,v)
&= -(u-v)^2 E + pKw(\iu u)w(\iu v) + K\left[ 1 + u^2 - uv + k^2 uv + v^2 + k^2 u^2 v^2 \right] \\
&> -(u-v)^2 K + Kw(\iu u)w(\iu v) + K\left[ 1 + u^2 - uv + k^2 uv + v^2 + k^2 u^2 v^2 \right] \\
&= K \left[ w(\iu u)w(\iu v) + 1 + (1 + k^2) uv + k^2 u^2 v^2 \right]
\end{align*}
This formula is almost sufficent. The only term that could be negative is the one featuring $uv$. However, a lower bound for the square root terms is
\[
w(\iu u) = \sqrt{1 + (1+k^2)u^2 + k^2u^4} > \sqrt{(1+k^2)u^2} = \sqrt{(1+k^2)}\abs{u},
\]
so
\begin{align*}
U(p,k,u,v)
&> K \left[ (1+k^2)\abs{uv} + 1 + (1 + k^2) uv + k^2 u^2 v^2 \right] \\
&\geq K \left[ 1 + k^2 u^2 v^2 \right].
\end{align*}
This is positive, so $U$ is without zeroes. The second function, $V$, is almost immediate. Take $E<K$,
\[
V(p,k,v) > K \left[ -1 + pk w(\iu v) + 1 + k^2 v^2\right].
\]
This establishes that $V$ has no roots either.
\end{proof}
\end{lem}

% This result is of interest because it shows that if $p \leq 1$, then the $v$ and $v'$ derivatives of $T$ are nonzero:
% \begin{align*}
% \frac{π}{2}\Partial{T}{v} &= \frac{-p}{w(\iu v)(u-v)^2} U\bra{ \frac{1}{p},k,u,v }, \\
% \frac{π}{2}\Partial{T}{v'} &= \frac{p}{w'(v')(uv'-1)^2} V\bra{ \frac{1}{p},k,u,v' }.
% \end{align*}
% And if $p \geq 1$, then the $u$ and $u'$ derivatives of $T$ are nonzero:
% \begin{align*}
% \frac{π}{2}\Partial{T}{u} &= \frac{1}{w(\iu u)(u-v)^2} U\bra{ p,k,u,v }, \\
% \frac{π}{2}\Partial{T}{u'} &= \frac{-1}{w'(u')(1-u'v)^2} V\bra{ p,k,v,u' }.
% \end{align*}

\begin{lem}
\label{lem:range_T}
The range of $\tilde{T}$ on $\mathcal{A}(p)$ is $\R$.

\begin{proof}
Fix $p$, but also fix any value for $k$. By REF, $\abs{F_0(x;k)}$ is bounded by $K'$ and $\abs{E_0(x;k)}$ is bounded by $K'-E'$. Thus there is some constant, dependent on both $p$ and $k$, such that
\[
-C \leq 4p \left[ E F_0(v) - K E_0(v) \right]-4 \left[ E F_0(u) - K E_0(u) \right] \leq C.
\]
It follows from REF that
\[
- 4 K \left[\frac{pw(\iu v) + w(\iu u)}{u-v} + k(pv-u) \right] - C
\leq
2π T_0(p,k,u,v)
\leq
- 4 K \left[\frac{pw(\iu v) + w(\iu u)}{u-v} + k(pv-u) \right] + C,
\]
and so it is sufficent to show that the bracketed expression has range equal to the real line. But this is easy to show. Consider the limit as $u \to v^+$,
\[
\lim_{u \to v^+} \frac{pw(\iu v) + w(\iu u)}{u-v} + k(pv-u)
= k(p-1)v + (p+1) w(\iu v)\lim_{u \to v^+} \frac{1}{u-v} = +\infty.
\]
From the other side,
\[
\lim_{u \to v^-} \frac{pw(\iu v) + w(\iu u)}{u-v} + k(pv-u)
= k(p-1)v + (p+1) w(\iu v)\lim_{u \to v^-} \frac{1}{u-v} = -\infty.
\]
By continuity, $T_0$ and therefore $\tilde{T}$ obtains every value.
\end{proof}
\end{lem}



\begin{lem}
\label{lem:T_graph}
If $p \leq 1$ then $\tilde{\mathcal{A}}(p)$ is diffeomorphic to $\{(q,k,\tilde{u}) \in \R\times(0,1)\times\R \}$, such that for fixed $q_0$, the level set $\tilde{T} = q_0$ is diffeomorphic to $\{(q_0,k,\tilde{u}) \}$.

Likewise, if $p \geq 1$ then $\tilde{\mathcal{A}}(p)$ is diffeomorphic to $\{(q,k,\tilde{v}) \in \R\times(0,1)\times\R \}$, such that for fixed $q_0$, the level set $\tilde{T} = q_0$ is diffeomorphic to $\{(q_0,k,\tilde{v})$.

\begin{proof}
Fix a value of $p$ and consider the function $F(q, k,\tilde{u},\tilde{v}) = \tilde{T}(p,k,\tilde{u},\tilde{v}) - q$ on $\mathcal{\tilde{A}}(p)\times\R$. $F^{-1}(0)$ is a graph over $\mathcal{\tilde{A}}(p)$ given by $q=\tilde{T}$, so they are diffeomorphic. We will apply the implicit function theorem to show that $F^{-1}(0)$ can also be written as a graph over either $(q,k,\tilde{u})$ or $(q,k,\tilde{v})$, depening on the magnitude of $p$.

Suppose first that the fixed value of $p$ is greater than or equal to one. We compute the following formula for the derivative of $F$ with respect to $\tilde{u}$.
\begin{align*}
\Partial{F}{\tilde{u}}
= \Partial{\tilde{T}}{\tilde{u}}
&= \frac{du}{d\tilde{u}}\Partial{T}{u} \\
&= \frac{1}{2}\sec^2\bfrac{\tilde{u}}{2} \Partial{T}{u} \\
&= \frac{1}{2}\sec^2\bfrac{\tilde{u}}{2} \times \frac{2}{π} \times \frac{1}{w(\iu u)(u-v)^2} U\bra{ p,k,u,v } \\
&= \frac{1}{π}\frac{1 + u^2}{w(\iu u)(u-v)^2} U\bra{ p,k,u,v },
\end{align*}
which holds for $\tilde{u} \not\in π + 2π\Z$.As witnessed in Lemma \ref{lem:deriv no zeroes}, $U$ is never zero, and neither are the other three factors present. Hence $\partial \tilde{T} / \partial \tilde{u}$ is never zero on this open set. It remains to check it does not vanish when $\tilde{u} \in π + 2π\Z$. Observe
\begin{align*}
\lim_{u\to\infty} u^{-2} U(p,k,u,v)
&= \lim_{u\to\infty} -(1-u^{-1}v)^2 E + pKw(\iu v) \cdot u^{-2}w(\iu u) + K\left[ u^{-2} + 1 - u^{-1}v + k^2 u^{-1}v + u^{-2}v^2 + k^2 v^2 \right] \\
&= V(p,k,v).
\end{align*}
As $\tilde{T}$ is an analytic function, in particular its derivatives are continuous and so we may compute their value at these points by taking a limit. Therefore
\begin{align}
\lim_{\tilde{u}\to π + 2π\Z} \Partial{F}{\tilde{u}}
&=\frac{1}{π} \lim_{u \to \infty} \frac{1 + u^2}{w(\iu u)(u-v)^2} U\bra{ p,k,u,v } \\
&=\frac{1}{π} \lim_{u \to \infty} \frac{(1 + u^2)u^2}{w(\iu u)(u-v)^2} \times u^{-2}U\bra{ p,k,u,v } \\
&=\frac{1}{π} \frac{1}{k} V\bra{ p,k,v },
\end{align}
which is also nonzero. By the implicit function theorem, there is a function $h$ such that $F^{-1}(0)$ is a graph of the form $\{ (q, k, \tilde{u}, h(q,k,\tilde{u})) \mid q \in \text{Range} \tilde{T}, k \in (0,1), \tilde{u}\in\R \}$. By Lemma \ref{lem:range_T}, the range of $\tilde{T}$ is $\R$. Finally, if we hold $q$ fixed at $q_0$, then the level set $q_0 = \tilde{T}$ is paramterised by the remaining two coordinates $(k,\tilde{u})$.

When $p \leq 1$, we employ the following symmetry.
\begin{align*}
T_0(p,k,u,v)
&= 4p \left[ E F(\iu v;k) - K E(\iu v;k) \right] - 4\left[ E F(\iu u;k) - K E(\iu u;k) \right] - 4\iu K \frac{p w(\iu v) + w(\iu u)}{u-v} \\
&= -p\left\{ -4\left[ E F(\iu v;k) - K E(\iu v;k) \right] + \frac{4}{p}\left[ E F(\iu u;k) - K E(\iu u;k) \right] + 4\iu K \frac{ w(\iu v) + \frac{1}{p}w(\iu u)}{u-v} \right\}\\
&= -p T_0\bra{ \tfrac{1}{p}, k, v, u },
\end{align*}
so that it is now the $\tilde{v}$ derivative of $\tilde{T}$ that is nonvanishing. Again the implicit function theorem gives the result.
\end{proof}
\end{lem}

Recall that $\mathcal{S} \subset \mathcal{C}$ is the set of spectral curves that admit closing spectral data and that $\mathcal{\tilde{S}} \subset \mathcal{\mathcal{A}}$ is its preimage in the universal cover. Let $\mathcal{\tilde{A}}(p,q)$ denote the subset of $\mathcal{\tilde{A}}$ on which the values of $p(α,β) = p$ and $\tilde{T} = q$ are fixed. As we have shown that $\mathcal{S}$ is the space on which $p$ is a positive rational and $q$ is any rational (lemma \ref{lem:closing_conds}), it follows that
\[
\mathcal{\tilde{S}} = \coprod_{p\in \Q^+,\; q\in \Q} \mathcal{\tilde{A}}(p,q).
\]

There are several topological implications of the previous lemma. First, each level set $\mathcal{\tilde{A}}(p,q)$ is connected. More precisely each of these is diffeomorphic to a product of $(0,1)$ and $\R$. The above disjoint union is the decomposition of $\mathcal{\tilde{S}}$ into its connected components. By varying the value of $q$, we see that the $\mathcal{\tilde{A}}(p,q)$ foliate $\mathcal{A}(p)$, so $\mathcal{\tilde{S}}$ is arranged densely in $\mathcal{\tilde{A}}(p)$ in the same way that $\Q$ is arranged densely in $\R$.
\todo{phrase this better}

To understand the topology of $\mathcal{S}$, we must understand the action of the deck transformations as restricted to $\tilde{\mathcal{S}}$. There are two types of transformations to understand, the transformations of $\mathcal{\tilde{A}}$ over $\mathcal{A}$ and the transformations of the two fold covering of $\mathcal{A}$ over $\mathcal{C}$. The first sort are easy to understand; they are generated by the translation
\[
(p, k,\tilde{u},\tilde{v}) \mapsto (p, k, \tilde{u} + 2π, \tilde{v} + 2π).
\]
The second sort require a little more effort. On one hand, the deck transformations of $\mathcal{A}$ over $\mathcal{C}$ is simply the involution that swaps $α$ and $β$. The difficulty is what this action looks like when pulled back to the universal cover. By inspection of the defintions of $p$ and $k$, they are unchanged if the labbeling of the branch points is exchanged. We therefore must say what happens to $u$ and $v$. Our construction thus far relies on the standardisation in the definition of $f$ that sends $α$ to $1$ and $β$ to $k^{-1}$. Let $f_s$ be the map which instead standardises the roots of the spectral curve in the other way, taking $β$ to $1$. Consider the composition of $f_s \circ f^{-1}$. It is a M\"obius transformation that exchanges $1$ and $k^{-1}$ and also $-1$ and $-k^{-1}$. It therefore must be the map
\[
z \mapsto \frac{1}{kz}.
\]
Under the composite map $\iu u$ is taken to $-\iu u^{-1}$. Ditto for $\iu v$. Using the map $f_s$ instead of $f$ is equivalent to swapping the labels on the branch points and then applying $f$. Thus, under the label-swapping involution,
\[
(p,k,u,v) \mapsto \bra{ p,k, -(ku)^{-1}, -(kv)^{-1} }.
\]
This then carries on to the universal cover in the following way. For some integer $n$
\[
\tilde{u} = 2πn + 2\atan u \mapsto 2πn + 2\atan \bra{ -\frac{1}{ku} } = 2πn + π + 2\atan (ku).
\]
Applying the transformation again
\[
2πn + π + 2\atan (ku) \mapsto 2πn + π + 2\atan \bra{ -\frac{k}{ku} } = 2π(n+1) + 2\atan u = \tilde{u} + 2π.
\]
The same applies to $\tilde{v}$. This shows that deck transformations of $\mathcal{\tilde{A}}$ over $\mathcal{A}$ are actually a subgroup of the group generated by the label-swapping involution. Hence we can focus our attention solely on the latter.

How does the value of $T$ change when this transformation is precomposed? To see this, it is useful to consider the situation in the $ζ$-plane. Suppose for concreteness that $μ$ and $ν$ are chosen such that $ν,μ,1$ and $-1$ are arranged clockwise as shown in the following diagram. The principal choice of path $γ_+$ is shown in black, whereas its pullback under the exchange of the roots is shown in red. Let it be denoted $γ'_+$.

\begin{center}
\includegraphics[height=50mm]{thesis_graphics_temp/deck_T_1.png}
\end{center}

The difference between these two paths is homologous to a loop anticlockwise around the upper unit circle. So by the normalisation of $Θ^2$,
\[
\int_{γ'_+} Θ^2 - \int_{γ_+} Θ^2 = \int_{\S^1} = -2π\iu.
\]

Likewise if we consider the difference between the path $γ_-$ and its pullback $γ'_-$ we again have a anticlockwise loop of the upper unit circle.
\begin{center}
    \includegraphics[height=50mm]{thesis_graphics_temp/deck_T_1.png}
\end{center}
Hence
\[
\int_{γ'_-} Θ^2 - \int_{γ_-} Θ^2 = \int_{\S^1} = -2π\iu.
\]
Putting these together we conclude that the value of $T_0$ changes by $1-p$ under this transformation.
\[
2π\iu (T_0 \circ g - T_0) = \bra{ p\int_{γ'_-} Θ^2 - \int_{γ'_+} Θ^2 } - \bra{ p\int_{γ_-} Θ^2 - \int_{γ_+} Θ^2 } = p(-2π\iu) - (-2π\iu) = 2π\iu (1-p).
\]

To infer what the effect of the transformation is on $\tilde{T}$ however, we also must take into account how the coordinates $\tilde{u},\tilde{v}$ may have changed, and consequently altered their winding numbers. If the points $μ,ν$ have been arranged as described, then this restricts the arrangement of $\iu u$, $\iu v$. In particular, as one tranverses the unit circle clockwise, one traverses the imaginary axis in the $z$-plane from bottom to top. Correspondingly, $-\infty < 0 < u < v$.

As $u,v$ are both positive, it must be that $\tilde{u} \in (2πn, 2πn + π)$ and $\tilde{v} \in (2πm, 2πm + π)$ for some integers $n,m$. $2\atan(ku)$ will be between $0$ and $π$, so after the transformation $\tilde{u}$ will be in the range $(2πn +π, 2π(n+1))$. In otherwords its winding number has incresed by $1$. The same can be said for $\tilde{v}$. Putting these two parts together
\[
\tilde{T} \circ g - \tilde{T}
= \bra{ T_0 \circ g + 2(p(m+1)-(n+1)) } - \bra{ T_0 + 2(pm-n) }
= 1-p  +2(p-1) = p-1.
\]
This relation is a difference of analytic functions on an open set, so by continuation it applies everywhere. Thus the effect of the deck transformation on $\tilde{T}$ is to increase its value by $p-1$.







If we consider the surfaces $\mathcal{\tilde{A}}(p,q)$ parameterised by $(p,q,k,\tilde{x})$, where $\tilde{x}$ is mapped to either $\tilde{u}$ or $\tilde{v}$ depending on the magnitude of $p$, we see that under the generating deck transformation
\[
(p,q,k,\tilde{x}) \mapsto (p, q + (p-1), k, \tilde{x}')
\]
where $\tilde{x}' = 2πn+ π + 2\atan (k \tan (\tilde{x}/2))$. This provides an identification between $\mathcal{\tilde{A}}(p,q)$ and $\mathcal{\tilde{A}}(p,q + l(p-1))$ for any integral $l$.

\begin{thm}
For $p\neq 1$, $\mathcal{C}(p)$ is diffeomorphic to
\[
\{ (q,k,\tilde{x}) \in \bra{\R/ (p-1)\Z} \times (0,1) \times \R \},
\]
such that $\mathcal{S}(p) = \{ \mathcal{C}(p) \mid q \in \Q/ (p-1)\Z \}$.
\begin{proof}
Fix $p\neq 1$ and consider $\mathcal{C}(p)$. It is the quotient of $\mathcal{\tilde{A}}(p)$ by the covering transformations, which preserve \ref{} the value of $p$. By Lemma \ref{lem:T_graph}, $\mathcal{\tilde{A}}(p)$ is foliated by $\mathcal{\tilde{A}}(p,q)$ and we have just shown how different $\mathcal{\tilde{A}}(p,q)$ can be identified if their values of $q$ differ by a multiple of $p-1$. Hence it is sufficient to take one representative from each of $\R/(p-1)\Z$ to cover the image.

Lemma \ref{lem:T_graph} also demonstrates that $\mathcal{\tilde{S}}(p)$ is the union of $\mathcal{\tilde{A}}(p,q)$ with $q \in \Q$. $\mathcal{S}(p)$ is the image of $\mathcal{\tilde{S}}(p)$, so it is the subset of $\mathcal{C}(p)$ where $q$ is in the image of $\Q$, that is $\Q/(p-1)\Z$.
\end{proof}
\end{thm}

This leaves just one special case.

\begin{thm}
$\mathcal{C}(1)$ is diffeomorphic to
\[
\{ (q,k,\tilde{u}) \in \R \times (0,1) \times \bra{\R/2π\Z} \},
\]
such that $\mathcal{S}(1) = \{ \mathcal{C}(1) \mid q \in \Q \}$.

\begin{proof}
In this special case, we see that the action does not identify path components with different values of $q$; in fact $p-1$ is zero, so $q$ is fixed under the deck transformations. Consequently, the shifting action identifies values of $\tilde{x}$ that differ by $2π$ in the same path component, resulting in annuli.
\end{proof}
\end{thm}

These two theorems give an understanding of what the condition $T\in\Q$ means. If we think of the first condition, $p\in\Q$, it lead to a dense disjoint collection of subspaces $\mathcal{C}(p)$ on which the condition was met. We can then think of examining the effect of imposing the second condition on one of these spaces. For $p$ not equal to one, the subset of meeting this extra condition is a dense collection of strips. As can be seen in the pictures below, and proved in the next chapter, they should be thought of as intertwined helicoids. But for $p$ equal to one, instead we have a dense collection of annuli. We shall see that the parameter $p$ can be thought of as controlling the slope of the helicoids, with $p=1$ being the transition between right and left handed spirals, where it is `flat' and closes up.

In particular, $\mathcal{S}(1)$ seems special because it is not simply connected. However, on the level of spectral data this disappears and the moduli space is simply connected. But first, we revisit and improve on lemma \ref{lem:closing_conds}. In that lemma, we see that the closing conditions are necessary and sufficient conditions for the existence of spectral data. The proof is constructive, in that it finds a set of spectral data. However, to examine the moduli of spectral data, we must find the `minimal' set from which all other on that curve are generated.

\begin{lem}
On any curve $Σ \in \mathcal{S}$, there is a pair of differentials such that every closing differential is an integer combination of that pair.

\begin{proof}
The method of proof is similar to that of lemma \ref{lem:closing_conds}, but the construction must be refined a little. Let $Ψ$ be as described there. Recall, we write $p = n/m$ in lowest terms and define
\[
a = \frac{2π\iu n}{2\iu η(1)^+}, \qquad Ψ = a Θ^1.
\]
Suppose that $Θ$ is an exact closing differential. We assert that it must be an integer multiple of $Ψ$. To see this, note that it is a real scalar of $Θ^1$ and compute its integrals
\[
\int_{γ_+} Θ = 2\iu η(1)^+ b = 2π\iu \tilde{n}, \qquad
\int_{γ_-} Θ = -2\iu η(-1)^+ b = 2π\iu \tilde{m},
\]
for some $b\in\R$, $n',m' \in \Z$. By the definition of $p$
\[
p = - \frac{η(1)}{η(-1)} = \frac{\tilde{n}}{\tilde{m}},
\]
so we must have that $\tilde{n} = ln, \tilde{m}= lm$ for an integer $l$. It follows that $b= la$ and hence $Θ = l Ψ$.

We now turn our attention to the nonexact differentials. Suppose that for fixed paths $γ_+,γ_-$ the value of $T$ on the curve is $n'/m'$ is lowest terms. Define $g$ to be $m'/\gcd(m,m')$. Now, similar to lemma \ref{lem:closing_conds}, let $x,y$ be integers with $nx-my = 1$ and $Γ^+ = n'my / \gcd(m,m')$, $Γ^- = n'mx / \gcd(m,m')$. Then
\[
\frac{nΓ^- - m Γ^+}{gm}
= \frac{n n'mx - m n'my}{m'm}
= \frac{n'(nx - my)}{m'}
= \frac{n'}{m'},
\]
So we can solve for a differential $\tilde{Ψ}$ with period $2π\iu g$ and integrals $2π\iu Γ^+, 2π\iu Γ^-$. Now if $Θ$ is any closing differential on the curve, we may write it as $Θ = bΘ^1 + kΘ^2$ for $b\in\R$ and $k\in \Z$. Its period is $2π\iu k$ and its integrals are $2π\iu e, 2π\iu f$. Taking its integrals over $γ_+$ and $γ_-$ and eliminating $b$ leads to
\begin{align*}
\frac{n'}{m'} &= \frac{nf - me}{km} \\
n'k\bra{\frac{m}{\gcd(m,m')}} &= (nf - me)\bra{\frac{m'}{\gcd(m,m')}} \\
&= (nf-me)g.
\end{align*}
Now, by construction $g$ is coprime to $m$. Also, $m'$ is coprime to $n'$, so $g$ is also. Hence we see that $g$ must divide $k$. Consider the differential
\[
Θ - \frac{k}{g}\tilde{Ψ}.
\]
It is closing and exact, so a multiple of $Ψ$ by the first part of this proof. Rearranging we have finally written $Θ$ as an integer combination of $Ψ,\tilde{Ψ}$.
\end{proof}
\end{lem}

So the closing differentials are a $\Z\times\Z$-subbundle $\mathcal{M}$ of the $\Z\times\R$-bundle $\mathcal{B}$ over $\mathcal{S}$. $Ψ$ is a well defined function on all of $\mathcal{C}(p)$, but $\tilde{Ψ}$ is not. Its definition depends on the paths $γ_+$ and $γ_-$, and there is no consistent way to make a smooth choice of paths on the whole space. For $p\neq 1$, $\mathcal{S}(p)$ is simply connected however, so it is possible to extend these differentials to a frame for $\mathcal{M}(p)$, showing it to be diffeomorphic to $\Z^2 \times \mathcal{S}(p) \simeq \Z^2 \,\times\, (\Q/(1-p)\Z)\times(0,1)\times\R$.

For $p=1$, take a path component of $\mathcal{S}$. It is an annulus. $p=1=n/m$ implies that $m=n=1$ and $T_0 = m'/n'$ is well defined and constant on the whole annulus. Consider a simple nontrivial loop $\ell : [0,1]$. If we make a choice of $γ_+,γ_-$ on $Σ(\ell(t))$ such that the curves vary smoothly in $t$, what will be the change in $\tilde{Ψ}$ when we return to $\ell(1) = \ell(0)$? Let $\tilde{Ψ} = b Θ^1 + m' Θ^2$ be the initial differential, and let $\tilde{Ψ}'  = b' Θ^1 + m' Θ^2$ be its continuation to $\ell(1)$. From our previous compuations, each of
\[
\int_{γ_+} Θ^2, \int_{γ_-} Θ^2
\]
will be incremented by $2π\iu$ (or decremented, depending on the orientation of $\ell$). The difference $\tilde{Ψ}' - \tilde{Ψ} = (b'-b)Θ^1$ can be explicitly written
\begin{align*}
b'
&= \frac{1}{2\iu η(1)^+}\bra{ 2π\iu y n'm - m' \bra{\int_{γ_+} Θ^2 + 2π\iu} } \\
&= b - \frac{2π\iu}{2\iu η(1)^+}m' \\
&= b - am',
\end{align*}
so that $\tilde{Ψ}' - \tilde{Ψ} = -am'Θ^1 = - m' Ψ$. Recall that the period of $\tilde{Ψ}$ is $g = m' / \gcd(m,m') = m'$. This says that everytime you loop around the annulus, the nonexact differential shifts by $m' Ψ$. Thus the part of the bundle $\mathcal{M}$ over this annulus is simply connected. The basis $Ψ,\tilde{Ψ}$ is not unique, one may add an integer multiple of $Ψ$ to $\tilde{Ψ}$, but this shows that there is a path in the moduli space between bases that have their nonexact differentials differ by $m'\Z Ψ$. Thus the part of the bundle $\mathcal{M}$ over this annulus has $m'$ components.

There are some symmetries and special cases that also bear mention. First, we have already remarked upon and made use of the symmetry
\[
T(p,k,u,v) = -p T\bra{ \tfrac{1}{p}, k, v, u },
\]
but what is its interpretation? Looking at
\[
p = \frac{\abs{1-α}\abs{1-β}}{\abs{1+α}\abs{1+β}}
\mapsto \frac{\abs{1+α}\abs{1+β}}{\abs{1-α}\abs{1-β}},
\]
one would guess that it is induced by $(α,β) \mapsto (-α,-β)$. Indeed this can be seen to be the case, as $k$ is invariant under such a transformation, and the induced map between the spectral curves
\begin{align*}
χ: Σ(α,β) &\to Σ(-α,-β) \\
(ζ, η) &\mapsto (-ζ,-η)
\end{align*}
interchanges $1$ and $-1$, effectively swapping the roles of $u$ and $v$. The pullback of the differentials under the involution $χ$ would preserve the integrality of their integrals and so spectral data on $Σ(α,β)$ be transformed into spectral data on $Σ(-α,-β)$. The harmonic map $g(z) : \T^2 \to \S^3$ arises as the gauge transformation between the connections at $1$ and $-1$, so exchanging these points gives the inverted harmonic map $g(z)^{-1}$.

This can be seen directly in the genus zero case. Remember the form of the harmonic map
\[
g(z) = \exp (4w_R \abs{X}\hat{X}) \cdot \exp (-4w_I \abs{Y}\hat{Y}), \quad
\hat{X} = \begin{pmatrix}
0 & 1 \\
-1 & 0
\end{pmatrix}, \quad
\hat{Y} = \begin{pmatrix}
0 & e^{\iu ω} \\
-e^{-\iu ω} & 0
\end{pmatrix},
\]
for some angle $ω$ between the derivatives in the $1$ and $\iu$ directions, which determines the image up to $SO(4)$ rotation. Under inversion
\[
g(z)^{-1} = \exp (4w_I \abs{Y}\hat{Y}) \cdot \exp (-4w_R \abs{X}\hat{X}).
\]
In the genus zero case, the map between the spectral curves is $(ζ,η) \mapsto (-ζ,\iu η)$, so that the differentials transform as
\[
χ^* d \bra{ ζ^{-1}η (bζ - \bar{b}) } = d \bra{ ζ^{-1}η (\iu b ζ + \iu\bar{b}) }.
\]
In particular, the domain is rotated by $\iu$, so that $w'_I = w_R$ and $w'_R = - w_I$. Rotating so that again the derivative in the real direction is fixed
\[
g(z)^{-1} = \exp (-4w_I \abs{Y}\begin{pmatrix}
0 & 1 \\
-1 & 0
\end{pmatrix})
\cdot \exp (-4w_R \abs{X}\begin{pmatrix}
0 & -e^{-\iu ω} \\
e^{\iu ω} & 0
\end{pmatrix}).
\]
Thus we can see that the parameter $ω$ has been changed to $π-ω$. But on the other hand, examining the spectral curve the parameter $ω$ is given by (for some real $x$)
\[
\iu x e^{\iu ω} = \frac{α+1}{α-1},
\]
which under $α\mapsto -α$,
\[
\iu x e^{\iu ω'} = \frac{-α+1}{-α-1} = \bra{ \frac{α+1}{α-1} }^{-1} = \bra{\iu x e^{\iu ω} }^{-1}.
\]
Solving for $ω'$ gives $π-ω$, in complete agreement to before.

\todo{} THERE is a bunch of fiddling around here to make the two approaches both come to $π-ω$, because the inversion also changes the domain subtlety, and ends up reversing one of the basis vectors. If instead you only cared about the unsigned angle between the vectors (ie up to complementary angle), then you could skip the domain related stuff and just say $-ω$ and $π-ω$ represent the same angle, up to a sign. See how much Emma and John hate the handwaving in this example. Could go either way.

Returning to genus one, a special case of this are the spectral curves that are a fixed point of this transformation. In that case, $χ$ is an extra involution of the spectral curve and hence $β = -α$. These are exactly the genus one spectral curves that meet the conditions for the harmonic map to have as their image a totally geodesic two-sphere. Hitchin identifies a particular one parameter family of these maps as the Gauss maps of Delauney surfaces. We identify in which part of $\mathcal{S}$ they reside.

Firstly, as these are fixed points of $p \mapsto p^{-1}$, $p$ is one. The symmetry in $T_0$ now reads $T_0(1,k,u,v) = - T_0(1,k,v,u)$. The plane $β=-α$ is two dimensional, but we have three parameters $(k,u,v)$ in this relation, so there must be some relation between them. As the four branch points are linear, $μ=\hat{α}$ and $ν = -\hat{α}$. Directly
\[
z_0 = \frac{A+αB}{A-αB} = \frac{1+ \abs{α}}{1-\abs{α}},\quad
k = \bra{ \frac{1- \abs{α}}{1+\abs{α}} }^2.
\]
It follows from eqn \ref{eqn:f_2} that
\[
\iu v = f(-1)
= \bra{\frac{1+ \abs{α}}{1-\abs{α}}}^2 \bra{ \frac{1+ \abs{α}}{1-\abs{α}} \frac{1+\hat{α}}{1-\hat{α}}}^{-1}
= \frac{1}{k} f(1)^{-1},
\]
or concisely that $v= - (ku)^{-1}$. This is exactly the formula for the change of $u,v$ under the label swapping involution, as it should be because $χ$ is in effect swapping $(α,-α)$ amongst doing other things. Hence using the formula for the label swapping involution in addition to the $1/p$ symmetry
\[
T_0\bra{1,k,u,-(ku)^{-1}}
= T_0\bra{1,k,-(ku)^{-1},u} + 1-p
= T_0\bra{1,k,-(ku)^{-1},u}
= -T_0\bra{1,k,u,-(ku)^{-1}},
\]
from which we deduce that $T_0$ is zero. Therefore we have shown that $\{(α,-α)\}$ is the single annulus of $\mathcal{S}(0)$ where $T_0$ is zero, and these correspond to harmonic maps to the sphere.
