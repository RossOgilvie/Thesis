%!TEX root = thesis.tex

\section{Genus One Moduli Space}
% \epigraph{All analytic functions are alike; each non-analytic function is non-analytic in its own way.}

\subsubsection{Rambling}

An outline:
\begin{enumerate}
\item
Give the outline
\item
Compute a basis of differentials with the correct symmetries
\item
Compute their periods and a basis of differentials
\item
Introduce the various moduli spaces we will consider
\item
Formulate the conditions
\item
Reformulate in coordindates more suited to calculation.
\item
Talk about shape of this parameter space.
\item
Talk about $T$ and its multi-defined-ness. Universal cover
\item
Uniform coordinate
\item
Show derivative is nonzero
\item
Apply Implicit Function theorem in uniform coord and k
\item
Project down. Compute the deck transformations and quotient by them to get the topology
\item
Give the basis of differentials over each point. Show every closing diff meets lies in this span. Triviality of the p=1 differentials
\item
Corollories:
The spirally nature, the density
Something about the k=1 limit. Is it a single point?
something about the modulus $τ$ of the domain?
the p, 1/p symmetry corresponding to a reflection, via swapping 1 and -1.
interchanging the diffs swaps the orientation.
Known examples, special cases
\item
Go to the Winchester, have a pint, and wait for this to blow over.
\end{enumerate}

Vocab summary
\begin{enumerate}
    \item
    differentials = differentials with the correct poles and symmetries and imaginary periods
    \item
    closing differentials = differentials with integral periods meeting the closing conditions, alt name differentials satisfying the closing conditions
    \item
    $\mathcal{C}_1$ = the space of spectral curves = $(D^2\setminus Δ) / \Z_2$
    \item
    $\mathcal{A}_1$ = the space of spectral curves with a distinguished branch point = $D^2\setminus Δ$
    \item
    principal path over = a path that follows the recipe laid out for traversing from the two points over $1$ or $-1$. Do not exist on every spectral curve (convention, expand the convention?)
    \item
    $\mathcal{\tilde{A}}_1$ = the universal cover of $\mathcal{A}_1$
    \item
    $\mathcal{S}_1$ = the space of spectral curves that have spectral data, $\subset \mathcal{C}_1$
    \item
    $\mathcal{\tilde{S}}_1$ = the subspace of $\mathcal{\tilde{A}}_1$ that meets the conditions $p\in\Q^+, \tilde{T}\in\Q$
    \item
    Any of these followed by $(p)$, eg $\mathcal{A}_1(p)$ = the subspace where $p(α,β)$ is a fixed value $p$
    \item
    $\mathcal{\tilde{A}}_1(p,q)$ = the subspace where $p(α,β)$ is a fixed value $p$, and $\tilde{T}$ is $q$

    \item
    $\mathcal{B}$ = differentials with integral periods
    \item
    $\mathcal{B}(n)$ = differentials with period $2π\iu n$
    \item
    $\mathcal{B}^2$ = pairs of differentials with integral periods
    \item
    $\mathcal{B}^2(n,\tilde{n})$ = pairs of differentials with periods $2π\iu n$, $2π\iu \tilde{n}$ respectively
    \item
    $\mathcal{M}_1,\mathcal{M}_1(n,\tilde{n})$ = space of spectral data, resp. with specified periods
\end{enumerate}

Once we understand which curves admit spectral data, then we can understand the structure of the differentials on those curves.

The different moduli spaces we could be interested in

There are several interrelated moduli spaces of spectral data that we could be interested in. Spectral data is a triple $(Σ,Θ,\tilde{Θ})$ of a spectral curve $Σ$ and a pair of differentials on it. We know that every spectral curve has many differentials with with correct poles and symmetries. Spectral curves are parameterised by their branch points, and by symmetries we need only track branch points inside the unit circle. By counting degrees, we know there is a plane of such differentials with purely imaginary periods, but in general not every spectral curve will satisfy the further condition that there is a rank two lattice of differentials within this plane that have integral periods. Let the space of spectral curves that have differentials with integral periods be denoted $\mathcal{C}_1$. Within this space we could impose the further condition, the closing condition, on the differentials that requires their integrals over the marked points to also be integral. We shall call such differentials \emph{closing}, and the space of spectral curves with closing differentials $\mathcal{S}_1$.

One can note that the plane of differentials with imaginary periods is naturally thought of as a rank two vector bundle over the space of spectral curves. Thus the moduli of closing spectral data $\mathcal{M}_1$ can be thought of a bundle over $\mathcal{S}_1$. Our ultimate aim will be to describe this bundle.










\subsection{Overview}
\label{sub:Overview}

Recall that the space of smooth, genus one marked curves is denoted $\mathcal{C}_1$, and the space of those curves that admit spectral data is $\mathcal{S}_1 \subset \mathcal{S}_1$. Every marked curve of $\mathcal{C}_1$ admits differentials satisfying Conditions \ref{P:real curve}--\ref{P:imaginary periods} and that also have integral periods (Condition \ref{P:periods}) but not every curve admits differentials that further meet the closing conditions, Condition \ref{P:closing}. The closing conditions therefore can be viewed as defining equations for $\mathcal{S}_1$ within $\mathcal{C}_1$. We will investigate the topology of $\mathcal{S}_1$ and show it to be a dense collection of leaves of a foliation of $\mathcal{C}_1$. Most connected components will be simply connected, though there is also an exceptional family of annuli. The moduli space $\mathcal{M}_1$ of periodic spectral data will be shown to be a trivial bundle over $\mathcal{S}_1$.

A marked curve is determined by its branch points, and because of the real involution $ρ$, the branch points come in conjugate-inverse pairs.
Recall from equation \eqref{eqn:def Ag} that $\mathcal{A}_1$ is the space $\{ (α,β) \in D^2 \mid α \neq β \}$, where $D$ is the open unit disc. The action of $\Z_2$ that interchanges the two points is a free and transitive action. Thus the quotient space $\mathcal{A}_1/\Z_2$ is smooth, and it can be identified with the space of marked curves $\mathcal{C}_1$. $\mathcal{A}_1$ is a two-to-one cover of $\mathcal{C}_1$. We denote the covering map $Σ$, so that
\[
Σ(α,β) = \{ (ζ,η) \mid η^2 = P(ζ) = (ζ-α)(1-\bar{α}ζ)(ζ-β)(1-\bar{β}ζ) \},
\]
is the marked curve with branch points $\{α,\cji{α},β,\cji{β}\}$ and which has the standard scaling \eqref{eqn:def P}.

The main theorems of this chapter are Theorems \ref{thm:topology_curves} and \ref{thm:topology_curves_p1}, which together describe a folition of $\mathcal{C}_1$ and identifies the connected components of $\mathcal{S}_1$ as leaves of this foliation. This result is proved by the construction, in several stages, of a parameterisation of the foliation adapted to the closing conditions. First, in \textsection\ref{sub:Differentials} on each marked curve we find the differentials with integral periods. Having found these differentials, in \textsection\ref{sub:closing conditions} we reformulate the closing conditions to produce two real valued functions on $\mathcal{A}_1$ that are valued in $\Q$ exactly when a given spectral curve admits spectral data. The first function is strictly positive and has an explicit algebraic formula; its level sets foliate $\mathcal{A}_1$ into solid tori. The second function, $T$ (eqn \eqref{eqn:def_T}), is however multivalued and transcendental; it is dependent on paths of integration on the curve, and its formula contains elliptic integrals.

To analyse the two functions, especially the second, we introduce a new set of coordinates on $\mathcal{A}_1$ that are adapted to calculation (\textsection\ref{sub:Reformulate}). These new coordinates dovetail with the results of \textsection\ref{sub:EllipticContinuation}, allowing us to lift $T$ to a  single valued function $\tilde{T}$ on $\mathcal{\tilde{A}}_1$, the universal cover of $\mathcal{A}_1$. We then compute the derivatives of $\tilde{T}$ and use the implicit function theorem to prove that its level sets, denoted $\mathcal{\tilde{A}}_1(p,q)$, are graphs and foliate the space $\mathcal{\tilde{A}}_1$ (Lemma \ref{lem:T_graph}). This allows for an explicit description of $\mathcal{\tilde{S}}_1$, the preimage of $\mathcal{S}_1$ in $\mathcal{\tilde{A}}_1$. Finally, we push this foliation back down to $\mathcal{C}_1$. By uniqueness, $\mathcal{\tilde{A}}_1$ is also the universal cover of $\mathcal{C}_1$, so one recovers $\mathcal{S}_1$ as a quotient by covering transformations (\textsection\ref{sub:Topology}).

To close this chapter, in \textsection\ref{sub:Corollaries}, we show that $\mathcal{M}_1$ is a trivial bundle over $\mathcal{S}_1$. We consider some well-known examples of harmonic maps and show in which component of the moduli space they reside. Building on this, we identify a symmetry of $\mathcal{S}_1$ and provide a geometrical interpretation. We illustrate this interpretation for the analogous symmetry of genus zero spectral curves using the explicit equations for harmonic maps derived in the previous chapter.

We have introduced numerous spaces, so we summarise them and their relationship to one another in the diagram below. One starts out with the space of marked curves $\mathcal{C}_1$ in the top right corner. It is covered by the parameter space $\mathcal{A}_1$, which is in turn covered by its universal cover $\mathcal{\tilde{A}}_1$. Within each of these spaces, we may consider the subspaces on which the first function has the value $p\in\Q^+$. On the bottom line we have the statement that $\mathcal{\tilde{S}}_1$, the preimage of $\mathcal{S}_1$, is the union of those $\mathcal{\tilde{A}}_1(p,q)$ on which $\tilde{T}$ is rational. A double-headed arrow indicates a covering map and the $\Z$ or $\Z_2$ over it indicates the group of covering transformations.
\[
\begin{diagram}
    \mathcal{\tilde{A}}_1 &\rOnto^{\Z}&  \mathcal{A}_1  &\rOnto^{\Z_2}&  \mathcal{C}_1 \\
    \uInto  &&  \uInto  &&  \uInto  \\
    %%%%%%%%%%%%%%%%%%%%%%%%%%%%%%%%%%%%%
    \mathcal{\tilde{A}}_1(p)  &\rOnto^{\Z}&  \mathcal{A}_1(p)  &\rOnto^{\Z_2}&  \mathcal{C}_1(p) \\
    \uInto  &&  &&  \uInto  \\
    %%%%%%%%%%%%%%%%%%%%%%%%%%%%%%%%%%%%%
    \mathcal{\tilde{S}}_1(p) =  \coprod_{q\in\Q} \mathcal{\tilde{A}}_1(p,q)  &&  \rOnto  && \mathcal{S}_1(p)
\end{diagram}
\]












\subsection{Differentials of the Elliptic Spectral Curve}
\label{sub:Differentials}
As outlined in the previous section, it will be necessary to have explicit formulae for the differentials on a marked curve $Σ$ that have integral periods. In this section, we construct the differentials in four stages. First, we give a coordinate transformation $f$ of the marked curve $Σ$ to the Legendre normal form of an elliptic curve. On any marked curve it is possible to find differentials that meet conditions \ref{P:poles}--\ref{P:reality}. The second step is to therefore write down these differentials; the differentials $Θ$ on $Σ$ that have double poles and no residues over $ζ=0,\infty$, that with respect to the hyperelliptic involution $σ$ obey $σ^*Θ = - Θ$, and that are real with respect to $ρ$, which is to say $ρ^* Θ = - \bar{Θ}$. They form a three dimensional real vector space.

Next, we use the transformation $f$ to compute the periods of these differentials in terms of the Jacobi elliptic integrals. Finally, we leverage the standard relations among the Jacobi elliptic integrals to describe explicitly the differentials with integral periods, those which meet condition \ref{P:periods}. Any differential that meets all the conditions \ref{P:poles}--\ref{P:periods} may be written as a linear combination of two distinguished differentials, $Θ^E$ and $Θ^P$. The choice of these two differentials on $Σ(α,β)$ varies smoothly in the parameters $(α,β)$. The space of differentials with integral periods will be shown to be trivial bundles over $\mathcal{A}_1$ and $\mathcal{C_1}$.

With the standard scaling \eqref{eqn:def P}, any marked curve of genus one may be written as
\[
η^2 = (ζ-α)(1-\bar{α}ζ)(ζ-β)(1-\bar{β}ζ),
\]
with roots $α, \cji{α}, β$ and $\cji{β}$. On the other hand, every ellpitic curve may be transformed into the Legendre normal form
\[
w^2 = (1-z^2)(1-k^2z^2)
\]
where $k$ is a complex number called the elliptic modulus. For a given marked curve, how is one to compute the modulus and therefore determine the appropriate Jacobi form into which to transform? The answer lies in the consideration of the cross ratio of the roots
\[
[α,\cji{α};β,\cji{β}] = \frac{\abs{α-β}^2}{\abs{1-\bar{α}β}^2}. \labelthis{eqn:roots_cross_ratio}
\]
This is a real quantity, and so the four roots lie on a circle (or a line). Thus the roots of the Legendre form do also, which forces $k$ to be real. Any transformation between the curves must take branch points to branch points, so we must decide on a correspondence for the roots. There are twenty-four possible choices, but we choose the correspondence
\[
  \begin{array}{ l||c|c|c|c}
    ζ & α & \cji{α} & β & \cji{β} \\
    \hline
    z & 1 & -1 & k^{-1} & -k^{-1}
  \end{array}
\]
This correspondence has three properties that distinguish it from the others. By convention, $k \in (0,1)$, which rules out sixteen of the choices. Second, consider the behaviour of the curve as $k\to 1$. The Legendre form of the curve develops two nodes at $z=\pm 1$. This corresponds to forming nodes $α=β$ and $\cji{α} = \cji{β}$. But the value of
\[
[1,k^{-1};-1,-k^{-1}] = \frac{4k}{(k+1)^2}
\]
disagrees with eqn \eqref{eqn:roots_cross_ratio} in this limit, which rules out this correspondence and the three others with the same cross ratio. Finally, our choice of correspondences takes the unit disc to the right half plane. There is only one other correspondence that has all three of these properties, namely
\[
  \begin{array}{l || c|c|c|c}
    ζ & β & \cji{β} & α & \cji{α} \\
    \hline
    z & 1 & -1 & k^{-1} & -k^{-1}
  \end{array}
\]
The difference between the two is a choice of which root, $α$ or $β$, is mapped to $1$. This is the reason that we work with $\mathcal{A}_1$ and not $\mathcal{C}_1$. On the latter space, there would be no way to consistently make this choice. \todo{improve this point} From our correspondence, equating the cross ratios $[α,\cji{α};β,\cji{β}] = [1,-1;k^{-1},-k^{-1}]$ gives
\[
k = \frac{\abs{1-\bar{α}β}-\abs{α-β}}{\abs{1-\bar{α}β}+\abs{α-β}},
\labelthis{eqn:def_k}
\]
and the map, which we shall call $f$, can be computed from the relation $[α,\cji{α};β,ζ] = [1,-1;k^{-1},f(ζ)]$. Instead of solving this relation for $f$ immediately, we turn to understanding the geometry of the transformation so that we might produce a meaningful expression.

The unit circle in the $ζ$-plane plays an important role in the definition of a spectral curve, so it is natural to ask what is its image under $f$. The involution $ρ(ζ)$ fixes the unit circle, and exchanges the pairs of branch points $α,\cji{α}$ and $β,\cji{β}$. The corresponding antiholomorphic involution $\tilde{ρ}(z)$ in the $z$-plane that exchanges $1,-1$ and $k^{-1},-k^{-1}$ is $z\mapsto -\bar{z}$. Its fixed point set is the imaginary axis, which therefore must be the image of the unit circle under $f$.

As already mentioned, the four roots of the spectral curve lie on a circle (or a line), which we shall call the branch circle. Let two points at the intersection of the branch circle with the unit circle be $μ$ and $ν$, with $μ$ lying between $α$ and $\cji{α}$ and $ν$ lying between $β$ and $\cji{β}$ (see Figure \ref{fig:zeta plane}). Under $f$, the branch circle is mapped to the real axis. Therefore the $f(μ)$ and $f(ν)$ must lie on the intersection of the real and imaginary axes. Hence $f(μ) = 0$ and $f(ν) = \infty$.

A M\"obius tranformation, such as $f$, is determined up to scaling by the points it sends to $0$ and $\infty$, in this case $μ$ and $ν$. One other point is therefore needed to determine this scaling. We write $z_0 := f(0)$. Using the reality structure $\tilde{ρ}(z)$, we have $f(\infty) = -\bar{z_0}$. These points allow us to write concise formulae for $f$ and $f^{-1}$
\begin{align}
z = f(ζ) &= -\bar{z}_0 \frac{ζ - μ}{ζ - ν},
\label{eqn:f} \\
ζ = f^{-1}(z) &= ν \frac{z - z_0}{z + \bar{z_0}}.
\label{eqn:f_inv}
\end{align}

\makefigure{The $ζ$-plane, with points marked in red. The black labels are their images under $f$\label{fig:zeta plane}}{thesis_graphics_temp/zeta_plane.png}
\makefigure{The $z$-plane, with points marked in black. The red labels are their images under $f^{-1}$}{thesis_graphics_temp/z_plane.png}

\begin{lem}
\label{lem:coeff_f_smooth}
The functions $μ,ν,z_0,(z_0)^{-1}$ and $k$ are smooth functions of the parameters $(α,β)\in\mathcal{A}_1$. The function $f(α,β)(ζ)$ is a smooth function of $(α,β,ζ)\in\mathcal{A}_1 \times \CP^1$ whenever $ζ \neq ν$. Further $μ - ν$ is never zero.
\begin{proof}
Recall that $μ$ is the point such that $f(μ) = 0$. We may find a formula for $μ$ in terms of $α,β$ using the cross-ratio relation $[α,\cji{α};β,μ] = [1,-1;k^{-1},0]$. Rearranging gives
\[
μ = \frac{ (α-β)\abs{1-\bar{α}β} + α(1-\bar{α}β)\abs{α-β} }{ \bar{α}(α-β)\abs{1-\bar{α}β} + (1-\bar{α}β)\abs{α-β} }.
\]
This could fail to be a smooth function if $α-β$ or $1-\bar{α}β$ were zero, or if the denominator was zero. $α-β$ is never zero on $\mathcal{A}_1$ by definition. The factor $1-\bar{α}β$ is zero if and only $\cji{α}=β$ and, as both $α,β$ are inside the unit disc, this is impossible. Finally, the denominator is zero exactly when
\[
\bar{α} = - \frac{1-\bar{α}β}{\abs{1-\bar{α}β}} \frac{\abs{α-β}}{α-β}.
\]
But the right hand side is an element of the unit circle, so again this possibility is ruled out. The proof of smoothness is entirely similiar for
\begin{align*}
ν &= \frac{ (α-β)\abs{1-\bar{α}β} - α(1-\bar{α}β)\abs{α-β} }{ \bar{α}(α-β)\abs{1-\bar{α}β} - (1-\bar{α}β)\abs{α-β} }, \\
z_0 &= \frac{ α(\bar{α}-\bar{β})\abs{1-\bar{α}β} + (1-α\bar{β})\abs{α-β} }{ α(\bar{α}-\bar{β})\abs{1-\bar{α}β} - (1-α\bar{β})\abs{α-β} }, \\
(z_0)^{-1} &= \frac{ α(\bar{α}-\bar{β})\abs{1-\bar{α}β} - (1-α\bar{β})\abs{α-β} }{ α(\bar{α}-\bar{β})\abs{1-\bar{α}β} + (1-α\bar{β})\abs{α-β} }, \text{ and} \\
k &= \frac{\abs{1-\bar{α}β}-\abs{α-β}}{\abs{1-\bar{α}β}+\abs{α-β}}.
\end{align*}
By the formula \eqref{eqn:f}, we also conclude that $f$ is smooth so long as the denominator is nonzero.

The final claim is that $μ-ν$ is never zero. This is clear from the geometry, as the branch circle is a circle that passes through points both inside and outside the unit circle and so must intersect the unit circle at distinct two points. Algebraically, the difference vanishes only if
\[
2\bra{\abs{α}^2 - 1} (α-β)(1-\bar{α}β) = 0,
\]
and this has already shown not to occur on $\mathcal{A}_1$.
\end{proof}
\end{lem}

Because of the holomorphic involution $σ: η\to-η$, equations \eqref{eqn:f} and \eqref{eqn:f_inv} almost but not quite specify a relation between $η$ and $w$: there is a free sign choice to make. On the spectral curve $Σ = \{ (ζ,η) | η^2 = P(ζ) \}$ there are two disjoint circles in $Σ$ lying over the unit circle $ζ=1$. Thus there is a notion of the `positive' unit circle, the one on which $η$ is positive over $ζ=1$. At a point $(ζ,η)$ on the positive unit circle in $Σ$, we define the function $η^+$ to be the value of $η$. Algebraically, $η^+(ζ) = ζ\abs{ζ-α}\abs{ζ-β}$.

Under the transformation $f$ the unit circle is mapped to the imaginary axis. At the points over the imaginary axis in the Jacobi elliptic curve, those points $(z,w)$ for which $z=\iu u$, we have that $w = \pm \sqrt{1+u^2}\sqrt{1+k^2u^2}$. Again, it is possible to make a consistent choice of sign along these two disjoint circles. For the sake of being concrete, we choose the transformation between elliptic curves to map the positive unit circle to points over the imaginary axis where $w$ is positive. In a slight abuse of notation, we shall also use $f$ to denote the map between elliptic curves.

Having found the map $f$ that transforms the genus one marked curve $Σ$ into Jacobi normal form, we may now turn our attention to the other part of the spectral data, the differentials that satisfy conditions \ref{P:poles}--\ref{P:periods}. Let us first find the vector space of differentials that satisfy just conditions \ref{P:poles}--\ref{P:reality}, and write a basis for this space. Following the notation of Section \ref{sub:prelim diffs}, recall that all differentials on a genus one marked curve $Σ$ that have (at most) a double root over $ζ=0$ and $ζ=\infty$ may be written the form
\[
Θ = b(ζ)\frac{dζ}{ζ^2η},
\]
for a polynomial $b(ζ)$ of degree $4$. Condition \ref{P:reality} forces $b\in \mathcal{P}^4_\R$, the space of polynomials that are real with respect to $ρ$. Practically, if we write $b(ζ) = b_0 + \dots + b_4 ζ^4$, this condition forces $b_i = b_{4-i}$. Lastly, if we write out the equation of $Σ$ as $η^2 = P(ζ) = P_0 ζ + \dots + P_4 ζ^4$, $Θ$ has no residues exactly when $P_1b_0 - 2P_0b_1 = 0$ (equation \ref{eqn:residue condition}). If we count the degrees of freedom remaining, we may choose $(b_0,b_1)$ from a complex line and $b_2$ to be any real number. Hence for each marked curve $Σ$ there is a real three dimensional vector space $W$ of differentials that meet conditions \ref{P:poles}--\ref{P:reality}.

This presents an obvious choice of basis, but one that we will not choose. Instead, we shall choose a basis that it suited to computing the periods of the differentials so that we may be able to satisfy condition \ref{P:periods}, that the periods of the differentials lie in $2π\iu$. For this we turn to the the classical theory of elliptic curves, which has long studied differentials and their periods. It is standard to refer to differentials with double poles and no residues as differentials of the second kind. Condition \ref{P:poles} may therefore be rephrased as the differential must be of the second kind and have poles at the points of $\Sigma$ over $ζ=0$ and $\infty$. The standard Jacobi differential of the second kind is defined to be
\[
e := (1-k^2 z^2) \frac{dz}{w}.
\labelthis{eqn:def_e}
\]
Every differential of the second kind may be written as the linear conbination of $e$, the holomorphic differential $ω$,
\[
ω := \frac{dz}{w},
\]
and an exact differential. GIVE CITATION.\todo{this}

Note that the holomorphic differential $ω$ lies in the space $W$ and so makes for an obvious first basis vector. It accounts for every possible choice of $b_2$.
In genus one, there is a real and exact differential with double poles over $ζ=0,\infty$, namely
\[
Θ^E := \iu\; d\left( \frac{η}{ζ} \right).
% = \iu\left[ -αβ + \frac{1}{2}\left(α(1+\abs{β}^2) + β(1+\abs{α}^2)\right)ζ - \frac{1}{2}\left(\bar{α}(1+\abs{β}^2) + β(1+\abs{α}^2)\right)ζ^3 + \bar{α}\bar{β}ζ^4 \right]\frac{dζ}{ζ^2η}.
\]
We take it as the second basis vector. The superscript $E$ is a mnemonic for exact. Given that we already have taken $ω$ as a basis vector, we seek to complete the basis of $W$ with the sum of $e$ and an exact differential.

Recall equation \eqref{eqn:def_e}, the definition of $e$. It is real with respect to $\tilde{ρ}(z) = -\bar{z}$, but it has a pole at $z=\infty$ ($ζ=ν$), which not allowed under condition \ref{P:poles}. This pole can be moved to $ζ=\infty$ by adding an exact differential. We assert that
\[
e + d\left[ \frac{w}{z + \bar{z}_0} \right]
\]
has no pole at $z=\infty$. To check this let $z' = z^{-1}$, and expand $e$ about $z' = 0$.
\begin{align*}
w &= kz'^{-2} \bra{1 + O(z'^2) } \\
e &= kz'^{-2} dz' \bra{1-k^{-2}z'^2}\bra{1 + O(z'^2)} = kz'^{-2} dz' \bra{1 + O(z'^2)},
\end{align*}
whereas the exact differential has the following expansion
\begin{align*}
d\left[ \frac{w}{z + \bar{z}_0} \right]
&= d\left[ k z'^{-1} \bra{1  + O(z'^2)} \bra{1 - \bar{z}_0z' + \bar{z}_0^2z'^2  + \ldots} \right]\\
&= d\left[ k z'^{-1} \bra{1  + O(z')} \right]\\
&= -k z'^{-2}dz' \bra{1 + O(z'^2)},
\end{align*}
which shows that their sum is holomorphic at $z=\infty$. Unfortunately, this differential is not real with respect to $\tilde{ρ}$. To correct this deficency we shall have to add another exact differential. The set of exact differentials (not necessarily real) with the double poles over $ζ=0,\infty$ is
\[
\Set { C\,d\left[ \frac{w}{(z-z_0)(z + \bar{z}_0)} \right] }{ C\in \C},
\]
so we add a differential of this form to restore the reality. The differential
\[
e + d\left[ \frac{w}{z + \bar{z}_0}\right] + C\,d \left[\frac{w}{(z-z_0)(z + \bar{z}_0)}\right]
= e + d\left[ \frac{w(z - z_0 + C)}{(z-z_0)(z + \bar{z}_0)}\right]
\]
is real when $z_0 - C$ is an imaginary number. Thus we should choose $C \in \Re z_0 + i\R$. The apparent freedom to choose the imaginary part of $C$ is exactly adding a scalar of $Θ^E$ and so will not change the span of the resulting basis. There are two natural choices; taking $z_0 - C$ to be zero, or taking $C$ to be purely real. Both have their merits, but the latter choice ends up being superior as it makes the principal part perpendicular to the principal part of $Θ^E$, which introduces a symmetry that we will use later. Hence we take as our third basis differential
\[
λ := e + d\left[ \frac{w}{z + \bar{z}_0} \right] + \Re z_0\; d\left[ \frac{w}{(z-z_0)(z + \bar{z}_0)} \right]
= e + d\left[ \frac{w(z - \iu\,\Im z_0)}{(z-z_0)(z + \bar{z}_0)} \right].
\]

We are now in a position to compute the periods of our basis $ω,Θ^E,λ$. We choose a basis of homology of the marked curve $Σ$ following way. First, take the branch cuts to be along the branch circle, between $α$ and $β$ and between $\cji{α}$ and $\cji{β}$, in particular so that they do not cross the unit circle. Under $f$, this corresponds to the standard choice of $[1,k^{-1}]$ and $[-k^{-1},-1]$. We can speak of points of $Σ$ as being on the positive or negative sheet, where as before the positive sheet is where $η$ is positive over $ζ=1$.

For the loop $A$, start on the positive unit circle at $μ$, traverse in and around $α$ anticlockwise (crossing a branch cut), then cross the negative unit circle and continue anticlockwise around $\cji{α}$ before returning to the starting point. For the loop $B$, start at the same point we began $A$ and follow the unit circle clockwise. The image of $A$ under $f$ is the anticlockwise loop around $-1$ and $1$ with the left to right part of the path on the positive sheet. $f(B)$ is simply a traversal of the imaginary axis from bottom to top on the positive sheet. But this is homologous to the standard clockwise loop around $1$ and $k^{-1}$.

\makefigure{$A$ in red and $B$ in black}{thesis_graphics_temp/zeta_periods.png}

The loop $A$ is a real period, which means that the integral of a real differential over $A$ is real. Recall that a differential $Θ$ is real with respect to $ρ$ when $ρ^* Θ = - \bar{Θ}$. By construction, $ρ_* A = -A$. Together, we see that
\[
\bar{\int_A Θ}
= \int_A \bar{Θ}
= -\int_A ρ^* Θ
= -\int_{ρ_* A} Θ
= -\int_{-A} Θ
= \int_{A} Θ,
\]
which shows the $A$-period of $Θ$ to be real. Likewise, the integral of a real differential over $B$ is always an imaginary number.

We wish to be able to describe the differentials of $W$ that have having imaginary and integral periods. As $Θ^E$ is exact, any real multiple of it fits this description. Suppose there existed $Θ^P\in W$ that had a real period of zero and an imaginary period of $2π\iu$. If we had any other differential in $W$ with vanishing real period and an imaginary period of  $2π\iu l$, then by subtracting $lΘ^P$ the result would be an exact differential, and therefore would have to be a real multiple of $Θ^E$. In short, if we are able to find such a differential $Θ^P$, then any differential meeting conditions \ref{P:poles}--\ref{P:periods} would lie in $\R Θ^E + \Z Θ^P \subset W$.

Thus we have reduced the problem to finding a differential $Θ^P \in W$ with vanishing $A$-period and a $B$-period of $2π\iu$. The superscript $P$ is a mnemonic for period. Clearly, we may add a real multiple of $Θ^E$ to $Θ^P$ and it will also meet this requirement. Therefore let $Θ^P = Ωω + Λλ$, for some real constants $Ω,Λ$, be a combination of the other two basis vectors of $W = \R\langle ω,Θ^E,λ\rangle$.

It is useful at this point to summarise the standard elliptic integrals, the periods of the differentials $ω$ and $e$,
\begin{align*}
\int_{f(A)} ω &= 4K(k) &
\int_{f(B)} ω &= 2\iu K' \\
\int_{f(A)} e &= 4E(k) &
\int_{f(B)} e &= 2\iu(K'-E')
\end{align*}
where $K$ and $E$ are the complete elliptic integrals of the first and second kind, and the prime denotes not the derivative but instead the elliptic complement. The elliptic complement is by definition $k' = \sqrt{1-k^2}$, and $K'(k) = K(k')$ and $E'(k) = E(k')$. Further properties of elliptic integrals may be found in Appendix \ref{sec:Elliptic Integrals}. Note that $λ$ is the sum of $e$ and an exact differential and so has the same periods as $e$.

We require that
\begin{align*}
\int_A Θ^P &= 4KΩ + 4EΛ = 0 \\
\int_B Θ^P &= 2\iu K' Ω + 2\iu(K'-E')Λ = 2π\iu.
\end{align*}
From the first equation, we can write $Λ = - ΓK$ and $Ω = ΓE$ for some $Γ\in \R$. Substituting this into the second equation gives
\begin{align*}
π
&= K'ΓE - (K'-E')ΓK \\
&= Γ(K'E + KE' - KK') \labelthis{eqn:legendre_relation}\\
&= \frac{\pi}{2}Γ \\
Γ &= 2
\end{align*}
where \eqref{eqn:legendre_relation} uses Legendre's relation (see Lemma \ref{lem:Legendres relation}). Thus $Θ^P = 2Eω - 2Kλ$, or if we unwind these definitions a little
\[
Θ^P = 2E ω - 2Ke - 2K d\left[ \frac{w(z-\iu\,\Im z_0)}{(z-z_0)(z + \bar{z}_0)} \right].
\labelthis{eqn:theta2}
\]
This equation shows a nice division, with the first two terms providing the desired periods and the last term giving the required poles. Though the choices up to this point may seem arbitrary and contrived, they are not, as we can characterise the differential $Θ^P$ in the following way.

\begin{lem}
    \label{lem:theta2_characterisation}
The differential $Θ^P$ is the unique real differential on the marked curve $Σ$ with double poles and no residues over $ζ=0$ and $\infty$, with periods $0$ and $2π\iu$ over $A$ and $B$, and with principal part over $ζ=0$
\[
\pp Θ^P \in \iu \R \pp Θ^E.
\]

\begin{proof}
We first verify that $Θ^P$ has such properties, then verify uniqueness. The only property not yet demonstrated is the third one, concerning the principal part. We note that $ζ=0$ corresponds to $z=z_0$, so for some real scalar $r$
\[
\pp Θ^E
= \iu \pp d\bra{\frac{η}{ζ}}
= \iu r \pp d \bra{ \frac{w}{(z-z_0)(z + \bar{z}_0)} }
= -\iu r \frac{w(z_0)}{z_0 + \bar{z}_0}\frac{dz}{(z-z_0)^2}.
\]
On the other hand
\begin{align*}
\pp Θ^P
&= - 2K \pp d\left[ \frac{w(z-\iu\,\Im z_0)}{(z-z_0)(z + \bar{z}_0)} \right] \\
&= + 2K \frac{w(z_0)(z_0-\iu\,\Im z_0)}{z_0 + \bar{z}_0} \frac{dz}{(z-z_0)^2} \\
&= 2K \frac{w(z_0)(\Re z_0)}{z_0 + \bar{z}_0} \frac{dz}{(z-z_0)^2}.
\end{align*}
To establish uniqueness, suppose that $Θ$ was another such differetial. Then $Θ-Θ^P$ would be exact, real and have double poles with no residues. So for some real $s$
\[
Θ = Θ^P + s Θ^E.
\]
As taking principal part is linear, the third condition forces $s=0$.
\end{proof}
\end{lem}

As we have already remarked, having found this pair of differentials any other differential satisfying \ref{P:poles}--\ref{P:periods} may be written in the form $a Θ^E + n Θ^P$ for some $a\in\R$ and $n\in\Z$. We may think of the differentials satisfying \ref{P:poles}--\ref{P:periods} as forming a $\Z\times\R$-bundle over $\mathcal{A}_1$. This pair of differentials $Θ^E$ and $Θ^P$ vary smoothly with respect to $(α,β)$ and are always linearly independent so they trivialises that bundle.

Recall that $\mathcal{A}_1$ double covers $\mathcal{C}_1$; the points $(α,β)$ and $(β,α)$ in $\mathcal{A}_1$ correspond to the same marked curve $Σ\in \mathcal{C}_1$. If we have a section of differentials $Θ : (α,β) \mapsto Ω^1(Σ(α,β))$, this raises the question on how $Θ(α,β)$ and $Θ(β,α)$, which are on the same curve $Σ(α,β)$, differ. We observe directly that $Θ^E = \iu d (η/ζ)$ is invariant under the interchange of $α$ and $β$. Now we may use the characterisation of $Θ^P$ given by Lemma \ref{lem:theta2_characterisation} to conclude the same for it, because
\[
\pp Θ^P(β,α) \in \iu \R \pp Θ^E(β,α) = \iu \R \pp Θ^E(α,β)
\]
and so uniqueness forces $Θ^P(β,α) = Θ^P(α,β)$.

The consequence of this is that $(Θ^E,Θ^P)$ pushes forward to a well-defined basis of the differentials over $\mathcal{C}_1$ as well, and just as for $\mathcal{A}_1$ they trivialise the bundle $\mathcal{B}_1 \to \mathcal{C}_1$. Though we mainly concentrate on the subspace of $\mathcal{A}_1$ of spectral curves that admit spectral data, at the end of this chaper we will consider the space of spectral data within the total space of $(\mathcal{B}_1)^2$.


\subsection{The Closing Conditions}
\label{sub:closing conditions}
Recall the closing conditions, Condition \ref{P:closing}. These are the conditions that spectral data must meet in order that they correspond to a harmonic map of the torus, rather than a harmonic map of the plane (of finite type). If $(Σ,Θ,\tilde{Θ})$ is a triple of spectral data, $Θ$ satisfies the closing condition at $ζ=1$ if
\[
\int_{γ_{+}} Θ \in 2π\iu \Z,
\]
where $γ_+$ is a path that begins at $(1,η^+(1)^-)$ and ends at $(1,η^+(1))$, the two points on the spectral curve lying over $ζ=1$. Recall that we have defined $η^{\pm}(ζ) = \pmζ\abs{ζ-α}\abs{ζ-β}$ to be the value of $η$ on the positive and negative unit circles in $Σ$ respectively. However, the value of the integral is dependent on the particular path chosen. To see that this condition is none-the-less well defined, suppose that $γ$ and $γ'$ are two paths between the two points over $ζ=1$. Their difference $γ-γ'$ is a closed loop and homologous to an integral combination $aA + bB$ of periods $A$ and $B$. The difference in the values of the integrals is therefore
\[
\int_{γ} Θ - \int_{γ'} Θ
= \int_{γ - γ'} Θ
= a\int_{A} Θ + b\int_{B} Θ
= 2π\iu nb,
\]
because by the period condition \ref{P:periods} the real period of $Θ$ is zero and its imaginary period is a multiple of $2π\iu$. So although the value of the integral is dependent on the path, the condition that the value must lie in $2π\iu \Z$ is not. Likewise, the closing condition at $ζ=-1$ is defined by taking $γ_-$, a path from $(-1,η^-(-1))$ to $(-1,η^+(-1))$, and requiring the integral of $Θ$ over this path to lie in $2π\iu \Z$ also.

We have seen that every marked curve $Σ\in\mathcal{A}_1$ admits differentials that meet Condtions \ref{P:poles}--\ref{P:periods}, but it is not possible to find differentials on every curve that further satisfy Condition \ref{P:closing}. It will be our ongoing aim to find all such curves in $\mathcal{A}_1$.

Let us begin by formulating a condition on $(α,β)$ such that $Σ(α,β)$ admits an exact differential that meets the closing conditions. For an exact differential, such as $Θ^E$, the particular path of integration is irrelevant and the value of the integral is
\[
\int_{γ_{+}} Θ^E = i \left. d\bra{\frac{η}{ζ}} \right|_{(1, η^-(1))}^{(1, η^+(1))} = 2i η^+(1) = 2i \abs{1-α}\abs{1-β}
\]
And similiarly over the other marked point
\[
\int_{γ_{-}} Θ^E = -2i η^+(-1) = 2i \abs{1+α}\abs{1+β}.
\]
One can use these formulae to define an explicit condition that a marked curve must satisfy. Any exact differential with the properties of \ref{P:poles}--\ref{P:periods} must be a real multiple of $Θ^E$. Let the scaling factor be $a\in\R$. The two closing conditions applied to $a Θ^E$ are then
\begin{align*}
2\iu η^+(1) a &\in 2π\iu \Z, \\
-2\iu η^+(-1) a &\in 2π\iu \Z.
\end{align*}
Eliminating $a$ from the two equations, there is a common solution for $a$ if and only if
\[
S(α,β) := \frac{2\iu η^+(1)}{-2\iu η^-(-1)} = \frac{\abs{1-α}\abs{1-β}}{\abs{1+α}\abs{1+β}} \in \Q^+.
\labelthis{eqn:def_S}
\]
In other words, if and only if $S(α,β)$ is a positive rational. This gives the flavour of what we are aiming to achieve. We will produce two explicit functions such that a marked curve admits spectral data exactly when these functions take certain values. The two functions may be interpreted as defining equations for the subspace of spectral curves $\mathcal{S}_1$ within the space of all marked curves $\mathcal{C}_1$.

Before we plow ahead to differentials with periods, there is a simplification we can make. Suppose that we have a triple of spectral data $(Σ,Θ^1,Θ^2)$ such that the differentials $Θ^1,Θ^2$ have imaginary periods $2π\iu n^1, 2π\iu n^2$ respectively. Let $l>0$ be the greatest common denominator of $n^1$ and $n^2$, and by Bezout's identity let $x,y$ be the integers that satisfy
\[
xn^1 + yn^2 = l.
\]
Then consider the differentials $Ψ^E,Ψ^P$ defined by the following integer combination
\[
\vt{Ψ^E}{Ψ^P} =
\begin{pmatrix}
\tfrac{n^2}{l}    &   -\tfrac{n^1}{l} \\
x                       &   y
\end{pmatrix}
\vt{Θ^1}{Θ^2}
\]
The new pair of differentials are simpler in the sense that their imaginary periods are $0$ and $2π\iu l$ respectively. They also meet the closing condition, because they are an integer combination of differentials that do. And the integer matrix has determinant one, so is invertible over the integers. Further, the two differentials are linearly dependent exactly when $Ψ^E$ is zero. Hence,

\begin{lem}
\label{lem:exist spectral data}
A marked curve admits spectral data if and only if it admits a pair of nonzero differentials with imaginary periods $0$ and $2π\iu l$, for some positive integer $l$, that also satisfy the closing conditions \ref{P:closing}.
\end{lem}

Thus the condition above, $S\in\Q^+$, is a necessary conditon for a spectral curve to admit spectral data. To find a second necessary condition, concerning the existence of a  differential with imaginary period $2π\iu l$, we follow the same line of reasoning. For some real number $b$, we may write $Ψ^P = b Θ^E + l Θ^P$. Fix two paths $γ_+, γ_-$. The two closing conditions applied to $Ψ^P$ are then
\begin{align*}
2η^+(1) \iu b + l\int_{γ_+} Θ^P &= 2π\iu Γ^+ \in 2π\iu \Z, \\
-2η^+(-1) \iu b + l\int_{γ_-} Θ^P &= 2π\iu Γ^- \in 2π\iu \Z.
\end{align*}
Again elimination of $b$ yields the condition for a common solution to exist. This equation can be written as
\[
2π\iu T(α,β,γ_+,γ_-) := S(α,β) \int_{γ_-} Θ^P - \int_{γ_+} Θ^P = 2π \iu \frac{S(α,β) Γ^- - Γ^+}{l}.
\labelthis{eqn:def_T}
\]

\begin{lem}
\label{lem:closing_conds}
A marked curve admits a pair of nonzero differentials satisfying the closing conditions, one exact and one inexact, if and only if $S\in\Q^+$ and $T\in\Q$ for any paths $γ_+, γ_-$.

\begin{proof}
From the above discussion, these are a necessary conditions.

For the converse suppose that the are both rational, say $S = n/m$ and $T = n'/m'$. The rationality of $S$ ensures the consistent solution of an $a$, and hence the existence of $Ψ^E = aΘ^E$ satisfying the closing conditions. Explicitly
\[
a = \frac{2π\iu n}{2\iu η^+(1)} = \frac{2π\iu n}{-2\iu η^+(-1)}\frac{-2\iu η^+(-1)}{2\iu η^+(1)} = \frac{2π\iu n}{-2\iu η^+(-1)} \frac{1}{p} = \frac{2π\iu m}{-2\iu η^+(-1)}.
\]

To find a $Ψ^P$, we must first solve
\[
T = \frac{n'}{m'} = \frac{n'm}{m'm} = \frac{n Γ^- - mΓ^+}{lm}.
\]
We can see that we may take $l$ to be $m'$. Let $x,y$ be integers such that $xm-ny = 1$. Then we may take
\[
Γ^+ = n'mx,\;\; Γ^- = n'my,
\]
to obtain equality of the numerator. Hence the condition for the existence of a consistent solution for $b$ is met. The constant $b$ may be computed as
\[
b = \frac{1}{2\iu η^+(1)}\bra{ 2π\iu y n'm - m' \int_{γ_+} Θ^P }.
\]
In summary, we have found real constants $a,b$ and $l$ such that $Ψ^E = aΘ^E, Ψ^P = bΘ^E + lΘ^P$ solve the closing conditions, and so we have demonstrated that the spectral curve admits differentials satisfying conditions \ref{P:poles}--\ref{P:closing}.
\end{proof}
\end{lem}

Thus the space of genus one spectral curves that admit closing spectral data is defined by the equations $S(α,β) \in \Q^+, T(α,β) \in \Q$ inside $\mathcal{A}_1$. In general the closing conditions are difficult to work with, harder than even the period conditions. In the genus one case that we are dealing with the integrals of $Θ^E$ lead to an algebraic expression, as we have just seen in \eqref{eqn:def_S}, but the integrals of $Θ^P$ will lead to a transcendental conditions involving incomplete elliptic integrals.

Note that although the condition $T\in\Q$ is well defined, $T(α,β)$ is a multivalued function on $\mathcal{A}_1$. It it dependent on the paths of integration. As previously seen, different paths will change the integrals of $Θ^P$ by multiples of $2π\iu$ and therefore $T(α,β)$ is defined up to an element of $\Z\langle 1,S(α,β)\rangle \subset \Q$. So that we may work with a well-defined function, we will make some branch cuts on $\mathcal{A}_1$ and choose a principal branch of $T$.

Let us then make a principal choice of paths, and let the value of $T$ on these principal choices be denoted $T_0$. Consider the open dense subset $\mathcal{A}_1 \setminus \{ν = \pm 1\}$ of $\mathcal{A}_1$. On any marked curve corresponding to a point of $\mathcal{A}_1 \setminus \{ν = \pm 1\}$, let $γ_+$ be the path that begins at $(1,η^-(1))$, traverses the unit circle to the point $μ$ without crossing $ν$, follows the branch circle to $α$, circles this branch point anticlockwise, goes back along the arc (though on a different sheet now) to the unit circle, and back to $(1,η^+(1))$. Likewise for $γ_-$ from $(-1,η^-(-1))$ to $(-1,η^+(-1))$. These paths are illustrated in figure \ref{fig:gamma paths}. In the case $ν=\pm 1$, it would be impossible to `avoid' $ν$, so this case had to be excluded.

\makefigure{$γ_+$ in red and $γ_-$ in black\label{fig:gamma paths}}{thesis_graphics_temp/gamma_pm_zeta.png}

As $ν\neq 1$, we know that $f(1) \neq \infty$ and so $f(γ_+)$ lies in the plane (it doesn't cross $z=\infty$).
By design it is easy to describe these paths in terms of the $(z,w)$ coordinates.
Start from the point $f(1)$ on the imaginary axis and go to the origin. Go out along the real axis, around $z=1$ (which corresponds to $ζ=α$) and back again to the origin. Return along the imaginary axis to $f(1)$. This is illustrated in figure \ref{fig:f gamma paths}.

\makefigure{$f(γ_+)$ in red and $f(γ_-)$ in black\label{fig:f gamma paths}}{thesis_graphics_temp/gamma_pm_z.png}

Having fixed a choice of paths, we can write explicit formulae for the value of the integrals on these particular paths.
\begin{align*}
\int_{γ_+} ω
&= \bra{2\int_0^{f(1)} - 2\int_0^1} ω
= 2 F(f(1);k) - 2 K(k) \\
\int_{γ_+} e
&= 2 \tilde E(f(1);k) - 2 E(k) \\
\int_{γ_+} Θ^P
&= \int_{γ_+} (2Eω - 2Ke) - 2K \int_{γ_+} d\left[ \frac{w(z-\iu\,\Im z_0)}{(z-z_0)(z + \bar{z}_0)} \right] \\
&= 4 E(k) F(f(1);k) - 4 K(k) E(f(1);k) - 4K \frac{w(f(1))\,(f(1)-\iu\,\Im z_0)}{(f(1)-z_0)(f(1) + \bar{z}_0)}.\labelthis{eqn:gamma_plus}
\end{align*}
For the closing condition at $ζ=-1$, we have similarly
\begin{align*}
\int_{γ_-} Θ^P
&= 4 E(k) F(f(-1);k) - 4 K(k) E(f(-1);k) - 4K \frac{w(f(-1))\,(f(-1)-\iu\,\Im z_0)}{(f(-1)-z_0)(f(-1) + \bar{z}_0)}.\labelthis{eqn:gamma_minus}
\end{align*}

If one wishes, one could substitute these formulae into \eqref{eqn:def_T} to get $T_0$, though we hold off doing that until later. Previously, the comment was made that the particular algorithm to chose a path is not valid when $ν = \pm 1$. Indeed, the result of this can be seen directly in the formulae we have derived. When $ν$ takes either of these values, then one of $f(1)$ or $f(-1)$ will be inifinite. We also note that these integrals are purely imagiary, as we expected on theoretical grounds, because $f(1)$ and $f(-1)$ are purely imaginary, $F(z;k)$ and $E(z ;k)$ take the imaginary axis to itself and
\[
(f(1)-z_0)(f(1) + \bar{z}_0) = -(f(1)-z_0)(\overline{f(1) - z_0}) = - \abs{f(1) - z_0}^2.
\]
From the definition of $T_0$ in \eqref{eqn:def_T}, it is a real valued function.

To summarise our calculations to this point, we determined that every differential on a marked curve $Σ$ that satisfies condtions \ref{P:poles}--\ref{P:reality} must lie in a real three-dimensional vector space $W$. We found a basis $\{ω,Θ^E,λ\}$ of $W$, gave a basis $A,B$ for the homology of $Σ$ and computed the periods of the basis differentials with respect to this basis. This allowed us to find a differential $Θ^P\in \R\{ω,λ\} \subset W$ that satisfied \ref{P:periods}. We observed that every differential on $Σ$ that meet the conditions \ref{P:poles}--\ref{P:periods} was the sum of a real multiple of $Θ^E$ and an integer multiple of $Θ^P$.

With these two differentials we then attempted to further satisfy \ref{P:closing}. This was not possible for an arbitrary marked curve $Σ$, which lead us to define the functions $S(α,β)$ by \eqref{eqn:def_S} and $T(α,β)$ by \eqref{eqn:def_T}. Lemmata \ref{lem:exist spectral data} and \ref{lem:closing_conds} taken together imply that a marked curve admits spectral data exactly when the functions $S$ and $T$ simultaneously take rational values. The last part of this section noted that $T$ is a multivalued function, and so took a principal branch cut of it and derived the corresponding explicit formulae.

% Properly understood, $T$ is not a function on the space of spectral curves $\mathcal{A}_1$, but rather it is a function on the space of paths $(γ_+, γ_-)$, which we shall call $\mathcal{P}$. To be more precise, let $\mathcal{P}$ be the space $\{(γ_+, γ_-, Σ, α)\}$ where $α$ is one of the branch points inside the unit circle of some genus one spectral curve $Σ$ and $γ_+$ and $γ_-$ are paths $(0,1) \to Σ$ defined in the following way. $γ_+$ starts at $(1, η^-(1))$, monotonically tranverses the unit circle to $μ$ (that is, it does not pause or reverse direction), goes along the branch circle to $α$, encircles it, returns to $μ$ along the branch circle and retraces its path to $(1, η^+(1))$. In an analogous way, $γ_-$ starts at $(-1, η^-(-1))$ and finishes at $(-1, η^+(-1))$. We shall call such paths rigid. Every path on $Σ$ that connects the two points over $1$ or $-1$ is homologous to one of these rigid paths.
%
% $\mathcal{P}$ is naturally a $\Z^2$-bundle over the space $\mathcal{A}_1$. The projection map takes a tuple $(γ_+, γ_-, Σ, α)$ to the underlying spectral curve $(Σ,α)$. The fibre over any spectral curve $Σ$ is seen to be $\Z^2$, because if one fixes a point on the upper unit circle, not lying over $1,-1,μ$, the (signed) number of crossings of that point by a path uniquely determines that path. As an aside, the issue with points over $1,-1$ and $μ$ is that the paths may terminate there (in the obvious way for the first two, in the sense that the path may leave the unit circle in the latter case). One could probably introduce a convention to get around this, but it is not necessary to do so for our purposes.
%
% Viewed from this perspective, we consider $T$ to be a multivalued function on $\mathcal{A}_1$ defined locally up the addition of multiples of $p$ and $1$, and $T_0$ is a principal branch cut. We call $\tilde{T}$ the well defined version that is defined on $\mathcal{P}$. $\mathcal{P}$ is actually a cover of the universal cover of $\mathcal{A}_1$ (implying $\mathcal{P}$ is disconnected), a fact that will become apparent once we have adopted coordinates better adapted to the situation.









\subsection{Coordinates for $\mathcal{A}_1$}
\label{sub:Reformulate}

We would like to examine the subspace of $\mathcal{A}_1$ defined by the two conditions $S,T \in \Q$. This would be the set of genus one spectral curves. At its core, our method will invoke the implicit function theorem, for which the necessary computation will be the differentiation of the function $T$. It is therefore prudent to adopt a parameterisation of the space of marked curves $\mathcal{A}_1$ more suited to the task than $(α,β)$. We shall take as our first coordinate $p=S(α,β)$ itself, as then we can enforce the condition $p\in\Q$ simply by holding this coordinate constant. Elliptic integrals are the most difficult part of \eqref{eqn:gamma_plus} to differentiate, so to minimise our labour we choose the other three coordinates to be $k$, $\iu u = f(1)$ and $\iu v = f(-1)$. The content of the following lemma is that these four choices can essentially serve as coordinates for $\mathcal{A}_1$.

\begin{lem}
    \label{lem:change of parameters}
The following functions are diffeomorphisms:
\begin{align*}
φ_0 :\; &\mathcal{A}_1 \setminus \{ν = \pm 1\} \to \R^+ \times (0,1) \times \R \times \R \setminus \{u=v\}\\
&(α,β) \mapsto (p,k,u,v) = \bra{S(α,β), k(α,β), -\iu f(1), -\iu f(-1)}, \\
φ_1 :\; &\mathcal{A}_1 \setminus \{μ = 1 \text{ or } ν = -1\} \to \R^+ \times (0,1) \times \R \times \R \setminus \{u'v=1\}\\
&(α,β) \mapsto (p,k,u',v) = \bra{S(α,β), k(α,β), \iu f(1)^{-1}, -\iu f(-1)}, \\
φ_2 :\; &\mathcal{A}_1 \setminus \{μ = -1 \text{ or } ν = 1\} \to \R^+ \times (0,1) \times \R \times \R \setminus \{uv'=1\}\\
&(α,β) \mapsto (p,k,u,v') = \bra{S(α,β), k(α,β), -\iu f(1), \iu f(-1)^{-1}},
% φ_{--} :\; &\mathcal{A}_1 \setminus \{μ = -1, ν = -1\} \to \R^+ \times (0,1) \times \R \times \R\\
% &(α,β) \mapsto (p,k,u',v') = \bra{S(α,β), k(α,β), \iu f(1)^{-1}, \iu f(-1)^{-1}}, \\
\end{align*}
for the functions $S$ given by \eqref{eqn:def_S}, $k$ given by \eqref{eqn:def_k} and $f$ given by \eqref{eqn:f}. Also, the union of the domains covers $\mathcal{A}_1$.

\begin{proof}
First note that the exclusions from the codomains are correct. If, for example, $u=v$ then
\[
f(1) = \iu u = \iu v = f(-1),
\]
but $f$ is an invertible transformation. Likewise for the codomains of other two functions.

Next, $S(α,β)$ is smooth since $α$ and $β$ are inside the unit disc. The other functions are smooth by Lemma \ref{lem:coeff_f_smooth}. Hence the function $φ_0$ is smooth. The other two functions are necessary because the map $f$ is a M\"obius transformation, and so takes the value infinity. Indeed, we have seen that $f(ν) = \infty$, and so one of $u$ or $v$ is infinite when $ν=\pm 1$. Using \eqref{eqn:f}, on the subset of $\mathcal{A}_1$ where $μ\neq 1$
\[
-\iu u' = -\frac{1}{\bar{z}_0} \frac{1-ν}{1-μ},
\]
and where $μ\neq -1$
\[
-\iu v' = -\frac{1}{\bar{z}_0} \frac{1+ν}{1+μ},
\]
demonstrating that $φ_1$ and $φ_2$ are smooth functions on their respective domains of definition.

It remains to show that these functions have smooth inverses. As the parameters $α$ and $β$ are points in the $ζ$-plane, one method to derive the inverse functions is to express the transformation $f^{-1}(z)$ in term of our new parameters $(p,k,u,v)$. Then $α = f^{-1}(1)$ and $β = f^{-1}(k^{-1})$, entirely analogous to how the transformation $f(ζ)$ is determined by $(α,β)$ and $\iu u = f(1)$. As a M\"obius transformation is described, up to a scalar, by the points sent to $0$ and $\infty$, $f^{-1}(z)$ is a scalar multiple of
\[
\frac{z-z_0}{z + \conj{z_0}},
\]
(compare to \eqref{eqn:f_inv}). Thus the construction of $f^{-1}$ proceeds in two steps; first find $z_0$, then determine the correct scaling factor. Observe the following trick using cross ratios:
\[
\abs{\frac{α-1}{α+1}}
= \abs{\frac{α-1}{α+1}} \abs{\frac{0+1}{0-1}}
= \abs{ \cross{α}{0}{1}{-1} }
= \abs{ \cross{1}{z_0}{\iu u}{\iu v} }
= \abs{\frac{1-\iu u}{1 - \iu v}} \abs{\frac{z_0 - \iu v}{z_0 - \iu u}}
\]
The same trick gives a similar formula for $β$.
\[
\abs{\frac{β-1}{β+1}}
= \abs{\frac{1-k\iu u}{1 - k\iu v}} \abs{\frac{z_0 - \iu v}{z_0 - \iu u}}
\]
The terms on the left are the two factors of the expression for $S$. We will show that
$z_0$ lies on a particular circle determined by the parameters $(p,k,u,v)$. Let $z_0 = x+\iu y$. Then
\begin{align*}
p
&= \abs{\frac{α-1}{α+1}} \abs{\frac{β-1}{β+1}}
= \abs{\frac{1-\iu u}{1 - \iu v}}\abs{\frac{1 - k\iu u}{1 - k\iu v}} \abs{\frac{z_0 - \iu v}{z_0 - \iu u}}^2 \\
\abs{\frac{z_0 - \iu v}{z_0 - \iu u}} ^2
&= p \frac{\sqrt{1+v^2}}{\sqrt{1+u^2}}\frac{\sqrt{1+k^2v^2}}{\sqrt{1+k^2u^2}}
= p \frac{w(\iu v)}{w(\iu u)} \\
\abs{z_0 - \iu v} ^2 &= p \frac{w(\iu v)}{w(\iu u)} \abs{z_0 - \iu u}^2 \\
x^2 + y^2 &+ 2y \frac{puw(\iu v) - vw(\iu u)}{w(\iu u)-pw(\iu v)} + \frac{v^2w(\iu u) - pu^2w(\iu v)}{w(\iu u)-pw(\iu v)} = 0,
\end{align*}
where recall that $w$ is defined by the relation $w^2 = (1-z^2)(1-k^2z^2)$. The function $w(\iu t) = \sqrt{(1+t^2)(1+k^2t^2)}$ should be taken with a positive square root on the second line, since the left hand side is positive.

On the other hand, in the $ζ$-plane the points $-1,0,1$ all lie on a straight line that is perpendicular to the unit circle at both $-1$ and $1$, and that is invariant under the real involution $ρ$. Applying the M\"obius transformation $f$ we can therefore say that $\iu v, z_0$ and $\iu u$ all lie on a circle that is perpendicular to the imaginary axis and symmetric under reflection in the imaginary axis ($\tilde{ρ}(z) = -\bar{z}$). Therefore $z_0$ lies on the circle
\[
x^2 + \bra{ y - \frac{u+v}{2} }^2 = \frac{(u-v)^2}{4},
\]
which simplifies to the relation
\[
x^2 + y^2 = y(u+v) - uv. \labelthis{eqn:z_0_circle}
\]
Thus we have determined two circles that $z_0$ lies on. These two circles intersect in two points: $z_0$ and $-\conj{z_0}$. The precise formulas are
\[
x = \frac{\sqrt{pw(\iu u)w(\iu v)}}{pw(\iu v) + w(\iu u)} \abs{u-v},\; y = \frac{puw(\iu v) + vw(\iu u)}{pw(\iu v) + w(\iu u)}
\labelthis{eqn:formula xy}
\]
where the sign of $x$ is chosen to make $z_0$ lie in the right half of the $z$-plane. This choice amounts to choosing the branch points $α,β$ inside the unit circle. Note that these are smooth functions of $(p,k,u,v)$, because the term under the square root and the denominators are strictly positive functions.

Having found $z_0$ in terms of $(p,k,u,v)$ it remains to find the correct scaling of $f^{-1}$. We use the fact that $f^{-1}(\iu u) = 1$ and $f^{-1}(\iu v) = -1$ to conclude
\[
f^{-1}(z)
=  \frac{\iu u + \conj{z_0}}{\iu u - z_0} \frac{z-z_0}{z + \conj{z_0}}
=  -\frac{\iu v + \conj{z_0}}{\iu v - z_0} \frac{z-z_0}{z + \conj{z_0}}.
\]
As was previously presented, one can simply take $α = f^{-1}(1)$ and $β = f^{-1}(k^{-1})$ to give formula for the branch points in terms of the new parameters. A problem could potentially occur if $z_0$ were to equal $\iu u$ or $\iu v$, in which case the scaling factor would be $0/0$. This could only occur if $\Re{z_0}=0$, which itself only occurs if $u=v$. But we have already noted that this is impossible. Likewise the formula would be in trouble if $z_0 = -1, -k^{-1}$ (for then $α$ or $β$ would be infinite), but again this occurs only if $\Re{z_0} < 0$, which is excluded by our decision to take $z_0$ in the right half-plane.

Having constructed an inverse for $φ_0$, we also have to construct ones for $φ_1$ and $φ_2$. But we can easily do so by modifying formula \eqref{eqn:formula xy} for $x$ and $y$. Using the notation $w'(\iu t)^2 = (1+t^2)(k^2 + t^2)$ we have
\begin{align*}
x
&= \frac{\sqrt{pw(\iu u)w(\iu v)}}{pw(\iu v) + w(\iu u)} \abs{u-v}
= \frac{\sqrt{u^2 \times pw'(\iu u')w(\iu v)}}{pw(\iu v) + u^2 w'(\iu u')} \abs{u}\abs{1-u'v} \\
&= \frac{\sqrt{pw'(\iu u')w(\iu v)}}{pu'^2w(\iu v) + w'(\iu u')} \abs{1-u'v}.
\end{align*}
Likewise
\[
x
= \frac{\sqrt{pw(\iu u)w(\iu v')}}{pw'(\iu v') + v'^2w'(\iu u')} \abs{uv'-1},
\]
and
\[
y
= \frac{pu'w(\iu v) + vw'(\iu u')}{pu'^2w(\iu v) + w'(\iu u')}
= \frac{puw'(\iu v') + v'w(\iu u)}{pw'(\iu v') + v'^2w(\iu u)}. \labelthis{eqn:y_prime_version}
\]
These both give smooth formulae for $z_0$. Having made the change in formula for $z_0$, the same formula for $f^{-1}$ applies.

It is interesting to see that it was necessary to exclude the plane where $u=v$ (or $u'v=1$ or $uv'=1$), for otherwise $x$ would be zero, $z_0$ would be equal to $-\bar{z_0}$ and $f^{-1}(z)$ would be a constant function. We shall see later that these points correspond to the diagonal $\{α=β\} \subset D\times D$ and represent a degeneration of marked curves.

Finally, it is simple to see that the three domains cover $\mathcal{A}_1$. If there was some point $(α,β) \in \mathcal{A}_1$ that was not covered, then one could compute $μ(α,β)$ and $ν(α,β)$. But the intersection
\[
\{ν = \pm 1\}
\cap \{μ = 1 \text{ or } ν = -1 \}
\cap \{μ = -1 \text{ or } ν = 1 \} \subset \mathcal{A}_1
\]
consists of only those points where $μ=ν=1$ or $μ=ν=-1$, and Lemma \ref{lem:coeff_f_smooth} proves that $μ$ and $ν$ are never equal.
\end{proof}
\end{lem}

Though standard, it is perhaps still of some interest to consider the above geometrical argument in the limit $u\to\infty$ to assure ourselves that nothing singular is happening. Suppose that $u' = 0$, which is to say geometrically that $1$ is mapped to infinity by $f$. Then the transformation $f$ takes the line through $1,0,-1$ to a line perpendicular to the imaginary axis, cutting at $f(-1)$. This line is therefore horizontal and so $z_0$ and $\iu v$ have the same imaginary parts. This gives $y=v$ directly, as can be observed by setting $u'=0$ in eqn \eqref{eqn:y_prime_version}.

Recall that we defined a principal branch cut $T_0$ of $T$ on $\mathcal{A}\setminus\{ν=\pm 1\}$. This is exactly the domain of $φ_0$, and we have constructed a formula for the inverse transformation, so it is time to rewrite $T_0$ in terms of $(p,k,u,v)$. Particularly, we must compute the factors in the last terms of equations \eqref{eqn:gamma_plus} and \eqref{gamma_minus}. By direct computation
\begin{align*}
u-y &= \frac{w(\iu u)(u-v)}{pw(\iu v) + w(\iu u)} &
\abs{\iu u - z_0}^2 &= \frac{w(\iu u)(u-v)^2}{pw(\iu v) + w(\iu u)} \\
v-y &= - \frac{w(\iu v)(u-v)}{pw(\iu v) + w(\iu u)} &
\abs{\iu v - z_0}^2 &= \frac{w(\iu v)(u-v)^2}{pw(\iu v) + w(\iu u)}
\end{align*}
These expressions combine to give from \eqref{eqn:gamma_plus} and \eqref{eqn:gamma_minus}
\begin{align*}
- 4K \frac{w(f(1))\,(f(1)-\iu\,\Im z_0)}{(f(1)-z_0)(f(1) + \bar{z}_0)}
&= 4\iu K \frac{w(\iu u)\,(u-y)}{\abs{\iu u-z_0}^2}
= 4\iu K \frac{w(\iu u)}{u-v}, \\
\int_{γ_+} Θ^P
= 4 E(k) F(\iu u;k) &- 4 K(k) E(\iu u;k) + 4\iu K \frac{w(\iu u)}{u-v}.
\labelthis{eqn:gamma_plus2}\\
%%%%%%%%%%%%%%%%%%%%%%%%%%
- 4K \frac{w(f(-1))\,(f(-1)-\iu\,\Im z_0)}{(f(-1)-z_0)(f(-1) + \bar{z}_0)}
&= 4\iu K \frac{w(\iu v)\,(v-y)}{\abs{\iu v-z_0}^2}
= - 4\iu K \frac{w(\iu v)}{u-v},\\
\int_{γ_-} Θ^P
= 4 E(k) F(\iu v;k) &- 4 K(k) E(\iu v;k) - 4\iu K \frac{w(\iu v)}{u-v}.\labelthis{eqn:gamma_minus2}
\end{align*}
And hence that
\begin{align*}
2π\iu T_0(p,k,u,v)
&= p\left\{ 4 E F(\iu v;k) - 4K E(\iu v;k) - 4\iu K \frac{w(\iu v)}{u-v} \right\}
- \left\{ 4 E F(\iu u;k) - 4K E(\iu u;k) + 4\iu K \frac{w(\iu u)}{u-v} \right\} \\
&= 4p \left[ E F(\iu v;k) - K E(\iu v;k) \right] - 4\left[ E F(\iu u;k) - K E(\iu u;k) \right] - 4\iu K \frac{p w(\iu v) + w(\iu u)}{u-v}.
\labelthis{eqn:Teqn}
\end{align*}

Consider the parameter space $\mathcal{A}_1$ for a fixed value of $p$, denoted $\mathcal{A}_1(p)$. Topologically it is the product of an interval and an annulus, a feature not easily seen from the $α,β$ description. The interval component is obvious, it comes from $k\in (0,1)$. To see the annulus part, consider $\RInf \times \RInf$. Topologically $\RInf$ is just a circle, so this product is a torus. The line $u=v$ can be represented as the line where the toroidal and poloidal angles are equal, and removing this line leaves an annulus.

A more instructive way of visualising $\mathcal{A}_1$ is to think of it as a solid cylinder with a line along the central axis removed. One should think of the `radius' of point being given by $1-k$, so that the central axis is identified with the value $k=1$. To motivate this, consider formula \eqref{eqn:def_k} for $k$.
\[
k = \frac{\abs{1-\bar{α}β}-\abs{α-β}}{\abs{1-\bar{α}β}+\abs{α-β}}.
\]
In the limit as $α \to β$, this formula says that $k \to 1$. From the equation of $p$, for a fixed value of $p$, the subspace $α=β$ is an arc. In this visualisation we are imagining this arc as the central axis of the cylinder. In later chapters, the interesting structure of the moduli space in this limit will be investigated.

% Another way that this model is useful is it allows us to see how the space of spectral curves is not simply connected and the essential appearence of the multivalued behaviour of $T$. Take a small loop around the central axis, which is to say a certain path in the space of spectral curves with $k$ close to $1$. This implies that $α$ and $β$ must be close together, and so the branch circle is approximately a line. As we move around the axis, the points $μ,ν$ in the $ζ$-plane are moving around the unit circle. Every time $ν$ crosses either $1$ or $-1$, we have moved across a branch cut. After one full rotation, $α$ and $β$ have returned to their original positions, but $T$ has been incremented by $1+p$.
%
% Viewed in
%
% PERIOD BEHAVIOR HASN"T BEEN INTRODUCED YET \todo{reorder this}

The fact that the parameter space is not simply connected and that $T$ is not even a single valued function could obstruct our use of the implicit function theorem. The obvious way to correct this is to move the the unversal cover.

\begin{defn}
    \label{defn:mathcal tilde A}
The universal cover of $\mathcal{A}_1$ is
\[
\mathcal{\tilde{A}}_1 =
\{(p, k,\tilde{u},\tilde{v}) \in \R^+\times(0,1)\times\R\times\R \mid  \tilde{u} < \tilde{v} < \tilde{u} + 2π \},
\]
with the projection map $\mathcal{\tilde{A}}_1 \to \mathcal{A}_1$ is given by
\begin{align*}
    p &= p, \\
    k &= k, \\
    u = \tan \frac{\tilde{u}}{2},       &\quad
        u' = \cot \frac{\tilde{u}}{2},  \\
    v = \tan \frac{\tilde{v}}{2},       &\quad
        v' = \cot \frac{\tilde{v}}{2}.
\end{align*}
\end{defn}

The justification of this definition of $\mathcal{\tilde{A}}_1$ precedes in two steps. First, the universal cover of $\R_{>0} \times (0,1) \times \RInf\times\RInf$ is $\R_{>0} \times (0,1) \times \R\times\R$, with the two circle components being covered by the line via a standard half-tan mapping as witnessed in the formulae above. The second step is to recall that $\mathcal{A}_1$ is the complement of the line $u-v = 0$. When pulled back to the universal cover, this becomes a collections of lines $u-v \in 2π\Z$. Only one stripe is needed to cover $\mathcal{A}_1$, and this is $\mathcal{\tilde{A}}_1$ above.

It is straightforward to lift $T_0$ to a single valued function $\tilde{T}$ on $\mathcal{\tilde{A}}_1$. Recall the defintions of $F_0$ and $E_0$ from Appendix \ref{sec:Elliptic Integrals}.
\[
F_0(x ;k) = \Im F(\iu x; k), \qquad
E_0(x ;k) = \Im F(\iu x; k) - kx.
\]
Using these, we rewrite $T_0$ in the follwoing way.
\[
2π T_0(p,k,u,v) =
4p \left[ E F_0(v) - K E_0(v) \right]
-4 \left[ E F_0(u) - K E_0(u) \right]
- 4 K \left[p\left\{\frac{w(\iu v)}{u-v} + kv \right\} + \left\{\frac{w(\iu u)}{u-v} - ku \right\} \right] .
\]

\begin{lem}
The function
\[
\frac{w(\iu u)}{u-v} - ku
\]
is an analytic function on $\mathcal{A}_1$.

\begin{proof}
As $u-v \neq 0$ on $\mathcal{A}_1$, this is an anlytic function of these coordinates. It remains to show that it is simiarly analytic in the other coordinates required to cover $\mathcal{A}_1$. Firstly, examining this function when using the coordinate $v'$ gives
\[
\frac{w(\iu u)}{u-v} - ku = \frac{w(\iu u)v'}{uv'-1} - ku,
\]
which is analytic. Next, when using the coordinate $u'$ we have
\begin{align}
\frac{w(\iu u)}{u-v} - k u
&= \frac{w(\iu u) - ku^2}{u-v} + \frac{kuv}{u-v} \\
&= \frac{1 + (1+k^2)u^2}{(u-v) (w(\iu u) + ku^2)} + \frac{kv}{1-u'v} \\
&= \frac{u'((u')^2 + (1+k^2)} {(1-u'v) (w'(\iu u') + k)} + \frac{kv}{1-u'v},
\end{align}
where we have introduced the shorthand $w'(z')^2 = (1-(z')^2)(k^2 - (z')^2)$ to deal with this function at infinity. This is now an analytic function of $u'$. As these three coordinate patches cover all of $\mathcal{A}_1$, we are done and there is no need to consider the case using both $u'$ and $v'$, though this case is easy to derive from the last expression, and also clearly analytic.
\end{proof}
\end{lem}

In Appendix \ref{sub:EllipticContinuation}, extensions of $F_0(x;k)$ and $E_0(x;k)$ to the universal covers of $(k,x)\in (0,1)\times(\RInf)$ are constructed. They are denoted respectively as $\tilde{F}$ and $\tilde{E}$. Thus the first two brackets of $T_0$ can be lifted to the universal cover by simply swapping to those ready-made functions. The previous lemma has established that the third bracket is analytic, and so lifts to the universal cover without the need for modifcation at all. Therefore we define
\[
2π \tilde{T}(p,k,\tilde{u},\tilde{v})
= 4p \left[ E \tilde{F}(\tilde{v}) - K \tilde{E}(\tilde{v}) \right]
- 4 \left[ E \tilde{F}(\tilde{u}) - K \tilde{E}(\tilde{u}) \right]
- 4 K \left[p\left\{\frac{w(\iu v)}{u-v} + kv \right\}
+ \left\{\frac{w(\iu u)}{u-v} - ku \right\} \right] .
\]
If we wish to relate this to a specific local expression, it is simply a matter of substituting the correct expression for $\tilde{F}$ or $\tilde{E}$, \'a la \eqref{eqn:tildeF_period} and \eqref{eqn:tildeE_period}. In particular, define the winding number $W : \R \to \Z$ of a number $x$ to be the integer $W(x)$ such that $-π < x - 2πW(x) < π$. Then
\begin{align*}
2π \tilde{T}(p,k,\tilde{u},\tilde{v})
%%%%%%%%%%%%%%%%%%%%%%%%
&= 4p \left[ E (2K'W(\tilde{v})+F_0(v)) - K (2(K'-E')W(\tilde{v})+E_0(v)) \right] \\
&\quad - 4 \left[ E (2K'W(\tilde{u})+F_0(u)) - K (2(K'-E')W(\tilde{u})+E_0(u)) \right]
- 4 K \left[p\left\{\frac{w(\iu v)}{u-v} + kv \right\}
+ \left\{\frac{w(\iu u)}{u-v} - ku \right\} \right] \\
%%%%%%%%%%%%%%%%%%%%%%%%
&= 4p \left[ 2EK' - 2K(K'-E') \right]W(\tilde{v})
- 4 \left[ 2EK' - 2K(K'-E') \right]W(\tilde{u}) + T_0(p,k,u,v)\\
%%%%%%%%%%%%%%%%%%%%%%%%
\tilde{T}(p,k,\tilde{u},\tilde{v})
&= 2\bra{ pW(\tilde{v}) - W(\tilde{u}) } + T_0(p,k,u,v) .
\label{eqn:tilde T computable}
\end{align*}
Thus to do computations with the function $\tilde{T}$, one can continue to work with the function $T_0$ downstairs on $\mathcal{A}_1$ and make a note of the constants. In particular, $T \in \Q$ if and only if $\tilde{T} \in \Q$, and so we can seek to find moduli space as a subset of the universal cover.






















\subsection{The topology of the moduli space}
\label{sub:Topology}

In the interest of having manageable formulae, we recall the definition of $w'(z')^2 = (1 - (z')^2)(k^2 - (z')^2)$. As $F(z;k)$ and $E(z;k)$ are parameter integrals in $z$, we have that
\[
\Partial{}{u} F(\iu u; k) = \frac{\iu}{w(\iu u)},\;\;\;
\Partial{}{u} E(\iu u; k) = \iu\frac{1+k^2 u^2}{w(\iu u)},
\]
and
\[
\Partial{}{u} w(\iu u)
= \Partial{}{u} \sqrt{1+u^2}\sqrt{1+k^2 u^2}
= \frac{(1+k^2)u + 2k^2 u^3}{w(\iu u)}.
\]
The other derivatives of elliptic integrals are calculated in appendix \ref{sec:Elliptic Integrals}. Equipped with these tools, the calculation of the derivatives of $T_0$ is mechanical if tedious.
% \begin{align*}\label{dTdk}
% \frac{π}{2}\Partial{T_0}{k}
% &= \frac{1}{k(1-k^2)}\frac{1}{u-v} \left[ p \sqrt { \frac{1+v^2}{1+k^2v^2}} + \sqrt { \frac{1+u^2}{1+k^2u^2} } \right] \left[ -(1+k^2uv) E + (1-k^2)K \right]
% \end{align*}
\begin{equation}\label{dTdu}
\frac{π}{2}\Partial{T_0}{u}
= -\frac{E}{w(\iu u)} + \frac{pK w(\iu v)}{(u-v)^2} + \frac{K}{w(\iu u)(u-v)^2}\left[1 + u^2 - uv + v^2 + k^2 uv + k^2 u^2v^2 \right]
\end{equation}
% \begin{equation}\label{dTdv}
% \frac{π}{2}\Partial{T_0}{v}
% = \frac{pE}{w(\iu v)} - \frac{K w(\iu u)}{(u-v)^2} - \frac{pK}{w(\iu v)(u-v)^2}\left[1 + u^2 - uv + v^2 + k^2 uv + k^2 u^2v^2 \right]
% \end{equation}
While $T$ may be multivalued on $\mathcal{A}_1'$, its derivatives are not: the ambiguity of being locally defined only up to a constant is removed by differentiation. Henceforth we can drop the subscript $0$ on derivatives.

% \begin{equation}\label{dTdu'}
% \frac{π}{2}\Partial{T}{u'}
% = \frac{E}{w'(u')} + \frac{pK w(\iu v)}{(1-u'v)^2} + \frac{K}{w'(u')(1-u'v)^2}\left[1 + k^2v^2 + - u'v + k^2u'v + (u')^2 + (u')^2v^2 \right]
% \end{equation}
% \begin{equation}\label{dTdv'}
% \frac{π}{2}\Partial{T}{v'}
% = -\frac{pE}{w'(v')} + \frac{K w(\iu u)}{(uv'-1)^2} + \frac{pK}{w'(v')(uv'-1)^2}\left[1 + k^2u^2 - uv' + k^2uv' + (v')^2 + u^2(v')^2 \right]
% \end{equation}

\begin{lem}
    \label{lem:deriv no zeroes}
The functions
\[
U(p,k,u,v) := -(u-v)^2 E + pKw(\iu u)w(\iu v) + K\left[ 1 + u^2 - uv + k^2 uv + v^2 + k^2 u^2 v^2 \right]
\]
and
\[
V(p,k,v) := -E + pkKw(\iu v) + K\left[ 1 + k^2v^2 \right]
\]
have no zeroes for $p \geq 1$, $k\in (0,1)$, $u,v \in \R$.
\begin{proof}
We shall prove this statement by showing that the two functions are in fact always positive. The first step is to eliminate $E$. We apply the crude estimate that $K>E$, \todo{reference the appendix}
and also the assumption that $p\geq 1$, to simplify
\begin{align*}
U(p,k,u,v)
&= -(u-v)^2 E + pKw(\iu u)w(\iu v) + K\left[ 1 + u^2 - uv + k^2 uv + v^2 + k^2 u^2 v^2 \right] \\
&> -(u-v)^2 K + Kw(\iu u)w(\iu v) + K\left[ 1 + u^2 - uv + k^2 uv + v^2 + k^2 u^2 v^2 \right] \\
&= K \left[ w(\iu u)w(\iu v) + 1 + (1 + k^2) uv + k^2 u^2 v^2 \right]
\end{align*}
This formula is almost sufficent. The only term that could be negative is the one featuring $uv$. However, a lower bound for the square root terms is
\[
w(\iu u) = \sqrt{1 + (1+k^2)u^2 + k^2u^4} > \sqrt{(1+k^2)u^2} = \sqrt{(1+k^2)}\abs{u},
\]
so
\begin{align*}
U(p,k,u,v)
&> K \left[ (1+k^2)\abs{uv} + 1 + (1 + k^2) uv + k^2 u^2 v^2 \right] \\
&\geq K \left[ 1 + k^2 u^2 v^2 \right].
\end{align*}
This is positive, so $U$ is without zeroes. The second function, $V$, is almost immediate. Take $E<K$,
\[
V(p,k,v) > K \left[ -1 + pk w(\iu v) + 1 + k^2 v^2\right].
\]
This establishes that $V$ has no roots either.
\end{proof}
\end{lem}

% This result is of interest because it shows that if $p \leq 1$, then the $v$ and $v'$ derivatives of $T$ are nonzero:
% \begin{align*}
% \frac{π}{2}\Partial{T}{v} &= \frac{-p}{w(\iu v)(u-v)^2} U\bra{ \frac{1}{p},k,u,v }, \\
% \frac{π}{2}\Partial{T}{v'} &= \frac{p}{w'(v')(uv'-1)^2} V\bra{ \frac{1}{p},k,u,v' }.
% \end{align*}
% And if $p \geq 1$, then the $u$ and $u'$ derivatives of $T$ are nonzero:
% \begin{align*}
% \frac{π}{2}\Partial{T}{u} &= \frac{1}{w(\iu u)(u-v)^2} U\bra{ p,k,u,v }, \\
% \frac{π}{2}\Partial{T}{u'} &= \frac{-1}{w'(u')(1-u'v)^2} V\bra{ p,k,v,u' }.
% \end{align*}

\begin{lem}
\label{lem:range_T}
The range of $\tilde{T}$ on $\mathcal{A}_1(p)$ is $\R$.

\begin{proof}
Fix $p$, but also fix any value for $k$. By REF, $\abs{F_0(x;k)}$ is bounded by $K'$ and $\abs{E_0(x;k)}$ is bounded by $K'-E'$. Thus there is some constant, dependent on both $p$ and $k$, such that
\[
-C \leq 4p \left[ E F_0(v) - K E_0(v) \right]-4 \left[ E F_0(u) - K E_0(u) \right] \leq C.
\]
It follows from REF that
\[
- 4 K \left[\frac{pw(\iu v) + w(\iu u)}{u-v} + k(pv-u) \right] - C
\leq
2π T_0(p,k,u,v)
\leq
- 4 K \left[\frac{pw(\iu v) + w(\iu u)}{u-v} + k(pv-u) \right] + C,
\]
and so it is sufficent to show that the bracketed expression has range equal to the real line. But this is easy to show. Consider the limit as $u \to v^+$,
\[
\lim_{u \to v^+} \frac{pw(\iu v) + w(\iu u)}{u-v} + k(pv-u)
= k(p-1)v + (p+1) w(\iu v)\lim_{u \to v^+} \frac{1}{u-v} = +\infty.
\]
From the other side,
\[
\lim_{u \to v^-} \frac{pw(\iu v) + w(\iu u)}{u-v} + k(pv-u)
= k(p-1)v + (p+1) w(\iu v)\lim_{u \to v^-} \frac{1}{u-v} = -\infty.
\]
By continuity, $T_0$ and therefore $\tilde{T}$ obtains every value.
\end{proof}
\end{lem}



\begin{lem}
\label{lem:T_graph}
If $p \leq 1$ then $\tilde{\mathcal{A}_1}(p)$ is diffeomorphic to $\{(q,k,\tilde{u}) \in \R\times(0,1)\times\R \}$, such that for fixed $q_0$, the level set $\tilde{T} = q_0$ is diffeomorphic to $\{(q_0,k,\tilde{u}) \}$.

Likewise, if $p \geq 1$ then $\tilde{\mathcal{A}_1}(p)$ is diffeomorphic to $\{(q,k,\tilde{v}) \in \R\times(0,1)\times\R \}$, such that for fixed $q_0$, the level set $\tilde{T} = q_0$ is diffeomorphic to $\{(q_0,k,\tilde{v})$.

\begin{proof}
Fix a value of $p$ and consider the function $F(q, k,\tilde{u},\tilde{v}) = \tilde{T}(p,k,\tilde{u},\tilde{v}) - q$ on $\mathcal{\tilde{A}}_1(p)\times\R$. $F^{-1}(0)$ is a graph over $\mathcal{\tilde{A}}_1(p)$ given by $q=\tilde{T}$, so they are diffeomorphic. We will apply the implicit function theorem to show that $F^{-1}(0)$ can also be written as a graph over either $(q,k,\tilde{u})$ or $(q,k,\tilde{v})$, depening on the magnitude of $p$.

Suppose first that the fixed value of $p$ is greater than or equal to one. We compute the following formula for the derivative of $F$ with respect to $\tilde{u}$.
\begin{align*}
\Partial{F}{\tilde{u}}
= \Partial{\tilde{T}}{\tilde{u}}
&= \frac{du}{d\tilde{u}}\Partial{T}{u} \\
&= \frac{1}{2}\sec^2\bfrac{\tilde{u}}{2} \Partial{T}{u} \\
&= \frac{1}{2}\sec^2\bfrac{\tilde{u}}{2} \times \frac{2}{π} \times \frac{1}{w(\iu u)(u-v)^2} U\bra{ p,k,u,v } \\
&= \frac{1}{π}\frac{1 + u^2}{w(\iu u)(u-v)^2} U\bra{ p,k,u,v },
\end{align*}
which holds for $\tilde{u} \not\in π + 2π\Z$.As witnessed in Lemma \ref{lem:deriv no zeroes}, $U$ is never zero, and neither are the other three factors present. Hence $\partial \tilde{T} / \partial \tilde{u}$ is never zero on this open set. It remains to check it does not vanish when $\tilde{u} \in π + 2π\Z$. Observe
\begin{align*}
\lim_{u\to\infty} u^{-2} U(p,k,u,v)
&= \lim_{u\to\infty} -(1-u^{-1}v)^2 E + pKw(\iu v) \cdot u^{-2}w(\iu u) + K\left[ u^{-2} + 1 - u^{-1}v + k^2 u^{-1}v + u^{-2}v^2 + k^2 v^2 \right] \\
&= V(p,k,v).
\end{align*}
As $\tilde{T}$ is an analytic function, in particular its derivatives are continuous and so we may compute their value at these points by taking a limit. Therefore
\begin{align}
\lim_{\tilde{u}\to π + 2π\Z} \Partial{F}{\tilde{u}}
&=\frac{1}{π} \lim_{u \to \infty} \frac{1 + u^2}{w(\iu u)(u-v)^2} U\bra{ p,k,u,v } \\
&=\frac{1}{π} \lim_{u \to \infty} \frac{(1 + u^2)u^2}{w(\iu u)(u-v)^2} \times u^{-2}U\bra{ p,k,u,v } \\
&=\frac{1}{π} \frac{1}{k} V\bra{ p,k,v },
\end{align}
which is also nonzero. By the implicit function theorem, there is a function $h$ such that $F^{-1}(0)$ is a graph of the form $\{ (q, k, \tilde{u}, h(q,k,\tilde{u})) \mid q \in \text{Range} \tilde{T}, k \in (0,1), \tilde{u}\in\R \}$. By Lemma \ref{lem:range_T}, the range of $\tilde{T}$ is $\R$. Finally, if we hold $q$ fixed at $q_0$, then the level set $q_0 = \tilde{T}$ is paramterised by the remaining two coordinates $(k,\tilde{u})$.

When $p \leq 1$, we employ the following symmetry.
\begin{align*}
T_0(p,k,u,v)
&= 4p \left[ E F(\iu v;k) - K E(\iu v;k) \right] - 4\left[ E F(\iu u;k) - K E(\iu u;k) \right] - 4\iu K \frac{p w(\iu v) + w(\iu u)}{u-v} \\
&= -p\left\{ -4\left[ E F(\iu v;k) - K E(\iu v;k) \right] + \frac{4}{p}\left[ E F(\iu u;k) - K E(\iu u;k) \right] + 4\iu K \frac{ w(\iu v) + \frac{1}{p}w(\iu u)}{u-v} \right\}\\
&= -p T_0\bra{ \tfrac{1}{p}, k, v, u },
\end{align*}
so that it is now the $\tilde{v}$ derivative of $\tilde{T}$ that is nonvanishing. Again the implicit function theorem gives the result.
\end{proof}
\end{lem}

Recall that $\mathcal{S}_1 \subset \mathcal{C}_1$ is the set of spectral curves that admit closing spectral data and that $\mathcal{\tilde{S}}_1 \subset \mathcal{\mathcal{A}_1}$ is its preimage in the universal cover. Let $\mathcal{\tilde{A}}_1(p,q)$ denote the subset of $\mathcal{\tilde{A}}_1$ on which the values of $p(α,β) = p$ and $\tilde{T} = q$ are fixed. As we have shown that $\mathcal{S}_1$ is the space on which $p$ is a positive rational and $q$ is any rational (lemma \ref{lem:closing_conds}), it follows that
\[
\mathcal{\tilde{S}}_1 = \coprod_{p\in \Q^+,\; q\in \Q} \mathcal{\tilde{A}}_1(p,q).
\]

There are several topological implications of the previous lemma. First, each level set $\mathcal{\tilde{A}}_1(p,q)$ is connected. More precisely each of these is diffeomorphic to a product of $(0,1)$ and $\R$. The above disjoint union is the decomposition of $\mathcal{\tilde{S}}_1$ into its connected components. By varying the value of $q$, we see that the $\mathcal{\tilde{A}}_1(p,q)$ foliate $\mathcal{A}_1(p)$, so $\mathcal{\tilde{S}}_1$ is arranged densely in $\mathcal{\tilde{A}}_1(p)$ in the same way that $\Q$ is arranged densely in $\R$.
\todo{phrase this better}

To understand the topology of $\mathcal{S}_1$, we must understand the action of the deck transformations as restricted to $\tilde{\mathcal{S}_1}$. There are two types of transformations to understand, the transformations of $\mathcal{\tilde{A}}_1$ over $\mathcal{A}_1$ and the transformations of the two fold covering of $\mathcal{A}_1$ over $\mathcal{C}_1$. The first sort are easy to understand; they are generated by the translation
\[
(p, k,\tilde{u},\tilde{v}) \mapsto (p, k, \tilde{u} + 2π, \tilde{v} + 2π).
\]
The second sort require a little more effort. On one hand, the deck transformations of $\mathcal{A}_1$ over $\mathcal{C}_1$ is simply the involution that swaps $α$ and $β$. The difficulty is what this action looks like when pulled back to the universal cover. By inspection of the defintions of $p$ and $k$, they are unchanged if the labbeling of the branch points is exchanged. We therefore must say what happens to $u$ and $v$. Our construction thus far relies on the standardisation in the definition of $f$ that sends $α$ to $1$ and $β$ to $k^{-1}$. Let $f_s$ be the map which instead standardises the roots of the spectral curve in the other way, taking $β$ to $1$. Consider the composition of $f_s \circ f^{-1}$. It is a M\"obius transformation that exchanges $1$ and $k^{-1}$ and also $-1$ and $-k^{-1}$. It therefore must be the map
\[
z \mapsto \frac{1}{kz}.
\]
Under the composite map $\iu u$ is taken to $-\iu k^{-1}u^{-1}$ and $\iu v$ is taken to $-\iu k^{-1}v^{-1}$. Using the map $f_s$ instead of $f$ is equivalent to swapping the labels on the branch points and then applying $f$. Thus, under the label-swapping involution,
\[
(p,k,u,v) \mapsto \bra{ p,k, -(ku)^{-1}, -(kv)^{-1} }.
\]
This then carries on to the universal cover in the following way. For some integer $n$
\[
\tilde{u} = 2πn + 2\atan u \mapsto 2πn + 2\atan \bra{ -\frac{1}{ku} } = 2πn + π + 2\atan (ku).
\]
Applying the transformation again
\[
2πn + π + 2\atan (ku) \mapsto 2πn + π + 2\atan \bra{ -\frac{k}{ku} } = 2π(n+1) + 2\atan u = \tilde{u} + 2π.
\]
The same applies to $\tilde{v}$. This shows that deck transformations of $\mathcal{\tilde{A}}_1$ over $\mathcal{A}_1$ are actually a subgroup of the group generated by the label-swapping involution. Hence we can focus our attention solely on the latter.

How does the value of $T$ change when this transformation is precomposed? To see this, it is useful to consider the situation in the $ζ$-plane. Suppose for concreteness that $μ$ and $ν$ are chosen such that $ν,μ,1$ and $-1$ are arranged clockwise as shown in the following diagram. The principal choice of path $γ_+$ is shown in black, whereas its pullback under the exchange of the roots is shown in red. Let it be denoted $γ'_+$.

\begin{center}
\includegraphics[height=50mm]{thesis_graphics_temp/deck_T_1.png}
\end{center}

The difference between these two paths is homologous to a loop anticlockwise around the upper unit circle. So by the normalisation of $Θ^P$,
\[
\int_{γ'_+} Θ^P - \int_{γ_+} Θ^P = \int_{\S^1} = -2π\iu.
\]

Likewise if we consider the difference between the path $γ_-$ and its pullback $γ'_-$ we again have a anticlockwise loop of the upper unit circle.
\begin{center}
    \includegraphics[height=50mm]{thesis_graphics_temp/deck_T_1.png}
\end{center}
Hence
\[
\int_{γ'_-} Θ^P - \int_{γ_-} Θ^P = \int_{\S^1} = -2π\iu.
\]
Putting these together we conclude that the value of $T_0$ changes by $1-p$ under this transformation.
\[
2π\iu (T_0 \circ g - T_0) = \bra{ p\int_{γ'_-} Θ^P - \int_{γ'_+} Θ^P } - \bra{ p\int_{γ_-} Θ^P - \int_{γ_+} Θ^P } = p(-2π\iu) - (-2π\iu) = 2π\iu (1-p).
\]

To infer what the effect of the transformation is on $\tilde{T}$ however, we also must take into account how the coordinates $\tilde{u},\tilde{v}$ may have changed, and consequently altered their winding numbers. If the points $μ,ν$ have been arranged as described, then this restricts the arrangement of $\iu u$, $\iu v$. In particular, as one tranverses the unit circle clockwise, one traverses the imaginary axis in the $z$-plane from bottom to top. Correspondingly, $-\infty < 0 < u < v$.

As $u,v$ are both positive, it must be that $\tilde{u} \in (2πn, 2πn + π)$ and $\tilde{v} \in (2πm, 2πm + π)$ for some integers $n,m$. $2\atan(ku)$ will be between $0$ and $π$, so after the transformation $\tilde{u}$ will be in the range $(2πn +π, 2π(n+1))$. In otherwords its winding number has incresed by $1$. The same can be said for $\tilde{v}$. Putting these two parts together
\[
\tilde{T} \circ g - \tilde{T}
= \bra{ T_0 \circ g + 2(p(m+1)-(n+1)) } - \bra{ T_0 + 2(pm-n) }
= 1-p  +2(p-1) = p-1.
\]
This relation is a difference of analytic functions on an open set, so by continuation it applies everywhere. Thus the effect of the deck transformation on $\tilde{T}$ is to increase its value by $p-1$.







If we consider the surfaces $\mathcal{\tilde{A}}_1(p,q)$ parameterised by $(p,q,k,\tilde{x})$, where $\tilde{x}$ is mapped to either $\tilde{u}$ or $\tilde{v}$ depending on the magnitude of $p$, we see that under the generating deck transformation
\[
(p,q,k,\tilde{x}) \mapsto (p, q + (p-1), k, \tilde{x}')
\]
where $\tilde{x}' = 2πn+ π + 2\atan (k \tan (\tilde{x}/2))$. This provides an identification between $\mathcal{\tilde{A}}_1(p,q)$ and $\mathcal{\tilde{A}}_1(p,q + l(p-1))$ for any integral $l$.

\begin{thm}
\label{thm:topology_curves}
For $p\neq 1$, $\mathcal{C}_1(p)$ is diffeomorphic to
\[
\{ (q,k,\tilde{x}) \in \bra{\R/ (p-1)\Z} \times (0,1) \times \R \},
\]
such that $\mathcal{S}_1(p) = \{ \mathcal{C}_1(p) \mid q \in \Q/ (p-1)\Z \}$.
\begin{proof}
Fix $p\neq 1$ and consider $\mathcal{C}_1(p)$. It is the quotient of $\mathcal{\tilde{A}}_1(p)$ by the covering transformations, which preserve the value of $p$. By Lemma \ref{lem:T_graph}, $\mathcal{\tilde{A}}_1(p)$ is foliated by $\mathcal{\tilde{A}}_1(p,q)$ and we have just shown how different $\mathcal{\tilde{A}}_1(p,q)$ can be identified if their values of $q$ differ by a multiple of $p-1$. Hence it is sufficient to take one representative from each of $\R/(p-1)\Z$ to cover the image.

Lemma \ref{lem:T_graph} also demonstrates that $\mathcal{\tilde{S}}_1(p)$ is the union of $\mathcal{\tilde{A}}_1(p,q)$ with $q \in \Q$. $\mathcal{S}_1(p)$ is the image of $\mathcal{\tilde{S}}_1(p)$, so it is the subset of $\mathcal{C}_1(p)$ where $q$ is in the image of $\Q$, that is $\Q/(p-1)\Z$.
\end{proof}
\end{thm}

This leaves just one special case.

\begin{thm}
\label{thm:topology_curves_p1}
$\mathcal{C}_1(1)$ is diffeomorphic to
\[
\{ (q,k,\tilde{u}) \in \R \times (0,1) \times \bra{\R/2π\Z} \},
\]
such that $\mathcal{S}_1(1) = \{ \mathcal{C}_1(1) \mid q \in \Q \}$.

\begin{proof}
In this special case, we see that the action does not identify path components with different values of $q$; in fact $p-1$ is zero, so $q$ is fixed under the deck transformations. Consequently, the shifting action identifies values of $\tilde{x}$ that differ by $2π$ in the same path component, resulting in annuli.
\end{proof}
\end{thm}

These two theorems give an understanding of what the condition $T\in\Q$ means. If we think of the first condition, $p\in\Q$, it lead to a dense disjoint collection of subspaces $\mathcal{C}_1(p)$ on which the condition was met. We can then think of examining the effect of imposing the second condition on one of these spaces. For $p$ not equal to one, the subset of meeting this extra condition is a dense collection of strips. As can be seen in the pictures below, and proved in the next chapter, they should be thought of as intertwined helicoids. But for $p$ equal to one, instead we have a dense collection of annuli. We shall see that the parameter $p$ can be thought of as controlling the slope of the helicoids, with $p=1$ being the transition between right and left handed spirals, where it is `flat' and closes up.





\subsection{Corollaries}
\label{sub:Corollaries}

$\mathcal{S}_1(1)$ seems special because it is not simply connected. However, on the level of spectral data this disappears and the moduli space is simply connected. But first, we revisit and improve on lemma \ref{lem:closing_conds}. In that lemma, we see that the closing conditions are necessary and sufficient conditions for the existence of spectral data. The proof is constructive, in that it finds a set of spectral data. However, to examine the moduli of spectral data, we must find the `minimal' set from which all other on that curve are generated.

\begin{lem}
On any curve $Σ \in \mathcal{S}_1$, there is a pair of differentials such that every closing differential is an integer combination of that pair.

\begin{proof}
The method of proof is similar to that of lemma \ref{lem:closing_conds}, but the construction must be refined a little. Let $Ψ$ be as described there. Recall, we write $p = n/m$ in lowest terms and define
\[
a = \frac{2π\iu n}{2\iu η^+(1)}, \qquad Ψ = a Θ^E.
\]
Suppose that $Θ$ is an exact closing differential. We assert that it must be an integer multiple of $Ψ$. To see this, note that it is a real scalar of $Θ^E$ and compute its integrals
\[
\int_{γ_+} Θ = 2\iu η^+(1) b = 2π\iu \tilde{n}, \qquad
\int_{γ_-} Θ = -2\iu η^+(-1) b = 2π\iu \tilde{m},
\]
for some $b\in\R$, $n',m' \in \Z$. By the definition of $p$
\[
p = - \frac{η(1)}{η(-1)} = \frac{\tilde{n}}{\tilde{m}},
\]
so we must have that $\tilde{n} = ln, \tilde{m}= lm$ for an integer $l$. It follows that $b= la$ and hence $Θ = l Ψ$.

We now turn our attention to the nonexact differentials. Suppose that for fixed paths $γ_+,γ_-$ the value of $T$ on the curve is $n'/m'$ is lowest terms. Define $g$ to be $m'/\gcd(m,m')$. Now, similar to lemma \ref{lem:closing_conds}, let $x,y$ be integers with $nx-my = 1$ and $Γ^+ = n'my / \gcd(m,m')$, $Γ^- = n'mx / \gcd(m,m')$. Then
\[
\frac{nΓ^- - m Γ^+}{gm}
= \frac{n n'mx - m n'my}{m'm}
= \frac{n'(nx - my)}{m'}
= \frac{n'}{m'},
\]
So we can solve for a differential $Ψ^P$ with period $2π\iu g$ and integrals $2π\iu Γ^+, 2π\iu Γ^-$. Now if $Θ$ is any closing differential on the curve, we may write it as $Θ = bΘ^E + kΘ^P$ for $b\in\R$ and $k\in \Z$. Its period is $2π\iu k$ and its integrals are $2π\iu e, 2π\iu f$. Taking its integrals over $γ_+$ and $γ_-$ and eliminating $b$ leads to
\begin{align*}
\frac{n'}{m'} &= \frac{nf - me}{km} \\
n'k\bra{\frac{m}{\gcd(m,m')}} &= (nf - me)\bra{\frac{m'}{\gcd(m,m')}} \\
&= (nf-me)g.
\end{align*}
Now, by construction $g$ is coprime to $m$. Also, $m'$ is coprime to $n'$, so $g$ is also. Hence we see that $g$ must divide $k$. Consider the differential
\[
Θ - \frac{k}{g}Ψ^P.
\]
It is closing and exact, so a multiple of $Ψ$ by the first part of this proof. Rearranging we have finally written $Θ$ as an integer combination of $Ψ,Ψ^P$.
\end{proof}
\end{lem}

So the closing differentials are a $\Z\times\Z$-subbundle $\mathcal{M}_1$ of the $\Z\times\R$-bundle $\mathcal{B}$ over $\mathcal{S}_1$. $Ψ$ is a well defined function on all of $\mathcal{C}_1(p)$, but $Ψ^P$ is not. Its definition depends on the paths $γ_+$ and $γ_-$, and there is no consistent way to make a smooth choice of paths on the whole space. For $p\neq 1$, $\mathcal{S}_1(p)$ is simply connected however, so it is possible to extend these differentials to a frame for $\mathcal{M}_1(p)$, showing it to be diffeomorphic to $\Z^2 \times \mathcal{S}_1(p) \simeq \Z^2 \,\times\, (\Q/(1-p)\Z)\times(0,1)\times\R$.

For $p=1$, take a path component of $\mathcal{S}_1$. It is an annulus. $p=1=n/m$ implies that $m=n=1$ and $T_0 = m'/n'$ is well defined and constant on the whole annulus. Consider a simple nontrivial loop $\ell : [0,1]$. If we make a choice of $γ_+,γ_-$ on $Σ(\ell(t))$ such that the curves vary smoothly in $t$, what will be the change in $Ψ^P$ when we return to $\ell(1) = \ell(0)$? Let $Ψ^P = b Θ^E + m' Θ^P$ be the initial differential, and let $\tilde{Ψ}^P  = b' Θ^E + m' Θ^P$ be its continuation to $\ell(1)$. From our previous compuations, each of
\[
\int_{γ_+} Θ^P, \int_{γ_-} Θ^P
\]
will be incremented by $2π\iu$ (or decremented, depending on the orientation of $\ell$). The difference $\tilde{Ψ}^P - Ψ^P = (b'-b)Θ^E$ can be explicitly written
\begin{align*}
b'
&= \frac{1}{2\iu η^+(1)}\bra{ 2π\iu y n'm - m' \bra{\int_{γ_+} Θ^P + 2π\iu} } \\
&= b - \frac{2π\iu}{2\iu η^+(1)}m' \\
&= b - am',
\end{align*}
so that $\tilde{Ψ}^P - Ψ^P = -am'Θ^E = - m' Ψ$. Recall that the period of $Ψ^P$ is $g = m' / \gcd(m,m') = m'$. This says that everytime you loop around the annulus, the nonexact differential shifts by $m' Ψ$. Thus the part of the bundle $\mathcal{M}_1$ over this annulus is simply connected. The basis $Ψ,Ψ^P$ is not unique, one may add an integer multiple of $Ψ$ to $Ψ^P$, but this shows that there is a path in the moduli space between bases that have their nonexact differentials differ by $m'\Z Ψ$. Thus the part of the bundle $\mathcal{M}_1$ over this annulus has $m'$ components.

There are some symmetries and special cases that also bear mention. First, we have already remarked upon and made use of the symmetry
\[
T(p,k,u,v) = -p T\bra{ \tfrac{1}{p}, k, v, u },
\]
but what is its interpretation? Looking at
\[
p = \frac{\abs{1-α}\abs{1-β}}{\abs{1+α}\abs{1+β}}
\mapsto \frac{\abs{1+α}\abs{1+β}}{\abs{1-α}\abs{1-β}},
\]
one would guess that it is induced by $(α,β) \mapsto (-α,-β)$. Indeed this can be seen to be the case, as $k$ is invariant under such a transformation, and the induced map between the spectral curves
\begin{align*}
χ: Σ(α,β) &\to Σ(-α,-β) \\
(ζ, η) &\mapsto (-ζ,-η)
\end{align*}
interchanges $1$ and $-1$, effectively swapping the roles of $u$ and $v$. The pullback of the differentials under the involution $χ$ would preserve the integrality of their integrals and so spectral data on $Σ(α,β)$ be transformed into spectral data on $Σ(-α,-β)$. The harmonic map $g(z) : \T^2 \to \S^3$ arises as the gauge transformation between the connections at $1$ and $-1$, so exchanging these points gives the inverted harmonic map $g(z)^{-1}$.

This can be seen directly in the genus zero case. Remember the form of the harmonic map
\[
g(z) = \exp (4w_R \abs{X}\hat{X}) \cdot \exp (-4w_I \abs{Y}\hat{Y}), \quad
\hat{X} = \begin{pmatrix}
0 & 1 \\
-1 & 0
\end{pmatrix}, \quad
\hat{Y} = \begin{pmatrix}
0 & e^{\iu ω} \\
-e^{-\iu ω} & 0
\end{pmatrix},
\]
for some angle $ω$ between the derivatives in the $1$ and $\iu$ directions, which determines the image up to $SO(4)$ rotation. Under inversion
\[
g(z)^{-1} = \exp (4w_I \abs{Y}\hat{Y}) \cdot \exp (-4w_R \abs{X}\hat{X}).
\]
In the genus zero case, the map between the spectral curves is $(ζ,η) \mapsto (-ζ,\iu η)$, so that the differentials transform as
\[
χ^* d \bra{ ζ^{-1}η (bζ - \bar{b}) } = d \bra{ ζ^{-1}η (\iu b ζ + \iu\bar{b}) }.
\]
In particular, the domain is rotated by $\iu$, so that $w'_I = w_R$ and $w'_R = - w_I$. Rotating so that again the derivative in the real direction is fixed
\[
g(z)^{-1} = \exp (-4w_I \abs{Y}\begin{pmatrix}
0 & 1 \\
-1 & 0
\end{pmatrix})
\cdot \exp (-4w_R \abs{X}\begin{pmatrix}
0 & -e^{-\iu ω} \\
e^{\iu ω} & 0
\end{pmatrix}).
\]
Thus we can see that the parameter $ω$ has been changed to $π-ω$. But on the other hand, examining the spectral curve the parameter $ω$ is given by (for some real $x$)
\[
\iu x e^{\iu ω} = \frac{α+1}{α-1},
\]
which under $α\mapsto -α$,
\[
\iu x e^{\iu ω'} = \frac{-α+1}{-α-1} = \bra{ \frac{α+1}{α-1} }^{-1} = \bra{\iu x e^{\iu ω} }^{-1}.
\]
Solving for $ω'$ gives $π-ω$, in complete agreement to before.

\todo{} THERE is a bunch of fiddling around here to make the two approaches both come to $π-ω$, because the inversion also changes the domain subtlety, and ends up reversing one of the basis vectors. If instead you only cared about the unsigned angle between the vectors (ie up to complementary angle), then you could skip the domain related stuff and just say $-ω$ and $π-ω$ represent the same angle, up to a sign. See how much Emma and John hate the handwaving in this example. Could go either way.

Returning to genus one, a special case of this are the spectral curves that are a fixed point of this transformation. In that case, $χ$ is an extra involution of the spectral curve and hence $β = -α$. These are exactly the genus one spectral curves that meet the conditions for the harmonic map to have as their image a totally geodesic two-sphere. Hitchin identifies a particular one parameter family of these maps as the Gauss maps of Delauney surfaces. We identify in which part of $\mathcal{S}_1$ they reside.

Firstly, as these are fixed points of $p \mapsto p^{-1}$, $p$ is one. The symmetry in $T_0$ now reads $T_0(1,k,u,v) = - T_0(1,k,v,u)$. The plane $β=-α$ is two dimensional, but we have three parameters $(k,u,v)$ in this relation, so there must be some relation between them. As the four branch points are linear, $μ=\hat{α}$ and $ν = -\hat{α}$. Directly
\[
z_0 = \frac{A+αB}{A-αB} = \frac{1+ \abs{α}}{1-\abs{α}},\quad
k = \bra{ \frac{1- \abs{α}}{1+\abs{α}} }^2.
\]
It follows from \eqref{eqn:f} that
\[
\iu v = f(-1)
= \bra{\frac{1+ \abs{α}}{1-\abs{α}}}^2 \bra{ \frac{1+ \abs{α}}{1-\abs{α}} \frac{1+\hat{α}}{1-\hat{α}}}^{-1}
= \frac{1}{k} f(1)^{-1}
= \frac{1}{k}(\iu u)^{-1},
\]
or concisely that $v= - (ku)^{-1}$. This is exactly the formula for the change of $u,v$ under the label swapping involution, as it should be because $χ$ is in effect swapping $(α,-α)$ amongst doing other things. Hence using the formula for the label swapping involution in addition to the $1/p$ symmetry
\[
T_0\bra{1,k,u,-(ku)^{-1}}
= T_0\bra{1,k,-(ku)^{-1},u} + 1-p
= T_0\bra{1,k,-(ku)^{-1},u}
= -T_0\bra{1,k,u,-(ku)^{-1}},
\]
from which we deduce that $T_0$ is zero. Therefore we have shown that $\{(α,-α)\}$ is the only annulus of $\mathcal{S}_1(1)$ where $T_0$ is zero, and these correspond to harmonic maps to the sphere.
