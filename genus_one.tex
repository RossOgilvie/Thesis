%!TEX root = thesis.tex

\chapter{The Genus One Moduli Space}
\label{chp:Genus One}
% \epigraph{All analytic functions are alike; each non-analytic function is non-analytic in its own way.}
% \epigraph{Perfection is a journey, not a destination -- Me}

The aim of this chapter is describe the topology of the moduli space of spectral curves $\mathcal{S}_1$, which we consider as a subspace of the space of marked curves $\mathcal{C}_1$.
We will construct the universal cover $\mathcal{\tilde{C}}_1$ of $\mathcal{C}_1$ and recover $\mathcal{S}_1$ as the quotient of a certain subspace $\mathcal{\tilde{S}}_1 \subset \mathcal{\tilde{C}}_1$ by the group of covering transformations.
Recall that a deformation of spectral data is a defined to be a path in the moduli space.
We will prove that the path connected components of $\mathcal{S}_1$ are indexed by two rational numbers $p > 0$ and $q$. For $p\neq 1$ the components are ribbons $(0,1)\times \R$, whereas for $p=1$ the components are annuli.

The key to describing the topology of $\mathcal{S}_1$ lies in the construction of coordinate charts for $\mathcal{\tilde{C}}_1$ with the dual properties that the group of covering transformations acts by translation and also that the components of the preimage $\mathcal{\tilde{S}}_1$ of $\mathcal{S}_1$ in $\mathcal{\tilde{C}}_1$ are the subsets where two of the coordinates take rational values. To accomplish the first goal, we will construct a global coordinate chart on $\mathcal{\tilde{C}}_1$,
\[
\mathcal{\tilde{C}}_1 =
\Set{ \bra{p,k,\tilde{U},\tilde{V}} \in \R^+ \times (0,1) \times \R \times \R }
{ \tilde{U} < \tilde{V} < \tilde{U} + 2π },
\labelthis{eqn:intro coords}
\]
in which the covering transformations act as translations.
% These coordinates show that $\mathcal{\tilde{C}}_1$ the product of $\R^+\times(0,1)$ and a strip.
The group $\mathcal{G}$ of covering transformations is proved to be $\Z\langle \tilde{λ} \rangle$, where the transformation $\tilde{λ}$ acts on $\mathcal{\tilde{C}}_1$ by
\[
\tilde{λ} : \bra{p,k,\tilde{U},\tilde{V}} \mapsto \bra{p,k,\tilde{U} + π,\tilde{V} + π}.
\]
Therefore we see that $\mathcal{C}_1 = \mathcal{\tilde{C}}_1 / \mathcal{G}$ is the product of $\R^+\times(0,1)$ and a cylinder.

Within $\mathcal{\tilde{C}}_1$ we must characterise $\mathcal{\tilde{S}}_1$. We will produce two functions $S$ and $\tilde{T}$ on $\mathcal{\tilde{C}}_1$, with ranges $\R^+$ and $\R$ respectively, such that a marked curve belongs to $\mathcal{\tilde{S}}_1$ precisely when both $S$ and $\tilde{T}$ are rationally valued.
In fact, the first coordinate introduced by~\eqref{eqn:intro coords} is already given by the function $S$.
If $\mathcal{\tilde{C}}_1(p,q)$ is the level set $S = p$ and $\tilde{T} = q$ in $\mathcal{\tilde{C}}_1$, the space $\mathcal{\tilde{S}}_1$ is the union
\[
\mathcal{\tilde{S}}_1 =  \coprod_{p \in \Q^+, q\in\Q} \mathcal{\tilde{C}}_1(p,q).
\]
Crucially, the value of $S$ is fixed by the action of $\tilde{λ}$ and precomposing $\tilde{T}$ with $\tilde{λ}$ increases its value by $S-1$ (Lemma~\ref{lem:T shift}). Thus $\tilde{λ}$ maps $\mathcal{\tilde{C}}_1(p,q)$ to $\mathcal{\tilde{C}}_1(p,q + p-1)$. If $p$ and $q$ are rational, so too is $q + (p-1)$ and hence the group $\mathcal{G} = \Z\langle \tilde{λ}\rangle$ of covering transformations restricts to give a group action on $\mathcal{\tilde{S}}_1$. We can therefore conclude that $\mathcal{S}_1 \diffeo \mathcal{\tilde{S}}_1 / \mathcal{G}$.

\maketikzfigure{Sketch of some level sets of $\tilde{T}$ on a cross-section of $\mathcal{\tilde{C}}_1$. The transformation $\tilde{λ}$ acts as a translation in the given direction. It maps level sets to level sets. Note that in this figure that the $\tilde{V}$-derivative of $\tilde{T}$ does not vanish.}{tikz/universal_level_sets}

As $p$ is rational it is constant along any path in $\mathcal{\tilde{S}}_1$. When working with the path connected components of $\mathcal{\tilde{S}}_1$ we may therefore restrict ourselves to the submanifolds $\mathcal{\tilde{C}}_1(p)$ of $\mathcal{\tilde{C}}_1$ where $p$ is fixed.
Further, the covering transformations do not change $p$ so these submanifolds are $\mathcal{G}$-invariant. The three remaining coordinates $(k,\tilde{U},\tilde{V})$ form a global coordinate chart for each $\mathcal{\tilde{C}}_1(p)$. However, in order to produce a coordinate chart where both goals are met, that is $\tilde{λ}$ acts as translation and $\mathcal{\tilde{S}}_1(p) := \mathcal{\tilde{S}}_1 \cap \mathcal{\tilde{C}}_1(p)$ is the union of coordinate planes, in Lemma~\ref{lem:T_graph} we show that for each $p$ there is a coordinate chart where $q=\tilde{T}$ may be used as a coordinate in place of either $\tilde{U}$ or $\tilde{V}$. Specifically for $p \leq 1$,
\[
\left\{ \bra{q,k,\tilde{U}} \in \R \times (0,1) \times \R \right\}
\]
is a coordinate chart covering $\mathcal{\tilde{C}}_1(p)$, whereas for $p \geq 1$ a coordinate chart covering $\mathcal{\tilde{C}}_1(p)$ is
\[
\left\{ \bra{q,k,\tilde{V}} \in \R \times (0,1) \times \R \right\}.
\]
This shows that the above union of $\mathcal{\tilde{C}}_1(p,q)$ was a decomposition of $\mathcal{\tilde{S}}_1$ into path connected components. If we let $\tilde{X}$ stand for either $\tilde{U}$ or $\tilde{V}$ depending on the magnitude of $p$, then the action of $\tilde{λ}$ in each coordinate chart reads
\[
\tilde{λ} : \bra{p,q,k,\tilde{X}} \mapsto \bra{p,q + (p-1),k,\tilde{X} + π}.
\]

The main result of the chapter takes the quotient of each of these submanifolds $\mathcal{\tilde{C}}_1(p)$ by the group of covering transformations $\mathcal{G}$. For $p\neq 1$, in Theorem~\ref{thm:topology_curves} we have that
\[
\mathcal{C}_1(p)
\diffeo \mathcal{\tilde{C}}_1(p) / \mathcal{G}
= \left\{ \bra{[q],k,\tilde{X}} \in \R/(p-1)\Z \times (0,1) \times \R \right\}.
\]
If further $p\neq 1$ is rational, the moduli space $\mathcal{S}_1(p)$ is the subset where $[q] \in \Q/(p-1)\Z$. For $p=1$, by contrast, Theorem~\ref{thm:topology_curves_p1} shows that
\[
\mathcal{C}_1(1)
\diffeo \mathcal{\tilde{C}}_1(1) / \mathcal{G}
= \left\{ \bra{q,k,\left[\tilde{X}\right]} \in \R \times (0,1) \times \R/π\Z \right\},
\]
where again we recover $\mathcal{S}_1(1)$ as the subset where $q$ is rational.

In either case, we have a foliation of $\mathcal{C}_1(p)$ such that $\mathcal{S}_1(p)$ is a dense collection of leaves. Though the topology of the components changes from ribbons to annuli at $p=1$, we can understand each foliation as belonging to a family parametrised by $p$. As $k$ is unaffected by the quotient, we may consider only $(q,\tilde{X}) \in \R^2$, as shown in Figure~\ref{fig:level set quotient}. For each $p$, the translation $\tilde{λ}$ takes $(q,\tilde{X})$ to $(q + (p-1), \tilde{X} + π)$. Let us rotate each plane so that these translations are in the vertical direction. The lines of constant $q$ now make an angle $\atan \tfrac{p-1}{π}$ to the vertical. We may therefore take the quotient of this space, a cylinder, and for $p\neq 1$ the lines of constant $q$ become helices wrapped around this cylinder. The parameter $p$ changes the slope of helices. When $p=1$ however the lines of constant $q$ are in the direction of translation and close up into circles under the group action. This exceptional case $p=1$ is the transition from left-handed to right-handed helices.

% \makefigure{$q,\tilde{X}$ axes in black, direction of the translation $\tilde{λ}$ in blue, lines of constant $q$ in grey.
% \label{fig:level set quotient}}
% {thesis_graphics_temp/level_set_quotient.png}
\begin{figure}
    \includestandalone[width=\textwidth]{tikz/universal_quotient}
    \caption{On the left, a cross-section of $\mathcal{\tilde{C}}_1(p)$ for fixed $k$, with the $q$ and $\tilde{X}$ axes in black and direction of the translation $\tilde{λ}$ in blue. The grey lines are the lines of constant $q$. The angle $θ$ is given by $\atan \tfrac{p-1}{π}$. In this figure, $p \approx 1 + π > 1$.\\
    On the right, the result of taking the quotient by $\mathcal{G} = \Z\langle \tilde{λ}\rangle$. The plane has been rolled into a cylinder, and the level sets are helices. As $p$ is changed, the slope of the helices changes also. When $p=1$, they close-up into circles on the cylinder.}
    \label{fig:level set quotient}
\end{figure}

Now that we have given a picture of the results of this chapter, let us run through the stages of the proof. The intermediate way-point is the construction of the functions $S$ and $\tilde{T}$, which derive from the closing conditions, Conditions~\ref{P:closing}. The difference between a marked curve of genus one, a point of $\mathcal{C}_1$, and a spectral curve, a point of $\mathcal{S}_1$, is that the latter possesses a pair of linearly independent differentials that satisfy the closing conditions. For any genus $g$, every marked curve of $\mathcal{C}_g$ meets Conditions~\ref{P:real curve}--\ref{P:simple zeroes} and admits a plane $\mathcal{B}_Σ$ of differentials satisfying Conditions~\ref{P:poles}--\ref{P:imaginary periods}. Further, in the case of marked curves of genus one, we will show that every curve has differentials with integral periods, Condition~\ref{P:periods}. But not every curve admits differentials that meet the closing conditions.

We therefore proceed by the construction of coordinates adapted to the closing conditions and the action of the covering transformations.
In the initial three sections of this chapter we work with the parameter space $\mathcal{A}_1$, which we recall from equation~\eqref{eqn:def Ag} to be the space $\{ (α,β) \in D^2 \mid α \neq β \}$, where $D$ is the open unit disc.
We take the quotient space $\mathcal{A}_1/\Z_2$ as a model for the space of marked curves $\mathcal{C}_1$, where $\Z_2$ acts by permutation of $(α,β)$.
Using $Σ$ to denote the covering map, the curve
\[
Σ(α,β) = \{ (ζ,η) \mid η^2 = P(ζ) = (ζ-α)(1-\bar{α}ζ)(ζ-β)(1-\bar{β}ζ) \} \in \mathcal{C}_1,
\]
is the marked curve with branch points $\{α,\cji{α},β,\cji{β}\}$ and the standard scaling given by~\eqref{eqn:def P}.

For each point of $\mathcal{A}_1$, in Section~\ref{sec:Differentials} we give a transformation of the corresponding marked curve to an elliptic curve in Jacobi normal form. This allows us to find on each marked curve the differentials that satisfy~\ref{P:poles}--\ref{P:imaginary periods} and have integral periods. Having found these differentials, in Section~\ref{sec:closing conditions} we reformulate the closing conditions~\ref{P:closing} to produce a function $S$ and a multi-valued function $T$ on $\mathcal{A}_1$, such that they are valued in $\Q$ exactly when a given marked curve admits differentials that further satisfy the closing conditions. A marked curve with a pair of such differentials constitute a triple of spectral data, and so we call such a curve a spectral curve. The first function, $S$ (defined by~\eqref{eqn:def_S}), is strictly positive and has an explicit algebraic formula; its level sets foliate $\mathcal{A}_1$ into solid tori. The second function, $T$ (defined by~\eqref{eqn:def_T}), is however multi-valued and transcendental; it is dependent on paths of integration on the curve, and its formula contains elliptic integrals.

The functions $S$ and $T$ may be analysed by introducing a new set of coordinates on $\mathcal{A}_1$ that are suited to calculation (Section~\ref{sec:Reformulate}).
These new coordinates allows us to reasonably compute the derivatives of $T$. They also dovetail with the results of Section~\ref{sec:EllipticContinuation}, allowing us to lift $T$ to a  single valued function $\tilde{T}$ on $\mathcal{\tilde{C}}_1$, the universal cover of $\mathcal{A}_1$.
We then use the implicit function theorem to prove that the level sets $\mathcal{\tilde{C}}_1(p,q)$ of $S$ and $\tilde{T}$ are graphs and foliate the space $\mathcal{\tilde{C}}_1$ (Lemma~\ref{lem:T_graph}). This allows for an explicit description of $\mathcal{\tilde{S}}_1$, the preimage of $\mathcal{S}_1$ in $\mathcal{\tilde{C}}_1$, as a union as above.
Finally, we push this foliation back down to $\mathcal{C}_1$ and recover $\mathcal{S}_1$ as a quotient by the group of covering transformations $\mathcal{G}$ (Section~\ref{sec:Topology}).

To close the chapter, in Section~\ref{sec:Corollaries} we show that the moduli space of spectral data $\mathcal{M}_1$ is not a trivial bundle over $\mathcal{S}_1$, as $\mathcal{M}_0$ was over $\mathcal{S}_0$. We show however that the total space of the bundle is simply connected. We also consider some well-known examples of harmonic maps, namely the Gauss maps of Delaunay surfaces, and identify them with a specific path connected component of $\mathcal{S}_1$. Building on this, we identify a symmetry of $\mathcal{S}_g$, for any genus $g$, and provide a geometrical interpretation. We illustrate this interpretation for the analogous symmetry of genus zero spectral curves using the explicit equations for harmonic maps derived in the previous chapter.

We summarise the spaces that we have introduced and their relationships to one another in the diagram below. One starts out with the space of marked curves $\mathcal{C}_1$ in the top right corner. It is covered by the parameter space $\mathcal{A}_1$, which is in turn covered by the universal cover $\mathcal{\tilde{C}}_1$. On the next line we consider within each of these spaces the subspaces on which the function $S$ has the value $p\in\Q^+$, denoted by $p$ in parentheses. On the bottom line we have the statement that $\mathcal{\tilde{S}}_1$, the preimage of the space of spectral curves $\mathcal{S}_1$, is a union of level sets $\mathcal{\tilde{C}}_1(p,q)$, the subsets of $\mathcal{\tilde{C}}_1$ on which $S = p$ and $\tilde{T} = q$. The horizontal arrows represent covering maps, labelled with the group of covering transformations, whereas the vertical arrows represent inclusions.
\[
\begin{diagram}
    \mathcal{\tilde{C}}_1 &\rOnto^{2π \Z}&  \mathcal{A}_1  &\rOnto^{\Z_2}&  \mathcal{C}_1 \\
    \uTo  &&  \uTo  &&  \uTo  \\
    %%%%%%%%%%%%%%%%%%%%%%%%%%%%%%%%%%%%%
    \mathcal{\tilde{C}}_1(p)  &\rOnto^{2π \Z}&  \mathcal{A}_1(p)  &\rOnto^{\Z_2}&  \mathcal{C}_1(p) \\
    \uTo  &&  &&  \uTo  \\
    %%%%%%%%%%%%%%%%%%%%%%%%%%%%%%%%%%%%%
    \mathcal{\tilde{S}}_1(p) =  \coprod_{q\in\Q} \mathcal{\tilde{C}}_1(p,q)  && \rOnto && \mathcal{S}_1(p)
\end{diagram}
\]












\section{The Differentials of an Elliptic Marked Curve}
\label{sec:Differentials}
It will be necessary to have explicit formulae for the differentials on a marked curve $Σ$ that have integral periods. In this section, we construct such differentials in four stages. First, we give a coordinate transformation $f$ of the marked curve $Σ$ to the Jacobi normal form of an elliptic curve. On any marked curve it is possible to find differentials that meet conditions~\ref{P:poles}--\ref{P:reality}.
These conditions are:
\begin{enumerate}[label=(P.\arabic*)]
\setcounter{enumi}{3}
\item
$Θ$ has double poles and no residues over $ζ=0$ and $\infty$,
\item
that with respect to the hyperelliptic involution $σ$, $Θ$ obeys $σ^*Θ = - Θ$, and
\item
that $Θ$ is real with respect to $ρ$, which is to say $ρ^* Θ = - \bar{Θ}$.
\end{enumerate}
The second step is to therefore write down set of differentials meeting these three conditions; they form a three-dimensional real vector space $W$.

Next, we use the transformation $f$ to compute the periods of these differentials in terms of the Jacobi elliptic integrals. Finally, we leverage the standard relations among the Jacobi elliptic integrals to describe explicitly the differentials with integral periods, those which meet condition~\ref{P:periods}. Any differential that meets all the conditions~\ref{P:poles}--\ref{P:periods} may be written as a linear combination of two distinguished differentials, $Θ^E$ and $Θ^P$. The choice of these two differentials on $Σ(α,β)$ varies smoothly in the parameters $(α,β)$. The space of differentials with integral periods will be shown to be trivial bundles over $\mathcal{A}_1$ and $\mathcal{C_1}$.

Let us get to work. With the standard scaling~\eqref{eqn:def P}, any marked curve of genus one may be written as
\[
η^2 = P(ζ) := (ζ-α)(1-\bar{α}ζ)(ζ-β)(1-\bar{β}ζ),
\labelthis{eqn:genus one curve}
\]
with roots $α, \cji{α}, β$ and $\cji{β}$. On the other hand, every elliptic curve may be transformed into the Jacobi normal form
\[
w^2 = (1-z^2)(1-k^2z^2)
\]
where $k$ is a complex number called the elliptic modulus. For a given marked curve, how is one to compute the modulus and therefore determine the appropriate Jacobi form into which to transform? The answer lies in the cross ratio of the roots
\[
[α,\cji{α};β,\cji{β}] = \frac{\abs{α-β}^2}{\abs{1-\bar{α}β}^2}. \labelthis{eqn:roots_cross_ratio}
\]
This is a real quantity, and so the four roots lie on a circle (or a line). Thus the roots of the Jacobi form do also, which forces $k$ to be real. Any transformation between the curves must take branch points to branch points, so we must decide on a correspondence for the roots. There are twenty-four possible choices, but we choose the correspondence
\[
  \begin{array}{ l||c|c|c|c}
    ζ & α & \cji{α} & β & \cji{β} \\
    \hline
    z & 1 & -1 & k^{-1} & -k^{-1}
  \end{array}
  \labelthis{eqn:choice f}
\]
This correspondence has three properties that distinguish it from the others. By convention, $k \in (0,1)$, which rules out sixteen of the choices. Second, consider the behaviour of the curve as $k\to 1$. The Jacobi form of the curve develops two nodes at $z=\pm 1$. This corresponds to forming nodes $α=β$ and $\cji{α} = \cji{β}$. But the value of
\[
[1,k^{-1};-1,-k^{-1}] = \frac{4k}{(k+1)^2}
\]
disagrees with eqn~\eqref{eqn:roots_cross_ratio} in this limit, which rules out this correspondence and the three others with the same cross ratio. Finally, our choice of correspondences takes the interior of the unit disc to the right half plane, which will be our convention throughout this chapter.

There is only one other correspondence that has all three of these properties, namely
\[
  \begin{array}{l || c|c|c|c}
    ζ & β & \cji{β} & α & \cji{α} \\
    \hline
    z & 1 & -1 & k^{-1} & -k^{-1}
  \end{array}
\]
The difference between this correspondence and our preferred correspondence is the choice of which root, $α$ or $β$, is mapped to $1$. This is the reason that we must work with $\mathcal{A}_1$ and not $\mathcal{C}_1$. On the latter space, there would be no way to consistently make the choice. For example, take the path $t \mapsto Σ(0.5 e^{\iu t}, -0.5 e^{\iu t})$ in $\mathcal{C}_1$. This is a closed loop where the branch points are interchanged as $t$ varies from $0$ to $π$.

From our correspondence, equating the cross ratios $[α,\cji{α};β,\cji{β}]$ and\\ $[1,-1;k^{-1},-k^{-1}]$ gives
\[
k = \frac{\abs{1-\bar{α}β}-\abs{α-β}}{\abs{1-\bar{α}β}+\abs{α-β}},
\labelthis{eqn:def_k}
\]
and the map, which we shall call $f$, can be computed from the relation
\[
[α,\cji{α};β,ζ] = [1,-1;k^{-1},f(ζ)].
\]
Instead of solving this relation for $f$ immediately, we turn to understanding the geometry of the transformation so that we might produce a meaningful expression.

The unit circle in the $ζ$-plane plays an important role in the definition of a spectral curve, so it is natural to ponder its image under $f$. The involution $ρ(ζ)$ fixes the unit circle, and exchanges the pairs of branch points $α,\cji{α}$ and $β,\cji{β}$. The corresponding antiholomorphic involution $\tilde{ρ}(z)$ in the $z$-plane that exchanges $1,-1$ and $k^{-1},-k^{-1}$ is $z\mapsto -\bar{z}$. Its fixed point set is the imaginary axis, which therefore must be the image of the unit circle under $f$.

As already mentioned, the four roots of the spectral curve lie on a circle (or a line), which we shall call the branch circle. Let the two points at the intersection of the branch circle with the unit circle be $μ$ and $ν$, with $μ$ lying between $α$ and $\cji{α}$ and $ν$ lying between $β$ and $\cji{β}$ (see Figure~\ref{fig:zeta plane}). Under $f$, the branch circle is mapped to the real axis. Therefore the $f(μ)$ and $f(ν)$ must lie on the intersection of the real and imaginary axes. Hence $f(μ) = 0$ and $f(ν) = \infty$.

A Möbius transformation, such as $f$, is determined up to scaling by the points it sends to $0$ and $\infty$, in this case $μ$ and $ν$. One other point is therefore needed to determine this scaling. We write $z_0 := f(0)$. Using the reality structure $\tilde{ρ}(z)$, we have $f(\infty) = -\bar{z_0}$. These points allow us to write concise formulae for $f$ and $f^{-1}$
\begin{align}
z = f(ζ) &= -\bar{z}_0 \frac{ζ - μ}{ζ - ν},
\label{eqn:f} \\
ζ = f^{-1}(z) &= ν \frac{z - z_0}{z + \bar{z_0}}.
\label{eqn:f_inv}
\end{align}

\maketikzfigure{The $ζ$-plane, with points marked in blue. The black labels are their images under $f$. The red line is the circle through the branch points. \label{fig:zeta plane}}{tikz/zeta_plane}
\maketikzfigure{The $z$-plane, with points marked in black. The blue labels are their images under $f^{-1}$.}{tikz/z_plane}

\begin{lem}
\label{lem:coeff_f_smooth}
The functions $μ,ν,z_0,(z_0)^{-1}$ and $k$ are smooth functions of the parameters $(α,β)\in\mathcal{A}_1$. The function $f(α,β)(ζ)$ is a smooth function of $(α,β,ζ)\in\mathcal{A}_1 \times \CP^1$ whenever $ζ \neq ν$. Further, $μ - ν$ is never zero.
\begin{proof}
Recall that $μ$ is the point such that $f(μ) = 0$. We may find a formula for $μ$ in terms of $α$ and $β$ using the cross-ratio relation $[α,\cji{α};β,μ] = [1,-1;k^{-1},0]$. Rearranging gives
\[
μ = \frac{ (α-β)\abs{1-\bar{α}β} + α(1-\bar{α}β)\abs{α-β} }{ \bar{α}(α-β)\abs{1-\bar{α}β} + (1-\bar{α}β)\abs{α-β} }.
\]
This could fail to be a smooth function if $α-β$ or $1-\bar{α}β$ were zero, or if the denominator was zero. The factor $α-β$ is never zero on $\mathcal{A}_1$ by definition. The factor $1-\bar{α}β$ is zero if and only $\cji{α}=β$ and, as both $α$ and $β$ are inside the unit disc, this is impossible. Finally, the denominator is zero exactly when
\[
\bar{α} = - \frac{1-\bar{α}β}{\abs{1-\bar{α}β}} \frac{\abs{α-β}}{α-β}.
\]
But the right hand side is an element of the unit circle, so again this possibility is eliminated. The proof of smoothness is entirely similar for
\begin{align*}
ν &= \frac{ (α-β)\abs{1-\bar{α}β} - α(1-\bar{α}β)\abs{α-β} }{ \bar{α}(α-β)\abs{1-\bar{α}β} - (1-\bar{α}β)\abs{α-β} }, \\
z_0 &= \frac{ α(\bar{α}-\bar{β})\abs{1-\bar{α}β} + (1-α\bar{β})\abs{α-β} }{ α(\bar{α}-\bar{β})\abs{1-\bar{α}β} - (1-α\bar{β})\abs{α-β} },
\labelthis{eqn:def z0} \\
(z_0)^{-1} &= \frac{ α(\bar{α}-\bar{β})\abs{1-\bar{α}β} - (1-α\bar{β})\abs{α-β} }{ α(\bar{α}-\bar{β})\abs{1-\bar{α}β} + (1-α\bar{β})\abs{α-β} }, \text{ and} \\
k &= \frac{\abs{1-\bar{α}β}-\abs{α-β}}{\abs{1-\bar{α}β}+\abs{α-β}}.
\end{align*}
By the formula~\eqref{eqn:f}, we also conclude that $f$ is smooth so long as the denominator is nonzero.

The final claim is that $μ-ν$ is never zero. This is clear from the geometry, as the branch circle is a circle that passes through points both inside and outside the unit circle and so must intersect the unit circle at distinct two points. Algebraically, the difference vanishes only if
\[
2\bra{\abs{α}^2 - 1} (α-β)(1-\bar{α}β) = 0,
\]
and this has already shown not to occur on $\mathcal{A}_1$.
\end{proof}
\end{lem}

Because of the holomorphic involution $σ: η\to-η$, equations~\eqref{eqn:f} and~\eqref{eqn:f_inv} almost but not quite specify a relation between $η$ and $w$: there is a free sign choice to make. On the marked curve $Σ = \{ (ζ,η) | η^2 = P(ζ) \}$ there are two disjoint circles in $Σ$ lying over the unit circle in $\CP^1$.
At a point $(ζ,η)$ over the unit circle in $Σ$, we have that the value of $η$ is $\pm ζ\abs{ζ-α}\abs{ζ-β}$.
Thus there is a notion of the `positive' unit circle, the one on which $η$ is positive over $ζ=1$.
We define the function $η^+$ to be $ζ\abs{ζ-α}\abs{ζ-β}$.

Under the transformation $f$ the unit circle is mapped to the imaginary axis. At the points over the imaginary axis in the Jacobi elliptic curve, those points $(z,w)$ for which $z=\iu u$, we have that $w = \pm \sqrt{1+u^2}\sqrt{1+k^2u^2}$. Again, it is possible to make a consistent choice of sign along these two disjoint circles. For the sake of being concrete, we choose the transformation between elliptic curves to map the positive unit circle to points over the imaginary axis where $w$ is positive. In a slight abuse of notation, we shall also use $f$ to denote the map between elliptic curves.

Having found the map $f$ that transforms a genus one marked curve $Σ$ into Jacobi normal form, we may now turn our attention to the other part of the spectral data, the differentials that satisfy conditions~\ref{P:poles}--\ref{P:periods}. Let us first find the vector space of differentials that satisfy just conditions~\ref{P:poles}--\ref{P:reality}, and write a basis for this space. Following the notation of~\eqref{eqn:def b}, recall that all differentials on a genus one marked curve $Σ$ that have (at worst) a double pole over $ζ=0$ and $ζ=\infty$ may be written the form
\[
Θ = b(ζ)\frac{dζ}{ζ^2η},
\]
for a polynomial $b(ζ)$ of degree $4$. Condition~\ref{P:reality} forces $b\in \mathcal{P}^4_\R$, the space of polynomials that are real with respect to $ρ$. Practically, if we write $b(ζ) = b_0 + \dots + b_4 ζ^4$, this condition forces $b_i = b_{4-i}$. Lastly, if we write out the equation of $Σ$ as $η^2 = P(ζ) = P_0 ζ + \dots + P_4 ζ^4$, $Θ$ has no residues exactly when $P_1b_0 - 2P_0b_1 = 0$ (equation~\ref{eqn:residue condition}). If we count the degrees of freedom remaining, the last equation shows that we may choose $(b_0,b_1)$ from a complex line, whereas $b_2$ may be any real number. Hence for each marked curve $Σ$ there is a real three-dimensional vector space $W$ of differentials that meet conditions~\ref{P:poles}--\ref{P:reality}.

This presents an obvious choice of basis, but one that we will not choose. Instead, we shall choose a basis that it suited to computing the periods of the differentials so that we may be able to satisfy condition~\ref{P:periods}, that the periods of the differentials lie in $2π\iu$. For this we turn to the the classical theory of elliptic curves, which has long studied differentials and their periods. It is standard to refer to differentials with double poles and no residues as differentials of the second kind. Condition~\ref{P:poles} may therefore be rephrased as the differential must be of the second kind and have poles at the points of $\Sigma$ over $ζ=0$ and $\infty$. The standard Jacobi differential of the second kind is defined to be
\[
e := (1-k^2 z^2) \frac{dz}{w}.
\labelthis{eqn:def_e}
\]
Every differential of the second kind may be written as the linear combination of $e$, the holomorphic differential $ω$,
\[
ω := \frac{dz}{w},
\]
and an exact differential \cite[Art. 167]{Hancock1910}.

Note that the holomorphic differential $ω$ lies in the space $W$ and so makes for an obvious first basis vector. It accounts for every possible choice of $b_2$.
In genus one, there is a real and exact differential with double poles over $ζ=0$ and $\infty$, namely
\[
Θ^E := \iu\; d\left( \frac{η}{ζ} \right).
% = \iu\left[ -αβ + \frac{1}{2}\left(α(1+\abs{β}^2) + β(1+\abs{α}^2)\right)ζ - \frac{1}{2}\left(\bar{α}(1+\abs{β}^2) + β(1+\abs{α}^2)\right)ζ^3 + \bar{α}\bar{β}ζ^4 \right]\frac{dζ}{ζ^2η}.
\]
We take it as the second basis vector. The superscript $E$ is a mnemonic for exact. Given that we already have taken $ω$ as a basis vector, we seek to complete the basis of $W$ with the sum of $e$ and an exact differential.

Recall equation~\eqref{eqn:def_e}, the definition of $e$. It is real with respect to $\tilde{ρ}(z) = -\bar{z}$, but it has a pole at $z=\infty$ ($ζ=ν$), which not allowed under condition~\ref{P:poles}. This pole can be moved to $ζ=\infty$ by adding an exact differential. We assert that
\[
e + d\left[ \frac{w}{z + \bar{z}_0} \right]
\]
has no pole at $z=\infty$. To check this let $z' = z^{-1}$, and expand $e$ about $z' = 0$.
\begin{align*}
w &= kz'^{-2} \bra{1 + O(z'^2) } \\
e &= -k^2z'^{-2} \bra{1-k^{-2}z'^2} \times -z'^{-2} dz' \times k^{-1}z'^{2} \bra{1 + O(z'^2)} \\
&= kz'^{-2} dz' \bra{1 + O(z'^2)},
\end{align*}
whereas the exact differential has the following expansion
\begin{align*}
d\left[ \frac{w}{z + \bar{z}_0} \right]
&= d\left[ k z'^{-1} \bra{1  + O(z'^2)} \bra{1 - \bar{z}_0z' + \bar{z}_0^2z'^2  + \ldots} \right]\\
&= d\left[ k z'^{-1} \bra{1  + O(z')} \right]\\
&= -k z'^{-2}dz' \bra{1 + O(z'^2)},
\end{align*}
which shows that their sum is holomorphic at $z=\infty$. Unfortunately, this differential is not real with respect to $\tilde{ρ}$. To correct this deficiency we shall have to add another exact differential. The set of exact differentials (not necessarily real) with the double poles over $ζ=0,\infty$ is
\[
\Set { C\,d\left[ \frac{w}{(z-z_0)(z + \bar{z}_0)} \right] }{ C\in \C},
\]
so we add a differential of this form to restore the reality. The differential
\[
e + d\left[ \frac{w}{z + \bar{z}_0}\right] + C\,d \left[\frac{w}{(z-z_0)(z + \bar{z}_0)}\right]
= e + d\left[ \frac{(z - z_0 + C)w}{(z-z_0)(z + \bar{z}_0)}\right]
\]
is real when $z_0 - C$ is an imaginary number. Thus we should choose $C \in \Real z_0 + i\R$. The apparent freedom to choose the imaginary part of $C$ is exactly adding a scalar of $Θ^E$ and so will not change the span of the resulting basis. There are two natural choices; taking $z_0 - C$ to be zero, or taking $C$ to be purely real. Both have their merits, but the latter choice ends up being superior as it makes the principal part of this differential perpendicular to the principal part of $Θ^E$, which introduces a symmetry that we will use later. Hence we take as our third basis differential
\[
ε := e + d\left[ \frac{w}{z + \bar{z}_0} \right] + \Real z_0\; d\left[ \frac{w}{(z-z_0)(z + \bar{z}_0)} \right]
= e + d\left[ \frac{(z - \iu\,\Imag z_0)w}{(z-z_0)(z + \bar{z}_0)} \right].
\labelthis{eqn:def ε}
\]

We are now in a position to compute the periods of our basis $\{ ω,Θ^E,ε \}$. We choose a basis of homology of the marked curve $Σ$ following way. First, take the branch cuts to be along the branch circle, between $α$ and $β$ and between $\cji{α}$ and $\cji{β}$, in particular so that they do not cross the unit circle. Under $f$, this corresponds to the standard choice of $[1,k^{-1}]$ and $[-k^{-1},-1]$. We can speak of points of $Σ$ as being on the positive or negative sheet, where as before the positive sheet is where $η$ is positive over $ζ=1$.

For the loop $A$, start on the positive unit circle at $μ$, traverse in and around $α$ anticlockwise (crossing a branch cut), then cross the negative unit circle and continue anticlockwise around $\cji{α}$ before returning to the starting point. For the loop $B$, start at the same point we began $A$ and follow the unit circle clockwise. These loops are shown in Figure~\ref{fig:zeta periods}.

% \makefigure{$A$ in red and $B$ in blue}{thesis_graphics_temp/zeta_periods.png}
\maketikzfigure{The anticlockwise, real period $A$ in red and the clockwise, imaginary period $B$ in blue. \label{fig:zeta periods}}{tikz/zeta_periods}

The image of $A$ under $f$ is the anticlockwise loop around $-1$ and $1$ with the left to right part of the path on the positive sheet. The loop $f(B)$ is simply a traversal of the imaginary axis from bottom to top on the positive sheet. But this is homologous to a clockwise loop around $1$ and $k^{-1}$. Thus we have chosen the basis of homology $A,B$ such that their images under $f$ are the standard choice of homology on an elliptic curve in Jacobi normal form, as depicted in Figure~\ref{fig:standard periods}.

The loop $A$ is a real period, which means that the integral of a real differential over $A$ is real. To prove this, recall that a differential $Θ$ is real with respect to $ρ$ when $ρ^* Θ = - \bar{Θ}$. By construction, $ρ_* A = -A$. Together,
\[
\bar{\int_A Θ}
= \int_A \bar{Θ}
= -\int_A ρ^* Θ
= -\int_{ρ_* A} Θ
= -\int_{-A} Θ
= \int_{A} Θ,
\]
which shows the $A$-period of $Θ$ to be real. Likewise, the integral of a real differential over $B$ is always an imaginary number.

We wish to be able to describe the differentials of $W$ that have both imaginary and integral periods.
\begin{lem}
On a genus one marked curve $Σ$, suppose there is a differential $Θ^P$ satisfying Conditions~\ref{P:poles}--\ref{P:periods} such that
\[
\int_B Θ^P = 2π \iu.
\]
Then for any other differential $Θ$ satisfying Conditions~\ref{P:poles}--\ref{P:periods}, $Θ$ lies in $\R Θ^E + \Z Θ^P \subset W$.
\begin{proof}
It is always the case that the real periods of a holomorphic differential on a compact Riemann surface are non-zero \cite[Cor~VIII.4.3]{Miranda1995}. Therefore $\{ ω, Θ^E, Θ^P \}$ is a linearly independent set of differentials in $W$, and thus a basis. By satisfying~\ref{P:poles}--\ref{P:reality}, $Θ$ is forced to lie in $W$ and so is a combination of this basis.

If $Θ$ were exact, which is to say that its integrals over $A$ and $B$ were zero, it follows immediately that $Θ$ would have to be a real multiple of $Θ^E$. If $Θ$ were non-exact, then by~\ref{P:periods} it would have an imaginary period of $2π\iu l$, for some $l \in \Z$. By subtracting $lΘ^P$ the result would be an exact differential, and therefore would have to be a real multiple of $Θ^E$.
\end{proof}
\end{lem}

Thus we have reduced the problem of describing differentials on $Σ$ meeting Conditions~\ref{P:poles}--\ref{P:periods} to that of finding a differential $Θ^P \in W$ with vanishing $A$-period and a $B$-period of $2π\iu$. The superscript $P$ is a mnemonic for period. Clearly, we may add a real multiple of $Θ^E$ to $Θ^P$ without altering affecting this requirement. Therefore, for some real constants $a$ and $b$, let $Θ^P = aω + bε$ be a combination of the other two basis vectors of $W = \R\langle ω,Θ^E,ε\rangle$.

It is useful at this point to summarise the standard elliptic integrals, the periods of the differentials $ω$ and $e$,
\begin{align*}
\int_{f(A)} ω &= 4K(k) &
\int_{f(B)} ω &= 2\iu K' \\
\int_{f(A)} e &= 4E(k) &
\int_{f(B)} e &= 2\iu(K'-E')
\end{align*}
where $K$ and $E$ are the complete elliptic integrals of the first and second kind, and the prime denotes not the derivative but instead the elliptic complement. The elliptic complement is by definition $k' = \sqrt{1-k^2}$, and $K'(k) = K(k')$ and $E'(k) = E(k')$. Further properties of elliptic integrals may be found in Appendix~\ref{chp:Elliptic Integrals}. Note that $ε$ is the sum of $e$ and an exact differential and so has the same periods as $e$.

We require that
\begin{align*}
\int_A Θ^P &= 4Ka + 4Eb = 0 \\
\int_B Θ^P &= 2\iu K' a + 2\iu(K'-E')b = 2π\iu.
\end{align*}
From the first equation, we can write $a = cE$ and $b = - cK$ for some $c\in \R$. Substituting this into the second equation gives
\begin{align*}
π
&= cK'E - c(K'-E')K \\
&= c(K'E + KE' - KK') \labelthis{eqn:legendre_relation}\\
&= \frac{\pi}{2}c \\
c &= 2
\end{align*}
where~\eqref{eqn:legendre_relation} uses Legendre's relation (see Lemma~\ref{lem:Legendres relation}). Thus $Θ^P = 2Eω - 2Kε$, or if we unwind the definition of $ε$,~\eqref{eqn:def ε},
\[
Θ^P = 2E ω - 2Ke - 2K d\left[ \frac{(z-\iu\,\Imag z_0)w}{(z-z_0)(z + \bar{z}_0)} \right].
\labelthis{eqn:thetaP}
\]
This equation shows a nice division, with the first two terms providing the desired periods and the last term giving the required poles. Though the choices up to this point may seem arbitrary and contrived, they are not, as we can characterise the differential $Θ^P$ in the following way.

\begin{lem}
    \label{lem:theta2_characterisation}
The differential $Θ^P$ is the unique real differential on the marked curve $Σ$ with double poles and no residues over $ζ=0$ and $\infty$, with periods $0$ and $2π\iu$ over $A$ and $B$ respectively, and with its principal part over $ζ=0$ satisfying
\[
\pp Θ^P \in \iu \R \pp Θ^E.
\]

\begin{proof}
We first verify that $Θ^P$ has such properties, then verify uniqueness. The only property not yet demonstrated is the third one, concerning the principal part. We note that $ζ=0$ corresponds to $z=z_0$, so for some real scalar $r$
\[
\pp Θ^E
= \iu \pp d\bra{\frac{η}{ζ}}
= \iu \pp d \bra{ \frac{rw}{(z-z_0)(z + \bar{z}_0)} }
= -\iu r \frac{w(z_0)}{z_0 + \bar{z}_0}\frac{dz}{(z-z_0)^2}.
\]
And on other side we have
\begin{align*}
\pp Θ^P
&= - 2K \pp d\left[ \frac{(z-\iu\,\Imag z_0)w}{(z-z_0)(z + \bar{z}_0)} \right] \\
&= + 2K \frac{(z_0-\iu\,\Imag z_0)w(z_0)}{z_0 + \bar{z}_0} \frac{dz}{(z-z_0)^2} \\
&= 2K (\Real z_0) \frac{w(z_0)}{z_0 + \bar{z}_0} \frac{dz}{(z-z_0)^2}.
\end{align*}
To establish uniqueness, suppose that $Θ$ was another such differential. Then $Θ-Θ^P$ would be exact, real and have double poles with no residues. So for some real $s$
\[
Θ = Θ^P + s Θ^E.
\]
As taking principal part is a linear operation, the third condition forces $s=0$.
\end{proof}
\end{lem}

As we have already remarked, having found this pair of differentials any other differential satisfying~\ref{P:poles}--\ref{P:periods} may be written in the form $a Θ^E + n Θ^P$ for some $a\in\R$ and $n\in\Z$. We may think of the differentials satisfying~\ref{P:poles}--\ref{P:imaginary periods} as forming a rank 2 real vector bundle over $\mathcal{A}_1$ spanned by $Θ^E$ and $Θ^P$, and those differentials satisfying~\ref{P:poles}--\ref{P:periods} as forming a $\Z\times\R$-subbundle. This pair of differentials $Θ^E$ and $Θ^P$ vary smoothly with respect to $(α,β)$ and are always linearly independent so they trivialises that bundle.

Recall that $\mathcal{A}_1$ double covers $\mathcal{C}_1$; the points $(α,β)$ and $(β,α)$ in $\mathcal{A}_1$ correspond to the same marked curve $Σ\in \mathcal{C}_1$. Suppose we have a section of differentials $Θ : (α,β) \mapsto Ω^1(Σ(α,β))$. As $Σ(α,β)$ and $Σ(β,α)$ are the same curve, this raises the question of what the difference $Θ(α,β) - Θ(β,α)$ is. Asking this for $Θ^E(α,β)$ and $Θ^P(α,β)$, we observe directly that $Θ^E = \iu d (η/ζ)$ is invariant under the interchange of $α$ and $β$. Now we may use the characterisation given by Lemma~\ref{lem:theta2_characterisation} to conclude the same for $Θ^P$, because
\[
\pp Θ^P(β,α) \in \iu \R \pp Θ^E(β,α) = \iu \R \pp Θ^E(α,β)
\]
and so uniqueness forces $Θ^P(β,α) = Θ^P(α,β)$.

The consequence of this is that $\langle Θ^E,Θ^P \rangle$ pushes forward to a well-defined basis of the differentials over $\mathcal{C}_1$ as well, and just as for $\mathcal{A}_1$ they trivialise the bundle $\mathcal{B}_1 \to \mathcal{C}_1$. Though we mainly concentrate on the subspace of $\mathcal{C}_1$ of marked curves that admit spectral data, at the end of this chapter we will consider the space of spectral data within the total space of pairs of differentials
\[
\mathcal{B}_1 \times \mathcal{B}_1 = (\R\langle Θ^E,Θ^P \rangle)^2.
\]


\section{The Closing Conditions}
\label{sec:closing conditions}
The closing conditions, Conditions~\ref{P:closing}, are the conditions that spectral data must meet in order that they correspond to a harmonic map of the torus, rather than a harmonic map of the plane (of finite type). If $(Σ,Θ,\tilde{Θ})$ is a triple of spectral data, $Θ$ satisfies the closing condition at $ζ=1$ if
\[
\int_{γ_{+}} Θ \in 2π\iu \Z,
\]
where $γ_+$ is a path that begins at $(1,-η^+(1))$ and ends at $(1,η^+(1))$, the two points on the spectral curve lying over $ζ=1$. Recall that we have defined $η^+(ζ) = ζ\abs{ζ-α}\abs{ζ-β}$ to be the value of $η$ on the positive unit circle in $Σ$. However, the value of the integral is dependent on the particular path chosen. To see that this condition is none-the-less well defined, suppose that $γ$ and $γ'$ are two paths between the two points over $ζ=1$. Their difference $γ-γ'$ is a closed loop and homologous to an integral combination $aA + bB$ of the periods $A$ and $B$. The difference in the values of the integrals is therefore
\[
\int_{γ} Θ - \int_{γ'} Θ
= \int_{γ - γ'} Θ
= a\int_{A} Θ + b\int_{B} Θ
= 2π\iu nb,
\]
because by the period condition~\ref{P:periods} the real period of $Θ$ is zero and its imaginary period is a multiple of $2π\iu$. So although the value of the integral is dependent on the path, the condition that the value must lie in $2π\iu \Z$ is not. Likewise, the closing condition at $ζ=-1$ is defined by taking $γ_-$, a path from $(-1,-η^+(-1))$ to $(-1,η^+(-1))$, and requiring the integral of $Θ$ over this path to lie in $2π\iu \Z$ also.

We have seen that every marked curve $Σ\in\mathcal{C}_1$ admits differentials that meet Conditions~\ref{P:poles}--\ref{P:periods}, but it is not possible to find differentials on every curve that further satisfy Condition~\ref{P:closing}. It will be our ongoing aim to find all such curves in $\mathcal{C}_1$.

Let us begin by formulating a condition on $(α,β)$ which will determine when $Σ(α,β)$ admits an exact differential that meets the closing conditions. For an exact differential, such as $Θ^E$, the particular path of integration is irrelevant and the value of the integral is
\[
\int_{γ_{+}} Θ^E = i \left. d\bra{\frac{η}{ζ}} \right|_{(1, -η^+(1))}^{(1, η^+(1))} = 2i η^+(1) = 2i \abs{1-α}\abs{1-β}
\]
And similarly over the other marked point
\[
\int_{γ_{-}} Θ^E = -2i η^+(-1) = 2i \abs{1+α}\abs{1+β}.
\]
Any other exact differential with the properties~\ref{P:poles}--\ref{P:periods} must be a real multiple of $Θ^E$.
For a real scalar $a \in \R$, the two closing conditions applied to $a Θ^E$ are
\begin{equation}
\left.
\begin{aligned}
2\iu η^+(1) a &\in 2π\iu \Z, \\
-2\iu η^+(-1) a &\in 2π\iu \Z.
\end{aligned}
\quad
\right\}
\label{eqn:exact closing cond}
\end{equation}
Eliminating $a$ from the two equations, there is a common solution for $a$ if and only if
\[
S(α,β) := \frac{2\iu η^+(1)}{-2\iu η^+(-1)} = \frac{\abs{1-α}\abs{1-β}}{\abs{1+α}\abs{1+β}} \in \Q^+.
\labelthis{eqn:def_S}
\]
This gives the flavour of what we are aiming to achieve. We will produce two explicit functions such that a marked curve admits spectral data exactly when these functions take rational values. The two functions may be interpreted as defining equations for the subspace of spectral curves $\mathcal{S}_1$ within the space of all marked curves $\mathcal{C}_1$.

Before we plough ahead to differentials with periods, there is a simplification we can make. Suppose that we have a triple of spectral data $(Σ,Θ^1,Θ^2)$ such that the differentials $Θ^1$ and $Θ^2$ have imaginary periods $2π\iu l_1$ and $2π\iu l_2$ respectively. Let $l>0$ be the greatest common denominator of $l_1$ and $l_2$, and by Bézout's identity let $x$ and $y$ be integers that satisfy
\[
xl_1 + yl_2 = l.
\]
Then consider the differentials $Ψ^E$ and $Ψ^P$ defined by the following integer combination
\[
\vt{Ψ^E}{Ψ^P} =
\begin{pmatrix}
\tfrac{l_2}{l}    &   -\tfrac{l_1}{l} \\
x                       &   y
\end{pmatrix}
\vt{Θ^1}{Θ^2}
\]
The new pair of differentials are simpler in the sense that their imaginary periods are $0$ and $2π\iu l$ respectively. They also meet the closing condition, because they are an integer combination of differentials that do. And the integer matrix has determinant one, so is invertible over the integers. Further, the two differentials are linearly dependent exactly when $Ψ^E$ is zero. Hence,

\begin{lem}
\label{lem:exist spectral data}
A marked curve admits spectral data if and only if it admits a pair of nonzero differentials with imaginary periods $0$ and $2π\iu l$, for some positive integer $l$, that also satisfy the closing conditions~\ref{P:closing}.
\hfill\qedsymbol
\end{lem}

The the condition above, $S\in\Q^+$, is a necessary condition for a spectral curve to admit spectral data. To find a second necessary condition, concerning the existence of a  differential with imaginary period $2π\iu l$, we follow the same line of reasoning. For some real number $b$, we may write $Ψ^P = b Θ^E + l Θ^P$. Fix two paths $γ_+, γ_-$. The two closing conditions applied to $Ψ^P$ are then
\begin{equation}
\left.
\begin{aligned}
2\iu η^+(1) b + l\int_{γ_+} Θ^P &= 2π\iu Γ^P_+ \in 2π\iu \Z, \\
-2\iu η^+(-1) b + l\int_{γ_-} Θ^P &= 2π\iu Γ^P_- \in 2π\iu \Z.
\end{aligned}
\quad
\right\}
\label{eqn:period closing cond}
\end{equation}
Again elimination of $b$ yields the condition for a common solution to exist. This condition can be written as
\[
2π\iu T(α,β,γ_+,γ_-) := S(α,β) \int_{γ_-} Θ^P - \int_{γ_+} Θ^P
% = 2π \iu \frac{S(α,β) Γ^P_- - Γ^P_+}{l}
\in 2π\iu \Q,
\labelthis{eqn:def_T}
\]
using the definition of $S(α,β)$ to substitute for $-η^+(1)/η^+(-1)$. As was shown at the beginning of this section, whether the integral of a differential over $γ_+$ or $γ_-$ lies in $2π\iu \Z$ is independent on the particular path chosen between the marked points. In the same manner, if $S(α,β)$ is rational then the condition $T(α,β,γ_+,γ_-)\in\Q$ is independent of the choice of paths because a different path $γ_+$ or $γ_-$ will change the corresponding integral of $Θ^P$ by a multiple of $2π\iu$. Therefore $T(α,β)$ is defined up to an element of $\Z\langle 1,S(α,β)\rangle \subset \Q$. We may therefore consider $T(α,β)$, without a particular choice of path, as a multi-valued function on $\mathcal{A}_1$.

We shall prove that these two conditions are sufficient.

\begin{lem}
\label{lem:closing_conds}
A marked curve admits a pair of nonzero differentials, one exact and one inexact, which both satisfying conditions~\ref{P:poles}--\ref{P:periods} and also the closing conditions~\ref{P:closing} if and only if $S\in\Q^+$ and $T\in\Q$ for any paths $γ_+, γ_-$.

\begin{proof}
From the above discussion, these are a necessary conditions.

For the converse suppose that the two functions are both rational, say $S = n/m$ and $T = n'/m'$. Further assume that $n$ and $m$ are coprime, as we can always take them to be.
First we will attempt to solve~\eqref{eqn:exact closing cond} to produce an exact differential $Ψ^E$. The rationality of $S$ directly ensures the consistent solution of an $a$.
Namely, we may define
\[
a := \frac{2π\iu n}{2\iu η^+(1)}.
\]
Then observe that
\[
a = \frac{2π\iu n}{2\iu η^+(1)} = \frac{2π\iu n}{-2\iu η^+(-1)}\frac{-2\iu η^+(-1)}{2\iu η^+(1)} = \frac{2π\iu n}{-2\iu η^+(-1)} \frac{m}{n} = \frac{2π\iu m}{-2\iu η^+(-1)},
\]
and hence for $Ψ^E := aΘ^E$ its integrals over $γ_+$ and $γ_-$ are
\begin{align*}
\int_{γ_{+}} Ψ^E &= a \int_{γ_+} Θ^E = 2\iu η^+(1)a = 2π \iu n \\
\int_{γ_{-}} Ψ^E &= a \int_{γ_-} Θ^E = -2i η^+(-1)a = 2\iu π m.
\end{align*}
This demonstrates the existence of an exact differential $Ψ^E$ that satisfies the closing conditions~\ref{P:closing}.

To find a non-exact differential $Ψ^P$ that also satisfies the closing conditions, we must solve a similar equation. For some real number $b$ and integer $l$, suppose that $Ψ^P = bΘ^E + l Θ^P$. We must find such $b$ and $l$ so that~\eqref{eqn:period closing cond} holds for some integers $Γ^P_+$ and $Γ^P_-$. Recall that $T$ is a rational number $n'/m'$ and so eliminating $b$ from~\eqref{eqn:period closing cond} gives
\begin{align*}
\frac{1}{2\iu η^+(1)} \bra{2π\iu Γ^P_+ - l\int_{γ_+} Θ^P}
&= \frac{1}{-2\iu η^+(-1)} \bra{2π\iu Γ^P_- - l\int_{γ_-} Θ^P} \\
2π\iu Γ^P_+ - l\int_{γ_+} Θ^P
&= \frac{2\iu η^+(1)}{-2\iu η^+(-1)} \bra{2π\iu Γ^P_- - l\int_{γ_-} Θ^P} \\
2π\iu \bra{ S Γ^P_- - Γ^P_+ }
&= l \bra{S \int_{γ_-} Θ^P - \int_{γ_+} Θ^P} \\
\frac{ n Γ^P_- - m Γ^P_+ }{m}
&= l \frac{n'}{m'} \\
m'(n Γ^P_- - m Γ^P_+)
&= lmn'.
\labelthis{eqn:period diff existence}
\end{align*}
If one were simply interested in getting a solution to this equation, one could take $l=m'$. However in Lemma~\ref{lem:minimal differentials} we will need the minimal solution to this equation. To avoid repeating ourselves, let us do the necessary extra work now. By considering the integer factorisation of each side, we see that a solution is only possible if $m'$ divides the right hand side. Therefore, $l$ must be at least
\[
l = \frac{m'}{\gcd(m',mn')}.
\]
We may then divide through by $m'$. To then solve this equation for $Γ^P_+$ and $Γ^P_-$, as $n$ and $m$ are coprime let $x$ and $y$ be integers such that $nx - my = 1$. By Bézout's Identity, the solution set is
\[
\Set{ (Γ^P_+, Γ^P_-) = \bra{\frac{mn'}{\gcd(m',mn')}y + mr, \frac{mn'}{\gcd(m',mn')}x + nr} }{ r \in \Z }.
\]
Therefore we may take
\[
Γ^P_+ = \frac{mn'}{\gcd(m',mn')}y,\qquad Γ^P_- = \frac{mn'}{\gcd(m',mn')}x,
\]
to obtain equality in~\eqref{eqn:period diff existence}. Hence, as for the exact differential, we may define
\[
b := \frac{1}{2\iu η^+(1)}\bra{ 2π\iu \frac{mn'}{\gcd(m',mn')}y - \frac{m'}{\gcd(m',mn')} \int_{γ_+} Θ^P }.
\labelthis{eqn:period differential scaling}
\]
Observe,
\begin{align*}
b
&= \frac{1}{2\iu η^+(1)}\bra{ 2π\iu \frac{mn'}{\gcd(m',mn')}y - \frac{m'}{\gcd(m',mn')} \int_{γ_+} Θ^P } \\
&= \frac{1}{-2\iu η^+(-1)}\frac{m}{n}\Bigg( 2π\iu \frac{mn'}{\gcd(m',mn')}\frac{nx-1}{m} \\
&\pushright{ - \frac{m'}{\gcd(m',mn')} \bra{\frac{n}{m}\int_{γ_-} Θ^P - 2π\iu \frac{n'}{m'} } \Bigg) }\\
%%%%%%%%%%%%%%%%%
% &= \frac{1}{-2\iu η^+(-1)}\Bigg( 2π\iu \frac{mn'}{\gcd(m',mn')}\frac{my}{n} - \frac{m'}{\gcd(m',mn')} \int_{γ_-} Θ^P \\
% &\pushright{ + 2π\iu \frac{m}{n}\frac{n'}{\gcd(m',mn')} \Bigg) } \\
%%%%%%%%%%%%%%%%%
% &= \frac{1}{-2\iu η^+(-1)}\Bigg( 2π\iu \frac{mn'}{\gcd(m',mn')}\frac{nx-1}{n} - \frac{m'}{\gcd(m',mn')} \int_{γ_-} Θ^P \\
% &\pushright{ + 2π\iu \frac{mn'}{n\gcd(m',mn')} \Bigg) } \\
%%%%%%%%%%%%%%%%%
&= \frac{1}{-2\iu η^+(-1)}\Bigg( 2π\iu \frac{mn'}{\gcd(m',mn')}x - \frac{m'}{\gcd(m',mn')} \int_{γ_-} Θ^P \\
&\pushright{ - 2π\iu \frac{mn'}{n\gcd(m',mn')} + 2π\iu \frac{mn'}{n\gcd(m',mn')} \Bigg) } \\
%%%%%%%%%%%%%%%%%
&= \frac{1}{-2\iu η^+(-1)}\bra{ 2π\iu \frac{mn'}{\gcd(m',mn')}x - \frac{m'}{\gcd(m',mn')} \int_{γ_-} Θ^P}.
\end{align*}
With these definitions of $b$ and $l$, it follows that $Ψ^P = b Θ^E + l Θ^P$ satisfy the closing conditions:
\begin{align*}
\int_{γ_+} Ψ^P
&= 2\iu η^+(1) b + l \int_{γ_+} Θ^P \\
&= 2π\iu \frac{mn'}{\gcd(m',mn')}y - \frac{m'}{\gcd(m',mn')} \int_{γ_+} Θ^P  + \frac{m'}{\gcd(m',mn')} \int_{γ_+} Θ^P \\
&= 2π\iu \frac{mn'}{\gcd(m',mn')}y \in 2π\iu \Z \\
\int_{γ_-} Ψ^P
&= -2\iu η^+(-1) b + l \int_{γ_-} Θ^P
% &= 2π\iu \frac{mn'}{\gcd(m',mn')}x - \frac{m'}{\gcd(m',mn')} \int_{γ_-} Θ^P + \frac{m'}{\gcd(m',mn')} \int_{γ_-} Θ^P \\
= 2π\iu \frac{mn'}{\gcd(m',mn')}x \in 2π\iu \Z.
\end{align*}
In summary, we have found real constants $a$ and $b$ and an integer $l$ such that $Ψ^E = aΘ^E$ and $Ψ^P = bΘ^E + lΘ^P$ satisfy the closing conditions.
\end{proof}
\end{lem}

Thus the space of genus one spectral curves that admit closing spectral data is defined by the equations $S(α,β) \in \Q^+$ and $T(α,β) \in \Q$ inside $\mathcal{A}_1$. For higher genus spectral curves, the closing conditions are difficult to work with, harder than even the period conditions. In the genus one case the integrals of $Θ^E$ lead to an algebraic expression, as we have just seen in~\eqref{eqn:def_S}, but the integrals of $Θ^P$ will lead to a transcendental conditions involving incomplete elliptic integrals.

Immediately before this lemma we observed that although the condition $T\in\Q$ is well defined, $T(α,β)$ is a multi-valued function on $\mathcal{A}_1$ which is dependent on the paths of integration. So that we may work with a well-defined function, we will make some branch cuts on $\mathcal{A}_1$ and choose a principal branch of $T$. On each curve we will have to make a choice of paths $γ_+$ and $γ_-$, and we shall refer to these choices as principal paths.

\labelpara{para:principal paths}

Consider the open dense subset $\mathcal{A}_1 \setminus \{ν = \pm 1\}$ of $\mathcal{A}_1$. On any marked curve $Σ(α,β)$ corresponding to a point of $\mathcal{A}_1 \setminus \{ν = \pm 1\}$, let $\symbf{γ}_+ = \symbf{γ}_+(α,β)$ be the path that begins at $(1,-η^+(1))$, traverses the unit circle to the point $μ$ without crossing $ν$, follows the branch circle to $α$, circles this branch point anticlockwise, goes back along the arc to the unit circle (though on a different sheet now), and back to $(1,η^+(1))$. This path is illustrated in Figure~\ref{fig:gamma paths}.

Likewise choose $\symbf{γ}_- = \symbf{γ}_-(α,β)$ to be the path from $(-1,-η^+(-1))$ to $(-1,η^+(-1))$ along the unit and branch circles that does not cross $ν$. In the case $ν=\pm 1$, it would be impossible to `avoid' $ν$, so this case had to be excluded. Note that these paths depend on the order of the branch points $(α,β) \in \mathcal{A}_1$, since $ν$ is defined to be the intersection of the unit and branch circles which lies between $β$ and $\cji{β}$.

\begin{defn}
\label{def:def_T_0}
The principal branch cut $T_0$ of $T$ is defined on $\mathcal{A}_1 \setminus \{ν = \pm 1\}$ to be
\[
T_0 (α,β) := T(α,β,\symbf{γ}_+,\symbf{γ}_-) = \frac{1}{2π\iu} \bra{S(α,β) \int_{\symbf{γ}_-} Θ^P - \int_{\symbf{γ}_+} Θ^P}.
\labelthis{eqn:def_T_0}
\]
\end{defn}

% \makefigure{$γ_+$ in red and $γ_-$ in black\label{fig:gamma paths}}{thesis_graphics_temp/gamma_pm_zeta.png}
\maketikzfigure{The principal path $\symbf{γ}_+$, in red, starts at $1$ on the lower sheet, traverses around $α$ and then returns to $1$ on the upper sheet. There is a branch cut between $α$ and $β$. Likewise, the principal path $\symbf{γ}_-$ is marked in blue. \label{fig:gamma paths}}{tikz/gamma_pm_zeta}

As $ν\neq 1$, we know that $f(1) \neq \infty$ and so $f(γ_+)$ and $f(γ_-)$ lie in the plane (they do not pass through $z=\infty$).
By design it is easy to describe these paths in terms of the $(z,w)$ coordinates.
For example, we may describe the path $f(γ_+)$ as follows. Start from the point $f(1)$ on the imaginary axis and go to the origin. Go out along the real axis, around $z=1$ (which corresponds to $ζ=α$) and back again to the origin. Return along the imaginary axis to $f(1)$. Both paths are illustrated in Figure~\ref{fig:f gamma paths}.

% \makefigure{$f(γ_+)$ in red and $f(γ_-)$ in black\label{fig:f gamma paths}}{thesis_graphics_temp/gamma_pm_z.png}
\makefigure{The paths $f(\symbf{γ}_+)$ in red and $f(\symbf{γ}_-)$ in blue. \label{fig:f gamma paths}}{tikz/gamma_pm_z}

Having fixed a choice of paths, we can express the integrals of the differentials $ω$ and $e$ along these particular paths in terms of Legendre elliptic integrals,
\begin{align*}
\int_{\symbf{γ}_+} ω
&= \bra{2\int_0^{f(1)} - 2\int_0^1} ω
= 2 F(f(1);k) - 2 K(k) \\
\int_{\symbf{γ}_+} e
&= 2 \tilde E(f(1);k) - 2 E(k),
\end{align*}
and so integrating~\eqref{eqn:thetaP} gives a formula for $Θ^P$ over $\symbf{γ}_+$,
\begin{multline*}
\int_{\symbf{γ}_+} Θ^P
= \int_{\symbf{γ}_+} (2Eω - 2Ke) - 2K \int_{\symbf{γ}_+} d\left[ \frac{(z-\iu\,\Imag z_0)w}{(z-z_0)(z + \bar{z}_0)} \right] \\
= 4 E(k) F(f(1);k) - 4 K(k) E(f(1);k) - 4K \frac{(f(1)-\iu\,\Imag z_0)\,w(f(1))}{(f(1)-z_0)(f(1) + \bar{z}_0)}.\labelthis{eqn:gamma_plus}
\end{multline*}
For the integral of $Θ^P$ over the path $\symbf{γ}_-$ for $ζ=-1$, we have similarly
\begin{multline*}
\int_{\symbf{γ}_-} Θ^P
= 4 E(k) F(f(-1);k) - 4 K(k) E(f(-1);k) \\
- 4K \frac{(f(-1)-\iu\,\Imag z_0)\,w(f(-1))}{(f(-1)-z_0)(f(-1) + \bar{z}_0)}.\labelthis{eqn:gamma_minus}
\end{multline*}
These two formulae may be substituted into~\eqref{eqn:def_T_0} to compute $T_0$. In the next section however, we will make a change of coordinates that simplifies these formulae.

Previously, the comment was made that the particular algorithm to chose a path is not valid when $ν = \pm 1$. Indeed, the result of this can be seen directly in the formulae we have derived. When $ν$ takes either of these values, then one of $f(1)$ or $f(-1)$ will be infinite. We also note that these integrals are purely imaginary, as we expected on theoretical grounds, because $f(1)$ and $f(-1)$ are purely imaginary, $F(z;k)$ and $E(z ;k)$ take the imaginary axis to itself and
\[
(f(1)-z_0)(f(1) + \bar{z}_0) = -(f(1)-z_0)(\overline{f(1) - z_0}) = - \abs{f(1) - z_0}^2.
\]
The function $T_0$ is therefore real valued.

To summarise our calculations up to this point, we determined that every differential on a marked curve $Σ$ that satisfies conditions~\ref{P:poles}--\ref{P:reality} must lie in a real three-dimensional vector space $W$. We found a basis $\{ω,Θ^E,\}$ of $W$, gave a basis $A,B$ for the homology of $Σ$ and computed the periods of the basis differentials. This allowed us to find a differential $Θ^P\in \R\{ω,λ\} \subset W$ that satisfied~\ref{P:periods}. We observed that every differential on $Σ$ that meet the conditions~\ref{P:poles}--\ref{P:periods} was the sum of a real multiple of $Θ^E$ and an integer multiple of $Θ^P$.

With these two differentials we then attempted to further satisfy~\ref{P:closing}. This was not possible for an arbitrary marked curve $Σ$, which lead us to define the functions $S(α,β)$ by~\eqref{eqn:def_S} and $T(α,β)$ by~\eqref{eqn:def_T}. Lemmata~\ref{lem:exist spectral data} and~\ref{lem:closing_conds} taken together imply that a marked curve admits spectral data exactly when the functions $S$ and $T$ simultaneously take rational values. The last part of this section noted that $T$ is a multi-valued function, and so took a principal branch of it and derived the corresponding explicit formulae.

% Properly understood, $T$ is not a function on the space of spectral curves $\mathcal{A}_1$, but rather it is a function on the space of paths $(γ_+, γ_-)$, which we shall call $\mathcal{P}$. To be more precise, let $\mathcal{P}$ be the space $\{(γ_+, γ_-, Σ, α)\}$ where $α$ is one of the branch points inside the unit circle of some genus one spectral curve $Σ$ and $γ_+$ and $γ_-$ are paths $(0,1) \to Σ$ defined in the following way. $γ_+$ starts at $(1, η^-(1))$, monotonically traverses the unit circle to $μ$ (that is, it does not pause or reverse direction), goes along the branch circle to $α$, encircles it, returns to $μ$ along the branch circle and retraces its path to $(1, η^+(1))$. In an analogous way, $γ_-$ starts at $(-1, η^-(-1))$ and finishes at $(-1, η^+(-1))$. We shall call such paths rigid. Every path on $Σ$ that connects the two points over $1$ or $-1$ is homologous to one of these rigid paths.
%
% $\mathcal{P}$ is naturally a $\Z^2$-bundle over the space $\mathcal{A}_1$. The projection map takes a tuple $(γ_+, γ_-, Σ, α)$ to the underlying spectral curve $(Σ,α)$. The fibre over any spectral curve $Σ$ is seen to be $\Z^2$, because if one fixes a point on the upper unit circle, not lying over $1,-1,μ$, the (signed) number of crossings of that point by a path uniquely determines that path. As an aside, the issue with points over $1,-1$ and $μ$ is that the paths may terminate there (in the obvious way for the first two, in the sense that the path may leave the unit circle in the latter case). One could probably introduce a convention to get around this, but it is not necessary to do so for our purposes.
%
% Viewed from this perspective, we consider $T$ to be a multi-valued function on $\mathcal{A}_1$ defined locally up the addition of multiples of $p$ and $1$, and $T_0$ is a principal branch cut. We call $\tilde{T}$ the well defined version that is defined on $\mathcal{P}$. $\mathcal{P}$ is actually a cover of the universal cover of $\mathcal{A}_1$ (implying $\mathcal{P}$ is disconnected), a fact that will become apparent once we have adopted coordinates better adapted to the situation.









\section{Coordinates for \texorpdfstring{$\mathcal{A}_1$}{A1}}
\label{sec:Reformulate}

In the previous section, we found a condition for a point of $\mathcal{A}_1$ to correspond a spectral curve, namely the functions $S$ and $T$ must be rationally valued. Thus on $\mathcal{S}_1$, the space of spectral curves, these functions must be (locally) constant. To understand $\mathcal{S}_1$ we should therefore be examining the joint level sets of $S$ and $T$. This will require invoking the implicit function theorem, for which the necessary computation will be the differentiation of the two functions.

It is therefore prudent to adopt a parametrisation of the space of marked curves $\mathcal{A}_1 \subset D\times D$ that is suited to the task of differentiating $T(α,β)$.
Elliptic integrals are the most difficult part of~\eqref{eqn:gamma_plus} to differentiate, so to minimise our labour we choose three coordinates to be $k$, $\iu u = f(1)$ and $\iu v = f(-1)$. For the final coordinate we shall take $p=S(α,β)$ itself, as then we can enforce the condition $S\in\Q^+$ simply by holding this coordinate constant.

However, these coordinates $(p,k,u,v)$ only cover part of $\mathcal{A}_1$, since $f$ is a Möbius function $\CP^1 \to \CP^1$ and so, for example, there will be points of $\mathcal{A}_1$ where $\iu u = f(1)$ is infinite. To cover these cases we must introduce the additional coordinates $u' = u^{-1}$ and $v' = v^{-1}$. The purpose of Lemma~\ref{lem:change of parameters} is that verify that these are in fact coordinates for $\mathcal{A}_1$ and that together they cover it.

After having established this coordinate change, the remainder of the section is devoted to calculations. First we rewrite the formulae derived thus far in terms of our new coordinates: equations~\eqref{eqn:gamma_plus2} and~\eqref{eqn:gamma_minus2} are rewrites of~\eqref{eqn:gamma_plus} and~\eqref{eqn:gamma_minus} respectively. From these it is feasible to compute the $u$-derivative of $T_0$,~\eqref{eqn:dTdu}, and in thereby in Lemma~\ref{lem:deriv no zeroes} we essentially prove that the derivative does not vanish.

\begin{lem}
\label{lem:change of parameters}
The following functions are diffeomorphisms:
\begin{align*}
φ_0 :\; &\mathcal{A}_1 \setminus \{ν = \pm 1\} \to \R^+ \times (0,1) \times \R \times \R \setminus \{u=v\}\\
&(α,β) \mapsto (p,k,u,v) := \bra{S(α,β), k(α,β), -\iu f(1), -\iu f(-1)}, \\
~\\
φ_1 :\; &\mathcal{A}_1 \setminus \{μ = 1 \text{ or } ν = -1\} \to \R^+ \times (0,1) \times \R \times \R \setminus \{u'v=1\}\\
&(α,β) \mapsto (p,k,u',v) := \bra{S(α,β), k(α,β), \iu f(1)^{-1}, -\iu f(-1)}, \\
~\\
φ_2 :\; &\mathcal{A}_1 \setminus \{μ = -1 \text{ or } ν = 1\} \to \R^+ \times (0,1) \times \R \times \R \setminus \{uv'=1\}\\
&(α,β) \mapsto (p,k,u,v') := \bra{S(α,β), k(α,β), -\iu f(1), \iu f(-1)^{-1}},
% φ_{--} :\; &\mathcal{A}_1 \setminus \{μ = -1, ν = -1\} \to \R^+ \times (0,1) \times \R \times \R\\
% &(α,β) \mapsto (p,k,u',v') = \bra{S(α,β), k(α,β), \iu f(1)^{-1}, \iu f(-1)^{-1}}, \\
\end{align*}
for the functions $S$ given by~\eqref{eqn:def_S}, $k$ given by~\eqref{eqn:def_k} and $f$ given by~\eqref{eqn:f}. Also, the union of the domains covers $\mathcal{A}_1$.

\begin{proof}
First note that the exclusions from the codomains are correct. Were, for example, $u$ and $v$ to be equal then
\[
f(1) = \iu u = \iu v = f(-1),
\]
but $f$ is an invertible transformation and this thus this would be a contradiction. Likewise for the codomains of other two functions.

Next, $S(α,β)$ is smooth since $α$ and $β$ are inside the unit disc. The other functions are smooth by Lemma~\ref{lem:coeff_f_smooth}. Hence the function $φ_0$ is smooth. The other two functions are necessary because the map $f$ is a Möbius transformation, and so takes the value infinity. Indeed, we have seen that $f(ν) = \infty$, and so one of $u$ or $v$ is infinite when $ν=\pm 1$. Using~\eqref{eqn:f}, on the subset of $\mathcal{A}_1$ where $μ\neq 1$
\[
-\iu u' = -\frac{1}{\bar{z}_0} \frac{1-ν}{1-μ},
\]
and where $μ\neq -1$
\[
-\iu v' = -\frac{1}{\bar{z}_0} \frac{1+ν}{1+μ},
\]
demonstrating that $φ_1$ and $φ_2$ are smooth functions on their respective domains of definition.

It remains to show that these functions have smooth inverses. As the parameters $α$ and $β$ are points in the $ζ$-plane, one method to derive the inverse functions is to express the transformation $f^{-1}(z)$ in term of our new parameters $(p,k,u,v)$. Then $α = f^{-1}(1)$ and $β = f^{-1}(k^{-1})$, entirely analogous to how the transformation $f(ζ)$ is determined by $(α,β)$ and the coordinate $u$ is defined by $\iu u = f(1)$. As a Möbius transformation is described, up to a scalar, by the points sent to $0$ and $\infty$, $f^{-1}(z)$ is a scalar multiple of
\[
\frac{z-z_0}{z + \bar{z_0}},
\]
(cf.~\eqref{eqn:f_inv}). Thus the construction of $f^{-1}$ proceeds in two steps; first find $z_0$, then determine the correct scaling factor. To find $z_0$, we will identify it as the intersection of two circles: one arising from $S$ and one arising from the geometry of the Möbius transformation $f$. In~\eqref{eqn:def_S}, the definition of $S$, there are two ratios. We may use the following trick using the cross ratio to write each ratio in terms of the new coordinates and $z_0$. Observe
\[
\abs{\frac{α-1}{α+1}}
= \abs{\frac{α-1}{α+1}} \abs{\frac{0+1}{0-1}}
= \abs{ \cross{α}{0}{1}{-1} }
= \abs{ \cross{1}{z_0}{\iu u}{\iu v} }
= \abs{\frac{1-\iu u}{1 - \iu v}} \abs{\frac{z_0 - \iu v}{z_0 - \iu u}}
\]
The same trick gives a similar formula for $β$.
\[
\abs{\frac{β-1}{β+1}}
= \abs{\frac{1-k\iu u}{1 - k\iu v}} \abs{\frac{z_0 - \iu v}{z_0 - \iu u}}
\]
We will show that together these imply that $z_0$ lies on a particular circle determined by the parameters $(p,k,u,v)$.
We have that
\begin{align*}
p = S(α,β)
&= \abs{\frac{α-1}{α+1}} \abs{\frac{β-1}{β+1}}
= \abs{\frac{1-\iu u}{1 - \iu v}}\abs{\frac{1 - k\iu u}{1 - k\iu v}} \abs{\frac{z_0 - \iu v}{z_0 - \iu u}}^2 \\
%%%%%%%%%%%%%%%%%%%%%%%
% \abs{\frac{z_0 - \iu v}{z_0 - \iu u}} ^2
% &= p \frac{\sqrt{1+v^2}}{\sqrt{1+u^2}}\frac{\sqrt{1+k^2v^2}}{\sqrt{1+k^2u^2}}
% = p \frac{w(\iu v)}{w(\iu u)} \\
%%%%%%%%%%%%%%%%%%%%%%%
\abs{z_0 - \iu v} ^2
&= p \frac{\sqrt{1+v^2}}{\sqrt{1+u^2}}\frac{\sqrt{1+k^2v^2}}{\sqrt{1+k^2u^2}} \abs{z_0 - \iu u}^2
= p \frac{w(\iu v)}{w(\iu u)} \abs{z_0 - \iu u}^2,
\end{align*}
where we recall that $w$ is defined by the relation $w^2 = (1-z^2)(1-k^2z^2)$.
% The function $w(\iu t) = \sqrt{(1+t^2)(1+k^2t^2)}$ should be taken with a positive square root on the second line above, since the left hand side is positive.
This equation for $z_0$ defines a circle. Explicitly, if we decompose $z_0 = x+\iu y$ then
\[
x^2 + y^2 + 2y \frac{puw(\iu v) - vw(\iu u)}{w(\iu u)-pw(\iu v)} + \frac{v^2w(\iu u) - pu^2w(\iu v)}{w(\iu u)-pw(\iu v)} = 0,
\]
which is centred on the imaginary axis.

On the other hand, in the $ζ$-plane the points $-1$, $0$, and $1$ all lie on a straight line that is perpendicular to the unit circle at both $-1$ and $1$, and that is invariant under the real involution $ρ$. Applying the Möbius transformation $f$ we can therefore say that $\iu v$, $z_0$, and $\iu u$ all lie on a circle that is perpendicular to the imaginary axis and symmetric under reflection in the imaginary axis. Therefore $z_0$ lies on the circle
\[
x^2 + \bra{ y - \frac{u+v}{2} }^2 = \frac{(u-v)^2}{4},
\]
which simplifies to the relation
\[
x^2 + y^2 = y(u+v) - uv. \labelthis{eqn:z_0_circle}
\]
Thus we have determined two circles that $z_0$ lies on. As these two circles are both centred on the imaginary axis, they intersect in two points: $z_0$ and $-\bar{z_0}$. We may solve for $z_0$, giving
\[
x = \frac{\sqrt{pw(\iu u)w(\iu v)}}{pw(\iu v) + w(\iu u)} \abs{u-v},\; y = \frac{puw(\iu v) + vw(\iu u)}{pw(\iu v) + w(\iu u)},
\labelthis{eqn:formula xy}
\]
where the sign of $x$ is chosen to make $z_0$ lie in the right half of the $z$-plane. This choice amounts to choosing the branch points $α$ and $β$ inside the unit circle. Note that these are smooth functions of $(p,k,u,v)$, because the term under the square root and the denominators are strictly positive functions, and $u-v$ is not zero by the definition of the codomain of $φ_0$.

Having found $z_0$ in terms of $(p,k,u,v)$ it remains to find the correct scaling of $f^{-1}$. We use the fact that $f^{-1}(\iu u) = 1$ and $f^{-1}(\iu v) = -1$ to conclude
\[
f^{-1}(z)
=  \frac{\iu u + \bar{z_0}}{\iu u - z_0} \frac{z-z_0}{z + \bar{z_0}}
=  -\frac{\iu v + \bar{z_0}}{\iu v - z_0} \frac{z-z_0}{z + \bar{z_0}}.
\]
As was previously presented, one can simply take $α = f^{-1}(1)$ and $β = f^{-1}(k^{-1})$ to give formulae for the branch points in terms of the new parameters. A problem could potentially occur if $z_0$ were to equal $\iu u$ or $\iu v$, in which case the scaling factor would be $0/0$. This could only occur if $\Real z_0 = 0$, which itself only occurs if $u=v$. But we have already noted that this is impossible. Likewise the formula would be ill-defined if $z_0 = -1$ or $-k^{-1}$ (for then $α$ or $β$ would be infinite), but again this occurs only if $\Real z_0 < 0$, which is excluded by our decision to take $z_0$ in the right half-plane.

Now that we have constructed an inverse for $φ_0$, we must also construct inverses for $φ_1$ and $φ_2$. But we may do so by modifying the formula for $x$ and $y$, given in equation~\eqref{eqn:formula xy}, to define them for the primed coordinates $u'$ or $v'$. Such a definition extends smoothly to the points where $u'$ or $v'$ is zero. Using the notation $w'(\iu t)^2 = (1+t^2)(k^2 + t^2)$ we have
\begin{align*}
x
&= \frac{\sqrt{pw(\iu u)w(\iu v)}}{pw(\iu v) + w(\iu u)} \abs{u-v}
= \frac{\sqrt{u^2 \times pw'(\iu u')w(\iu v)}}{pw(\iu v) + u^2 w'(\iu u')} \abs{u}\abs{1-u'v} \\
&= \frac{\sqrt{pw'(\iu u')w(\iu v)}}{pu'^2w(\iu v) + w'(\iu u')} \abs{1-u'v}.
\end{align*}
Likewise
\[
x
= \frac{\sqrt{pw(\iu u)w(\iu v')}}{pw'(\iu v') + v'^2w'(\iu u')} \abs{uv'-1},
\]
and
\[
y
= \frac{pu'w(\iu v) + vw'(\iu u')}{pu'^2w(\iu v) + w'(\iu u')}
= \frac{puw'(\iu v') + v'w(\iu u)}{pw'(\iu v') + v'^2w(\iu u)}. \labelthis{eqn:y_prime_version}
\]
Having made these changes in formula for $z_0$, the same formula for $f^{-1}$ applies without modification.

It is interesting to see that it was necessary to exclude the plane where $u=v$ (or $u'v=1$ or $uv'=1$), for otherwise $x$ would be zero, $z_0$ would be equal to $-\bar{z_0}$ and $f^{-1}(z)$ would be a constant function. We shall see later that these points correspond to the diagonal $\{α=β\} \subset D\times D$ and represent a degeneration of marked curves.

Finally, it remains to be demonstrated that the three domains cover $\mathcal{A}_1$. If there was some point $(α,β) \in \mathcal{A}_1$ that was not covered, then one could compute $μ(α,β)$ and $ν(α,β)$. But the intersection
\[
\{ν = \pm 1\}
\cap \{μ = 1 \text{ or } ν = -1 \}
\cap \{μ = -1 \text{ or } ν = 1 \} \subset \mathcal{A}_1
\]
consists of only those points where $μ=ν=1$ or $μ=ν=-1$, and Lemma~\ref{lem:coeff_f_smooth} proves that $μ$ and $ν$ are never equal.
\end{proof}
\end{lem}

Though standard, it is perhaps still of some interest to consider the above geometrical argument in the limit $u\to\infty$ to assure ourselves that nothing singular is happening. Suppose that $u' = 0$, which is to say geometrically that $1$ is mapped to infinity by $f$. Then the transformation $f$ takes the line through $1$, $0$, and $-1$ to a line perpendicular to the imaginary axis, cutting at $f(-1)$. This line is therefore horizontal and so $z_0$ and $\iu v$ have the same imaginary parts. This gives $y=v$ directly, as can be observed by setting $u'=0$ in~\eqref{eqn:y_prime_version}.

Now that we have establish that these are valid changes of coordinates, we can put them to work. First we will show how to compute $T_0$ in these coordinates, and then develop related coordinates for the universal cover $\mathcal{\tilde{C}}_1$. Recall that we defined a principal branch cut $T_0$ of $T$ on $\mathcal{A}_1\setminus\{ν=\pm 1\}$. This is exactly the domain of $φ_0$, and in equation~\eqref{eqn:formula xy} we have a formula for the inverse.

To rewrite $T_0$ in terms of $(p,k,u,v)$, we must compute the factors in the last terms of equations~\eqref{eqn:gamma_plus} and~\eqref{eqn:gamma_minus}. Note that in the previous lemma we have denoted $\Imag z_0$ by $y$, and that by definition $\iu u = f(1)$ and $\iu v = f(-1)$. Therefore, by direct computation
\begin{align*}
u-y &= \frac{w(\iu u)(u-v)}{pw(\iu v) + w(\iu u)} &
\abs{\iu u - z_0}^2 &= \frac{w(\iu u)(u-v)^2}{pw(\iu v) + w(\iu u)} \\
v-y &= - \frac{w(\iu v)(u-v)}{pw(\iu v) + w(\iu u)} &
\abs{\iu v - z_0}^2 &= \frac{w(\iu v)(u-v)^2}{pw(\iu v) + w(\iu u)},
\end{align*}
and thus
\begin{align*}
- 4K \frac{w(f(1))\,(f(1)-\iu\,\Imag z_0)}{(f(1)-z_0)(f(1) + \bar{z}_0)}
&= 4\iu K \frac{w(\iu u)\,(u-y)}{\abs{\iu u-z_0}^2}
= 4\iu K \frac{w(\iu u)}{u-v}, \\
- 4K \frac{w(f(-1))\,(f(-1)-\iu\,\Imag z_0)}{(f(-1)-z_0)(f(-1) + \bar{z}_0)}
&= 4\iu K \frac{w(\iu v)\,(v-y)}{\abs{\iu v-z_0}^2}
= - 4\iu K \frac{w(\iu v)}{u-v}.
\end{align*}
We may make these replacements in~\eqref{eqn:gamma_plus} and~\eqref{eqn:gamma_minus} to yield
\begin{align*}
\int_{\symbf{γ}_+} Θ^P
= 4 E(k) F(\iu u;k) &- 4 K(k) E(\iu u;k) + 4\iu K \frac{w(\iu u)}{u-v},
\labelthis{eqn:gamma_plus2}\\
%%%%%%%%%%%%%%%%%%%%%%%%%%
\int_{\symbf{γ}_-} Θ^P
= 4 E(k) F(\iu v;k) &- 4 K(k) E(\iu v;k) - 4\iu K \frac{w(\iu v)}{u-v}, \labelthis{eqn:gamma_minus2}
\end{align*}
and hence by the equation~\eqref{eqn:def_T} we finally arrive at
\begin{align*}
&2π\iu T_0(p,k,u,v) \\
&= 4p\left[ E F(\iu v;k) - K E(\iu v;k) - \iu K \frac{w(\iu v)}{u-v} \right]
- 4\left[ E F(\iu u;k) - K E(\iu u;k) + \iu K \frac{w(\iu u)}{u-v} \right] \\
&= 4p \left[ E F(\iu v;k) - K E(\iu v;k) \right] - 4\left[ E F(\iu u;k) - K E(\iu u;k) \right]
- 4\iu K \frac{p w(\iu v) + w(\iu u)}{u-v}.
\labelthis{eqn:Teqn}
\end{align*}
Be aware that we have omitted the elliptic modulus $k$ in the complete elliptic integrals, as is standard, so that $K=K(k)$ and $E= E(k)$.

The purpose for constructing these new coordinates was to make it feasible to compute the derivatives of $T$. While $T$ may be a multi-valued function on $\mathcal{A}_1$, its derivative is not. For a fixed value of $p$ the ambiguity present in $T$ is locally a constant, which is removed by differentiation. In particular then let us compute the $u$-derivative of $T_0$, a principal branch cut of $T$, from the explicit formula in~\eqref{eqn:Teqn}. The $v$-derivative is similar, but later we will employ a symmetry in $T_0$ to avoid the need to compute it directly. As $F(z;k)$ and $E(z;k)$ are parameter integrals in $z$, we have that
\[
\Partial{}{u} F(\iu u; k) = \frac{\iu}{w(\iu u)},\;\;\;
\Partial{}{u} E(\iu u; k) = \iu\frac{1+k^2 u^2}{w(\iu u)},
\]
and we recall the definition $w(\iu u) = \sqrt{(1+u^2)(1+k^2 u^2)}$, so it follows elementarily that
\[
\Partial{}{u} w(\iu u)
= \Partial{}{u} \sqrt{1+u^2}\sqrt{1+k^2 u^2}
= \frac{(1+k^2)u + 2k^2 u^3}{w(\iu u)}.
\]
Equipped with the derivatives of these factors, the calculation of the derivative of $T_0$ is mechanical if tedious.
% \begin{align*}\label{dTdk}
% \frac{π}{2}\Partial{T_0}{k}
% &= \frac{1}{k(1-k^2)}\frac{1}{u-v} \left[ p \sqrt { \frac{1+v^2}{1+k^2v^2}} + \sqrt { \frac{1+u^2}{1+k^2u^2} } \right] \left[ -(1+k^2uv) E + (1-k^2)K \right]
% \end{align*}
\[
\frac{π}{2}\Partial{T_0}{u}
= -\frac{E}{w(\iu u)} + \frac{pK w(\iu v)}{(u-v)^2} + \frac{K}{w(\iu u)(u-v)^2}\left[1 + u^2 - uv + v^2 + k^2 uv + k^2 u^2v^2 \right]
\labelthis{eqn:dTdu}
\]
% \begin{equation}\label{dTdv}
% \frac{π}{2}\Partial{T_0}{v}
% = \frac{pE}{w(\iu v)} - \frac{K w(\iu u)}{(u-v)^2} - \frac{pK}{w(\iu v)(u-v)^2}\left[1 + u^2 - uv + v^2 + k^2 uv + k^2 u^2v^2 \right]
% \end{equation}
% \begin{equation}\label{dTdu'}
% \frac{π}{2}\Partial{T}{u'}
% = \frac{E}{w'(u')} + \frac{pK w(\iu v)}{(1-u'v)^2} + \frac{K}{w'(u')(1-u'v)^2}\left[1 + k^2v^2 + - u'v + k^2u'v + (u')^2 + (u')^2v^2 \right]
% \end{equation}
% \begin{equation}\label{dTdv'}
% \frac{π}{2}\Partial{T}{v'}
% = -\frac{pE}{w'(v')} + \frac{K w(\iu u)}{(uv'-1)^2} + \frac{pK}{w'(v')(uv'-1)^2}\left[1 + k^2u^2 - uv' + k^2uv' + (v')^2 + u^2(v')^2 \right]
% \end{equation}
From this formula, we factor out the common denominator $[w(\iu u)(u-v)^2]^{-1}$ to define a function $L$,
\begin{multline*}
L(p,k,u,v) := -(u-v)^2 E + pKw(\iu u)w(\iu v) \\
+ K\left[ 1 + u^2 - uv + k^2 uv + v^2 + k^2 u^2 v^2 \right].
\labelthis{eqn:def L}
\end{multline*}
We shall need the value of the derivative `at infinity' too, and so let us also define
\begin{align*}
L'(p,k,v)
&:= \lim_{u\to\infty} u^{-2} L(p,k,u,v) \\
&= \lim_{u\to\infty} \Big(-(1-u^{-1}v)^2 E + pKw(\iu v) \cdot u^{-2}w(\iu u) \\
&\qquad+ K\left[ u^{-2} + 1 - u^{-1}v + k^2 u^{-1}v + u^{-2}v^2 + k^2 v^2 \right] \Big) \\
&= -E + pkKw(\iu v) + K\left[ 1 + k^2v^2 \right].
\labelthis{eqn:def L'}
\end{align*}
The final lemma of this section shows these functions $L$ and $L'$ are non-vanishing, which will be used to later prove that certain derivatives of $T$ are non-vanishing also. The lemma below is not completely sufficient to establish this latter fact by itself, because $T_0$ is only defined in the $(p,k,u,v)$ coordinate patch.

\begin{lem}
\label{lem:deriv no zeroes}
The functions $L$ and $L'$, defined by~\eqref{eqn:def L} and~\eqref{eqn:def L'} respectively are strictly positive for $p \geq 1$, $k\in (0,1)$, and $u,v \in \R$.

\begin{proof}
For this proof, we shall draw upon several inequalities that are explained in Section~\ref{sec:Inequalities}. The first step is to eliminate $E$. We apply the crude estimate that $K>E$, from~\eqref{eqn:K_E_values}, and also the assumption that $p\geq 1$ to simplify
\begin{align*}
L&(p,k,u,v)\\
&= -(u-v)^2 E + pKw(\iu u)w(\iu v) + K\left[ 1 + u^2 - uv + k^2 uv + v^2 + k^2 u^2 v^2 \right] \\
&> -(u-v)^2 K + Kw(\iu u)w(\iu v) + K\left[ 1 + u^2 - uv + k^2 uv + v^2 + k^2 u^2 v^2 \right] \\
&= K \left[ w(\iu u)w(\iu v) + 1 + (1 + k^2) uv + k^2 u^2 v^2 \right]
\end{align*}
This formula is almost sufficient. The only term that could be negative is the one featuring $uv$. However, a lower bound for the square root terms is
\[
w(\iu u) = \sqrt{1 + (1+k^2)u^2 + k^2u^4} > \sqrt{(1+k^2)u^2} = \sqrt{(1+k^2)}\abs{u},
\]
so applying this to both $w(\iu u)$ and $w(\iu v)$ gives
\begin{align*}
L(p,k,u,v)
&> K \left[ (1+k^2)\abs{uv} + 1 + (1 + k^2) uv + k^2 u^2 v^2 \right] \\
&\geq K \left[ 1 + k^2 u^2 v^2 \right].
\end{align*}
As $K > π/2$, this is strictly positive, as required. The proof of the positivity of the second function, $L'$, is almost immediate. Again using $K>E$,
\[
L'(p,k,v) > K \left[ -1 + pk w(\iu v) + 1 + k^2 v^2\right].
\]
This establishes that $L'$ is positive too.
\end{proof}
\end{lem}

% This result is of interest because it shows that if $p \leq 1$, then the $v$ and $v'$ derivatives of $T$ are nonzero:
% \begin{align*}
% \frac{π}{2}\Partial{T}{v} &= \frac{-p}{w(\iu v)(u-v)^2} U\bra{ \frac{1}{p},k,u,v }, \\
% \frac{π}{2}\Partial{T}{v'} &= \frac{p}{w'(v')(uv'-1)^2} V\bra{ \frac{1}{p},k,u,v' }.
% \end{align*}
% And if $p \geq 1$, then the $u$ and $u'$ derivatives of $T$ are nonzero:
% \begin{align*}
% \frac{π}{2}\Partial{T}{u} &= \frac{1}{w(\iu u)(u-v)^2} U\bra{ p,k,u,v }, \\
% \frac{π}{2}\Partial{T}{u'} &= \frac{-1}{w'(u')(1-u'v)^2} V\bra{ p,k,v,u' }.
% \end{align*}
























\section{The Topology of the Moduli Space}
\label{sec:Topology}

The purpose of this chapter overall is to describe the set of genus one spectral curves $\mathcal{S}_1$. The coordinates constructed in the previous chapter are apt for computation, but because $T$ is a multi-valued function any work on its level sets in $\mathcal{A}_1$ will be limited to local results. The way forward is to transition to the universal cover $\mathcal{\tilde{C}}_1$ of $\mathcal{A}_1$, which is covered by coordinates $(p,k,\tilde{u},\tilde{v})$. The universal cover allows us to pull back the multi-valued function $T$ to a single valued function $\tilde{T}$. This global function $\tilde{T}$ will allow us to gain global results about $\mathcal{S}_1$.

More precisely,
in Lemma~\ref{lem:T_graph} we demonstrate that for any values $p \in \R^+$, $q\in\R$ the level set $\mathcal{\tilde{C}}_1(p,q)$ defined by $p = S$ and $q = \tilde{T}$ is a graph over two of the coordinates. This follows by an application of the implicit function theorem and relies on the non-vanishing result from Lemma~\ref{lem:deriv no zeroes}. Hence each level set is diffeomorphic to a ribbon $(0,1)\times\R$.
% Moreover, that lemma shows that for fixed $p$ these strips $\mathcal{\tilde{C}}_1(p,q)$ foliate $\mathcal{\tilde{C}}_1(p)$, the subspace of $\mathcal{\tilde{C}}_1$ on which $p$ is a fixed constant.

It will follow from the observation that $\tilde{T}$ is rational exactly when $T$ is that the preimage $\mathcal{\tilde{S}}_1$ of $\mathcal{S}_1$ in the universal cover $\mathcal{\tilde{C}}_1$ may be written as the disjoint union of level sets,
\[
\mathcal{\tilde{S}}_1 = \coprod_{p\in \Q^+,\; q\in \Q} \mathcal{\tilde{C}}_1(p,q).
\]
The second half of this section seeks to recover $\mathcal{S}_1$ from $\mathcal{\tilde{S}}_1$. To do so, we investigate the action of the group $\mathcal{G}$, the covering transformations of $\mathcal{\tilde{C}}_1$ over $\mathcal{C}_1$, on these level sets. This culminates in Theorems~\ref{thm:topology_curves} and~\ref{thm:topology_curves_p1}, wherein we take the quotient of $\mathcal{\tilde{S}}_1$ by this group and thereby enumerate the path connected components of the moduli space $\mathcal{S}_1$ of genus one spectral curves and describe the topology of each component.




To motivate the definition of the universal cover in Lemma~\ref{lem:mathcal tilde C}, and to provide context for Figures~\ref{fig:p05 plot}--\ref{fig:p2 plot}, we will deduce the topology of $\mathcal{A}_1$ from the coordinates $(p,k,u,v)$ introduced in the previous section.
These coordinates embed $\mathcal{A}_1$ into $\R^+\times(0,1)\times \mathbb{T}^2$ in the following way.
The first two factors $\R^+ \times (0,1)$ are obvious, they come from $p \in R^+$ and $k\in (0,1)$. To see the torus part, consider
\[
\{(u,v) \in \R \times \R \} \cup
\{(u',v) \in \R \times \R \} \cup
\{(u,v') \in \R \times \R \} \cup
\{(u',v') \in \R \times \R \}
\]
with the identifications $u' = u^{-1}$ and $v'=v^{-1}$. This is the product of two circles, which is to say a torus $\mathbb{T}^2$. Specifically $\mathcal{A}_1$ is the subset of this torus where $u\neq v$. The line $u=v$ can be represented as the line where the toroidal and poloidal angles are equal, and removing this line leaves an annulus. Moreover, consider the subsets $\mathcal{A}_1(p)$ of the parameter space $\mathcal{A}_1$ for which the coordinate $p$ is fixed. This shows that topologically it is the product of an interval and an annulus, a feature not as easily seen from the $(α,β)$ description.

A more instructive way of visualising $\mathcal{A}_1(p)$ is to think of it as a solid cylinder with a line along the central axis removed. One should think of the `radius' of point being given by $1-k$, so that the central axis is identified with the value $k=1$. To motivate this, consider formula~\eqref{eqn:def_k} for $k$.
\[
k = \frac{\abs{1-\bar{α}β}-\abs{α-β}}{\abs{1-\bar{α}β}+\abs{α-β}}.
\]
In the limit as $α \to β$, this formula says that $k \to 1$. From the equation of $S$, the subspace of $D\times D$ where $α=β$ and $S(α,β) = p$ is an arc. In this visualisation we are imagining this arc as the central axis of the cylinder. In Section~\ref{sec:Interior}, the interesting structure of the moduli space in this limit will be investigated.

The fact that the parameter space is not simply connected is fundamentally tied to the fact that $T$ is not a single valued function. We have defined a principal branch cut $T_0$ of $T$ and given a formula, but the more natural way to correct this deficiency is to move to the universal cover. By constructing a lift $\tilde{T}$ of $T$ which is single valued, we will be able to treat the level sets $T\in\Q$ globally and thereby acquire complete description of the topology of the space $\mathcal{S}_1$, significantly more than Theorem~\ref{thm:moduli manifold} which is a local result showing it to be a surface.

\begin{lem}
    \label{lem:mathcal tilde C}
The universal cover of $\mathcal{A}_1$ is
\[
\mathcal{\tilde{C}}_1 =
\{(p, k,\tilde{u},\tilde{v}) \in \R^+\times(0,1)\times\R\times\R \mid  \tilde{u} < \tilde{v} < \tilde{u} + 2π \},
\]
with the projection map $\tilde{π} : \mathcal{\tilde{C}}_1 \to \mathcal{A}_1$ is given by
\begin{align*}
    p &= p, \\
    k &= k, \\
    u = \tan \frac{\tilde{u}}{2},       &\quad
        u' = \cot \frac{\tilde{u}}{2},  \\
    v = \tan \frac{\tilde{v}}{2},       &\quad
        v' = \cot \frac{\tilde{v}}{2}.
\end{align*}

\begin{proof}
The justification of this definition of $\mathcal{\tilde{C}}_1$ precedes in two steps. First, we have already observed that $\mathcal{A}_1$ embeds into $\R^+ \times (0,1) \times \mathbb{T}^2$. The universal cover of this larger space is $\R^+\times(0,1)\times\R^2$, with the covering map given above using the standard $2π$-periodic $\tan$ mapping of $\R$ to $\S^1$. The second step is to recall that $\mathcal{A}_1$ is the complement of the hyperplane $u-v = 0$. When pulled back to the universal cover, this hyperplane becomes a collection of hyperplanes $\tilde{u}-\tilde{v} \in 2π\Z$. Thus we may take a simply connected region of the complement $\tilde{u}-\tilde{v} \not\in 2π\Z$ to cover $\mathcal{A}_1$ and this is $\mathcal{\tilde{C}}_1$ above.
\end{proof}
\end{lem}

It is straightforward to lift $T_0$, defined on $\mathcal{A}_1\setminus\{ν = \pm 1\} \subset \mathcal{A}_1$, to a single valued function $\tilde{T}$ on $\mathcal{\tilde{C}}_1$. Recall the definitions of $F_0$ and $E_0$ from~\ref{defn:F0 and E0}:
\[
F_0(x ;k) = \Imag F(\iu x; k), \qquad
E_0(x ;k) = \Imag F(\iu x; k) - kx.
\]
Using these, we rewrite $T_0$ in the following way.
\begin{align*}
2π T_0(p,k,u,v) =
4p& \left[ E F_0(v; k) - K E_0(v; k) \right]
-4 \left[ E F_0(u; k) - K E_0(u; k) \right] \\
&- 4 K \left[p\left(\frac{w(\iu v)}{u-v} + kv \right) + \left(\frac{w(\iu u)}{u-v} - ku \right) \right] .
\labelthis{eqn:T0 rewrite}
\end{align*}

In Section~\ref{sec:EllipticContinuation}, analytic extensions of $F_0(x;k)$ and $E_0(x;k)$ are constructed. They are denoted respectively as $\tilde{F}$ and $\tilde{E}$. Thus the first two brackets of $T_0$ in~\eqref{eqn:T0 rewrite} can be lifted to the universal cover by replacing $F_0$ and $E_0$ with their extensions. The following lemma will establish that the third bracket is analytic, and so lifts to the universal cover without the need for modification at all. Therefore we define
\begin{align*}
2π \tilde{T}(p,k,\tilde{u},\tilde{v})
:= 4p &\left[ E \tilde{F}(\tilde{v};k) - K \tilde{E}(\tilde{v};k) \right]
- 4 \left[ E \tilde{F}(\tilde{u};k) - K \tilde{E}(\tilde{u};k) \right] \\
&- 4 K \left[p\left(\frac{w(\iu v)}{u-v} + kv \right)
+ \left(\frac{w(\iu u)}{u-v} - ku \right) \right] .
\end{align*}

\begin{lem}
The function
\[
\frac{w(\iu u)}{u-v} - ku
\]
defined on $\mathcal{A}_1 \setminus \{ν = \pm 1\}$ extends to an analytic function on $\mathcal{A}_1$.

\begin{proof}
As $u-v \neq 0$ on $\mathcal{A}_1 \setminus \{ν = \pm 1\}$, this is an analytic function of the coordinates $(p,k,u,v)$. It remains to show that it is similarly analytic in the other coordinates required to cover $\mathcal{A}_1$. Firstly, examining this function when using the coordinate $v'$ gives
\[
\frac{w(\iu u)}{u-v} - ku = \frac{w(\iu u)v'}{uv'-1} - ku,
\]
which is analytic. Next, when using the coordinate $u'$ we have
\begin{align}
\frac{w(\iu u)}{u-v} - k u
&= \frac{w(\iu u) - ku^2}{u-v} + \frac{kuv}{u-v} \\
&= \frac{1 + (1+k^2)u^2}{(u-v) (w(\iu u) + ku^2)} + \frac{kv}{1-u'v} \\
&= \frac{u'((u')^2 + (1+k^2))} {(1-u'v) (w'(\iu u') + k)} + \frac{kv}{1-u'v},
\end{align}
where we have again used the auxiliary function $w'(\iu t)^2 = (1+t^2)(k^2 + t^2)$. This is a sum of analytic functions of $u'$. As these three coordinate patches cover all of $\mathcal{A}_1$ we are done.
\end{proof}
\end{lem}

To actually compute the value of the function $\tilde{T}$, it is simply a matter of substituting the correct expression for $\tilde{F}$ or $\tilde{E}$.
If we define the winding number $\Wind : \R \to \Z$ of a number $x$ to be the integer $\Wind(x)$ such that $-π < x - 2π\Wind(x) < π$, then recall~\eqref{eqn:tildeF_period} and~\eqref{eqn:tildeE_period},
\begin{align*}
\tilde{F}(\tilde{x};k) &= 2\Wind(\tilde{x})K' + F_0\bra{\tan \frac{\tilde{x}}{2};k}, \\
\tilde{E}(\tilde{x}; k) &= 2\Wind(\tilde{x})(K'-E') + E_0\bra{\tan \frac{\tilde{x}}{2};k} .
\end{align*}
Then
\begin{align*}
2π \tilde{T}(p,k,\tilde{u},\tilde{v})
%%%%%%%%%%%%%%%%%%%%%%%%
% &= 4p \left[ E (2K'W(\tilde{v})+F_0(v)) - K (2(K'-E')W(\tilde{v})+E_0(v)) \right] \\
% &\quad - 4 \left[ E (2K'W(\tilde{u})+F_0(u)) - K (2(K'-E')W(\tilde{u})+E_0(u)) \right]
% - 4 K \left[p\left\{\frac{w(\iu v)}{u-v} + kv \right\}
% + \left\{\frac{w(\iu u)}{u-v} - ku \right\} \right] \\
%%%%%%%%%%%%%%%%%%%%%%%%
&= 2π T_0(p,k,u,v) + 4p \left[ 2EK' - 2K(K'-E') \right]\Wind(\tilde{v}) \\
&\qquad - 4 \left[ 2EK' - 2K(K'-E') \right]\Wind(\tilde{u}) \\
%%%%%%%%%%%%%%%%%%%%%%%%
\tilde{T}(p,k,\tilde{u},\tilde{v})
&= T_0(p,k,u,v) + 2\bra{ p\Wind(\tilde{v}) - \Wind(\tilde{u}) },
\labelthis{eqn:tilde T computable}
\end{align*}
using Legendre's relation (see Section~\ref{sec:Legendre's Relation}).
Thus to do computations with the function $\tilde{T}$, for the most part one can continue to work with the function $T_0$ downstairs on $\mathcal{A}_1$ and keep track of the winding numbers. For example, the following lemma uses this fact to motivate us looking at the level sets of $\tilde{T}$ as a proxy for looking at the level sets of $T$.
\begin{lem}
\label{lem:tilde T rational}
If $p$ is rational, $T\circ\tilde{π} \in \Q$ if and only if $\tilde{T} \in \Q$.
\hfill\qedsymbol
\end{lem}

This relationship between $\tilde{T}$ and $T_0$ is also used in the next lemma to show that the range of $\tilde{T}$ is $\R$, which is used in Lemma~\ref{lem:T_graph}.
\begin{lem}
\label{lem:range_T}
The range of $\tilde{T}$ on $\mathcal{\tilde{C}}_1$ is $\R$.

\begin{proof}
We shall prove below that the range of $T_0$ on $\mathcal{\tilde{C}}_1 \setminus \{ν = \pm 1\}$ is $\R$. Because on each component of the preimage of $\mathcal{\tilde{C}}_1 \setminus \{ν = \pm 1\}$ in $\mathcal{\tilde{C}}_1$ equation~\eqref{eqn:tilde T computable} tells us that $\tilde{T}$ and $T_0\circ \tilde{π}$ differ by a constant, it follows that the range of $\tilde{T}$ is also $\R$.

Fix $p$, but also fix any value for $k$. By~\eqref{eqn:F0_E0_uniform_bounds}, the magnitude $\abs{F_0(x;k)}$ is bounded by $K'$ and $\abs{E_0(x;k)}$ is bounded by $K'-E'$. Thus there is some constant, dependent on both $p$ and $k$ but independent of $u$ and $v$, such that
\[
-C \leq 4p \left[ E F_0(v) - K E_0(v) \right]-4 \left[ E F_0(u) - K E_0(u) \right] \leq C.
\]
Substituting this into~\eqref{eqn:T0 rewrite},the definition of $T_0$,
\begin{gather*}
- 4 K \left[\frac{pw(\iu v) + w(\iu u)}{u-v} + k(pv-u) \right] - C \\
\leq
2π T_0(p,k,u,v)
\leq \\
- 4 K \left[\frac{pw(\iu v) + w(\iu u)}{u-v} + k(pv-u) \right] + C,
\end{gather*}
and so it is sufficient to show that for any fixed $p$ and $k$ the bracketed expression has range equal to the real line. But this is easy to show. Consider the limit as $u \to v^+$,
\[
\lim_{u \to v^+} \frac{pw(\iu v) + w(\iu u)}{u-v} + k(pv-u)
= k(p-1)v + (p+1) w(\iu v)\lim_{u \to v^+} \frac{1}{u-v} = +\infty.
\]
From the other side,
\[
\lim_{u \to v^-} \frac{pw(\iu v) + w(\iu u)}{u-v} + k(pv-u)
= k(p-1)v + (p+1) w(\iu v)\lim_{u \to v^-} \frac{1}{u-v} = -\infty.
\]
By continuity $T_0$ obtains every value.
\end{proof}
\end{lem}









Let us make two definitions. Define $\mathcal{\tilde{C}}_1(p)$ to be the subspace of $\mathcal{\tilde{C}}_1$ on which $S \circ \tilde{π} = p$ is a fixed constant. We further denote the subset of $\mathcal{\tilde{C}}_1(p)$ on which the value of $\tilde{T}$ is $q$ by $\mathcal{\tilde{C}}_1(p,q)$.

The spaces $\mathcal{\tilde{C}}_1(p)$ are simple to understand. They are just the points of $\mathcal{\tilde{C}}_1$ where the first coordinate is fixed, and so they are simply covered by the three remaining coordinates $(k,\tilde{u},\tilde{v}) \in (0,1)\times\R\times\R$. As demonstrated by the following lemma, the level sets $\mathcal{\tilde{C}}_1(p,q)$ are similarly well behaved.





\begin{lem}
\label{lem:T_graph}
If $p \leq 1$ then there is a diffeomorphism between $\tilde{\mathcal{A}_1}(p)$ and
\[
\{(q,k,\tilde{u}) \in \R\times(0,1)\times\R \},
\]
such that fixing a value $q$ gives the level set $\mathcal{\tilde{C}}_1(p,q)$, on which $S\circ \tilde{π} = p$ and $\tilde{T} = q$. Likewise, if $p \geq 1$ then there is a diffeomorphism between $\tilde{\mathcal{A}_1}(p)$ and
\[
\{(q,k,\tilde{v}) \in \R\times(0,1)\times\R \},
\]
such that again fixing a value $q$ gives the level set $\mathcal{\tilde{C}}_1(p,q)$.
In either case, the level sets are ribbons $(0,1)\times \R$.

\begin{proof}
Fix a value of $p$ and consider the function $G(q, k,\tilde{u},\tilde{v}) = \tilde{T}(p,k,\tilde{u},\tilde{v}) - q$ on $\mathcal{\tilde{C}}_1(p)\times\R$. $G^{-1}(0)$ is a graph over $\mathcal{\tilde{C}}_1(p)$ given by $q=\tilde{T}$, so they are diffeomorphic. We will apply the implicit function theorem to show that $G^{-1}(0)$ can also be written as a graph over either $(q,k,\tilde{u})$ or $(q,k,\tilde{v})$, depending on the magnitude of $p$.

Suppose first that the fixed value of $p$ is greater than or equal to one. We compute the following formula for the derivative of $G$ with respect to $\tilde{u}$.
\begin{align*}
\Partial{G}{\tilde{u}}
= \Partial{\tilde{T}}{\tilde{u}}
&= \frac{du}{d\tilde{u}}\Partial{T_0}{u} \\
&= \frac{1}{2}\sec^2\bfrac{\tilde{u}}{2} \Partial{T_0}{u} \\
&= \frac{1}{2}\sec^2\bfrac{\tilde{u}}{2} \times \frac{2}{π} \times \frac{1}{w(\iu u)(u-v)^2} L\bra{ p,k,u,v } \\
&= \frac{1}{π}\frac{1 + u^2}{w(\iu u)(u-v)^2} L\bra{ p,k,u,v },
\end{align*}
which holds for $\tilde{u} \not\in π + 2π\Z$, using~\eqref{eqn:def L}. As witnessed in Lemma~\ref{lem:deriv no zeroes}, $L$ is never zero, and neither are the other three factors present. Hence $\partial \tilde{G} / \partial \tilde{u}$ is never zero on this open set.


It remains to check it does not vanish when $\tilde{u} \in π + 2π\Z$. Recall the definition of $L'$ from~\eqref{eqn:def L'},
\[
L'(p,k,v) :=
\lim_{u\to\infty} u^{-2} L(p,k,u,v).
\]
As $\tilde{T}$ is an analytic function its derivatives are continuous and so we may compute their value at these points by taking a limit. Therefore
\begin{align*}
\lim_{\tilde{u}\to π + 2π\Z} \Partial{G}{\tilde{u}}
&=\frac{1}{π} \lim_{u \to \infty} \frac{1 + u^2}{w(\iu u)(u-v)^2} L\bra{ p,k,u,v } \\
&=\frac{1}{π} \lim_{u \to \infty} \frac{(1 + u^2)u^2}{w(\iu u)(u-v)^2} \times u^{-2}L\bra{ p,k,u,v } \\
&=\frac{1}{π} \frac{1}{k} L'\bra{ p,k,v },
\end{align*}
which is also nonzero by Lemma~\ref{lem:deriv no zeroes}. The implicit function theorem states that there is a function $h$ such that $G^{-1}(0)$ is a graph of the form $\{ (q, k, \tilde{u}, h(q,k,\tilde{u})) \mid q \in \text{Range } \tilde{T}, k \in (0,1), \tilde{u}\in\R \}$. By Lemma~\ref{lem:range_T}, the range of $\tilde{T}$ is $\R$. Finally, if we hold $q$ fixed, then the level set $\mathcal{\tilde{C}}_1(p,q)$ is parametrised by the remaining two coordinates $(k,\tilde{u})$.

When $p \leq 1$, we employ the following symmetry.
\begin{align*}
T_0&(p,k,u,v) \\
&= 4p \left[ E F(\iu v;k) - K E(\iu v;k) \right] - 4\left[ E F(\iu u;k) - K E(\iu u;k) \right] \\
&\hspace{8cm} - 4\iu K \frac{p w(\iu v) + w(\iu u)}{u-v} \\
&= -p\Big( -4\left[ E F(\iu v;k) - K E(\iu v;k) \right] + \frac{4}{p}\left[ E F(\iu u;k) - K E(\iu u;k) \right] \\
&\hspace{8cm}+ 4\iu K \frac{ w(\iu v)+ \frac{1}{p}w(\iu u)}{u-v} \Big)\\
&= -p T_0\bra{ \tfrac{1}{p}, k, v, u },
\labelthis{eqn:T_0 symmetry}
\end{align*}
so that it is now the $\tilde{v}$ derivative of $\tilde{T}$ that is non-vanishing. Again the implicit function theorem gives the result.
\end{proof}
\end{lem}





With this lemma in hand, we are within striking distance of results about $\mathcal{S}_1$. Recall that $\mathcal{S}_1 \subset \mathcal{C}_1$ is the set of spectral curves, those marked curves that admit spectral data. We will recover $\mathcal{S}_1$ as the quotient of the level sets of $\tilde{T}$ by the group $\mathcal{G}$ of covering transformations $\mathcal{\tilde{C}}_1$ over $\mathcal{C}_1$.

Define $\mathcal{\tilde{S}}_1 \subset \mathcal{\tilde{C}}_1$ to be the preimage of $\mathcal{S}_1$ in the universal cover. As we have shown that $\mathcal{S}_1$ is the subspace of $\mathcal{C}_1$ on which $S$ is a positive rational and $T$ is any rational (Lemma~\ref{lem:closing_conds}) and that a point in $\mathcal{\tilde{C}}_1$ is the preimage of a point of $\mathcal{S}_1$ exactly when $p$ is a positive rational and $\tilde{T} \in \Q$ (Lemma~\ref{lem:tilde T rational}), it follows that
\[
\mathcal{\tilde{S}}_1 = \coprod_{p\in \Q^+,\; q\in \Q} \mathcal{\tilde{C}}_1(p,q).
\labelthis{eqn:def tilde S}
\]

There are several topological implications of the previous lemma. First, each level set $\mathcal{\tilde{C}}_1(p,q)$ is connected. More precisely each of these is diffeomorphic to a product of $(0,1)$ and $\R$. Therefore the above disjoint union is the decomposition of $\mathcal{\tilde{S}}_1$ into its path connected components. By varying the value of $q$, we see that the $\mathcal{\tilde{C}}_1(p,q)$ foliate $\mathcal{\tilde{C}}_1(p)$, so $\mathcal{\tilde{S}}_1$ is arranged densely in $\mathcal{\tilde{C}}_1(p)$, analogously to how $\Q$ is arranged densely in $\R$.

To understand the topology of $\mathcal{S}_1$, we must understand the action of the covering transformations of $\mathcal{\tilde{C}}_1 \to \mathcal{A}_1 \to \mathcal{C}_1$ as restricted to $\tilde{\mathcal{S}}_1$, and so we must first describe the group of covering transformations $\mathcal{G}$. The covering transformations of $\mathcal{A}_1$ over $\mathcal{C}_1$ are easy to understand; besides the identity there is only
\[
% ZUUL
λ : (α,β) \mapsto (β,α),
\]
which swaps the labelling of the branch points inside the unit disc. We will use $λ$ as a stepping stone to $\mathcal{G}$.

% There are two types of transformations to understand, the transformations of $\mathcal{\tilde{C}}_1$ over $\mathcal{A}_1$ and the transformations of the two fold covering of $\mathcal{A}_1$ over $\mathcal{C}_1$. The first sort are easy to understand; they are generated by the translation
% \[
% (p, k,\tilde{u},\tilde{v}) \mapsto (p, k, \tilde{u} + 2π, \tilde{v} + 2π).
% \]
% The second sort require a little more effort. On one hand, the deck transformations of $\mathcal{A}_1$ over $\mathcal{C}_1$ is simply the involution that swaps $α$ and $β$.

\labelpara{para:def f_s}

Let us describe $λ$ in terms of the $(p,k,u,v)$ coordinates.
By inspection of the definitions of $p = S(α,β)$ and $k = k(α,β)$, equations \eqref{eqn:def_S} and \eqref{eqn:def_k}, they are unchanged if the labelling of the branch points is exchanged. We therefore must say what happens to $u$ and $v$. Recall that $\iu u = f(1)$ and $\iu v = f(-1)$. Our construction thus far relies on a definition of $f$ that sends $α$ to $1$ and $β$ to $k^{-1}$. Let $f_s$ be the Möbius transformation which instead standardises the branch points of the marked curve in the other way, taking $β$ to $1$ and $α$ to $k^{-1}$. Equivalently, $f_s$ is the result of first swapping the order of branch points $(α,β)$ and then applying $f$. Consider the composition of $f_s \circ f^{-1}$. It is a Möbius transformation that exchanges $1$ and $k^{-1}$ and also $-1$ and $-k^{-1}$. It therefore must be the map
\[
z \mapsto \frac{1}{kz}.
\]
Under the map $f_s \circ f^{-1}$ the point $\iu u$ is taken to $-\iu (ku)^{-1}$ and $\iu v$ is taken to $-\iu (kv)^{-1}$. Thus, under the label-swapping involution $λ$,
\[
λ: (p,k,u,v) \mapsto \bra{ p,k, -(ku)^{-1}, -(kv)^{-1} }.
\labelthis{eqn:def_lambda}
\]
To gain a geometric understanding of $λ$, we imagine should again consider $u$ and $u' = u^{-1}$ as coordinates on $\RP^1$. The action of $λ$ above on $u$ is then a distorted rotation of the circle. To see this clearly, we may rescale $u$ in the following manner. Let $U=\sqrt{k} u$ and $U' = \tfrac{1}{\sqrt{k}} u'$. Then
\[
U
\mapsto \sqrt{k} λ\bra{\frac{U}{\sqrt{k}} }
% = -\sqrt{k} \frac{1}{k}\frac{1}{\frac{U}{\sqrt{k}}}
= - \frac{1}{U} = -U',
\]
which is half-rotation of the circle (as shown in Figure~\ref{fig:circle rotation}).

\maketikzfigure{The circle on the left is $\RP^1$ with four points marked. The map $U \mapsto -U^{-1}$ is applied to produce the circle on the right, with corresponding points coloured the same. We can see that the circle on the right is rotated by half. \label{fig:circle rotation}}{tikz/circle_rotation}

We shall use this rescaled coordinate $U$, and likewise $V = \sqrt{k} v$, to construct coordinates on $\mathcal{\tilde{C}}_1$ such that the covering transformations are simply translations. This is motivated by the fact that translating the line $\R$ rotates the quotient $\R/\Z$.

In analogy to Lemma~\ref{lem:mathcal tilde C}, we define coordinates $\tilde{U}$ and $\tilde{V}$ on $\mathcal{\tilde{C}}_1$ to be
\begin{align*}
U = \tan \frac{\tilde{U}}{2},       &\quad
    U' = \cot \frac{\tilde{U}}{2},  \\
V = \tan \frac{\tilde{V}}{2},       &\quad
    V' = \cot \frac{\tilde{V}}{2}.
\end{align*}
Using $U = \sqrt{k}u$, the change of coordinates from $\tilde{u}$ to $\tilde{U}$ is
\begin{align*}
    \tan \frac{\tilde{U}}{2} &= U = \sqrt{k} u = \sqrt{k} \tan \frac{\tilde{u}}{2} \\
    \tilde{U} &= 2π\Wind(\tilde{u}) + 2 \atan \left[ \sqrt{k} \tan \frac{\tilde{u}}{2} \right],
\end{align*}
and similarly
\[
\tilde{V} = 2π\Wind(\tilde{v}) + 2 \atan \left[ \sqrt{k} \tan \frac{\tilde{v}}{2} \right].
\]
Observe that these formula fix certain points, namely multiples of $\Z$.
Hence $\tilde{u}$ and $\tilde{U}$ have the same winding number.
Recall also from Lemma~\ref{lem:mathcal tilde C} that the range of the coordinates $\tilde{u}$ and $\tilde{v}$ is restricted to $\tilde{u} < \tilde{v} < \tilde{u} + 2π$.
Since $\tan$ and $\atan$ are increasing functions, it follows that the transformation is order preserving, so we may derive that $\tilde{U} < \tilde{V} < \tilde{V} + 2π$ as well.

These coordinates $\tilde{U}$ and $\tilde{V}$ can be used interchangeably with $\tilde{u}$ and $\tilde{v}$. Consider for example the coordinates $(p,k,\tilde{u},\tilde{v})$ on $\mathcal{\tilde{C}}_1$ defined in Lemma~\ref{lem:mathcal tilde C}.
The determinant of the Jacobian of the change of coordinates to $(p,k,\tilde{U},\tilde{V})$ is
\[
\det \mathrm{Jac}\,
= \begin{vmatrix}
1 & 0 & 0 & 0\\
0 & 1 & 0 & 0 \\
0 & \Partial{\tilde{U}}{k} & \Partial{\tilde{U}}{\tilde{u}} & 0 \\
0 & \Partial{\tilde{V}}{k} & 0 & \Partial{\tilde{V}}{\tilde{v}}
\end{vmatrix}
= \frac{\sqrt{k}}{\cos^2 \tfrac{\tilde{u}}{2} + k \sin^2 \tfrac{\tilde{u}}{2}} \times \frac{\sqrt{k}}{\cos^2 \tfrac{\tilde{v}}{2} + k \sin^2 \tfrac{\tilde{v}}{2}} \neq 0,
\]
so this is indeed a valid change of coordinates. The same is true for the coordinates on $\mathcal{\tilde{C}}_1(p)$ provided by Lemma~\ref{lem:mathcal tilde C}. For $p \leq 1$, we may change coordinates from $(p,q,k,\tilde{u})$ to $(p,q,k,\tilde{U})$, and for $p \geq 1$ from $(p,q,k,\tilde{v})$ to $(p,q,k,\tilde{V})$.

First let us use the coordinate $(p,k,\tilde{U},\tilde{V})$ on $\mathcal{\tilde{C}}_1$ to find the group $\mathcal{G}$ of covering transformations $\mathcal{\tilde{C}}_1 \to \mathcal{C}_1$. We can divide the covering transformations into two types: those that push forward to $\mathcal{A}_1$ to give the identity and those that push forward to give $λ$. Recall that the projection of the universal cover $\mathcal{\tilde{C}}_1$ to $\mathcal{A}_1$ in part reads $U = \tan \tilde{U}/2$. This is the standard covering of the circle by $\R$, so the covering transformations that push forward to the identity are simply
\[
\Set{ (p,k,\tilde{U},\tilde{V}) \mapsto (p,k,\tilde{U} + 2π n,\tilde{V} +2π n) }
{ n \in \Z }.
\]

To find those covering transformations of $\mathcal{G}$ that push forward to $λ$, take a point $(α,β) \in \mathcal{A}_1$ with some $U$ coordinate. Then the $\tilde{U}$ coordinate of any point of $\mathcal{\tilde{C}}_1$ that lies over $(β,α)$ must lie in
\[
2\atan \bra{ -\frac{1}{U} } + 2π\Z
= 2\bra{\pm\frac{π}{2} + \atan U} + 2π\Z
\]
and so is $2πm + π + \tilde{U}$ for some integer $m$. Therefore, these covering transformations lie within the set
\[
\Set {(p,k,\tilde{U},\tilde{V}) \mapsto (p,k, 2πm + π + \tilde{U}, 2πn + π + \tilde{V})} {m,n \in \Z}.
\]
Not every element of this set is a well defined map on $\mathcal{\tilde{C}}_1$ however, because we require that $\tilde{U} < \tilde{V} < \tilde{U} + 2π$. Hence above we must choose $m=n$. The group of covering transformations $\mathcal{G}$ is generated by
\[
\tilde{λ} : (p,k,\tilde{U},\tilde{V}) \mapsto (p,k, \tilde{U} + π, \tilde{V} + π),
\labelthis{eqn:tilde lambda action}
\]
and we may write $\mathcal{G} = \Z\langle \tilde{λ}\rangle$. If we apply this transformation twice, we see that $\tilde{λ}^2$ generates the subgroup of transformations that pushforward to the identity transformation on $\mathcal{A}_1$. It is unsurprising that this subgroup is index two in $\mathcal{G}$ because the group of covering transformations of $\mathcal{A}_1$ over $\mathcal{C}_1$ has just two elements.






If we wish to see the effect of the covering transformations as restricted to $\mathcal{\tilde{S}}_1$, we must determine how the value of $\tilde{T}$ changes when it is precomposed with $\tilde{λ}$, since $\mathcal{\tilde{S}}_1$ is a collection of its level sets.

\begin{lem}
\label{lem:T shift}
The effect of precomposing $\tilde{T}$ with $\tilde{λ}$ is to increase its value by $S-1$. That is,
\[
\tilde{T} \circ \tilde{λ} - \tilde{T}
= S-1.
\]
\begin{proof}
It is fruitful to consider first the effect of $λ$ on $T_0$ in the $ζ$-plane. Suppose that $μ$ and $ν$ are chosen such that $ν,μ,1$ and $-1$ are arranged clockwise as shown in Figure~\ref{fig:deck_T_1}. The principal choice of path $\symbf{γ}_+ = \symbf{γ}_+(α,β)$ is shown in red, whereas the principal choice after swapping the labels of the roots is shown in blue. Let it be denoted $γ'_+ = \symbf{γ}_+ \circ λ = \symbf{γ}_+(β,α)$.

% \makefigure{The principal path $\symbf{γ}_+$ in black and path $γ'_+ = \symbf{γ}_+ \circ λ$ in red. \label{fig:deck_T_1}}{thesis_graphics_temp/deck_T_1.png}
\maketikzfigure{The principal path $\symbf{γ}_+$ in red and path $γ'_+ = \symbf{γ}_+ \circ λ$ in blue. \label{fig:deck_T_1}}{tikz/deck_T_1}

The difference between these two paths is homologous to a loop anticlockwise around the upper unit circle. So by the construction of $Θ^P$,
\[
\int_{γ'_+} Θ^P - \int_{\symbf{γ}_+} Θ^P = \int_{\S^1} Θ^P = -2π\iu.
\]
Likewise if we consider the difference between the principal path $\symbf{γ}_-$ and the path $γ'_- = \symbf{γ}_- \circ λ$ we again have a anticlockwise loop of the upper unit circle, shown in Figure~\ref{fig:deck_T_-1}.
Hence
\[
\int_{γ'_-} Θ^P - \int_{\symbf{γ}_-} Θ^P = \int_{\S^1} Θ^P = -2π\iu.
\]

% \makefigure{The principal path $\symbf{γ}_-$ in black and $γ'_- = \symbf{γ}_- \circ λ$ in red. \label{fig:deck_T_-1}}{thesis_graphics_temp/deck_T_-1.png}
\maketikzfigure{The principal path $\symbf{γ}_-$ in red and $γ'_- = \symbf{γ}_- \circ λ$ in blue. \label{fig:deck_T_-1}}{tikz/deck_T_-1}

Putting these together we conclude that the value of $T_0$ changes by $1-S$ under this transformation at these points:
\begin{align*}
2π\iu (T_0 \circ λ - T_0)
&= \bra{ S\int_{γ'_-} Θ^P - \int_{γ'_+} Θ^P } - \bra{ S\int_{\symbf{γ}_-} Θ^P - \int_{\symbf{γ}_+} Θ^P } \\
&= S(-2π\iu) - (-2π\iu) \\
&= 2π\iu (1-S).
\end{align*}
To infer the effect of the transformation on $\tilde{T}$ however, we also must take into account how the coordinates $\tilde{u}$ and $\tilde{v}$ may have changed, and consequently any alteration to their winding numbers. If the points $μ$ and $ν$ have been arranged as described, then this restricts the arrangement of $\iu u$ and $\iu v$. By definition,
\[
0 = f(μ),\;\; \iu u = f(1),\;\; \iu v = f(-1),\;\; \infty = f(ν),
\]
and as one traverses the unit circle clockwise in the $ζ$-plane, one traverses the imaginary axis in the $z$-plane upwards. The clockwise arrangement of $μ$, $1$, $-1$, and $ν$ therefore corresponds to $0 < u < v < \infty$.

As $u$ and $v$ are both positive, it must be that $\tilde{u} \in (2πn, 2πn + π)$ and $\tilde{v} \in (2πm, 2πm + π)$ for some integers $n$ and $m$. Hence $\tilde{U}$ and $\tilde{V}$ also lie in those two intervals respectively. Under applying the transformation $\tilde{λ}$, the coordinate $\tilde{U}$ will be translated by $π$ and so lie in $(2π(n+1) - π, 2π(n+1))$. In other words its winding number has increased by $1$. The same can be said for $\tilde{V}$. As $\tilde{u}$ and $\tilde{v}$ have the same winding numbers as $\tilde{U}$ and $\tilde{V}$, combining the effect of $λ$ on $T_0$ with the change of winding number in \eqref{eqn:tilde T computable} shows
\begin{align*}
\tilde{T} \circ \tilde{λ} - \tilde{T}
&= \bra{ T_0\circ λ + 2(S(m+1)-(n+1)) } - \bra{ T_0 + 2(Sm-n) } \\
&= 1-S + 2(S-1) \\
&= S-1.
\end{align*}
This relation a between analytic functions on an open set, so by continuation it applies everywhere.
\end{proof}
\end{lem}

Thus the effect of the covering transformation $\tilde{λ}$ on points in the level set $\mathcal{\tilde{C}}_1(p,q)$, where $\tilde{T} = q$, is to move them into the $\mathcal{\tilde{C}}_1(p,q + p-1)$ level set. To be precise, fix a value for $p$ and take a point $(p,k,\tilde{U},\tilde{V}) \in \mathcal{\tilde{C}}_1(p)$.
If $p\leq 1$ let $\tilde{X}$ be $\tilde{U}$ and otherwise take it to be $\tilde{V}$. We know by Lemma~\ref{lem:T_graph} and the above change of coordinates that $(q,k,\tilde{X})$, where $q = \tilde{T}$, are coordinates for $\mathcal{\tilde{C}}_1(p)$. In particular, $\mathcal{\tilde{C}}_1(p)$ is foliated by the level sets $\mathcal{\tilde{C}}_1(p,q)$ of $\tilde{T}$. Under the covering transformation $\tilde{λ}$,
\[
\bra{p,q,k,\tilde{X}} \mapsto \bra{p, q + (p-1), k, \tilde{X} + π}.
\labelthis{eqn:group action}
\]
Viewing the cosets of the group of covering transformations as an equivalence relation, for $p\neq 1$ they provide an identification between the level set $\mathcal{\tilde{C}}_1(p,q)$ and the level set $\mathcal{\tilde{C}}_1(p,q + l(p-1))$ for any integer $l$.

\begin{thm}
\label{thm:topology_curves}
For $p\neq 1$, the space of marked curves $\mathcal{C}_1(p)$ is diffeomorphic to
\[
\left\{ \bra{[q],k,\tilde{X}} \in \bra{\R/ (p-1)\Z} \times (0,1) \times \R \right\},
\]
such that the subspace of spectral curves $\mathcal{S}_1(p)$ is
\[
\mathcal{S}_1(p) = \{ \mathcal{C}_1(p) \mid [q] \in \Q/ (p-1)\Z \}.
\]
\begin{proof}
Fix $p\neq 1$ and consider $\mathcal{C}_1(p)$. It is the quotient of $\mathcal{\tilde{C}}_1(p)$ by the group of covering transformations $\mathcal{G} = \Z\langle\tilde{λ}\rangle$. By Lemma~\ref{lem:T_graph}, $\mathcal{\tilde{C}}_1(p)$ is foliated by $\mathcal{\tilde{C}}_1(p,q)$ and we have just shown in~\eqref{eqn:group action} how different $\mathcal{\tilde{C}}_1(p,q)$ can be identified if their values of $q$ differ by a multiple of $p-1$. Hence it is sufficient to take one representative from each element of $\R/(p-1)\Z$ to cover the image.

Lemma~\ref{lem:tilde T rational} also demonstrates that $\mathcal{\tilde{S}}_1(p)$ is the union of those level sets $\mathcal{\tilde{C}}_1(p,q)$ with $q \in \Q$. $\mathcal{S}_1(p)$ is the image of $\mathcal{\tilde{S}}_1(p)$ under the covering map, so it is the subset of $\mathcal{C}_1(p)$ where $q$ is in the image of $\Q$, that is $q \in \Q/(p-1)\Z$.
\end{proof}
\end{thm}

This leaves just one special case, where $p=1$. We see that the action of $\tilde{λ}$ on $\mathcal{\tilde{C}}_1(1)$ fixes the value of $\tilde{T}$ and so does not identify different level sets $q=\tilde{T}$. Instead, the group action creates an equivalence relation on each level set. Specifically, the action of $\tilde{λ}$ on $\mathcal{\tilde{C}}_1(1,q)$ given by~\eqref{eqn:group action} reads
\[
\bra{1,q,k,\tilde{X}} \mapsto \bra{1, q, k, \tilde{X} + π}.
\]
and so in particular only the fourth coordinates is changed. Using these coordinates $(1,q,k,\tilde{X})$ on $\mathcal{\tilde{C}}_1(1)$, it is trivial to deduce the quotient by $\mathcal{G}$.

\begin{thm}
\label{thm:topology_curves_p1}
The space of marked curves $\mathcal{C}_1(1)$ is diffeomorphic to
\[
\left\{ \bra{q,k,\left[\tilde{X}\right]} \in \R \times (0,1) \times \R/π\Z \right\},
\]
the product of $\R$ and an annulus, such that the subspace of spectral curves $\mathcal{S}_1(1)$ is the restriction of the first component of the product to $\Q$.

\begin{proof}
The orbit of a point $(1,q,k,\tilde{X})$ of $\mathcal{\tilde{C}}_1(1)$ is
\[
\bra{1,q,k,\tilde{X}+π\Z},
\]
so the quotient sends $\tilde{X} \in \R$ to $\left[\tilde{X}\right] \in \R/π\Z$.

As in the previous theorem, by construction of the coordinates $(p,q,k,\tilde{X})$ and Lemma~\ref{lem:tilde T rational}, a point is in $\mathcal{S}_1 \subset \mathcal{C}_1$ exactly when it is the image of a point where $p$ and $q$ are rational. As $p=1$ in this case, the points of $\mathcal{S}_1(1)$ are those where $q\in\Q$.
\end{proof}
\end{thm}

These two theorems complete our quest to gain an understanding of the subspace of spectral curves $\mathcal{S}_1$ within $\mathcal{C}_1$. First we characterised when a marked curve is a spectral curve by way of a necessary and sufficient condition: that the two real valued functions $S(α,β)$ and $T(α,β)$ take rational values. The first condition, $S\in\Q^+$, lead immediately to a dense disjoint collection of subspaces $\mathcal{C}_1(p)$ on which the condition was met, where $p$ was the value of $S$.

Then we moved to the universal cover $\mathcal{\tilde{C}}_1$, so that we could construct a well defined pullback $\tilde{T}$ of the multi-valued function $T$. We then considered the subspaces of $\mathcal{\tilde{C}}_1(p)$ where $\tilde{T}$ was constant. These were shown to be ribbons in Lemma~\ref{lem:T_graph}. The function $\tilde{T}$ is rational exactly when $T$ is, so the preimage of $\mathcal{S}_1$ in its universal cover is the union of the level sets $\mathcal{\tilde{C}}_1(p,q)$ of $p = S$ and $q = \tilde{T}$ where $p$ and $q$ are rational.

The final step was to push these level sets back down to $\mathcal{C}_1$. To do this, one must quotient out by the action of the group of covering transformations. This group $\mathcal{G}$ was shown to be $\Z\langle\tilde{λ}\rangle$ and its action, described by~\eqref{eqn:group action}, was that of translations. This allowed us to describe the quotients in the two Theorems~\ref{thm:topology_curves} and~\ref{thm:topology_curves_p1}. For $p$ not equal to one, $\mathcal{S}_1(p)$ is a dense collection of ribbons within $\mathcal{C}_1$. As can be seen in Figures~\ref{fig:p05 plot}--\ref{fig:p2 plot} below, they should be thought of as being intertwined around a central axis, similar to a family of helicoids. But for $p$ equal to one, instead we have a dense collection of annuli $\mathcal{S}_1(1)$.

Unlike helicoids however, cross-sections perpendicular to the central axis to not meet the axis. Instead they spiral infinitely closer. In this respect, the behave like the cone of a spiral. This aspect is especially prominent in Figure~\ref{fig:p1 plot}, which one could think of as a family of cones with a common vertex (though the plotting software has difficulty for $k\approx 1$, so the `cones' appear to be heading towards the white hole in the middle of the figure but are truncated. The centre of this white hole is the vertex). This vertex structure is proved in Section \ref{sec:Interior}.

We saw in the introduction to this chapter that the parameter $p$ can be thought of as controlling the slope of the level sets as they wind around the central axis. From this point of view, $p=1$ is the intermediate case between right and left handed spirals where the slope is `flat' and the level sets `close up'.

\begin{figure}
    \includegraphics[width=\textwidth]{thesis_graphics/moduli_plot_p05.png}
    \caption{One component of the moduli space $\mathcal{S}_1(0.5)$. Notice that the ribbon is wrapped in a left handed direction around a central axis.}
    \label{fig:p05 plot}
\end{figure}

\begin{figure}
    \includegraphics[width=\textwidth]{thesis_graphics/moduli_plot_p1.png}
    \caption{Select components of the moduli space $\mathcal{S}_1(1)$, namely those components on which $T$ is $-3$, $-1$, $0$, $1$, or $3$. The component on which $T$ is $0$ is the disc in the middle.}
    \label{fig:p1 plot}
\end{figure}

\begin{figure}
    \includegraphics[width=\textwidth]{thesis_graphics/moduli_plot_p2.png}
    \caption{One component of the moduli space $\mathcal{S}_1(2)$. Notice, contra Figure~\ref{fig:p05 plot}, that the ribbon is wrapped in a right handed direction around a central axis.}
    \label{fig:p2 plot}
\end{figure}



\section{Corollaries}
\label{sec:Corollaries}

In the last section, we deduced the topology of the path components of the moduli space of spectral curves. Most components are ribbons, but the components of $\mathcal{S}_1(1)$ are annuli. In this section we will first investigate the moduli space $\mathcal{M}_1$ of spectral data $(Σ,Θ^1,Θ^2)$. For $p\neq 1$, over each component of $\mathcal{S}_1(p)$ it is a trivial bundle. Its bundle structure over the components of $\mathcal{S}_1(1)$ is more complicated, but we will prove that the total space of the bundle is simply connected.

Second, we shall extend the symmetry exhibited in~\eqref{eqn:T_0 symmetry} to a general transformation $χ$ on the space of spectral curves of a fixed genus $\mathcal{S}_g$ and give a geometric interpretation. We will illustrate this interpretation by observing how it applies to the space $\mathcal{S}_0$ of spectral curves with genus zero. Finally, we will examine the special case of harmonic maps to a $2$-sphere and show that those with a genus one spectral curve can be identified with a particular path component of $\mathcal{S}_1(1)$.

Before we can prove results about the moduli space of spectral data, we must revisit and improve on Lemma~\ref{lem:closing_conds}. In that lemma, we saw that the rationality of the values of the functions $S$ and $T$ are necessary and sufficient conditions for the existence of spectral data. The proof was constructive, in that it finds a set of spectral data on any marked curve that meets both conditions. However, to examine the moduli of spectral data, we must find a `minimal' set of spectral data from which all other on that curve may generated.

\begin{lem}
\label{lem:minimal differentials}
On any curve $Σ \in \mathcal{S}_1$, there is a pair of differentials $Ψ^E$ and $Ψ^P$ such that every differential that satisfies conditions~\ref{P:poles}--\ref{P:closing} is an integer combination of that pair.

\begin{proof}
The method of proof is similar to that of Lemma~\ref{lem:closing_conds}. As $Σ$ is a spectral curve, if we fix paths $γ_+$ and $γ_-$ we know that the functions $S$ and $T$ take rational values. Let $S = n/m$ and $T= n'/m'$, where $n$ and $m$ are coprime. Recall the definitions of $Ψ^E$ and $Ψ^P$ as derived in that lemma, namely $Ψ^E = a Θ^E$ and $Ψ^P = b Θ^E + l Θ^P$, for constants
\begin{align*}
a &:= \frac{2π\iu n}{2\iu η^+(1)}, \\
l &:= \frac{m'}{\gcd(m',mn')}, \\
b &:= \frac{1}{2\iu η^+(1)}\bra{ 2π\iu \frac{mn'}{\gcd(m',mn')}y - \frac{m'}{\gcd(m',mn')} \int_{γ_+} Θ^P }.
\labelthis{eqn:def Psi coeff}
\end{align*}
We shall prove this lemma first for exact differentials. Suppose that $Θ$ is an exact differential meeting conditions~\ref{P:poles}--\ref{P:closing}. We assert that it must be an integer multiple of $Ψ^E$. To see this, note that it is a real scalar of $Θ^E$ and compute its integrals over $γ_+$ and $γ_-$
\begin{align*}
\int_{γ_+} Θ &= a' \int_{γ_+} Θ^E = 2\iu η^+(1) a' =: 2π\iu Γ_+ \\
\int_{γ_-} Θ &= a' \int_{γ_-} Θ^E = -2\iu η^+(-1) a' =: 2π\iu Γ_-,
\end{align*}
for some $a'\in\R$ and $Γ_+, Γ_- \in \Z$. By the definition of $S(α,β)$
\[
S = \frac{n}{m} = - \frac{η^+(1)}{η^+(-1)} = \frac{Γ_+}{Γ_-},
\]
so since $n/m$ is a simplified fraction we must have that $Γ_+ = cn$ and $Γ_- = cm$ for an integer $c$. It follows that $a' = ca$ and hence $Θ = c Ψ^E$.

% We now turn our attention to a non-exact differentials. Similar to Lemma~\ref{lem:closing_conds}, let $x$ and $y$ be integers with $nx-my = 1$. Then
% \[
% \frac{nΓ^- - m Γ^+}{gm}
% = \frac{n n'mx - m n'my}{m'm}
% = \frac{n'(nx - my)}{m'}
% = \frac{n'}{m'},
% \]
% So we can solve~\eqref{eqn:period differential scaling} for a differential $Ψ^P$ with period $2π\iu g$ and integrals $2π\iu Γ^+, 2π\iu Γ^-$.

Now if $Θ$ is any differential that satisfies the closing conditions, we may write it as $Θ = b'Θ^E + l'Θ^P$ for $b'\in\R$ and $l'\in \Z$. Its imaginary period is $2π\iu l'$ and let its integrals over $γ_+$ and $γ_-$ be $2π\iu Γ_+$ and $2π\iu Γ_-$. Similar to Lemma~\ref{lem:closing_conds}, eliminating $b'$ from~\eqref{eqn:period closing cond} leads to
\begin{align*}
% \frac{n'}{m'} &= \frac{nΓ_- - mΓ_+}{l'm} \\
l'mn' &= m'(nΓ_- - mΓ_+) \\
l'\frac{mn'}{\gcd(m',mn')} &= \frac{m'}{\gcd(m',mn')}(nΓ_- - mΓ_+) \\
&= l(nΓ_- - mΓ_+)
\end{align*}
Now, by construction $l$ is coprime to $mn'$. Hence we see that $l$ must divide $l'$. Finally, consider the differential
\[
Θ - \frac{l'}{l}Ψ^P.
\]
It meets conditions~\ref{P:poles}--\ref{P:closing} and is exact, so must be an integer multiple of $Ψ^E$ by the first part of this proof. Rearranging, we have written $Θ$ as an integer combination of $Ψ^E$ and $Ψ^P$.
\end{proof}
\end{lem}

Morally, the differentials $Ψ^E$ and $Ψ^P$ should be used locally as a frame for the space of differentials meeting~\ref{P:poles}--\ref{P:closing}. These differentials form a lattice, so each fibre of the bundle $\mathcal{M}_1$ over $\mathcal{S}_1$ is a discrete space. The global structure of the bundle is therefore given by the monodromy action on the lattice; which determines fibres connect to one another.

Before we can make this precise however there is a technical point that must be addressed, namely that $\mathcal{S}_1$ is only an immersed submanifold.
This causes issues if one tries to describe bundles over $\mathcal{S}_1$ as subspaces of bundles over $\mathcal{C}_1$.
For example, we demonstrated at the end of Section~\ref{sec:Differentials}, using Lemma~\ref{lem:theta2_characterisation}, that the differentials meeting conditions~\ref{P:poles}--\ref{P:imaginary periods} form a rank two vector bundle $\mathcal{B}_1$ over $\mathcal{C}_1$ framed by $\langle Θ^E,Θ^P \rangle$.
If we try to consider $Ψ^E$ as a section of this bundle restricted to $\mathcal{S}_1$ then it is not even a continuous function!
Indeed, consider a sequence of points
\[
Σ_j \in \mathcal{S}_1\bra{1+j^{-1}},
\]
with a limit $Σ_\infty \in \mathcal{S}_1(1)$. The integers $j$ and $j+1$ are always coprime, so in the above notation $S = n/m = (j+1)/j$ and the differential $Ψ^E_j$ on $Σ_j$ is defined by
\[
Ψ^E_j = \frac{2π\iu (j+1)}{2\iu η^+(1)} Θ^E,
\]
which does not have a well defined limit, whereas on the other hand
\[
Ψ^E_\infty = \frac{2π\iu}{2\iu η^+(1)} Θ^E.
\]

Instead, we must consider bundles over each path connected component of $\mathcal{S}_1$ separately, each of which is an embedded submanifold of $\mathcal{C}_1$. This is the more natural choice anyway if one recalls that deformations have been defined to be paths in the moduli space.
Let us label the path connected components of $\mathcal{S}_1$. Following Theorem~\ref{thm:topology_curves}, for each $p\in \Q^+\setminus\{1\}$ and $[q] \in \Q / (p-1) \Z$, there is a ribbon-like component $\mathcal{S}_1(p,[q]) \diffeo (0,1)\times\R$. By Theorem~\ref{thm:topology_curves_p1}, for $p=1$ there is an annuli component $\mathcal{S}_1(1,q)$ for each rational number $q$.

To examine the local triviality of the bundle $\mathcal{M}_1$ over each component, fix a path connected component $\mathcal{X}$ of $\mathcal{S}_1$. For any fixed value of $p \in \Q^+$, the differential $Ψ^E$ is a well-defined smooth function on $\mathcal{C}_1(p)$, and as $\mathcal{X}$ is entirely contained within some $\mathcal{C}_1(p)$ it follows that $Ψ^E$ is well-defined and smooth on $\mathcal{X}$. The same is not necessarily true for $Ψ^P$.
Locally though, in a simply connected neighbourhood $\mathcal{V} \subset \mathcal{X}$ it is possible to choose paths $γ_+$ and $γ_-$ on each spectral curve that vary smoothly with changes of the branch points. Hence by Lemma~\ref{lem:minimal differentials}, the moduli space of differentials satisfying~\ref{P:poles}--\ref{P:closing} is the trivial $\Z^2$-bundle over $\mathcal{V}$ framed by $Ψ^E$ and $Ψ^P$.

From this we may establish the local structure of $\mathcal{M}_1$.
We recall from Chapter~\ref{chp:Genus Zero} the definition of the integer matrices $\Mat_2^*\Z = \Set{ M \in \Mat_2\Z }{\det M \neq 0}$. There the moduli space of spectral data with a genus zero spectral curve was described as the product $\mathcal{M}_0 = D \times \Mat_2^*\Z$. Similarly, we may describe the moduli space of spectral data whose spectral curve lies in $\mathcal{V}$ as $\mathcal{V} \times \Mat_2^*\Z$, where the differentials $(Θ^1,Θ^2)$ on any curve may be described by a matrix via
\[
\begin{pmatrix}
Θ^1 \\ Θ^2
\end{pmatrix}
=
\begin{pmatrix}
b_1 & l_1 \\
b_2 & l_2
\end{pmatrix}
\begin{pmatrix}
Ψ^E \\ Ψ^P
\end{pmatrix},
\]
using the frame $(Ψ^E,Ψ^P)$. The non-vanishing of the determinant of this matrix is equivalent to Condition~\ref{P:linear independence}, that the pair of differentials $(Θ^1,Θ^2)$ is linearly independent.

With the local structure established, we can now turn our attention to the global structure of $\mathcal{M}_1$. Just as we decomposed $\mathcal{S}_1$ into path components $\mathcal{S_1}(p,[q])$ and $\mathcal{S}_1(1,q)$, we may similarly decompose $\mathcal{M}_1$ according to which component its spectral curve belongs. As the path connected components $\mathcal{X} = \mathcal{S}_1(p,[q])$ are simply connected, immediately we have that
\[
\mathcal{M}_1(p, [q]) = \mathcal{S}_1(p,[q]) \times \Mat_2^*\Z.
\]

For $p=1$, let us fix a path component $\mathcal{S}_1(1,q)$ of $\mathcal{S}_1(1)$.
This component is an annulus and not simply connected. The differential $Ψ^P$ is not well-defined on all of $\mathcal{S}_1(1,q)$ because its definition depends on a choice of paths $γ_+$ and $γ_-$, and it is not possible to make a choice of paths that varies continuously as the branch points of the spectral curve are moved. For example, recall the family of curves $t \mapsto Σ(0.5 e^{\iu t}, -0.5 e^{\iu t})$ in $\mathcal{C}_1$. If one takes $γ_+$ to be a path that wraps around $ζ = 0.5$ on $Σ(0)$ (as the principal choice of path $\symbf{γ}_+$ does), then as $t$ is increased to $π$, the path now passes around the other branch point $ζ=-0.5$ on the same spectral curve $Σ(π) = Σ(0)$.

Consider therefore a nontrivial loop $\ell : [0,1] \to \mathcal{S}_1(1,q)$ that wraps around the annulus once. If we make a choice of $γ_+(t)$ and $γ_-(t)$ on each $Σ(\ell(t))$ such that the paths vary smoothly in $t$ then note in particular that $γ_+(0)$ and $γ_+(1)$ will be different paths on the same curve. One may ask: if we construct $Ψ^P$ using $γ_+(t)$ and $γ_-(t)$ then what will be the change in $Ψ^P$ when we return to $\ell(1) = \ell(0)$?

From~\eqref{eqn:group action} it follows that $T = n'/m'$ is well defined and constant on the whole annulus. We may simplify so that it is a reduced fraction. Writing $S=1=n/m$ as a reduced fraction implies that $m=n=1$. The period $l$ of $Ψ^P$ is therefore $m' / \gcd(m',mn') = m'$.
Let $Ψ^P(0) = b(0)Θ^E + m' Θ^P$ be the differential on $Σ(\ell(0))$. As the periods of the differential are integral, they cannot change along the path $\ell$ and so
\[
Ψ^P(t) = b(t) Θ^E + m' Θ^P.
\]
As we move from the start of $\ell$ to the end, from our previous computation in Lemma~\ref{lem:T shift}, each of
\[
\int_{γ_+(t)} Θ^P \;\text{ and } \int_{γ_-(t)} Θ^P
\]
will be incremented or decremented (depending on the orientation of $\ell$) by $2π\iu$. The difference $Ψ^P(1) - Ψ^P(0) = (b(1)-b(0))Θ^E$ can be explicitly computed from~\eqref{eqn:def Psi coeff} as
\begin{align*}
b(1)
&= \frac{1}{2\iu η^+(1)}\bra{ 2π\iu y n'm - m' \bra{\int_{γ_+(1)} Θ^P} } \\
&= \frac{1}{2\iu η^+(1)}\bra{ 2π\iu y n'm - m' \bra{\int_{γ_+(0)} Θ^P + 2π\iu} } \\
&= b(0) - \frac{2π\iu}{2\iu η^+(1)}m' \\
&= b(0) - am',
\end{align*}
so that $Ψ^P(1) - Ψ^P(0) = -am' Θ^E = - m' Ψ^E$. This equations shows that every time you loop around the annulus, the non-exact differential $Ψ^P$ shifts by $m' Ψ^E$.

Translating the shift of $Ψ^P$ into a statement about $\Mat_2^*\Z$, given a tuple of spectral data $(Σ,Θ^1,Θ^2)$ in $\mathcal{M}_1(1,q)$, we may write $Θ^i = b_i Ψ^E + l_i Ψ^P$ for integers $b_i$ and $l_i$. One is free to vary $Σ$ within $\mathcal{S}_1(1,q)$, but the effect of looping around on the differentials is
\begin{align*}
\begin{pmatrix}
Θ^1 \\ Θ^2
\end{pmatrix}
&=
\begin{pmatrix}
b_1 & l_1 \\
b_2 & l_2
\end{pmatrix}
\begin{pmatrix}
Ψ^E \\ Ψ^P
\end{pmatrix} \\
&\mapsto
\begin{pmatrix}
b_1 & l_1 \\
b_2 & l_2
\end{pmatrix}
\begin{pmatrix}
    Ψ^E \\ Ψ^P - m' Ψ^E
    \end{pmatrix} \\
&=
\begin{pmatrix}
b_1 & l_1 \\
b_2 & l_2
\end{pmatrix}
\begin{pmatrix}
    1 & 0 \\
    -m' & 1
\end{pmatrix}
\begin{pmatrix}
    Ψ^E \\ Ψ^P
    \end{pmatrix}.
\end{align*}
In short, if $B_q$ is the subgroup of $\Mat_2^*\Z$ of matrices of the form
\[
\begin{pmatrix}
1 & 0 \\
m'\Z & 1
\end{pmatrix},
\]
then the connected components of $\mathcal{M}_1(1,q)$ are enumerated by the right $B_q$-orbits of $\Mat_2^*\Z$.

Finally, we can prove the space $\mathcal{M}_1(1,q)$ is simply connected. Given any closed path $\ell$ in it, we may project this loop down to $\mathcal{S}_1(1,q)$. If the projection of the loop is null-homotopic, then it is contained in a simply connected neighbourhood $\mathcal{V}$ of $\mathcal{S}_1(1,q)$ and we may use the frame $\langle Ψ^E,Ψ^P \rangle$ to lift to a null-homotopy of $\ell$ in $\mathcal{M}_1(1,q)$. If the projection is non-trivial, then it winds a certain number of times around the annulus. The above calculation shows that if the differentials are unchanged from the beginning to end of the path then either $m' = 0$ or $l_1=l_2 = 0$. The former is excluded because by definition $n'/m' = q$ and so $m'$ is never zero. The latter implies that the differentials are both multiples of $Θ^E$, which by~\ref{P:linear independence} contradicts their linear independence. This demonstrates that every closed path in $\mathcal{M}_1(1,q)$ is null-homotopic, and so it is simply connected.

It is therefore the case that each connected component of $\mathcal{M}_1(1,q)$ is diffeomorphic to the universal cover of the annulus $\mathcal{S}_1(1,q)$, which is to say that each of them is a ribbon $(0,1)\times \R$. The connected components of $\mathcal{M}_1(p,[q])$ were also ribbons. In summary, $\mathcal{M}_1$ is the disjoint union
\begin{gather*}
\mathcal{M}_1
=
\coprod_{q \in \Q} \mathcal{M}_1(1,q)
\;\;\amalg \coprod_{\substack{p \in \Q^+,\, p \neq 1\\ [q] \in \Q/(p-1)\Z}}
\mathcal{M}_1(p,[q]) \\
=
\coprod_{q \in \Q} \Big[ (0,1)\times\R \times \bra{\Mat_2^*\Z / B_q} \Big]
\;\amalg
\coprod_{\substack{p \in \Q^+,\, p \neq 1\\ [q] \in \Q/(p-1)\Z}}
\Big[ (0,1)\times\R \times \Mat_2^*\Z \Big].
\end{gather*}


There are some symmetries and special cases that also bear mention. First, we have already remarked upon, and made use of, the symmetry in~\eqref{eqn:T_0 symmetry},
\[
T_0(p,k,u,v) = -p T_0\bra{ \tfrac{1}{p}, k, v, u },
\]
but how should one interpret it geometrically? Looking at the transformation
\[
p = \frac{\abs{1-α}\abs{1-β}}{\abs{1+α}\abs{1+β}}
\mapsto \frac{1}{p} = \frac{\abs{1+α}\abs{1+β}}{\abs{1-α}\abs{1-β}},
\]
the natural guess would be that it is induced by
\begin{align*}
χ: \mathcal{A}_1 &\to \mathcal{A}_1 \\
(α,β) &\mapsto (-α,-β).
\labelthis{eqn:def chi}
\end{align*}
Indeed this can be seen to be the case, as $k$ is invariant under such a transformation, and the associated map between the spectral curves
\begin{align*}
χ_{(α,β)}: Σ(α,β) &\to Σ(-α,-β) \\
(ζ, η) &\mapsto (-ζ,-η)
\end{align*}
interchanges $1$ and $-1$, in effect swapping the roles of $u = -\iu f(1)$ and $v = -\iu f(-1)$. The pullback of the differentials under the map $χ_{(α,β)}$ preserves the integrality of the periods and the integrals over $γ_+$ and $γ_-$ and so spectral data on $Σ(α,β)$ is transformed into spectral data on $Σ(-α,-β)$.
The exact same reasoning applies in general to marked curves of any genus,
\begin{align*}
χ:&& \mathcal{A}_g &\to \mathcal{A}_g \\
&&(α_1,α_2,\dots,α_g) &\mapsto (-α_1,-α_2,\dots,-α_g), \\
χ_{(α_1,α_2,\dots,α_g)}:&& Σ(α_1,α_2,\dots,α_g) &\to Σ(-α_1,-α_2,\dots,-α_g) \\
&&(ζ, η) &\mapsto (-ζ,\iu^g η).
\end{align*}
The harmonic map $g(z) : \mathbb{T}^2 \to \S^3$ arises from the spectral data as the gauge transformation between the connections corresponding to $ζ=1$ and $ζ=-1$ in~\eqref{eqn:flat connections}, so exchanging these points with $χ_{(α_1,α_2,\dots,α_g)}$ gives the inverted map $g(z)^{-1}$. It is also harmonic \cite[Prop~8.2]{Uhlenbeck1989}.

This can be seen directly in the genus zero case, which was treated in Chapter~\ref{chp:Genus Zero}. Recall that the equation of any harmonic map corresponding to spectral data with a genus zero spectral curve may, as in~\eqref{eqn:genus zero simple map}, be written
\[
g(w_R + \iu w_I) = \exp (-4w_R X) \exp (4w_I Y),
\]
for
\[
X = \norm{X}\begin{pmatrix}
0 & 1 \\
-1 & 0
\end{pmatrix}, \quad
Y = \norm{Y}\begin{pmatrix}
0 & e^{\iu δ} \\
-e^{-\iu δ} & 0
\end{pmatrix}.
\]
The norms, $\norm{X}$ and $\norm{Y}$, come from the identification of $\su_2$ with $\R^3$ using the standard basis $\{σ_1,σ_2,σ_3\}$ defined in~\eqref{eqn:su2 basis}. Geometrically, $δ$ is the angle between $X$ and $Y$. Under inversion,
\[
g(w)^{-1} = \exp (-4w_I Y) \exp (4w_R X).
\]
To bring this back into the form of~\eqref{eqn:genus zero simple map}, we must perform two operations. First, we must change coordinates on the domain so that the real part is in the first factor. The multiplication $\tilde{w} = \iu w$ accomplishes this:
\[
g(\tilde{w})^{-1} = \exp (4\tilde{w}_R Y) \exp (4\tilde{w}_I X).
\]
Second we must rotate the image so that $-Y$ is aligned with $σ_2$ and $X$ lies in the plane spanned by $σ_2$ and $σ_3$. This may be achieved by $\SU(2)$ conjugation:
\begin{align*}
&\begin{pmatrix}
\iu e^{\iu δ/2} & 0 \\
0 & -\iu e^{-\iu δ/2}
\end{pmatrix}^{-1}
\norm{Y}\begin{pmatrix}
0 & -e^{\iu δ} \\
e^{-\iu δ} & 0
\end{pmatrix}
\begin{pmatrix}
\iu e^{\iu δ/2} & 0 \\
0 & -\iu e^{-\iu δ/2}
\end{pmatrix}
=
\norm{Y}\begin{pmatrix}
0 & 1 \\
-1 & 0
\end{pmatrix}
\\
&\begin{pmatrix}
\iu e^{\iu δ/2} & 0 \\
0 & -\iu e^{-\iu δ/2}
\end{pmatrix}^{-1}
\norm{X}\begin{pmatrix}
0 & 1 \\
-1 & 0
\end{pmatrix}
\begin{pmatrix}
\iu e^{\iu δ/2} & 0 \\
0 & -\iu e^{-\iu δ/2}
\end{pmatrix}
=
\norm{X}\begin{pmatrix}
0 & e^{\iu(π- δ)} \\
-e^{\iu (π-δ)} & 0
\end{pmatrix}.
\end{align*}
Thus the angle parameter has become $π-δ$. As an aside, as seen in Figure~\ref{fig:genus0 linked} this angle parameter determines the image of the harmonic map up to an $SO(4)$ rotation of $\S^3$. In particular, these maps $g$ and $g^{-1}$ have congruent images.

Recall that the parameter $x$ of $g$ was defined by~\eqref{eqn:conformal type} to be $\norm{Y}/\norm{X}$. The value of this ratio is inverted for $g^{-1}$. Now using~\eqref{eqn:def branch point genus zero} to determine the branch point of the spectral curve associated to $g^{-1}$ we have
\[
\frac{\frac{1}{x} e^{\iu (π-δ)} - \iu}{\frac{1}{x} e^{\iu (π-δ)} + \iu}
= \frac{-e^{-\iu δ} - \iu x}{-e^{-\iu δ} + \iu x}
= \frac{-1 - \iu x e^{\iu δ}}{-1 + \iu x e^{\iu δ}}
= -\frac{x e^{\iu δ}-\iu}{x e^{\iu δ} + \iu} = -α,
\]
which is to say the branch point of the spectral curve associated to $g^{-1}$ is the negative of the branch point of the spectral curve associated to $g$, as asserted above.

% One can observe these same effects at the level of spectral data. The map between $Σ(α)$ and $Σ(-α)$ induced by $ζ \mapsto -ζ$ is $χ(ζ,η) = (-ζ,\iu η)$. Pulling back the basis of differentials, defined by~\eqref{eqn:genus zero differential basis},
% \begin{align*}
% χ^* Ψ^1
% &= d\, \Big\{ (-ζ)^{-1}(r^1 + \bar{r^1}(-ζ)) \iu η \Big\} \\
% &= d\, \Big\{ ζ^{-1}(-\iu r^1 + \iu\bar{r^1}ζ) η \Big\} \\
% χ^* Ψ^2
% &= d\, \Big\{ ζ^{-1}(-\iu r^2 + \iu\bar{r^2}ζ) η \Big\},
% \end{align*}
% where
% \begin{align*}
% r^1 &= \frac{π}{2}\bra{ \frac{1}{\abs{1+α}} + \iu \frac{1}{\abs{1-α}} } \\
% r^2 &= \frac{π}{2}\bra{ \frac{1}{\abs{1+α}} - \iu \frac{1}{\abs{1-α}} }.
% \end{align*}
% If $\tilde{r}^1, \tilde{r}^2$ represents the generators of lattice of principal parts of the differentials on $Σ(-α)$, then
% \[
% -\iu r^1
% = -\iu \frac{π}{2}\bra{ \frac{1}{\abs{1+α}} + \iu \frac{1}{\abs{1-α}} }
% = \frac{π}{2}\bra{ \frac{1}{\abs{1+(-α)}} - \iu \frac{1}{\abs{1-(-α)}} }
% = \tilde{r}^2,
% \]
% and
% \[
% -\iu r^2
% = -\iu \frac{π}{2}\bra{ \frac{1}{\abs{1+α}} - \iu \frac{1}{\abs{1-α}} }
% = -\frac{π}{2}\bra{ \frac{1}{\abs{1+(-α)}} + \iu \frac{1}{\abs{1-(-α)}} }
% = -\tilde{r}^1.
% \]
% Recall the interpretation of span of the principal parts as the universal cover of the domain, with the lattice generated by these differentials corresponding to tthe lattice of periods. The previous equation then shows a rotation by $\iu$, corresponding to how we had to define $\tilde{w} = \iu w$.

Returning to genus one spectral curves, of special interest are the spectral curves that are a fixed point of this transformation $χ: Σ(α,β) \mapsto Σ(-α,-β)$. For these spectral curves $χ_{(α,β)}$ is an extra involution and hence $β=-α$. These are exactly the genus one marked curves that would meet the conditions, if they admit spectral data, for the associated harmonic map to have a totally geodesic two-sphere as its image (see discussion at end of Section~\ref{sec:construction}). Hitchin \cite[p693]{Hitchin1990} identifies a particular one parameter family of these maps as the Gauss maps of Delaunay surfaces. We shall show that all such marked curves in fact admit spectral data and further identify in which component of $\mathcal{S}_1$ they reside.

Suppose that $Σ(α,-α)$ is a curve that is fixed by $χ$. As $χ$ sends $p \mapsto p^{-1}$, it follows that $p$ is one. To show that $Σ(α,-α)$ admits spectral data it remains to shows that $T$ is rationally valued at this point $(α,-α) \in \mathcal{A}_1$.
Computing the value of $T$ from its principal branch cut $T_0$ requires us to know which coordinates $(1,k,u,v)$ correspond to $(α,-α)$.
The annulus $\{ (α,-α) \in \mathcal{A}_1\}$ is two-dimensional, but there are three parameters $(k,u,v)$, so there must be some relation between them. As the four branch points lie on a line it follows that $μ=\hat{α}$ and $ν = -\hat{α}$, where $\hat{α}$ is the unit vector of $α$. Directly from~\eqref{eqn:def z0} and~\eqref{eqn:def_k},
\[
z_0 = \frac{1+ \abs{α}}{1-\abs{α}},\quad
k = \bra{ \frac{1- \abs{α}}{1+\abs{α}} }^2.
\]
It follows from~\eqref{eqn:f} that
\[
\iu v = f(-1)
= \bra{\frac{1+ \abs{α}}{1-\abs{α}}}^2 \bra{ \frac{1+ \abs{α}}{1-\abs{α}} \frac{1+\hat{α}}{1-\hat{α}}}^{-1}
= \frac{1}{k} f(1)^{-1}
= \frac{1}{k}(\iu u)^{-1},
\]
or concisely that $v= - (ku)^{-1}$. But $u \mapsto -(ku)^{-1}$ is exactly the formula for the change of $u$ under the label swapping involution $λ$, described by~\eqref{eqn:def_lambda}. Thus we see that $χ$ and $λ$ act in the same way on $\{(α,-α) \in \mathcal{A}_1)$, as one would expect because they are both swapping $α$ and $-α$, the two branch points inside the unit disc. Precomposing $T_0$ with $λ$ shifts its value by $1-p$, which in this case is zero. Hence using this together with the fact that precomposing $T_0$ with $χ$ negates the function shows that at a point of $\{(α,-α) = (1,k,u,-(ku)^{-1})\in \mathcal{A}_1\}$,
\begin{align*}
T_0\bra{1,k,u,-(ku)^{-1}}
&= T_0\bra{1,k,-(ku)^{-1},u} + 1-p \\
% &= T_0\bra{1,k,-(ku)^{-1},u} \\
&= -T_0\bra{1,k,u,-(ku)^{-1}},
\end{align*}
from which we deduce that $T_0$ is zero. Conversely, the disjoint annuli that constitute $\mathcal{S}_1(1)$ are determined uniquely by the value $T_0$ takes on them.
Therefore we have shown that $\{Σ(α,-α) \in \mathcal{C}_1\}$ is the only annulus of $\mathcal{S}_1(1)$ where $T_0$ is zero, and these are exactly the harmonic maps to the sphere with a genus one spectral curve.
